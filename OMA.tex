% Options for packages loaded elsewhere
\PassOptionsToPackage{unicode}{hyperref}
\PassOptionsToPackage{hyphens}{url}
%
\documentclass[
]{book}
\usepackage{amsmath,amssymb}
\usepackage{lmodern}
\usepackage{iftex}
\ifPDFTeX
  \usepackage[T1]{fontenc}
  \usepackage[utf8]{inputenc}
  \usepackage{textcomp} % provide euro and other symbols
\else % if luatex or xetex
  \usepackage{unicode-math}
  \defaultfontfeatures{Scale=MatchLowercase}
  \defaultfontfeatures[\rmfamily]{Ligatures=TeX,Scale=1}
\fi
% Use upquote if available, for straight quotes in verbatim environments
\IfFileExists{upquote.sty}{\usepackage{upquote}}{}
\IfFileExists{microtype.sty}{% use microtype if available
  \usepackage[]{microtype}
  \UseMicrotypeSet[protrusion]{basicmath} % disable protrusion for tt fonts
}{}
\makeatletter
\@ifundefined{KOMAClassName}{% if non-KOMA class
  \IfFileExists{parskip.sty}{%
    \usepackage{parskip}
  }{% else
    \setlength{\parindent}{0pt}
    \setlength{\parskip}{6pt plus 2pt minus 1pt}}
}{% if KOMA class
  \KOMAoptions{parskip=half}}
\makeatother
\usepackage{xcolor}
\usepackage{color}
\usepackage{fancyvrb}
\newcommand{\VerbBar}{|}
\newcommand{\VERB}{\Verb[commandchars=\\\{\}]}
\DefineVerbatimEnvironment{Highlighting}{Verbatim}{commandchars=\\\{\}}
% Add ',fontsize=\small' for more characters per line
\usepackage{framed}
\definecolor{shadecolor}{RGB}{248,248,248}
\newenvironment{Shaded}{\begin{snugshade}}{\end{snugshade}}
\newcommand{\AlertTok}[1]{\textcolor[rgb]{0.94,0.16,0.16}{#1}}
\newcommand{\AnnotationTok}[1]{\textcolor[rgb]{0.56,0.35,0.01}{\textbf{\textit{#1}}}}
\newcommand{\AttributeTok}[1]{\textcolor[rgb]{0.77,0.63,0.00}{#1}}
\newcommand{\BaseNTok}[1]{\textcolor[rgb]{0.00,0.00,0.81}{#1}}
\newcommand{\BuiltInTok}[1]{#1}
\newcommand{\CharTok}[1]{\textcolor[rgb]{0.31,0.60,0.02}{#1}}
\newcommand{\CommentTok}[1]{\textcolor[rgb]{0.56,0.35,0.01}{\textit{#1}}}
\newcommand{\CommentVarTok}[1]{\textcolor[rgb]{0.56,0.35,0.01}{\textbf{\textit{#1}}}}
\newcommand{\ConstantTok}[1]{\textcolor[rgb]{0.00,0.00,0.00}{#1}}
\newcommand{\ControlFlowTok}[1]{\textcolor[rgb]{0.13,0.29,0.53}{\textbf{#1}}}
\newcommand{\DataTypeTok}[1]{\textcolor[rgb]{0.13,0.29,0.53}{#1}}
\newcommand{\DecValTok}[1]{\textcolor[rgb]{0.00,0.00,0.81}{#1}}
\newcommand{\DocumentationTok}[1]{\textcolor[rgb]{0.56,0.35,0.01}{\textbf{\textit{#1}}}}
\newcommand{\ErrorTok}[1]{\textcolor[rgb]{0.64,0.00,0.00}{\textbf{#1}}}
\newcommand{\ExtensionTok}[1]{#1}
\newcommand{\FloatTok}[1]{\textcolor[rgb]{0.00,0.00,0.81}{#1}}
\newcommand{\FunctionTok}[1]{\textcolor[rgb]{0.00,0.00,0.00}{#1}}
\newcommand{\ImportTok}[1]{#1}
\newcommand{\InformationTok}[1]{\textcolor[rgb]{0.56,0.35,0.01}{\textbf{\textit{#1}}}}
\newcommand{\KeywordTok}[1]{\textcolor[rgb]{0.13,0.29,0.53}{\textbf{#1}}}
\newcommand{\NormalTok}[1]{#1}
\newcommand{\OperatorTok}[1]{\textcolor[rgb]{0.81,0.36,0.00}{\textbf{#1}}}
\newcommand{\OtherTok}[1]{\textcolor[rgb]{0.56,0.35,0.01}{#1}}
\newcommand{\PreprocessorTok}[1]{\textcolor[rgb]{0.56,0.35,0.01}{\textit{#1}}}
\newcommand{\RegionMarkerTok}[1]{#1}
\newcommand{\SpecialCharTok}[1]{\textcolor[rgb]{0.00,0.00,0.00}{#1}}
\newcommand{\SpecialStringTok}[1]{\textcolor[rgb]{0.31,0.60,0.02}{#1}}
\newcommand{\StringTok}[1]{\textcolor[rgb]{0.31,0.60,0.02}{#1}}
\newcommand{\VariableTok}[1]{\textcolor[rgb]{0.00,0.00,0.00}{#1}}
\newcommand{\VerbatimStringTok}[1]{\textcolor[rgb]{0.31,0.60,0.02}{#1}}
\newcommand{\WarningTok}[1]{\textcolor[rgb]{0.56,0.35,0.01}{\textbf{\textit{#1}}}}
\usepackage{longtable,booktabs,array}
\usepackage{calc} % for calculating minipage widths
% Correct order of tables after \paragraph or \subparagraph
\usepackage{etoolbox}
\makeatletter
\patchcmd\longtable{\par}{\if@noskipsec\mbox{}\fi\par}{}{}
\makeatother
% Allow footnotes in longtable head/foot
\IfFileExists{footnotehyper.sty}{\usepackage{footnotehyper}}{\usepackage{footnote}}
\makesavenoteenv{longtable}
\usepackage{graphicx}
\makeatletter
\def\maxwidth{\ifdim\Gin@nat@width>\linewidth\linewidth\else\Gin@nat@width\fi}
\def\maxheight{\ifdim\Gin@nat@height>\textheight\textheight\else\Gin@nat@height\fi}
\makeatother
% Scale images if necessary, so that they will not overflow the page
% margins by default, and it is still possible to overwrite the defaults
% using explicit options in \includegraphics[width, height, ...]{}
\setkeys{Gin}{width=\maxwidth,height=\maxheight,keepaspectratio}
% Set default figure placement to htbp
\makeatletter
\def\fps@figure{htbp}
\makeatother
\setlength{\emergencystretch}{3em} % prevent overfull lines
\providecommand{\tightlist}{%
  \setlength{\itemsep}{0pt}\setlength{\parskip}{0pt}}
\setcounter{secnumdepth}{5}
\usepackage{booktabs}
% Solution from:
% https://github.com/ulyngs/oxforddown/commit/ee17c2fff6d5df37df972936dd9df25866c088bb

\usepackage{fvextra}
\DefineVerbatimEnvironment{Highlighting}{Verbatim}{breaklines,commandchars=\\\{\}}
\usepackage{booktabs}
\usepackage{longtable}
\usepackage{array}
\usepackage{multirow}
\usepackage{wrapfig}
\usepackage{float}
\usepackage{colortbl}
\usepackage{pdflscape}
\usepackage{tabu}
\usepackage{threeparttable}
\usepackage{threeparttablex}
\usepackage[normalem]{ulem}
\usepackage{makecell}
\usepackage{xcolor}
\ifLuaTeX
  \usepackage{selnolig}  % disable illegal ligatures
\fi
\usepackage[]{natbib}
\bibliographystyle{apalike}
\IfFileExists{bookmark.sty}{\usepackage{bookmark}}{\usepackage{hyperref}}
\IfFileExists{xurl.sty}{\usepackage{xurl}}{} % add URL line breaks if available
\urlstyle{same} % disable monospaced font for URLs
\hypersetup{
  pdftitle={Orchestrating Microbiome Analysis},
  hidelinks,
  pdfcreator={LaTeX via pandoc}}

\title{Orchestrating Microbiome Analysis}
\author{}
\date{\vspace{-2.5em}\textbf{Authors:} Leo Lahti {[}aut{]}, Tuomas Borman {[}aut, cre{]}, Henrik Eckermann {[}ctb{]}, Sudarshan Shetty {[}aut{]}, Felix GM Ernst {[}aut{]} \textbf{Version:} 0.98.14 \textbf{Modified:} 2023-04-10 \textbf{Compiled:} 2023-06-12 \textbf{Environment:} R version 4.3.0 (2023-04-21), Bioconductor 3.17 \textbf{License:} CC BY-NC-SA 3.0 US \textbf{Copyright:} \textbf{Source:} \url{https://github.com/microbiome/OMA}}

\begin{document}
\maketitle

{
\setcounter{tocdepth}{1}
\tableofcontents
}
\hypertarget{welcome}{%
\chapter*{Welcome}\label{welcome}}
\addcontentsline{toc}{chapter}{Welcome}

You are reading the online book, \href{microbiome.github.io/OMA}{\textbf{Orchestrating Microbiome Analysis
with R and Bioconductor}} \citep{OMA}, where we
walk through common strategies and workflows in microbiome data
science.

The book shows through concrete examples how you can take advantage of
the latest developments in R/Bioconductor for the manipulation,
analysis, and reproducible reporting of hierarchical and heterogeneous
microbiome profiling data sets. The book was borne out of necessity,
while updating microbiome analysis tools to work with Bioconductor
classes that provide support for multi-modal data collections. Many of
these techniques are generic and widely applicable in other contexts
as well.

This work has been heavily influenced by other similar resources, in
particular the Orchestrating Single-Cell Analysis with Bioconductor
\citep{Amezquita2020}, \href{http://joey711.github.io/phyloseq/tutorials-index}{phyloseq
tutorials}
\citep{Callahan2016} and \href{https://microbiome.github.io/tutorials/}{microbiome
tutorials} \citep{Shetty2019}.
This book extends these resources to teach the grammar of Bioconductor
workflows in the context of microbiome data science. As such, it
supports the adoption of general skills in the analysis of large,
hierarchical, and multi-modal data collections. We focus on microbiome
analysis tools, including entirely new, partially updated as well as
previously established methods.

This online resource and its associated ecosystem of microbiome data
science tools are a result of a community-driven development process,
and welcoming new contributors. Several individuals have
\href{https://github.com/microbiome/OMA/graphs/contributors}{contributed}
methods, workflows and improvements as acknowledged in the
Introduction. You can find more information on how to find us online
and join the developer community through the project homepage at
\href{https://microbiome.github.io}{microbiome.github.io}. This online
resource has been written in RMarkdown with the bookdown R
package. The material is \textbf{free to use} with the \href{https://creativecommons.org/licenses/by-nc/3.0/us/}{Creative Commons
Attribution-NonCommercial
3.0} License.

\begin{center}\rule{0.5\linewidth}{0.5pt}\end{center}

\hypertarget{part-introduction}{%
\part{Introduction}\label{part-introduction}}

\hypertarget{intro}{%
\chapter{Introduction}\label{intro}}

This work - \href{microbiome.github.io/OMA}{\textbf{Orchestrating Microbiome Analysis with R and
Bioconductor}} \citep{OMA} - contributes novel
methods and educational resources for microbiome data science. It
aims to teach the grammar of Bioconductor workflows in the context of
microbiome data science. We show through concrete examples how to use
the latest developments and data analytical strategies in
R/Bioconductor for the manipulation, analysis, and reproducible
reporting of hierarchical, heterogeneous, and multi-modal microbiome
profiling data. The data science methodology is tightly integrated
with the broader R/Bioconductor ecosystem that focuses on the
development of high-quality open research software for life
sciences (\citet{Gentleman2004}, \citet{Huber2015}). The support for modularity and
interoperability is a key to efficient resource sharing and
collaborative development both within and across research fields. The
central data infrastructure, the \texttt{SummarizedExperiment} data container
and its derivatives, have already been widely adopted in microbiome
research, single cell sequencing, and in other fields, allowing a
rapid adoption and extensions of emerging data science techniques
across application domains.

We assume that the reader is already familiar with R programming. For
references and tips on introductory material for R and Bioconductor,
see Chapter \ref{resources}. This online resource and its associated
ecosystem of microbiome data science tools are a result of a
community-driven development process, and welcoming new users and
contributors. You can find more information on how to find us online
and join the developer community through the project homepage at
\href{https://microbiome.github.io}{microbiome.github.io}.

The book is organized into three parts. We start by introducing the
material and link to further resources for learning R and
Bioconductor. We describe the key data infrastructure, the
\texttt{TreeSummarizedExperiment} class that provides a container for
microbiome data, and how to get started by loading microbiome data set
in the context of this new framework. The second section, \emph{Focus
Topics}, covers the common steps in microbiome data analysis,
beginning with the most common steps and progressing to more
specialized methods in subsequent sections. Third, \emph{Workflows},
provides case studies for the various datasets used throughout the
book. Finally, \emph{Appendix}, links to further resources and
acknowledgments.

\hypertarget{packages}{%
\chapter{Packages}\label{packages}}

The Bioconductor microbiome data science framework consists of:

\begin{itemize}
\tightlist
\item
  \textbf{data containers}, designed to organize multi-assay microbiome data
\item
  \textbf{R packages} that provide dedicated methods for analysing such data
\item
  \textbf{community} of users and developers
\end{itemize}

This section provides an overview of the package ecosystem. Section
\ref{example-data} links to various open microbiome data resources
that support this framework.

\hypertarget{package-installation}{%
\section{Package installation}\label{package-installation}}

You can install all packages that are required to run every example in this book via the following command:

\begin{Shaded}
\begin{Highlighting}[]
\FunctionTok{source}\NormalTok{(}\StringTok{"https://raw.githubusercontent.com/microbiome/OMA/master/install\_packages.R"}\NormalTok{)}
\end{Highlighting}
\end{Shaded}

\hypertarget{packages_specific}{%
\subsection{Installing specific packages}\label{packages_specific}}

You can install R packages of your choice with the following command
line procedure.

\textbf{Bioconductor release version} is the most stable and tested version
but may miss some of the latest methods and updates. It can be
installed with:

\begin{Shaded}
\begin{Highlighting}[]
\NormalTok{BiocManager}\SpecialCharTok{::}\FunctionTok{install}\NormalTok{(}\StringTok{"microbiome/mia"}\NormalTok{)}
\end{Highlighting}
\end{Shaded}

\textbf{Bioconductor development version} requires the installation of the
latest R beta version. This is primarily recommended for those who
already have experience with R/Bioconductor and need access to the
latest updates.

\begin{Shaded}
\begin{Highlighting}[]
\NormalTok{BiocManager}\SpecialCharTok{::}\FunctionTok{install}\NormalTok{(}\StringTok{"microbiome/mia"}\NormalTok{, }\AttributeTok{version=}\StringTok{"devel"}\NormalTok{)}
\end{Highlighting}
\end{Shaded}

\textbf{Github development version} provides access to the latest but
potentially unstable features. This is useful when you want access to
all available tools.

\begin{Shaded}
\begin{Highlighting}[]
\NormalTok{devtools}\SpecialCharTok{::}\FunctionTok{install\_github}\NormalTok{(}\StringTok{"microbiome/mia"}\NormalTok{)}
\end{Highlighting}
\end{Shaded}

\hypertarget{package-ecosystem}{%
\section{Package ecosystem}\label{package-ecosystem}}

Methods for the analysis and manipulation of
\texttt{(Tree)SummarizedExperiment} and \texttt{MultiAssayExperiment} data
containers are available through a number of R packages. Some of these
are listed below. If you know more tips on such packages, data
sources, or other resources, kindly \href{https://microbiome.github.io}{let us
know} through the issues, pull requests,
or online channels.

\hypertarget{mia-family-of-methods}{%
\subsection{mia family of methods}\label{mia-family-of-methods}}

\begin{itemize}
\tightlist
\item
  \href{microbiome.github.io/mia}{mia}: Microbiome analysis tools \citep{R-mia}
\item
  \href{microbiome.github.io/miaViz}{miaViz}: Microbiome analysis specific visualization \citep{Ernst2022}
\item
  \href{microbiome.github.io/miaSim}{miaSim}: Microbiome data simulations \citep{Simsek2021}
\item
  \href{microbiome.github.io/miaTime}{miaTime}: Microbiome time series analysis \citep{Lahti2021}
\end{itemize}

\hypertarget{sub-tree-methods}{%
\subsection{Tree-based methods}\label{sub-tree-methods}}

\begin{itemize}
\tightlist
\item
  \href{http://bioconductor.org/packages/devel/bioc/html/philr.html}{philr} (\citet{Silverman2017})
\end{itemize}

\hypertarget{sub-diff-abund}{%
\subsection{Differential abundance}\label{sub-diff-abund}}

\begin{itemize}
\tightlist
\item
  \href{https://bioconductor.org/packages/devel/bioc/html/ANCOMBC.html}{ANCOMBC} for differential abundance analysis
\item
  \href{https://bioconductor.org/packages/release/bioc/vignettes/benchdamic/inst/doc/intro.html}{benchdamic} for benchmarking differential abundance methods
\item
  \href{https://cran.r-project.org/web/packages/MicrobiomeStat/}{LinDA} for differential abundance analysis
\item
  \href{https://cran.r-project.org/web/packages/GUniFrac/}{ZicoSeq} for differential abundance analysis
\item
  \href{https://www.bioconductor.org/packages/release/bioc/html/ALDEx2.html}{ALDEx2} for differential abundance analysis
\item
  \href{https://www.bioconductor.org/packages/release/bioc/html/phyloseq.html}{phyloseq} for data preparation into phyloseq format for differential abundance analysis, such as ANCOMBC requires the input data is phyloseq format
\end{itemize}

\hypertarget{sub-manipulation}{%
\subsection{Manipulation}\label{sub-manipulation}}

\begin{itemize}
\tightlist
\item
  \href{https://bioconductor.org/packages/release/bioc/html/MicrobiotaProcess.html}{MicrobiotaProcess} for analyzing microbiome and other ecological data within the tidy framework
\end{itemize}

\hypertarget{further-options}{%
\subsection{Further options}\label{further-options}}

\begin{itemize}
\tightlist
\item
  \href{https://microsud.github.io/Tools-Microbiome-Analysis/}{Tools for Microbiome
  Analysis}
  site listed over 130 R packages for microbiome data science in

  \begin{enumerate}
  \def\labelenumi{\arabic{enumi}.}
  \setcounter{enumi}{2022}
  \tightlist
  \item
    Many of these are not in Bioconductor, or do not directly
    support the data containers used in this book but can be used with
    minor modifications.
  \end{enumerate}
\end{itemize}

\hypertarget{containers}{%
\chapter{Microbiome Data}\label{containers}}

\hypertarget{data-science-framework}{%
\section{Data science framework}\label{data-science-framework}}

The building blocks of the framework are \textbf{data container}
(SummarizedExperiment and its derivatives), \textbf{packages} from various
developers using the TreeSE container, open \textbf{demonstration data
sets}, in a separate chapter \ref{example-data}, and \textbf{online
tutorials} including this online book as well as the various package
vignettes and other materials.

\includegraphics[width=18.67in]{general/figures/FigureOverviewV2_mod}

\hypertarget{data-containers}{%
\section{Data containers}\label{data-containers}}

\texttt{SummarizedExperiment} (\texttt{SE}) \citep{R-SummarizedExperiment} is a generic and highly optimized container for complex data
structures. It has become a common choice for analysing various types
of biomedical profiling data, such as RNAseq, ChIp-Seq, microarrays,
flow cytometry, proteomics, and single-cell
sequencing.

{[}\texttt{TreeSummarizedExperiment}{]} (\texttt{TreeSE}) \citep{R-TreeSummarizedExperiment} was developed as an extension to incorporate hierarchical
information (such as phylogenetic trees and sample hierarchies) and
reference sequences.

{[}\texttt{MultiAssayExperiment}{]} (\texttt{MAE}) \citep{Ramos2017} provides an organized way to bind several different data
structures together in a single object. For example, we can bind
microbiome data (in \texttt{TreeSE} format) with metabolomic profiling data
(in \texttt{SE}) format, with shared sample metadata. This is convenient and
robust for instance in subsetting and other data manipulation
tasks. Microbiome data can be part of multiomics experiments and
analysis strategies and we want to outline the understanding in which
we think the packages explained and used in this book relate to these
experiment layouts using the \texttt{TreeSummarizedExperiment} and classes
beyond.

This section provides an introductions to these data containers. In
microbiome data science, these containers link taxonomic abundance
tables with rich side information on the features and
samples. Taxonomic abundance data can be obtained by 16S rRNA amplicon
or metagenomic sequencing, phylogenetic microarrays, or by other
means. Many microbiome experiments include multiple versions and types
of data generated independently or derived from each other through
transformation or agglomeration. We start by providing recommendations
on how to represent different varieties of multi-table data within the
\texttt{TreeSummarizedExperiment} class.

The options and recommendations are summarized in Table \ref{tab:options}.

\hypertarget{assay-data}{%
\subsection{Assay data}\label{assay-data}}

The original count-based taxonomic abundance tables may have different
transformations, such as logarithmic, Centered Log-Ratio (CLR), or relative
abundance. These are typically stored in \emph{\textbf{assays}}.

\begin{Shaded}
\begin{Highlighting}[]
\FunctionTok{library}\NormalTok{(mia)}
\FunctionTok{data}\NormalTok{(GlobalPatterns, }\AttributeTok{package=}\StringTok{"mia"}\NormalTok{)}
\NormalTok{tse }\OtherTok{\textless{}{-}}\NormalTok{ GlobalPatterns}
\FunctionTok{assays}\NormalTok{(tse)}
\end{Highlighting}
\end{Shaded}

\begin{verbatim}
## List of length 1
## names(1): counts
\end{verbatim}

The \texttt{assays} slot contains the experimental data as multiple count matrices. The result of \texttt{assays} is a list of matrices.

\begin{Shaded}
\begin{Highlighting}[]
\FunctionTok{assays}\NormalTok{(tse)}
\end{Highlighting}
\end{Shaded}

\begin{verbatim}
## List of length 1
## names(1): counts
\end{verbatim}

Individual assays can be accessed via \texttt{assay}

\begin{Shaded}
\begin{Highlighting}[]
\FunctionTok{assay}\NormalTok{(tse, }\StringTok{"counts"}\NormalTok{)[}\DecValTok{1}\SpecialCharTok{:}\DecValTok{5}\NormalTok{,}\DecValTok{1}\SpecialCharTok{:}\DecValTok{7}\NormalTok{]}
\end{Highlighting}
\end{Shaded}

\begin{verbatim}
##        CL3 CC1 SV1 M31Fcsw M11Fcsw M31Plmr M11Plmr
## 549322   0   0   0       0       0       0       0
## 522457   0   0   0       0       0       0       0
## 951      0   0   0       0       0       0       1
## 244423   0   0   0       0       0       0       0
## 586076   0   0   0       0       0       0       0
\end{verbatim}

To illustrate the use of multiple assays, the relative abundance data can be
calculated and stored along the original count data using \texttt{relAbundanceCounts}.

\begin{Shaded}
\begin{Highlighting}[]
\NormalTok{tse }\OtherTok{\textless{}{-}} \FunctionTok{relAbundanceCounts}\NormalTok{(tse)}
\FunctionTok{assays}\NormalTok{(tse)}
\end{Highlighting}
\end{Shaded}

\begin{verbatim}
## List of length 2
## names(2): counts relabundance
\end{verbatim}

Now there are two assays available in the \texttt{tse} object, \texttt{counts} and
\texttt{relabundance}.

\begin{Shaded}
\begin{Highlighting}[]
\FunctionTok{assay}\NormalTok{(tse, }\StringTok{"relabundance"}\NormalTok{)[}\DecValTok{1}\SpecialCharTok{:}\DecValTok{5}\NormalTok{,}\DecValTok{1}\SpecialCharTok{:}\DecValTok{7}\NormalTok{]}
\end{Highlighting}
\end{Shaded}

\begin{verbatim}
##        CL3 CC1 SV1 M31Fcsw M11Fcsw M31Plmr   M11Plmr
## 549322   0   0   0       0       0       0 0.000e+00
## 522457   0   0   0       0       0       0 0.000e+00
## 951      0   0   0       0       0       0 2.305e-06
## 244423   0   0   0       0       0       0 0.000e+00
## 586076   0   0   0       0       0       0 0.000e+00
\end{verbatim}

Here the dimension of the count data remains unchanged. This is in
fact, a requirement for any \texttt{SummarizedExperiment} object.

\hypertarget{coldata}{%
\subsection{colData}\label{coldata}}

\texttt{colData} contains data on the samples.

\begin{Shaded}
\begin{Highlighting}[]
\FunctionTok{colData}\NormalTok{(tse)}
\end{Highlighting}
\end{Shaded}

\begin{verbatim}
## DataFrame with 26 rows and 7 columns
##         X.SampleID   Primer Final_Barcode Barcode_truncated_plus_T
##           <factor> <factor>      <factor>                 <factor>
## CL3        CL3      ILBC_01        AACGCA                   TGCGTT
## CC1        CC1      ILBC_02        AACTCG                   CGAGTT
## SV1        SV1      ILBC_03        AACTGT                   ACAGTT
## M31Fcsw    M31Fcsw  ILBC_04        AAGAGA                   TCTCTT
## M11Fcsw    M11Fcsw  ILBC_05        AAGCTG                   CAGCTT
## ...            ...      ...           ...                      ...
## TS28         TS28   ILBC_25        ACCAGA                   TCTGGT
## TS29         TS29   ILBC_26        ACCAGC                   GCTGGT
## Even1        Even1  ILBC_27        ACCGCA                   TGCGGT
## Even2        Even2  ILBC_28        ACCTCG                   CGAGGT
## Even3        Even3  ILBC_29        ACCTGT                   ACAGGT
##         Barcode_full_length SampleType
##                    <factor>   <factor>
## CL3             CTAGCGTGCGT      Soil 
## CC1             CATCGACGAGT      Soil 
## SV1             GTACGCACAGT      Soil 
## M31Fcsw         TCGACATCTCT      Feces
## M11Fcsw         CGACTGCAGCT      Feces
## ...                     ...        ...
## TS28            GCATCGTCTGG      Feces
## TS29            CTAGTCGCTGG      Feces
## Even1           TGACTCTGCGG      Mock 
## Even2           TCTGATCGAGG      Mock 
## Even3           AGAGAGACAGG      Mock 
##                                        Description
##                                           <factor>
## CL3     Calhoun South Carolina Pine soil, pH 4.9  
## CC1     Cedar Creek Minnesota, grassland, pH 6.1  
## SV1     Sevilleta new Mexico, desert scrub, pH 8.3
## M31Fcsw M3, Day 1, fecal swab, whole body study   
## M11Fcsw M1, Day 1, fecal swab, whole body study   
## ...                                            ...
## TS28                                       Twin #1
## TS29                                       Twin #2
## Even1                                      Even1  
## Even2                                      Even2  
## Even3                                      Even3
\end{verbatim}

\hypertarget{rowdata}{%
\subsection{rowData}\label{rowdata}}

\texttt{rowData} contains data on the features of the analyzed samples. Of particular
interest to the microbiome field, this is used to store taxonomic information.

\begin{Shaded}
\begin{Highlighting}[]
\FunctionTok{rowData}\NormalTok{(tse)}
\end{Highlighting}
\end{Shaded}

\begin{verbatim}
## DataFrame with 19216 rows and 7 columns
##            Kingdom        Phylum        Class        Order        Family
##        <character>   <character>  <character>  <character>   <character>
## 549322     Archaea Crenarchaeota Thermoprotei           NA            NA
## 522457     Archaea Crenarchaeota Thermoprotei           NA            NA
## 951        Archaea Crenarchaeota Thermoprotei Sulfolobales Sulfolobaceae
## 244423     Archaea Crenarchaeota        Sd-NA           NA            NA
## 586076     Archaea Crenarchaeota        Sd-NA           NA            NA
## ...            ...           ...          ...          ...           ...
## 278222    Bacteria           SR1           NA           NA            NA
## 463590    Bacteria           SR1           NA           NA            NA
## 535321    Bacteria           SR1           NA           NA            NA
## 200359    Bacteria           SR1           NA           NA            NA
## 271582    Bacteria           SR1           NA           NA            NA
##              Genus                Species
##        <character>            <character>
## 549322          NA                     NA
## 522457          NA                     NA
## 951     Sulfolobus Sulfolobusacidocalda..
## 244423          NA                     NA
## 586076          NA                     NA
## ...            ...                    ...
## 278222          NA                     NA
## 463590          NA                     NA
## 535321          NA                     NA
## 200359          NA                     NA
## 271582          NA                     NA
\end{verbatim}

\hypertarget{rowtree}{%
\subsection{rowTree}\label{rowtree}}

Phylogenetic trees also play an important role in the microbiome field. The
\texttt{TreeSummarizedExperiment} class can keep track of features and node
relations via two functions, \texttt{rowTree} and \texttt{rowLinks}.

A tree can be accessed via \texttt{rowTree} as \texttt{phylo} object.

\begin{Shaded}
\begin{Highlighting}[]
\FunctionTok{rowTree}\NormalTok{(tse)}
\end{Highlighting}
\end{Shaded}

\begin{verbatim}
## 
## Phylogenetic tree with 19216 tips and 19215 internal nodes.
## 
## Tip labels:
##   549322, 522457, 951, 244423, 586076, 246140, ...
## Node labels:
##   , 0.858.4, 1.000.154, 0.764.3, 0.995.2, 1.000.2, ...
## 
## Rooted; includes branch lengths.
\end{verbatim}

The links to the individual features are available through \texttt{rowLinks}.

\begin{Shaded}
\begin{Highlighting}[]
\FunctionTok{rowLinks}\NormalTok{(tse)}
\end{Highlighting}
\end{Shaded}

\begin{verbatim}
## LinkDataFrame with 19216 rows and 5 columns
##           nodeLab   nodeNum nodeLab_alias    isLeaf   whichTree
##       <character> <integer>   <character> <logical> <character>
## 1          549322         1       alias_1      TRUE       phylo
## 2          522457         2       alias_2      TRUE       phylo
## 3             951         3       alias_3      TRUE       phylo
## 4          244423         4       alias_4      TRUE       phylo
## 5          586076         5       alias_5      TRUE       phylo
## ...           ...       ...           ...       ...         ...
## 19212      278222     19212   alias_19212      TRUE       phylo
## 19213      463590     19213   alias_19213      TRUE       phylo
## 19214      535321     19214   alias_19214      TRUE       phylo
## 19215      200359     19215   alias_19215      TRUE       phylo
## 19216      271582     19216   alias_19216      TRUE       phylo
\end{verbatim}

Please note that there can be a 1:1 relationship between tree nodes and
features, but this is not a must-have. This means there can be features, which
are not linked to nodes, and nodes, which are not linked to features. To change
the links in an existing object, the \texttt{changeTree} function is available.

\hypertarget{alt-exp}{%
\subsection{Alternative experiments}\label{alt-exp}}

\emph{\textbf{Alternative experiments}} differ from transformations as they can
contain complementary data, which is no longer tied to the same
dimensions as the assay data. However, the number of samples (columns)
must be the same.

This can come into play, for instance, when one has taxonomic abundance
profiles quantified with different measurement technologies, such as
phylogenetic microarrays, amplicon sequencing, or metagenomic
sequencing. Such alternative experiments that concern the same samples
can be stored as

\begin{enumerate}
\def\labelenumi{\arabic{enumi}.}
\tightlist
\item
  Separate \emph{assays} assuming that the taxonomic information can be mapped
  between features directly 1:1; or
\item
  Data in the \emph{altExp} slot of the \texttt{TreeSummarizedExperiment}, if the feature
  dimensions differ. Each element of the \emph{altExp} slot is a \texttt{SummarizedExperiment}
  or an object from a derived class with independent feature data.
\end{enumerate}

As an example, we show how to store taxonomic abundance tables
agglomerated at different taxonomic levels. However, the data could as
well originate from entirely different measurement sources as long as
the samples are matched.

\begin{Shaded}
\begin{Highlighting}[]
\CommentTok{\# Agglomerate the data to Phylym level}
\NormalTok{tse\_phylum }\OtherTok{\textless{}{-}} \FunctionTok{agglomerateByRank}\NormalTok{(tse, }\StringTok{"Phylum"}\NormalTok{)}
\CommentTok{\# both have the same number of columns (samples)}
\FunctionTok{dim}\NormalTok{(tse)}
\end{Highlighting}
\end{Shaded}

\begin{verbatim}
## [1] 19216    26
\end{verbatim}

\begin{Shaded}
\begin{Highlighting}[]
\FunctionTok{dim}\NormalTok{(tse\_phylum)}
\end{Highlighting}
\end{Shaded}

\begin{verbatim}
## [1] 67 26
\end{verbatim}

\begin{Shaded}
\begin{Highlighting}[]
\CommentTok{\# Add the new table as an alternative experiment}
\FunctionTok{altExp}\NormalTok{(tse, }\StringTok{"Phylum"}\NormalTok{) }\OtherTok{\textless{}{-}}\NormalTok{ tse\_phylum}
\FunctionTok{altExpNames}\NormalTok{(tse)}
\end{Highlighting}
\end{Shaded}

\begin{verbatim}
## [1] "Phylum"
\end{verbatim}

\begin{Shaded}
\begin{Highlighting}[]
\CommentTok{\# Pick a sample subset: this acts on both altExp and assay data}
\NormalTok{tse[,}\DecValTok{1}\SpecialCharTok{:}\DecValTok{10}\NormalTok{]}
\end{Highlighting}
\end{Shaded}

\begin{verbatim}
## class: TreeSummarizedExperiment 
## dim: 19216 10 
## metadata(0):
## assays(2): counts relabundance
## rownames(19216): 549322 522457 ... 200359 271582
## rowData names(7): Kingdom Phylum ... Genus Species
## colnames(10): CL3 CC1 ... M31Tong M11Tong
## colData names(7): X.SampleID Primer ... SampleType Description
## reducedDimNames(0):
## mainExpName: NULL
## altExpNames(1): Phylum
## rowLinks: a LinkDataFrame (19216 rows)
## rowTree: 1 phylo tree(s) (19216 leaves)
## colLinks: NULL
## colTree: NULL
\end{verbatim}

\begin{Shaded}
\begin{Highlighting}[]
\FunctionTok{dim}\NormalTok{(}\FunctionTok{altExp}\NormalTok{(tse[,}\DecValTok{1}\SpecialCharTok{:}\DecValTok{10}\NormalTok{],}\StringTok{"Phylum"}\NormalTok{))}
\end{Highlighting}
\end{Shaded}

\begin{verbatim}
## [1] 67 10
\end{verbatim}

For more details of altExp have a look at the \href{https://bioconductor.org/packages/release/bioc/vignettes/SingleCellExperiment/inst/doc/intro.html}{Intro vignette} of the
\texttt{SingleCellExperiment} package \citep{R-SingleCellExperiment}.

\hypertarget{mae}{%
\subsection{MultiAssayExperiments}\label{mae}}

\emph{\textbf{Multiple experiments}} relate to complementary measurement types,
such as transcriptomic or metabolomic profiling of the microbiome or
the host. Multiple experiments can be represented using the same
options as alternative experiments, or by using the
\texttt{MultiAssayExperiment} class \citep{Ramos2017}. Depending on how the
datasets relate to each other the data can be stored as:

\begin{enumerate}
\def\labelenumi{\arabic{enumi}.}
\tightlist
\item
  Separate \emph{altExp} if the samples can be matched directly 1:1; or
\item
  As \texttt{MultiAssayExperiment} objects, in which the connections between
  samples are defined through a \texttt{sampleMap}. Each element on the
  \texttt{experimentsList} of an \texttt{MultiAssayExperiment} is \texttt{matrix} or
  \texttt{matrix}-like objects, including \texttt{SummarizedExperiment} objects, and
  the number of samples can differ between the elements.
\end{enumerate}

\begin{Shaded}
\begin{Highlighting}[]
\CommentTok{\#TODO: Find the right dataset to explain a non 1:1 sample relationship}
\end{Highlighting}
\end{Shaded}

For information have a look at the \href{https://bioconductor.org/packages/release/bioc/vignettes/MultiAssayExperiment/inst/doc/MultiAssayExperiment.html}{intro vignette} of the \texttt{MultiAssayExperiment} package.

\begin{longtable}[]{@{}rlrr@{}}
\caption{\label{tab:options} \textbf{Recommended options for storing multiple data tables in microbiome studies} The \emph{assays} are best suited for data transformations (one-to-one match between samples and columns across the assays). The \emph{alternative experiments} are particularly suitable for alternative versions of the data that are of same type but may have a different number of features (e.g.~taxonomic groups); this is for instance the case with taxonomic abundance tables agglomerated at different levels (e.g.~genus vs.~phyla) or alternative profiling technologies (e.g.~amplicon sequencing vs.~shallow shotgun metagenomics). For alternative experiments one-to-one match between samples (cols) is libraryd but the alternative experiment tables can have different numbers of features (rows). Finally, elements of the \emph{MultiAssayExperiment} provide the most flexible way to incorporate multi-omic data tables with flexible numbers of samples and features. We recommend these conventions as the basis for methods development and application in microbiome studies.}\tabularnewline
\toprule()
Option & Rows (features) & Cols (samples) & Recommended \\
\midrule()
\endfirsthead
\toprule()
Option & Rows (features) & Cols (samples) & Recommended \\
\midrule()
\endhead
assays & match & match & Data transformations \\
altExp & free & match & Alternative experiments \\
MultiAssay & free & free (mapping) & Multi-omic experiments \\
\bottomrule()
\end{longtable}

\hypertarget{example-data}{%
\section{Demonstration data}\label{example-data}}

Open demonstration data for testing and benchmarking purposes is
available from multiple locations. This chapter introduces some
options. The other chapters of this book provide ample examples about
the use of the data.

\hypertarget{package-data}{%
\subsection{Package data}\label{package-data}}

The \texttt{mia} R package contains example data sets that are direct
conversions from the alternative \texttt{phyloseq} container to the
\texttt{TreeSummarizedExperiment} container.

List the \href{https://microbiome.github.io/mia/reference/index.html}{available
datasets} in
the \texttt{mia} package:

\begin{Shaded}
\begin{Highlighting}[]
\FunctionTok{library}\NormalTok{(mia)}
\FunctionTok{data}\NormalTok{(}\AttributeTok{package=}\StringTok{"mia"}\NormalTok{)}
\end{Highlighting}
\end{Shaded}

Load the \texttt{GlobalPatterns} data from the \texttt{mia} package:

\begin{Shaded}
\begin{Highlighting}[]
\FunctionTok{data}\NormalTok{(}\StringTok{"GlobalPatterns"}\NormalTok{, }\AttributeTok{package=}\StringTok{"mia"}\NormalTok{)}
\NormalTok{GlobalPatterns}
\end{Highlighting}
\end{Shaded}

\begin{verbatim}
## class: TreeSummarizedExperiment 
## dim: 19216 26 
## metadata(0):
## assays(1): counts
## rownames(19216): 549322 522457 ... 200359 271582
## rowData names(7): Kingdom Phylum ... Genus Species
## colnames(26): CL3 CC1 ... Even2 Even3
## colData names(7): X.SampleID Primer ... SampleType Description
## reducedDimNames(0):
## mainExpName: NULL
## altExpNames(0):
## rowLinks: a LinkDataFrame (19216 rows)
## rowTree: 1 phylo tree(s) (19216 leaves)
## colLinks: NULL
## colTree: NULL
\end{verbatim}

Check the documentation for this data set:

\hypertarget{hintikka-desc}{%
\subsubsection{HintikkaXOData}\label{hintikka-desc}}

\href{https://microbiome.github.io/microbiomeDataSets/reference/HintikkaXOData.html}{HintikkaXOData}
is derived from a study about the effects of fat diet and prebiotics on the
microbiome of rat models \citep{Hintikka2021}. It is available in the MAE data
container for R. The dataset is briefly presented in
\href{https://microbiome.github.io/outreach/hintikkaxo_presentation.html}{these slides}.

\hypertarget{experimenthub-data}{%
\subsection{ExperimentHub data}\label{experimenthub-data}}

\href{https://bioconductor.org/packages/release/bioc/vignettes/ExperimentHub/inst/doc/ExperimentHub.html}{ExperimentHub}
provides a variety of data resources, including the
\href{https://bioconductor.org/packages/release/data/experiment/html/microbiomeDataSets.html}{microbiomeDataSets}
package \citep{Morgan2021, microlahti2021}.

A table of the available data sets is available through the
\texttt{availableDataSets} function.

\begin{Shaded}
\begin{Highlighting}[]
\FunctionTok{library}\NormalTok{(microbiomeDataSets)}
\FunctionTok{availableDataSets}\NormalTok{()}
\end{Highlighting}
\end{Shaded}

\begin{verbatim}
##             Dataset
## 1  GrieneisenTSData
## 2    HintikkaXOData
## 3       LahtiMLData
## 4        LahtiMData
## 5       LahtiWAData
## 6      OKeefeDSData
## 7 SilvermanAGutData
## 8        SongQAData
## 9   SprockettTHData
\end{verbatim}

All data are downloaded from ExperimentHub and cached for local
re-use. Check the \href{https://microbiome.github.io/microbiomeDataSets/reference/index.html}{man pages of each
function}
for a detailed documentation of the data contents and references. Let
us retrieve a \emph{\href{https://bioconductor.org/packages/3.17/MultiAssayExperiment}{MultiAssayExperiment}} data set:

\begin{Shaded}
\begin{Highlighting}[]
\CommentTok{\# mae \textless{}{-} HintikkaXOData()}
\CommentTok{\# Since HintikkaXOData is now added to mia, we can load it directly from there}
\CommentTok{\# We suggest to check other datasets from microbiomeDataSets}
\FunctionTok{data}\NormalTok{(HintikkaXOData)}
\NormalTok{mae }\OtherTok{\textless{}{-}}\NormalTok{ HintikkaXOData}
\end{Highlighting}
\end{Shaded}

Data is available in \emph{\href{https://bioconductor.org/packages/3.17/SummarizedExperiment}{SummarizedExperiment}}, \texttt{r\ Biocpkg("TreeSummarizedExperiment")} and \texttt{r\ Biocpkg("MultiAssayExperiment")} data containers; see the separate
page on \href{https://microbiome.github.io/OMA/multitable.html}{alternative
containers} for more
details.

\hypertarget{curated-metagenomic-data}{%
\subsection{Curated metagenomic data}\label{curated-metagenomic-data}}

\href{https://bioconductor.org/packages/release/data/experiment/html/curatedMetagenomicData.html}{curatedMetagenomicData}
is a large collection of curated human microbiome data sets, provided as
\texttt{(Tree)SummarizedExperiment} objects \citep{Pasolli2017}. The resource
provides curated human microbiome data including gene families, marker
abundance, marker presence, pathway abundance, pathway coverage, and
relative abundance for samples from different body sites. See the
package homepage for more details on data availability and access.

As one example, let us retrieve the Vatanen (2016) \citep{Vatanen2016} data
set. This is a larger collection with a bit longer download time.

\begin{Shaded}
\begin{Highlighting}[]
\FunctionTok{library}\NormalTok{(curatedMetagenomicData)}
\NormalTok{tse }\OtherTok{\textless{}{-}} \FunctionTok{curatedMetagenomicData}\NormalTok{(}\StringTok{"Vatanen*"}\NormalTok{, }\AttributeTok{dryrun =} \ConstantTok{FALSE}\NormalTok{, }\AttributeTok{counts =} \ConstantTok{TRUE}\NormalTok{)}
\end{Highlighting}
\end{Shaded}

\hypertarget{other-data-sources}{%
\subsection{Other data sources}\label{other-data-sources}}

The current collections provide access to vast microbiome data
resources. The output has to be converted into TreeSE/MAE separately.

\begin{itemize}
\tightlist
\item
  \href{https://github.com/beadyallen/MGnifyR}{MGnifyR} provides access to \href{https://www.ebi.ac.uk/metagenomics/}{EBI/MGnify}
\item
  \href{https://github.com/cran/qiitr}{qiitr} provides access to \href{https://qiita.com/about}{QIITA}
\end{itemize}

\hypertarget{loading-experimental-microbiome-data}{%
\section{Loading experimental microbiome data}\label{loading-experimental-microbiome-data}}

\hypertarget{s-workflow}{%
\subsection{16S workflow}\label{s-workflow}}

Result of amplicon sequencing is a large number of files that include all the sequences
that were read from samples. Those sequences need to be matched with taxa. Additionally,
we need to know how many times each taxa were found from each sample.

There are several algorithms to do that, and DADA2 is one of the most common.
You can find DADA2 pipeline tutorial, for example,
\href{https://benjjneb.github.io/dada2/tutorial.html}{here}.
After the DADA2 portion of the tutorial is completed, the data is stored into \emph{phyloseq} object
(Bonus: Handoff to phyloseq). To store the data to \emph{TreeSummarizedExperiment},
follow the example below.

You can find full workflow script without further explanations and comments from
\href{https://github.com/microbiome/OMA/blob/master/dada2_workflow.Rmd}{here}

Load required packages.

\begin{Shaded}
\begin{Highlighting}[]
\FunctionTok{library}\NormalTok{(mia)}
\FunctionTok{library}\NormalTok{(ggplot2)}

\FunctionTok{library}\NormalTok{(}\StringTok{"BiocManager"}\NormalTok{)}
\FunctionTok{library}\NormalTok{(}\StringTok{"Biostrings"}\NormalTok{)}

\FunctionTok{library}\NormalTok{(Biostrings)}
\end{Highlighting}
\end{Shaded}

Create arbitrary example sample metadata like it was done in the tutorial. Usually,
sample metadata is imported as a file.

\begin{Shaded}
\begin{Highlighting}[]
\NormalTok{samples.out }\OtherTok{\textless{}{-}} \FunctionTok{rownames}\NormalTok{(seqtab.nochim)}
\NormalTok{subject }\OtherTok{\textless{}{-}} \FunctionTok{sapply}\NormalTok{(}\FunctionTok{strsplit}\NormalTok{(samples.out, }\StringTok{"D"}\NormalTok{), }\StringTok{\textasciigrave{}}\AttributeTok{[}\StringTok{\textasciigrave{}}\NormalTok{, }\DecValTok{1}\NormalTok{)}
\NormalTok{gender }\OtherTok{\textless{}{-}} \FunctionTok{substr}\NormalTok{(subject,}\DecValTok{1}\NormalTok{,}\DecValTok{1}\NormalTok{)}
\NormalTok{subject }\OtherTok{\textless{}{-}} \FunctionTok{substr}\NormalTok{(subject,}\DecValTok{2}\NormalTok{,}\DecValTok{999}\NormalTok{)}
\NormalTok{day }\OtherTok{\textless{}{-}} \FunctionTok{as.integer}\NormalTok{(}\FunctionTok{sapply}\NormalTok{(}\FunctionTok{strsplit}\NormalTok{(samples.out, }\StringTok{"D"}\NormalTok{), }\StringTok{\textasciigrave{}}\AttributeTok{[}\StringTok{\textasciigrave{}}\NormalTok{, }\DecValTok{2}\NormalTok{))}
\NormalTok{samdf }\OtherTok{\textless{}{-}} \FunctionTok{data.frame}\NormalTok{(}\AttributeTok{Subject=}\NormalTok{subject, }\AttributeTok{Gender=}\NormalTok{gender, }\AttributeTok{Day=}\NormalTok{day)}
\NormalTok{samdf}\SpecialCharTok{$}\NormalTok{When }\OtherTok{\textless{}{-}} \StringTok{"Early"}
\NormalTok{samdf}\SpecialCharTok{$}\NormalTok{When[samdf}\SpecialCharTok{$}\NormalTok{Day}\SpecialCharTok{\textgreater{}}\DecValTok{100}\NormalTok{] }\OtherTok{\textless{}{-}} \StringTok{"Late"}
\FunctionTok{rownames}\NormalTok{(samdf) }\OtherTok{\textless{}{-}}\NormalTok{ samples.out}
\end{Highlighting}
\end{Shaded}

Convert data into right format and create a \emph{TreeSE} object.

\begin{Shaded}
\begin{Highlighting}[]
\CommentTok{\# Create a list that contains assays}
\NormalTok{counts }\OtherTok{\textless{}{-}} \FunctionTok{t}\NormalTok{(seqtab.nochim)}
\NormalTok{counts }\OtherTok{\textless{}{-}} \FunctionTok{as.matrix}\NormalTok{(counts)}
\NormalTok{assays }\OtherTok{\textless{}{-}} \FunctionTok{SimpleList}\NormalTok{(}\AttributeTok{counts =}\NormalTok{ counts)}

\CommentTok{\# Convert colData and rowData into DataFrame}
\NormalTok{samdf }\OtherTok{\textless{}{-}} \FunctionTok{DataFrame}\NormalTok{(samdf)}
\NormalTok{taxa }\OtherTok{\textless{}{-}} \FunctionTok{DataFrame}\NormalTok{(taxa)}

\CommentTok{\# Create TreeSE}
\NormalTok{tse }\OtherTok{\textless{}{-}} \FunctionTok{TreeSummarizedExperiment}\NormalTok{(}\AttributeTok{assays =}\NormalTok{ assays,}
                                \AttributeTok{colData =}\NormalTok{ samdf,}
                                \AttributeTok{rowData =}\NormalTok{ taxa}
\NormalTok{                                )}

\CommentTok{\# Remove mock sample like it is also done in DADA2 pipeline tutorial}
\NormalTok{tse }\OtherTok{\textless{}{-}}\NormalTok{ tse[ , }\FunctionTok{colnames}\NormalTok{(tse) }\SpecialCharTok{!=} \StringTok{"mock"}\NormalTok{]}
\end{Highlighting}
\end{Shaded}

Add sequences into \emph{referenceSeq} slot and convert rownames into simpler format.

\begin{Shaded}
\begin{Highlighting}[]
\CommentTok{\# Convert sequences into right format}
\NormalTok{dna }\OtherTok{\textless{}{-}}\NormalTok{ Biostrings}\SpecialCharTok{::}\FunctionTok{DNAStringSet}\NormalTok{( }\FunctionTok{rownames}\NormalTok{(tse) )}
\CommentTok{\# Add sequences into referenceSeq slot}
\FunctionTok{referenceSeq}\NormalTok{(tse) }\OtherTok{\textless{}{-}}\NormalTok{ dna}
\CommentTok{\# Convert rownames into ASV\_number format}
\FunctionTok{rownames}\NormalTok{(tse) }\OtherTok{\textless{}{-}} \FunctionTok{paste0}\NormalTok{(}\StringTok{"ASV"}\NormalTok{, }\FunctionTok{seq}\NormalTok{( }\FunctionTok{nrow}\NormalTok{(tse) ))}
\NormalTok{tse}
\end{Highlighting}
\end{Shaded}

\begin{verbatim}
## class: TreeSummarizedExperiment 
## dim: 232 20 
## metadata(0):
## assays(1): counts
## rownames(232): ASV1 ASV2 ... ASV231 ASV232
## rowData names(7): Kingdom Phylum ... Genus Species
## colnames(20): F3D0 F3D1 ... F3D9 Mock
## colData names(4): Subject Gender Day When
## reducedDimNames(0):
## mainExpName: NULL
## altExpNames(0):
## rowLinks: NULL
## rowTree: NULL
## colLinks: NULL
## colTree: NULL
## referenceSeq: a DNAStringSet (232 sequences)
\end{verbatim}

\hypertarget{import-from-external-files}{%
\subsection{Import from external files}\label{import-from-external-files}}

Microbiome (taxonomic) profiling data is commonly distributed in
various file formats. You can import such external data files as a
(Tree)SummarizedExperiment object, but the details depend on the file
format. Here, we provide examples for common formats.

\hypertarget{csv-import}{%
\subsubsection{CSV import}\label{csv-import}}

\textbf{CSV data tables} can be imported with the standard R functions,
then converted to the desired format. For detailed examples, you can
check the \href{https://bioconductor.org/help/course-materials/2019/BSS2019/04_Practical_CoreApproachesInBioconductor.html}{Bioconductor course
material}
by Martin Morgan. You can also check the \href{https://github.com/microbiome/OMA/tree/master/data}{example
files} and
construct your own CSV files accordingly.

Recommendations for the CSV files are the following. File names are
arbitrary; we refer here to the same names as in the examples:

\begin{itemize}
\item
  Abundance table (\texttt{assay\_taxa.csv}): data matrix (features x
  samples); first column provides feature IDs, the first row provides
  sample IDs; other values should be numeric (abundances).
\item
  Row data (\texttt{rowdata\_taxa.csv}): data table (features x info); first
  column provides feature IDs, the first row provides column headers;
  this file usually contains the taxonomic mapping between different
  taxonomic levels. Ideally, the feature IDs (row names) match one-to-one with
  the abundance table row names.
\item
  Column data (\texttt{coldata.csv}): data table (samples x info); first
  column provides sample IDs, the first row provides column headers;
  this file usually contains the sample metadata/phenodata (such as
  subject age, health etc). Ideally, the sample IDs match one-to-one with
  the abundance table column names.
\end{itemize}

After you have set up the CSV files, you can read them in R:

\begin{Shaded}
\begin{Highlighting}[]
\NormalTok{count\_file  }\OtherTok{\textless{}{-}} \StringTok{"data/assay\_taxa.csv"}
\NormalTok{tax\_file    }\OtherTok{\textless{}{-}} \StringTok{"data/rowdata\_taxa.csv"}
\NormalTok{sample\_file }\OtherTok{\textless{}{-}} \StringTok{"data/coldata.csv"}

\CommentTok{\# Load files}
\NormalTok{counts  }\OtherTok{\textless{}{-}} \FunctionTok{read.csv}\NormalTok{(count\_file, }\AttributeTok{row.names=}\DecValTok{1}\NormalTok{)   }\CommentTok{\# Abundance table (e.g. ASV data; to assay data)}
\NormalTok{tax     }\OtherTok{\textless{}{-}} \FunctionTok{read.csv}\NormalTok{(tax\_file, }\AttributeTok{row.names=}\DecValTok{1}\NormalTok{)     }\CommentTok{\# Taxonomy table (to rowData)}
\NormalTok{samples }\OtherTok{\textless{}{-}} \FunctionTok{read.csv}\NormalTok{(sample\_file, }\AttributeTok{row.names=}\DecValTok{1}\NormalTok{)  }\CommentTok{\# Sample data (to colData)}
\end{Highlighting}
\end{Shaded}

After reading the data in R, ensure the following:

\begin{itemize}
\item
  abundance table (\texttt{counts}): numeric \texttt{matrix}, with feature IDs as
  rownames and sample IDs as column names
\item
  rowdata (\texttt{tax}): \texttt{DataFrame}, with feature IDs as rownames. If this
  is a \texttt{data.frame} you can use the function \texttt{DataFrame()} to change
  the format. Column names are free but in microbiome analysis they
  usually they refer to taxonomic ranks. The rownames in rowdata
  should match with rownames in abundance table.
\item
  coldata (\texttt{samples}): \texttt{DataFrame}, with sample IDs as rownames. If
  this is a \texttt{data.frame} you can use the function \texttt{DataFrame()} to
  change the format. Column names are free. The rownames in coldata
  should match with colnames in abundance table.
\end{itemize}

\textbf{Always ensure that the tables have rownames!} The \emph{TreeSE} constructor compares
rownames and ensures that, for example, right samples are linked with right patient.

Also ensure that the row and column names match one-to-one between
abundance table, rowdata, and coldata:

\begin{Shaded}
\begin{Highlighting}[]
\CommentTok{\# Match rows and columns}
\NormalTok{counts }\OtherTok{\textless{}{-}}\NormalTok{ counts[}\FunctionTok{rownames}\NormalTok{(tax), }\FunctionTok{rownames}\NormalTok{(samples)]}

\CommentTok{\# Let us ensure that the data is in correct (numeric matrix) format:}
\NormalTok{counts }\OtherTok{\textless{}{-}} \FunctionTok{as.matrix}\NormalTok{(counts)}
\end{Highlighting}
\end{Shaded}

If you hesitate about the format of the data, you can compare to one
of the available demonstration data sets, and make sure that your data
components have the same format.

There are many different source files and many different ways to read
data in R. One can do data manipulation in R as well. Investigate the
entries as follows.

\begin{Shaded}
\begin{Highlighting}[]
\CommentTok{\# coldata rownames match assay colnames}
\FunctionTok{all}\NormalTok{(}\FunctionTok{rownames}\NormalTok{(samples) }\SpecialCharTok{==} \FunctionTok{colnames}\NormalTok{(counts)) }\CommentTok{\# our data set}
\end{Highlighting}
\end{Shaded}

\begin{verbatim}
## [1] TRUE
\end{verbatim}

\begin{Shaded}
\begin{Highlighting}[]
\FunctionTok{class}\NormalTok{(samples) }\CommentTok{\# should be data.frame or DataFrame}
\end{Highlighting}
\end{Shaded}

\begin{verbatim}
## [1] "data.frame"
\end{verbatim}

\begin{Shaded}
\begin{Highlighting}[]
\CommentTok{\# rowdata rownames match assay rownames}
\FunctionTok{all}\NormalTok{(}\FunctionTok{rownames}\NormalTok{(tax) }\SpecialCharTok{==} \FunctionTok{rownames}\NormalTok{(counts)) }\CommentTok{\# our data set}
\end{Highlighting}
\end{Shaded}

\begin{verbatim}
## [1] TRUE
\end{verbatim}

\begin{Shaded}
\begin{Highlighting}[]
\FunctionTok{class}\NormalTok{(tax) }\CommentTok{\# should be data.frame or DataFrame}
\end{Highlighting}
\end{Shaded}

\begin{verbatim}
## [1] "data.frame"
\end{verbatim}

\begin{Shaded}
\begin{Highlighting}[]
\CommentTok{\# Counts }
\FunctionTok{class}\NormalTok{(counts) }\CommentTok{\# should be a numeric matrix}
\end{Highlighting}
\end{Shaded}

\begin{verbatim}
## [1] "matrix" "array"
\end{verbatim}

\hypertarget{constructing-treesummarizedexperiment}{%
\subsection{Constructing TreeSummarizedExperiment}\label{constructing-treesummarizedexperiment}}

Now let us create the TreeSE object from the input data tables. Here
we also convert the data objects in their preferred formats:

\begin{itemize}
\tightlist
\item
  counts --\textgreater{} numeric matrix
\item
  rowData --\textgreater{} DataFrame
\item
  colData --\textgreater{} DataFrame
\end{itemize}

The \texttt{SimpleList} could be used to include multiple alternative assays, if
necessary.

\begin{Shaded}
\begin{Highlighting}[]
\CommentTok{\# Create a TreeSE}
\NormalTok{tse\_taxa }\OtherTok{\textless{}{-}} \FunctionTok{TreeSummarizedExperiment}\NormalTok{(}\AttributeTok{assays =}  \FunctionTok{SimpleList}\NormalTok{(}\AttributeTok{counts =}\NormalTok{ counts),}
                                     \AttributeTok{colData =} \FunctionTok{DataFrame}\NormalTok{(samples),}
                                     \AttributeTok{rowData =} \FunctionTok{DataFrame}\NormalTok{(tax))}

\NormalTok{tse\_taxa}
\end{Highlighting}
\end{Shaded}

\begin{verbatim}
## class: TreeSummarizedExperiment 
## dim: 12706 40 
## metadata(0):
## assays(1): counts
## rownames(12706): GAYR01026362.62.2014 CVJT01000011.50.2173 ...
##   JRJTB:03787:02429 JRJTB:03787:02478
## rowData names(7): Phylum Class ... Species OTU
## colnames(40): C1 C2 ... C39 C40
## colData names(6): Sample Rat ... Fat XOS
## reducedDimNames(0):
## mainExpName: NULL
## altExpNames(0):
## rowLinks: NULL
## rowTree: NULL
## colLinks: NULL
## colTree: NULL
\end{verbatim}

Now you should have a ready-made TreeSE data object that can be used in downstream analyses.

\hypertarget{constructing-multiassayexperiment}{%
\subsection{Constructing MultiAssayExperiment}\label{constructing-multiassayexperiment}}

To construct a \emph{MultiAssayExperiment} object, just combine multiple \emph{TreeSE} data containers.
Here we import metabolite data from the same study.

\begin{Shaded}
\begin{Highlighting}[]
\NormalTok{count\_file }\OtherTok{\textless{}{-}} \StringTok{"data/assay\_metabolites.csv"}
\NormalTok{sample\_file }\OtherTok{\textless{}{-}} \StringTok{"data/coldata.csv"}

\CommentTok{\# Load files}
\NormalTok{counts  }\OtherTok{\textless{}{-}} \FunctionTok{read.csv}\NormalTok{(count\_file, }\AttributeTok{row.names=}\DecValTok{1}\NormalTok{)  }
\NormalTok{samples }\OtherTok{\textless{}{-}} \FunctionTok{read.csv}\NormalTok{(sample\_file, }\AttributeTok{row.names=}\DecValTok{1}\NormalTok{)}

\CommentTok{\# Create a TreeSE for the metabolite data}
\NormalTok{tse\_metabolite }\OtherTok{\textless{}{-}} \FunctionTok{TreeSummarizedExperiment}\NormalTok{(}\AttributeTok{assays =} \FunctionTok{SimpleList}\NormalTok{(}\AttributeTok{concs =} \FunctionTok{as.matrix}\NormalTok{(counts)),}
                                           \AttributeTok{colData =} \FunctionTok{DataFrame}\NormalTok{(samples))}

\NormalTok{tse\_metabolite}
\end{Highlighting}
\end{Shaded}

\begin{verbatim}
## class: TreeSummarizedExperiment 
## dim: 38 40 
## metadata(0):
## assays(1): concs
## rownames(38): Butyrate Acetate ... Malonate 1,3-dihydroxyacetone
## rowData names(0):
## colnames(40): C1 C2 ... C39 C40
## colData names(6): Sample Rat ... Fat XOS
## reducedDimNames(0):
## mainExpName: NULL
## altExpNames(0):
## rowLinks: NULL
## rowTree: NULL
## colLinks: NULL
## colTree: NULL
\end{verbatim}

Now we can combine these two experiments into \emph{MAE}.

\begin{Shaded}
\begin{Highlighting}[]
\CommentTok{\# Create an ExperimentList that includes experiments}
\NormalTok{experiments }\OtherTok{\textless{}{-}} \FunctionTok{ExperimentList}\NormalTok{(}\AttributeTok{microbiome =}\NormalTok{ tse\_taxa, }
                              \AttributeTok{metabolite =}\NormalTok{ tse\_metabolite)}

\CommentTok{\# Create a MAE}
\NormalTok{mae }\OtherTok{\textless{}{-}} \FunctionTok{MultiAssayExperiment}\NormalTok{(}\AttributeTok{experiments =}\NormalTok{ experiments)}

\NormalTok{mae}
\end{Highlighting}
\end{Shaded}

\begin{verbatim}
## A MultiAssayExperiment object of 2 listed
##  experiments with user-defined names and respective classes.
##  Containing an ExperimentList class object of length 2:
##  [1] microbiome: TreeSummarizedExperiment with 12706 rows and 40 columns
##  [2] metabolite: TreeSummarizedExperiment with 38 rows and 40 columns
## Functionality:
##  experiments() - obtain the ExperimentList instance
##  colData() - the primary/phenotype DataFrame
##  sampleMap() - the sample coordination DataFrame
##  `$`, `[`, `[[` - extract colData columns, subset, or experiment
##  *Format() - convert into a long or wide DataFrame
##  assays() - convert ExperimentList to a SimpleList of matrices
##  exportClass() - save data to flat files
\end{verbatim}

\hypertarget{import-functions-for-standard-formats}{%
\subsection{Import functions for standard formats}\label{import-functions-for-standard-formats}}

Specific import functions are provided for:

\begin{itemize}
\tightlist
\item
  Biom files (see \texttt{help(mia::loadFromBiom)})
\item
  QIIME2 files (see \texttt{help(mia::loadFromQIIME2)})
\item
  Mothur files (see \texttt{help(mia::loadFromMothur)})
\end{itemize}

\hypertarget{biom-import}{%
\subsubsection{Biom import}\label{biom-import}}

This example shows how Biom files are imported into a
\texttt{TreeSummarizedExperiment} object.

The data is from following publication:
Tengeler AC \emph{et al.} (2020) \href{https://doi.org/10.1186/s40168-020-00816-x}{\textbf{Gut microbiota from persons with
attention-deficit/hyperactivity disorder affects the brain in
mice}}.

The data set consists of 3 files:

\begin{itemize}
\tightlist
\item
  biom file: abundance table and taxonomy information
\item
  csv file: sample metadata
\item
  tree file: phylogenetic tree
\end{itemize}

Store the data in your desired local directory (for instance, \emph{data/} under the
working directory), and define source file paths

\begin{Shaded}
\begin{Highlighting}[]
\NormalTok{biom\_file\_path }\OtherTok{\textless{}{-}} \StringTok{"data/Aggregated\_humanization2.biom"}
\NormalTok{sample\_meta\_file\_path }\OtherTok{\textless{}{-}} \StringTok{"data/Mapping\_file\_ADHD\_aggregated.csv"}
\NormalTok{tree\_file\_path }\OtherTok{\textless{}{-}} \StringTok{"data/Data\_humanization\_phylo\_aggregation.tre"}
\end{Highlighting}
\end{Shaded}

Now we can load the biom data into a SummarizedExperiment (SE) object.

\begin{Shaded}
\begin{Highlighting}[]
\FunctionTok{library}\NormalTok{(mia)}

\CommentTok{\# Imports the data}
\NormalTok{se }\OtherTok{\textless{}{-}} \FunctionTok{loadFromBiom}\NormalTok{(biom\_file\_path)}

\CommentTok{\# Check}
\NormalTok{se}
\end{Highlighting}
\end{Shaded}

\begin{verbatim}
## class: TreeSummarizedExperiment 
## dim: 151 27 
## metadata(0):
## assays(1): counts
## rownames(151): 1726470 1726471 ... 17264756 17264757
## rowData names(6): taxonomy1 taxonomy2 ... taxonomy5 taxonomy6
## colnames(27): A110 A111 ... A38 A39
## colData names(0):
## reducedDimNames(0):
## mainExpName: NULL
## altExpNames(0):
## rowLinks: NULL
## rowTree: NULL
## colLinks: NULL
## colTree: NULL
\end{verbatim}

The \texttt{assays} slot includes a list of abundance tables. The imported
abundance table is named as ``counts''. Let us inspect only the first
cols and rows.

\begin{Shaded}
\begin{Highlighting}[]
\FunctionTok{assays}\NormalTok{(se)}\SpecialCharTok{$}\NormalTok{counts[}\DecValTok{1}\SpecialCharTok{:}\DecValTok{3}\NormalTok{, }\DecValTok{1}\SpecialCharTok{:}\DecValTok{3}\NormalTok{]}
\end{Highlighting}
\end{Shaded}

\begin{verbatim}
##           A110  A111  A12
## 1726470  17722 11630    0
## 1726471  12052     0 2679
## 17264731     0   970    0
\end{verbatim}

The \texttt{rowdata} includes taxonomic information from the biom file. The \texttt{head()} command
shows just the beginning of the data table for an overview.

\texttt{knitr::kable()} is for printing the information more nicely.

\begin{Shaded}
\begin{Highlighting}[]
\FunctionTok{head}\NormalTok{(}\FunctionTok{rowData}\NormalTok{(se))}
\end{Highlighting}
\end{Shaded}

\begin{verbatim}
## DataFrame with 6 rows and 6 columns
##             taxonomy1          taxonomy2           taxonomy3
##           <character>        <character>         <character>
## 1726470  "k__Bacteria   p__Bacteroidetes      c__Bacteroidia
## 1726471  "k__Bacteria   p__Bacteroidetes      c__Bacteroidia
## 17264731 "k__Bacteria   p__Bacteroidetes      c__Bacteroidia
## 17264726 "k__Bacteria   p__Bacteroidetes      c__Bacteroidia
## 1726472  "k__Bacteria p__Verrucomicrobia c__Verrucomicrobiae
## 17264724 "k__Bacteria   p__Bacteroidetes      c__Bacteroidia
##                      taxonomy4              taxonomy5           taxonomy6
##                    <character>            <character>         <character>
## 1726470       o__Bacteroidales      f__Bacteroidaceae     g__Bacteroides"
## 1726471       o__Bacteroidales      f__Bacteroidaceae     g__Bacteroides"
## 17264731      o__Bacteroidales  f__Porphyromonadaceae g__Parabacteroides"
## 17264726      o__Bacteroidales      f__Bacteroidaceae     g__Bacteroides"
## 1726472  o__Verrucomicrobiales f__Verrucomicrobiaceae     g__Akkermansia"
## 17264724      o__Bacteroidales      f__Bacteroidaceae     g__Bacteroides"
\end{verbatim}

These taxonomic rank names (column names) are not real rank
names. Let's replace them with real rank names.

In addition to that, the taxa names include, e.g., '\,``k\_\_' before the name, so let's
make them cleaner by removing them.

\begin{Shaded}
\begin{Highlighting}[]
\FunctionTok{names}\NormalTok{(}\FunctionTok{rowData}\NormalTok{(se)) }\OtherTok{\textless{}{-}} \FunctionTok{c}\NormalTok{(}\StringTok{"Kingdom"}\NormalTok{, }\StringTok{"Phylum"}\NormalTok{, }\StringTok{"Class"}\NormalTok{, }\StringTok{"Order"}\NormalTok{, }
                        \StringTok{"Family"}\NormalTok{, }\StringTok{"Genus"}\NormalTok{)}

\CommentTok{\# Goes through the whole DataFrame. Removes \textquotesingle{}.*[kpcofg]\_\_\textquotesingle{} from strings, where [kpcofg] }
\CommentTok{\# is any character from listed ones, and .* any character.}
\NormalTok{rowdata\_modified }\OtherTok{\textless{}{-}}\NormalTok{ BiocParallel}\SpecialCharTok{::}\FunctionTok{bplapply}\NormalTok{(}\FunctionTok{rowData}\NormalTok{(se), }
                                           \AttributeTok{FUN =}\NormalTok{ stringr}\SpecialCharTok{::}\NormalTok{str\_remove, }
                                           \AttributeTok{pattern =} \StringTok{\textquotesingle{}.*[kpcofg]\_\_\textquotesingle{}}\NormalTok{)}

\CommentTok{\# Genus level has additional \textquotesingle{}\textbackslash{}"\textquotesingle{}, so let\textquotesingle{}s delete that also}
\NormalTok{rowdata\_modified }\OtherTok{\textless{}{-}}\NormalTok{ BiocParallel}\SpecialCharTok{::}\FunctionTok{bplapply}\NormalTok{(rowdata\_modified, }
                                           \AttributeTok{FUN =}\NormalTok{ stringr}\SpecialCharTok{::}\NormalTok{str\_remove, }
                                           \AttributeTok{pattern =} \StringTok{\textquotesingle{}}\SpecialCharTok{\textbackslash{}"}\StringTok{\textquotesingle{}}\NormalTok{)}

\CommentTok{\# rowdata\_modified is a list, so it is converted back to DataFrame format. }
\NormalTok{rowdata\_modified }\OtherTok{\textless{}{-}} \FunctionTok{DataFrame}\NormalTok{(rowdata\_modified)}

\CommentTok{\# And then assigned back to the SE object}
\FunctionTok{rowData}\NormalTok{(se) }\OtherTok{\textless{}{-}}\NormalTok{ rowdata\_modified}

\CommentTok{\# Now we have a nicer table}
\FunctionTok{head}\NormalTok{(}\FunctionTok{rowData}\NormalTok{(se))}
\end{Highlighting}
\end{Shaded}

\begin{verbatim}
## DataFrame with 6 rows and 6 columns
##              Kingdom          Phylum            Class              Order
##          <character>     <character>      <character>        <character>
## 1726470     Bacteria   Bacteroidetes      Bacteroidia      Bacteroidales
## 1726471     Bacteria   Bacteroidetes      Bacteroidia      Bacteroidales
## 17264731    Bacteria   Bacteroidetes      Bacteroidia      Bacteroidales
## 17264726    Bacteria   Bacteroidetes      Bacteroidia      Bacteroidales
## 1726472     Bacteria Verrucomicrobia Verrucomicrobiae Verrucomicrobiales
## 17264724    Bacteria   Bacteroidetes      Bacteroidia      Bacteroidales
##                       Family           Genus
##                  <character>     <character>
## 1726470       Bacteroidaceae     Bacteroides
## 1726471       Bacteroidaceae     Bacteroides
## 17264731  Porphyromonadaceae Parabacteroides
## 17264726      Bacteroidaceae     Bacteroides
## 1726472  Verrucomicrobiaceae     Akkermansia
## 17264724      Bacteroidaceae     Bacteroides
\end{verbatim}

We notice that the imported biom file did not contain the sample meta data
yet, so it includes an empty data frame.

\begin{Shaded}
\begin{Highlighting}[]
\FunctionTok{head}\NormalTok{(}\FunctionTok{colData}\NormalTok{(se))}
\end{Highlighting}
\end{Shaded}

\begin{verbatim}
## DataFrame with 6 rows and 0 columns
\end{verbatim}

Let us add a sample metadata file.

\begin{Shaded}
\begin{Highlighting}[]
\CommentTok{\# We use this to check what type of data it is}
\CommentTok{\# read.table(sample\_meta\_file\_path)}

\CommentTok{\# It seems like a comma separated file and it does not include headers}
\CommentTok{\# Let us read it and then convert from data.frame to DataFrame}
\CommentTok{\# (required for our purposes)}
\NormalTok{sample\_meta }\OtherTok{\textless{}{-}} \FunctionTok{DataFrame}\NormalTok{(}\FunctionTok{read.table}\NormalTok{(sample\_meta\_file\_path, }\AttributeTok{sep =} \StringTok{","}\NormalTok{, }\AttributeTok{header =} \ConstantTok{FALSE}\NormalTok{))}

\CommentTok{\# Add sample names to rownames}
\FunctionTok{rownames}\NormalTok{(sample\_meta) }\OtherTok{\textless{}{-}}\NormalTok{ sample\_meta[,}\DecValTok{1}\NormalTok{]}

\CommentTok{\# Delete column that included sample names}
\NormalTok{sample\_meta[,}\DecValTok{1}\NormalTok{] }\OtherTok{\textless{}{-}} \ConstantTok{NULL}

\CommentTok{\# We can add headers}
\FunctionTok{colnames}\NormalTok{(sample\_meta) }\OtherTok{\textless{}{-}} \FunctionTok{c}\NormalTok{(}\StringTok{"patient\_status"}\NormalTok{, }\StringTok{"cohort"}\NormalTok{, }\StringTok{"patient\_status\_vs\_cohort"}\NormalTok{, }\StringTok{"sample\_name"}\NormalTok{)}

\CommentTok{\# Then it can be added to colData}
\FunctionTok{colData}\NormalTok{(se) }\OtherTok{\textless{}{-}}\NormalTok{ sample\_meta}
\end{Highlighting}
\end{Shaded}

Now \texttt{colData} includes the sample metadata.

\begin{Shaded}
\begin{Highlighting}[]
\FunctionTok{head}\NormalTok{(}\FunctionTok{colData}\NormalTok{(se))}
\end{Highlighting}
\end{Shaded}

\begin{verbatim}
## DataFrame with 6 rows and 4 columns
##      patient_status      cohort patient_status_vs_cohort sample_name
##         <character> <character>              <character> <character>
## A110           ADHD    Cohort_1            ADHD_Cohort_1        A110
## A12            ADHD    Cohort_1            ADHD_Cohort_1         A12
## A15            ADHD    Cohort_1            ADHD_Cohort_1         A15
## A19            ADHD    Cohort_1            ADHD_Cohort_1         A19
## A21            ADHD    Cohort_2            ADHD_Cohort_2         A21
## A23            ADHD    Cohort_2            ADHD_Cohort_2         A23
\end{verbatim}

Now, let's add a phylogenetic tree.

The current data object, se, is a SummarizedExperiment object. This
does not include a slot for adding a phylogenetic tree. In order to do
this, we can convert the SE object to an extended TreeSummarizedExperiment
object which includes also a \texttt{rowTree} slot.

TreeSummarizedExperiment contains also other additional slots and features which
is why we recommend to use \texttt{TreeSE}.

\begin{Shaded}
\begin{Highlighting}[]
\NormalTok{tse }\OtherTok{\textless{}{-}} \FunctionTok{as}\NormalTok{(se, }\StringTok{"TreeSummarizedExperiment"}\NormalTok{)}

\CommentTok{\# tse includes same data as se}
\NormalTok{tse}
\end{Highlighting}
\end{Shaded}

\begin{verbatim}
## class: TreeSummarizedExperiment 
## dim: 151 27 
## metadata(0):
## assays(1): counts
## rownames(151): 1726470 1726471 ... 17264756 17264757
## rowData names(6): Kingdom Phylum ... Family Genus
## colnames(27): A110 A12 ... A35 A38
## colData names(4): patient_status cohort patient_status_vs_cohort
##   sample_name
## reducedDimNames(0):
## mainExpName: NULL
## altExpNames(0):
## rowLinks: NULL
## rowTree: NULL
## colLinks: NULL
## colTree: NULL
\end{verbatim}

Next, let us read the tree data file and add it to the R data object (tse).

\begin{Shaded}
\begin{Highlighting}[]
\CommentTok{\# Reads the tree file}
\NormalTok{tree }\OtherTok{\textless{}{-}}\NormalTok{ ape}\SpecialCharTok{::}\FunctionTok{read.tree}\NormalTok{(tree\_file\_path)}

\CommentTok{\# Add tree to rowTree}
\FunctionTok{rowTree}\NormalTok{(tse) }\OtherTok{\textless{}{-}}\NormalTok{ tree}

\CommentTok{\# Check}
\NormalTok{tse}
\end{Highlighting}
\end{Shaded}

\begin{verbatim}
## class: TreeSummarizedExperiment 
## dim: 151 27 
## metadata(0):
## assays(1): counts
## rownames(151): 1726470 1726471 ... 17264756 17264757
## rowData names(6): Kingdom Phylum ... Family Genus
## colnames(27): A110 A12 ... A35 A38
## colData names(4): patient_status cohort patient_status_vs_cohort
##   sample_name
## reducedDimNames(0):
## mainExpName: NULL
## altExpNames(0):
## rowLinks: a LinkDataFrame (151 rows)
## rowTree: 1 phylo tree(s) (151 leaves)
## colLinks: NULL
## colTree: NULL
\end{verbatim}

Now \texttt{rowTree} includes a phylogenetic tree:

\begin{Shaded}
\begin{Highlighting}[]
\FunctionTok{head}\NormalTok{(}\FunctionTok{rowTree}\NormalTok{(tse))}
\end{Highlighting}
\end{Shaded}

\hypertarget{conversions-between-data-formats-in-r}{%
\subsection{Conversions between data formats in R}\label{conversions-between-data-formats-in-r}}

If the data has already been imported in R in another format, it
can be readily converted into \texttt{TreeSummarizedExperiment}, as shown in our next
example. Note that similar conversion functions to
\texttt{TreeSummarizedExperiment} are available for multiple data formats via
the \texttt{mia} package (see makeTreeSummarizedExperimentFrom* for phyloseq,
Biom, and DADA2).

\begin{Shaded}
\begin{Highlighting}[]
\FunctionTok{library}\NormalTok{(mia)}

\CommentTok{\# phyloseq example data}
\FunctionTok{data}\NormalTok{(GlobalPatterns, }\AttributeTok{package=}\StringTok{"phyloseq"}\NormalTok{) }
\NormalTok{GlobalPatterns\_phyloseq }\OtherTok{\textless{}{-}}\NormalTok{ GlobalPatterns}
\NormalTok{GlobalPatterns\_phyloseq}
\end{Highlighting}
\end{Shaded}

\begin{verbatim}
## phyloseq-class experiment-level object
## otu_table()   OTU Table:         [ 19216 taxa and 26 samples ]
## sample_data() Sample Data:       [ 26 samples by 7 sample variables ]
## tax_table()   Taxonomy Table:    [ 19216 taxa by 7 taxonomic ranks ]
## phy_tree()    Phylogenetic Tree: [ 19216 tips and 19215 internal nodes ]
\end{verbatim}

\begin{Shaded}
\begin{Highlighting}[]
\CommentTok{\# convert phyloseq to TSE}
\NormalTok{GlobalPatterns\_TSE }\OtherTok{\textless{}{-}} \FunctionTok{makeTreeSummarizedExperimentFromPhyloseq}\NormalTok{(GlobalPatterns\_phyloseq) }
\NormalTok{GlobalPatterns\_TSE}
\end{Highlighting}
\end{Shaded}

\begin{verbatim}
## class: TreeSummarizedExperiment 
## dim: 19216 26 
## metadata(0):
## assays(1): counts
## rownames(19216): 549322 522457 ... 200359 271582
## rowData names(7): Kingdom Phylum ... Genus Species
## colnames(26): CL3 CC1 ... Even2 Even3
## colData names(7): X.SampleID Primer ... SampleType Description
## reducedDimNames(0):
## mainExpName: NULL
## altExpNames(0):
## rowLinks: a LinkDataFrame (19216 rows)
## rowTree: 1 phylo tree(s) (19216 leaves)
## colLinks: NULL
## colTree: NULL
\end{verbatim}

We can also convert \texttt{TreeSummarizedExperiment} objects into \texttt{phyloseq}
with respect to the shared components that are supported by both
formats (i.e.~taxonomic abundance table, sample metadata, taxonomic
table, phylogenetic tree, sequence information). This is useful for
instance when additional methods are available for \texttt{phyloseq}.

\begin{Shaded}
\begin{Highlighting}[]
\CommentTok{\# convert TSE to phyloseq}
\NormalTok{GlobalPatterns\_phyloseq2 }\OtherTok{\textless{}{-}} \FunctionTok{makePhyloseqFromTreeSummarizedExperiment}\NormalTok{(GlobalPatterns\_TSE) }
\NormalTok{GlobalPatterns\_phyloseq2}
\end{Highlighting}
\end{Shaded}

\begin{verbatim}
## phyloseq-class experiment-level object
## otu_table()   OTU Table:         [ 19216 taxa and 26 samples ]
## sample_data() Sample Data:       [ 26 samples by 7 sample variables ]
## tax_table()   Taxonomy Table:    [ 19216 taxa by 7 taxonomic ranks ]
## phy_tree()    Phylogenetic Tree: [ 19216 tips and 19215 internal nodes ]
\end{verbatim}

Conversion is possible between other data formats. Interested readers can refer to the following functions:
* \href{https://microbiome.github.io/mia/reference/makeTreeSummarizedExperimentFromDADA2.html}{makeTreeSummarizedExperimentFromDADA2}
* \href{https://microbiome.github.io/mia/reference/makeSummarizedExperimentFromBiom.html}{makeSummarizedExperimentFromBiom}
* \href{https://microbiome.github.io/mia/reference/loadFromMetaphlan.html}{loadFromMetaphlan}
* \href{https://microbiome.github.io/mia/reference/loadFromQIIME2.html}{readQZA}

\hypertarget{part-focus-topics}{%
\part{Focus Topics}\label{part-focus-topics}}

\hypertarget{datamanipulation}{%
\chapter{Data Manipulation}\label{datamanipulation}}

\hypertarget{tidying-and-subsetting}{%
\section{Tidying and subsetting}\label{tidying-and-subsetting}}

\hypertarget{tidy-data}{%
\subsection{Tidy data}\label{tidy-data}}

For several custom analysis and visualization packages, such as those from
\texttt{tidyverse}, the \texttt{SE} data can be converted to a long data.frame format with
\texttt{meltAssay}.

\begin{Shaded}
\begin{Highlighting}[]
\FunctionTok{library}\NormalTok{(mia)}
\FunctionTok{data}\NormalTok{(GlobalPatterns, }\AttributeTok{package=}\StringTok{"mia"}\NormalTok{)}
\NormalTok{tse }\OtherTok{\textless{}{-}}\NormalTok{ GlobalPatterns}
\NormalTok{tse }\OtherTok{\textless{}{-}} \FunctionTok{transformCounts}\NormalTok{(tse, }\AttributeTok{MARGIN =} \StringTok{"samples"}\NormalTok{, }\AttributeTok{method=}\StringTok{"relabundance"}\NormalTok{)}
\NormalTok{molten\_tse }\OtherTok{\textless{}{-}}\NormalTok{ mia}\SpecialCharTok{::}\FunctionTok{meltAssay}\NormalTok{(tse,}
                        \AttributeTok{add\_row\_data =} \ConstantTok{TRUE}\NormalTok{,}
                        \AttributeTok{add\_col\_data =} \ConstantTok{TRUE}\NormalTok{,}
                        \AttributeTok{assay.type =} \StringTok{"relabundance"}\NormalTok{)}
\NormalTok{molten\_tse}
\end{Highlighting}
\end{Shaded}

\begin{verbatim}
## # A tibble: 499,616 x 17
##    FeatureID SampleID relabundance Kingdom Phylum       Class Order Family Genus
##    <fct>     <fct>           <dbl> <chr>   <chr>        <chr> <chr> <chr>  <chr>
##  1 549322    CL3                 0 Archaea Crenarchaeo~ Ther~ <NA>  <NA>   <NA> 
##  2 549322    CC1                 0 Archaea Crenarchaeo~ Ther~ <NA>  <NA>   <NA> 
##  3 549322    SV1                 0 Archaea Crenarchaeo~ Ther~ <NA>  <NA>   <NA> 
##  4 549322    M31Fcsw             0 Archaea Crenarchaeo~ Ther~ <NA>  <NA>   <NA> 
##  5 549322    M11Fcsw             0 Archaea Crenarchaeo~ Ther~ <NA>  <NA>   <NA> 
##  6 549322    M31Plmr             0 Archaea Crenarchaeo~ Ther~ <NA>  <NA>   <NA> 
##  7 549322    M11Plmr             0 Archaea Crenarchaeo~ Ther~ <NA>  <NA>   <NA> 
##  8 549322    F21Plmr             0 Archaea Crenarchaeo~ Ther~ <NA>  <NA>   <NA> 
##  9 549322    M31Tong             0 Archaea Crenarchaeo~ Ther~ <NA>  <NA>   <NA> 
## 10 549322    M11Tong             0 Archaea Crenarchaeo~ Ther~ <NA>  <NA>   <NA> 
## # i 499,606 more rows
## # i 8 more variables: Species <chr>, X.SampleID <fct>, Primer <fct>,
## #   Final_Barcode <fct>, Barcode_truncated_plus_T <fct>,
## #   Barcode_full_length <fct>, SampleType <fct>, Description <fct>
\end{verbatim}

\hypertarget{subsetting}{%
\subsection{Subsetting}\label{subsetting}}

\textbf{Subsetting} data helps to draw the focus of analysis on particular
sets of samples and / or features. When dealing with large data
sets, the subset of interest can be extracted and investigated
separately. This might improve performance and reduce the
computational load.

Load:

\begin{itemize}
\tightlist
\item
  mia
\item
  dplyr
\item
  knitr
\item
  data \texttt{GlobalPatterns}
\end{itemize}

Let us store \texttt{GlobalPatterns} into \texttt{tse} and check its original number of features (rows) and samples (columns). \textbf{Note}: when subsetting by sample, expect the number of columns to decrease; when subsetting by feature, expect the number of rows to decrease.

\begin{Shaded}
\begin{Highlighting}[]
\CommentTok{\# Store data into se and check dimensions}
\FunctionTok{data}\NormalTok{(}\StringTok{"GlobalPatterns"}\NormalTok{, }\AttributeTok{package=}\StringTok{"mia"}\NormalTok{)}
\NormalTok{tse }\OtherTok{\textless{}{-}}\NormalTok{ GlobalPatterns}
\CommentTok{\# Show dimensions (features x samples)}
\FunctionTok{dim}\NormalTok{(tse) }
\end{Highlighting}
\end{Shaded}

\begin{verbatim}
## [1] 19216    26
\end{verbatim}

\hypertarget{subset-by-sample-column-wise}{%
\subsubsection{Subset by sample (column-wise)}\label{subset-by-sample-column-wise}}

For the sake of demonstration, here we will extract a subset containing only the samples of human origin (feces, skin or tongue), stored as \texttt{SampleType} within \texttt{colData(tse)} and also in \texttt{tse}.

First, we would like to see all the possible values that \texttt{SampleType} can take on and how frequent those are:

\begin{Shaded}
\begin{Highlighting}[]
\CommentTok{\# Inspect possible values for SampleType}
\FunctionTok{unique}\NormalTok{(tse}\SpecialCharTok{$}\NormalTok{SampleType)}
\end{Highlighting}
\end{Shaded}

\begin{verbatim}
## [1] Soil               Feces              Skin               Tongue            
## [5] Freshwater         Freshwater (creek) Ocean              Sediment (estuary)
## [9] Mock              
## 9 Levels: Feces Freshwater Freshwater (creek) Mock ... Tongue
\end{verbatim}

\begin{Shaded}
\begin{Highlighting}[]
\CommentTok{\# Show the frequency of each value}
\NormalTok{tse}\SpecialCharTok{$}\NormalTok{SampleType }\SpecialCharTok{\%\textgreater{}\%} \FunctionTok{table}\NormalTok{()}
\end{Highlighting}
\end{Shaded}

\begin{table}
\centering
\resizebox{\linewidth}{!}{
\begin{tabular}{l|r}
\hline
. & Freq\\
\hline
Feces & 4\\
\hline
Freshwater & 2\\
\hline
Freshwater (creek) & 3\\
\hline
Mock & 3\\
\hline
Ocean & 3\\
\hline
Sediment (estuary) & 3\\
\hline
Skin & 3\\
\hline
Soil & 3\\
\hline
Tongue & 2\\
\hline
\end{tabular}}
\end{table}

\textbf{Note}: after subsetting, expect the number of columns to equal the
sum of the frequencies of the samples that you are interested
in. For instance, \texttt{ncols\ =\ Feces\ +\ Skin\ +\ Tongue\ =\ 4\ +\ 3\ +\ 2\ =\ 9}.

Next, we \emph{logical index} across the columns of \texttt{tse} (make sure to
leave the first index empty to select all rows) and filter for the
samples of human origin. For this, we use the information on the
samples from the meta data \texttt{colData(tse)}.

\begin{Shaded}
\begin{Highlighting}[]
\CommentTok{\# Subset by sample}
\NormalTok{tse\_subset\_by\_sample }\OtherTok{\textless{}{-}}\NormalTok{ tse[ , tse}\SpecialCharTok{$}\NormalTok{SampleType }\SpecialCharTok{\%in\%} \FunctionTok{c}\NormalTok{(}\StringTok{"Feces"}\NormalTok{, }\StringTok{"Skin"}\NormalTok{, }\StringTok{"Tongue"}\NormalTok{)]}

\CommentTok{\# Show dimensions}
\FunctionTok{dim}\NormalTok{(tse\_subset\_by\_sample)}
\end{Highlighting}
\end{Shaded}

\begin{verbatim}
## [1] 19216     9
\end{verbatim}

As a sanity check, the new object \texttt{tse\_subset\_by\_sample} should have
the original number of features (rows) and a number of samples
(columns) equal to the sum of the samples of interest (in this case
9).

Several characteristics can be used to subset by sample:

\begin{itemize}
\tightlist
\item
  origin
\item
  sampling time
\item
  sequencing method
\item
  DNA / RNA barcode
\item
  cohort
\end{itemize}

\hypertarget{subset-by-feature-row-wise}{%
\subsubsection{Subset by feature (row-wise)}\label{subset-by-feature-row-wise}}

Similarly, here we will extract a subset containing only the features
that belong to the phyla Actinobacteria and Chlamydiae, stored as
\texttt{Phylum} within \texttt{rowData(tse)}. However, subsetting by feature implies
a few more obstacles, such as the presence of \texttt{NA} elements and the
possible need for agglomeration.

As previously, we would first like to see all the possible values that
\texttt{Phylum} can take on and how frequent those are:

\begin{Shaded}
\begin{Highlighting}[]
\CommentTok{\# Inspect possible values for phylum}
\FunctionTok{unique}\NormalTok{(}\FunctionTok{rowData}\NormalTok{(tse)}\SpecialCharTok{$}\NormalTok{Phylum)}
\end{Highlighting}
\end{Shaded}

\begin{verbatim}
##  [1] "Crenarchaeota"    "Euryarchaeota"    "Actinobacteria"   "Spirochaetes"    
##  [5] "MVP-15"           "Proteobacteria"   "SBR1093"          "Fusobacteria"    
##  [9] "Tenericutes"      "ZB3"              "Cyanobacteria"    "GOUTA4"          
## [13] "TG3"              "Chlorobi"         "Bacteroidetes"    "Caldithrix"      
## [17] "KSB1"             "SAR406"           "LCP-89"           "Thermi"          
## [21] "Gemmatimonadetes" "Fibrobacteres"    "GN06"             "AC1"             
## [25] "TM6"              "OP8"              "Elusimicrobia"    "NC10"            
## [29] "SPAM"             NA                 "Acidobacteria"    "CCM11b"          
## [33] "Nitrospirae"      "NKB19"            "BRC1"             "Hyd24-12"        
## [37] "WS3"              "PAUC34f"          "GN04"             "GN12"            
## [41] "Verrucomicrobia"  "Lentisphaerae"    "LD1"              "Chlamydiae"      
## [45] "OP3"              "Planctomycetes"   "Firmicutes"       "OP9"             
## [49] "WPS-2"            "Armatimonadetes"  "SC3"              "TM7"             
## [53] "GN02"             "SM2F11"           "ABY1_OD1"         "ZB2"             
## [57] "OP11"             "Chloroflexi"      "SC4"              "WS1"             
## [61] "GAL15"            "AD3"              "WS2"              "Caldiserica"     
## [65] "Thermotogae"      "Synergistetes"    "SR1"
\end{verbatim}

\begin{Shaded}
\begin{Highlighting}[]
\CommentTok{\# Show the frequency of each value}
\FunctionTok{rowData}\NormalTok{(tse)}\SpecialCharTok{$}\NormalTok{Phylum }\SpecialCharTok{\%\textgreater{}\%} \FunctionTok{table}\NormalTok{()}
\end{Highlighting}
\end{Shaded}

\begin{table}
\centering
\resizebox{\linewidth}{!}{
\begin{tabular}{l|r}
\hline
. & Freq\\
\hline
ABY1\_OD1 & 7\\
\hline
AC1 & 1\\
\hline
Acidobacteria & 1021\\
\hline
Actinobacteria & 1631\\
\hline
AD3 & 9\\
\hline
Armatimonadetes & 61\\
\hline
Bacteroidetes & 2382\\
\hline
BRC1 & 13\\
\hline
Caldiserica & 3\\
\hline
Caldithrix & 10\\
\hline
CCM11b & 2\\
\hline
Chlamydiae & 21\\
\hline
Chlorobi & 64\\
\hline
Chloroflexi & 437\\
\hline
Crenarchaeota & 106\\
\hline
Cyanobacteria & 393\\
\hline
Elusimicrobia & 31\\
\hline
Euryarchaeota & 102\\
\hline
Fibrobacteres & 7\\
\hline
Firmicutes & 4356\\
\hline
Fusobacteria & 37\\
\hline
GAL15 & 2\\
\hline
Gemmatimonadetes & 191\\
\hline
GN02 & 8\\
\hline
GN04 & 7\\
\hline
GN06 & 2\\
\hline
GN12 & 1\\
\hline
GOUTA4 & 11\\
\hline
Hyd24-12 & 4\\
\hline
KSB1 & 6\\
\hline
LCP-89 & 2\\
\hline
LD1 & 2\\
\hline
Lentisphaerae & 21\\
\hline
MVP-15 & 5\\
\hline
NC10 & 9\\
\hline
Nitrospirae & 74\\
\hline
NKB19 & 16\\
\hline
OP11 & 6\\
\hline
OP3 & 30\\
\hline
OP8 & 27\\
\hline
OP9 & 4\\
\hline
PAUC34f & 3\\
\hline
Planctomycetes & 638\\
\hline
Proteobacteria & 6416\\
\hline
SAR406 & 21\\
\hline
SBR1093 & 9\\
\hline
SC3 & 8\\
\hline
SC4 & 8\\
\hline
SM2F11 & 5\\
\hline
SPAM & 22\\
\hline
Spirochaetes & 124\\
\hline
SR1 & 5\\
\hline
Synergistetes & 7\\
\hline
Tenericutes & 143\\
\hline
TG3 & 5\\
\hline
Thermi & 46\\
\hline
Thermotogae & 1\\
\hline
TM6 & 27\\
\hline
TM7 & 32\\
\hline
Verrucomicrobia & 470\\
\hline
WPS-2 & 20\\
\hline
WS1 & 5\\
\hline
WS2 & 2\\
\hline
WS3 & 70\\
\hline
ZB2 & 2\\
\hline
ZB3 & 2\\
\hline
\end{tabular}}
\end{table}

\textbf{Note}: after subsetting, expect the number of columns to equal the
sum of the frequencies of the feature(s) that you are interested
in. For instance, \texttt{nrows\ =\ Actinobacteria\ +\ Chlamydiae\ =\ 1631\ +\ 21\ =\ \ \ 1652}.

Depending on your research question, you might or might not need to
agglomerate the data in the first place: if you want to find the
abundance of each and every feature that belongs to Actinobacteria and
Chlamydiae, agglomeration is not needed; if you want to find the total
abundance of all features that belong to Actinobacteria or
Chlamydiae, agglomeration is recommended.

\hypertarget{non-agglomerated-data}{%
\paragraph{Non-agglomerated data}\label{non-agglomerated-data}}

Next, we \emph{logical index} across the rows of \texttt{tse} (make sure to leave
the second index empty to select all columns) and filter for the
features that fall in either Actinobacteria or Chlamydiae group. For this,
we use the information on the samples from the metadata
\texttt{rowData(tse)}.

The first term with the \texttt{\%in\%} operator includes all the features
of interest, whereas the second term after the AND operator \texttt{\&}
filters out all features that have an \texttt{NA} in place of the phylum variable.

\begin{Shaded}
\begin{Highlighting}[]
\CommentTok{\# Subset by feature}
\NormalTok{tse\_subset\_by\_feature }\OtherTok{\textless{}{-}}\NormalTok{ tse[}\FunctionTok{rowData}\NormalTok{(tse)}\SpecialCharTok{$}\NormalTok{Phylum }\SpecialCharTok{\%in\%} \FunctionTok{c}\NormalTok{(}\StringTok{"Actinobacteria"}\NormalTok{, }\StringTok{"Chlamydiae"}\NormalTok{) }\SpecialCharTok{\&} \SpecialCharTok{!}\FunctionTok{is.na}\NormalTok{(}\FunctionTok{rowData}\NormalTok{(tse)}\SpecialCharTok{$}\NormalTok{Phylum), ]}

\CommentTok{\# Show dimensions}
\FunctionTok{dim}\NormalTok{(tse\_subset\_by\_feature)}
\end{Highlighting}
\end{Shaded}

\begin{verbatim}
## [1] 1652   26
\end{verbatim}

As a sanity check, the new object, \texttt{tse\_subset\_by\_feature}, should have the original number of samples (columns) and a number of features (rows) equal to the sum of the features of interest (in this case, 1652).

\hypertarget{agglomerated-data}{%
\paragraph{Agglomerated data}\label{agglomerated-data}}

When total abundances of certain phyla are of relevance, the data is initially agglomerated by Phylum. Then, similar steps as in the case of non-agglomerated data are followed.

\begin{Shaded}
\begin{Highlighting}[]
\CommentTok{\# Agglomerate by phylum}
\NormalTok{tse\_phylum }\OtherTok{\textless{}{-}}\NormalTok{ tse }\SpecialCharTok{\%\textgreater{}\%} \FunctionTok{agglomerateByRank}\NormalTok{(}\AttributeTok{rank =} \StringTok{"Phylum"}\NormalTok{)}

\CommentTok{\# Subset by feature and remove NAs}
\NormalTok{tse\_phylum\_subset\_by\_feature }\OtherTok{\textless{}{-}}\NormalTok{ tse\_phylum[}\FunctionTok{rowData}\NormalTok{(tse\_phylum)}\SpecialCharTok{$}\NormalTok{Phylum }\SpecialCharTok{\%in\%} \FunctionTok{c}\NormalTok{(}\StringTok{"Actinobacteria"}\NormalTok{, }\StringTok{"Chlamydiae"}\NormalTok{) }\SpecialCharTok{\&} \SpecialCharTok{!}\FunctionTok{is.na}\NormalTok{(}\FunctionTok{rowData}\NormalTok{(tse\_phylum)}\SpecialCharTok{$}\NormalTok{Phylum), ]}

\CommentTok{\# Show dimensions}
\FunctionTok{dim}\NormalTok{(tse\_phylum\_subset\_by\_feature)}
\end{Highlighting}
\end{Shaded}

\begin{verbatim}
## [1]  2 26
\end{verbatim}

\textbf{Note}: as data was agglomerated, the number of rows should equal the
number of phyla used to index (in this case, just 2).

Alternatively:

\begin{Shaded}
\begin{Highlighting}[]
\CommentTok{\# Store features of interest into phyla}
\NormalTok{phyla }\OtherTok{\textless{}{-}} \FunctionTok{c}\NormalTok{(}\StringTok{"Phylum:Actinobacteria"}\NormalTok{, }\StringTok{"Phylum:Chlamydiae"}\NormalTok{)}
\CommentTok{\# subset by feature}
\NormalTok{tse\_phylum\_subset\_by\_feature }\OtherTok{\textless{}{-}}\NormalTok{ tse\_phylum[phyla, ]}
\CommentTok{\# Show dimensions}
\FunctionTok{dim}\NormalTok{(tse\_subset\_by\_feature)}
\end{Highlighting}
\end{Shaded}

\begin{verbatim}
## [1] 1652   26
\end{verbatim}

The code above returns the non-agglomerated version of the data.

Fewer characteristics can be used to subset by feature:

\begin{itemize}
\tightlist
\item
  Taxonomic rank
\item
  Meta-taxonomic group
\end{itemize}

For subsetting by kingdom, agglomeration does not apply, whereas for
the other ranks it can be applied if necessary.

\hypertarget{subset-by-sample-and-feature}{%
\subsubsection{Subset by sample and feature}\label{subset-by-sample-and-feature}}

Finally, we can subset data by sample and feature at once. The
resulting subset contains all the samples of human origin and all the
features of phyla Actinobacteria or Chlamydiae.

\begin{Shaded}
\begin{Highlighting}[]
\CommentTok{\# Subset by sample and feature and remove NAs}
\NormalTok{tse\_subset\_by\_sample\_feature }\OtherTok{\textless{}{-}}\NormalTok{ tse[}\FunctionTok{rowData}\NormalTok{(tse)}\SpecialCharTok{$}\NormalTok{Phylum }\SpecialCharTok{\%in\%} \FunctionTok{c}\NormalTok{(}\StringTok{"Actinobacteria"}\NormalTok{, }\StringTok{"Chlamydiae"}\NormalTok{) }\SpecialCharTok{\&} \SpecialCharTok{!}\FunctionTok{is.na}\NormalTok{(}\FunctionTok{rowData}\NormalTok{(tse)}\SpecialCharTok{$}\NormalTok{Phylum), tse}\SpecialCharTok{$}\NormalTok{SampleType }\SpecialCharTok{\%in\%} \FunctionTok{c}\NormalTok{(}\StringTok{"Feces"}\NormalTok{, }\StringTok{"Skin"}\NormalTok{, }\StringTok{"Tongue"}\NormalTok{)]}

\CommentTok{\# Show dimensions}
\FunctionTok{dim}\NormalTok{(tse\_subset\_by\_sample\_feature)}
\end{Highlighting}
\end{Shaded}

\begin{verbatim}
## [1] 1652    9
\end{verbatim}

\textbf{Note}: the dimensions of \texttt{tse\_subset\_by\_sample\_feature} agree with
those of the previous subsets (9 columns filtered by sample and 1652
rows filtered by feature).

If a study was to consider and quantify the presence of Actinobacteria
as well as Chlamydiae in different sites of the human body,
\texttt{tse\_subset\_by\_sample\_feature} might be a suitable subset to start
with.

\hypertarget{remove-empty-columns-and-rows}{%
\subsubsection{Remove empty columns and rows}\label{remove-empty-columns-and-rows}}

Sometimes data might contain, e.g., features that are not present in any of the samples.
This can occur, for example, after the data subsetting. In certain analyses, we might want to
remove those instances.

\begin{Shaded}
\begin{Highlighting}[]
\CommentTok{\# Agglomerate data at Genus level }
\NormalTok{tse\_genus }\OtherTok{\textless{}{-}} \FunctionTok{agglomerateByRank}\NormalTok{(tse, }\AttributeTok{rank =} \StringTok{"Genus"}\NormalTok{)}
\CommentTok{\# List bacteria that we want to include}
\NormalTok{genera }\OtherTok{\textless{}{-}} \FunctionTok{c}\NormalTok{(}\StringTok{"Class:Thermoprotei"}\NormalTok{, }\StringTok{"Genus:Sulfolobus"}\NormalTok{, }\StringTok{"Genus:Sediminicola"}\NormalTok{)}
\CommentTok{\# Subset data}
\NormalTok{tse\_genus\_sub }\OtherTok{\textless{}{-}}\NormalTok{ tse\_genus[genera, ]}

\NormalTok{tse\_genus\_sub}
\end{Highlighting}
\end{Shaded}

\begin{verbatim}
## class: TreeSummarizedExperiment 
## dim: 3 26 
## metadata(1): agglomerated_by_rank
## assays(1): counts
## rownames(3): Class:Thermoprotei Genus:Sulfolobus Genus:Sediminicola
## rowData names(7): Kingdom Phylum ... Genus Species
## colnames(26): CL3 CC1 ... Even2 Even3
## colData names(7): X.SampleID Primer ... SampleType Description
## reducedDimNames(0):
## mainExpName: NULL
## altExpNames(0):
## rowLinks: a LinkDataFrame (3 rows)
## rowTree: 1 phylo tree(s) (19216 leaves)
## colLinks: NULL
## colTree: NULL
\end{verbatim}

\begin{Shaded}
\begin{Highlighting}[]
\CommentTok{\# List total counts of each sample}
\FunctionTok{colSums}\NormalTok{(}\FunctionTok{assay}\NormalTok{(tse\_genus\_sub, }\StringTok{"counts"}\NormalTok{))}
\end{Highlighting}
\end{Shaded}

\begin{verbatim}
##      CL3      CC1      SV1  M31Fcsw  M11Fcsw  M31Plmr  M11Plmr  F21Plmr 
##        1        0        0        1        1        0        4        1 
##  M31Tong  M11Tong LMEpi24M SLEpi20M   AQC1cm   AQC4cm   AQC7cm      NP2 
##        7        3        0        2       64      105      136      222 
##      NP3      NP5  TRRsed1  TRRsed2  TRRsed3     TS28     TS29    Even1 
##     6433     1154        2        2        2        0        0        0 
##    Even2    Even3 
##        2        0
\end{verbatim}

Now we can see that certain samples do not include any bacteria. We can remove those.

\begin{Shaded}
\begin{Highlighting}[]
\CommentTok{\# Remove samples that do not contain any bacteria}
\NormalTok{tse\_genus\_sub }\OtherTok{\textless{}{-}}\NormalTok{ tse\_genus\_sub[ , }\FunctionTok{colSums}\NormalTok{(}\FunctionTok{assay}\NormalTok{(tse\_genus\_sub, }\StringTok{"counts"}\NormalTok{)) }\SpecialCharTok{!=} \DecValTok{0}\NormalTok{ ]}
\NormalTok{tse\_genus\_sub}
\end{Highlighting}
\end{Shaded}

\begin{verbatim}
## class: TreeSummarizedExperiment 
## dim: 3 18 
## metadata(1): agglomerated_by_rank
## assays(1): counts
## rownames(3): Class:Thermoprotei Genus:Sulfolobus Genus:Sediminicola
## rowData names(7): Kingdom Phylum ... Genus Species
## colnames(18): CL3 M31Fcsw ... TRRsed3 Even2
## colData names(7): X.SampleID Primer ... SampleType Description
## reducedDimNames(0):
## mainExpName: NULL
## altExpNames(0):
## rowLinks: a LinkDataFrame (3 rows)
## rowTree: 1 phylo tree(s) (19216 leaves)
## colLinks: NULL
## colTree: NULL
\end{verbatim}

The same action can also be applied to the features.

\begin{Shaded}
\begin{Highlighting}[]
\CommentTok{\# Take only those samples that are collected from feces, skin, or tongue}
\NormalTok{tse\_genus\_sub }\OtherTok{\textless{}{-}}\NormalTok{ tse\_genus[ , }\FunctionTok{colData}\NormalTok{(tse\_genus)}\SpecialCharTok{$}\NormalTok{SampleType }\SpecialCharTok{\%in\%} \FunctionTok{c}\NormalTok{(}\StringTok{"Feces"}\NormalTok{, }\StringTok{"Skin"}\NormalTok{, }\StringTok{"Tongue"}\NormalTok{)]}

\NormalTok{tse\_genus\_sub}
\end{Highlighting}
\end{Shaded}

\begin{verbatim}
## class: TreeSummarizedExperiment 
## dim: 1516 9 
## metadata(1): agglomerated_by_rank
## assays(1): counts
## rownames(1516): Class:Thermoprotei Genus:Sulfolobus ...
##   Genus:Coprothermobacter Phylum:SR1
## rowData names(7): Kingdom Phylum ... Genus Species
## colnames(9): M31Fcsw M11Fcsw ... TS28 TS29
## colData names(7): X.SampleID Primer ... SampleType Description
## reducedDimNames(0):
## mainExpName: NULL
## altExpNames(0):
## rowLinks: a LinkDataFrame (1516 rows)
## rowTree: 1 phylo tree(s) (19216 leaves)
## colLinks: NULL
## colTree: NULL
\end{verbatim}

\begin{Shaded}
\begin{Highlighting}[]
\CommentTok{\# What is the number of bacteria that are not present?}
\FunctionTok{sum}\NormalTok{(}\FunctionTok{rowSums}\NormalTok{(}\FunctionTok{assay}\NormalTok{(tse\_genus\_sub, }\StringTok{"counts"}\NormalTok{)) }\SpecialCharTok{==} \DecValTok{0}\NormalTok{)}
\end{Highlighting}
\end{Shaded}

\begin{verbatim}
## [1] 435
\end{verbatim}

We can see that there are bacteria that are not present in these samples we chose.
We can remove those bacteria from the data.

\begin{Shaded}
\begin{Highlighting}[]
\CommentTok{\# Take only those bacteria that are present}
\NormalTok{tse\_genus\_sub }\OtherTok{\textless{}{-}}\NormalTok{ tse\_genus\_sub[}\FunctionTok{rowSums}\NormalTok{(}\FunctionTok{assay}\NormalTok{(tse\_genus\_sub, }\StringTok{"counts"}\NormalTok{)) }\SpecialCharTok{\textgreater{}} \DecValTok{0}\NormalTok{, ]}

\NormalTok{tse\_genus\_sub}
\end{Highlighting}
\end{Shaded}

\begin{verbatim}
## class: TreeSummarizedExperiment 
## dim: 1081 9 
## metadata(1): agglomerated_by_rank
## assays(1): counts
## rownames(1081): Genus:Sulfolobus Order:NRP-J ...
##   Genus:Coprothermobacter Phylum:SR1
## rowData names(7): Kingdom Phylum ... Genus Species
## colnames(9): M31Fcsw M11Fcsw ... TS28 TS29
## colData names(7): X.SampleID Primer ... SampleType Description
## reducedDimNames(0):
## mainExpName: NULL
## altExpNames(0):
## rowLinks: a LinkDataFrame (1081 rows)
## rowTree: 1 phylo tree(s) (19216 leaves)
## colLinks: NULL
## colTree: NULL
\end{verbatim}

\hypertarget{splitting}{%
\subsection{Splitting}\label{splitting}}

You can split the data based on variables by using the functions \texttt{splitByRanks}
and \texttt{splitOn}.

\texttt{splitByRanks} splits the data based on taxonomic ranks. Since the elements of the output list
share columns, they can be stored into \texttt{altExp}.

\begin{Shaded}
\begin{Highlighting}[]
\FunctionTok{altExps}\NormalTok{(tse) }\OtherTok{\textless{}{-}} \FunctionTok{splitByRanks}\NormalTok{(tse)}
\FunctionTok{altExps}\NormalTok{(tse)}
\end{Highlighting}
\end{Shaded}

\begin{verbatim}
## List of length 7
## names(7): Kingdom Phylum Class Order Family Genus Species
\end{verbatim}

If you want to split the data based on another variable than taxonomic rank, use
\texttt{splitOn}. It works for row-wise and column-wise splitting.

\begin{Shaded}
\begin{Highlighting}[]
\FunctionTok{splitOn}\NormalTok{(tse, }\StringTok{"SampleType"}\NormalTok{)}
\end{Highlighting}
\end{Shaded}

\begin{verbatim}
## List of length 9
## names(9): Soil Feces Skin Tongue ... Ocean Sediment (estuary) Mock
\end{verbatim}

\hypertarget{add-or-modify-data}{%
\section{Add or modify data}\label{add-or-modify-data}}

The information contained by the \texttt{colData} of a \texttt{TreeSE} can be modified by
accessing the desired variables.

\begin{Shaded}
\begin{Highlighting}[]
\CommentTok{\# modify the Description entries}
\FunctionTok{colData}\NormalTok{(tse)}\SpecialCharTok{$}\NormalTok{Description }\OtherTok{\textless{}{-}} \FunctionTok{paste}\NormalTok{(}\FunctionTok{colData}\NormalTok{(tse)}\SpecialCharTok{$}\NormalTok{Description, }\StringTok{"modified description"}\NormalTok{)}

\CommentTok{\# view modified variable}
\FunctionTok{head}\NormalTok{(tse}\SpecialCharTok{$}\NormalTok{Description)}
\end{Highlighting}
\end{Shaded}

\begin{verbatim}
## [1] "Calhoun South Carolina Pine soil, pH 4.9 modified description"  
## [2] "Cedar Creek Minnesota, grassland, pH 6.1 modified description"  
## [3] "Sevilleta new Mexico, desert scrub, pH 8.3 modified description"
## [4] "M3, Day 1, fecal swab, whole body study modified description"   
## [5] "M1, Day 1, fecal swab, whole body study  modified description"  
## [6] "M3, Day 1, right palm, whole body study modified description"
\end{verbatim}

New information can also be added to the experiment by creating a new variable.

\begin{Shaded}
\begin{Highlighting}[]
\CommentTok{\# simulate new data}
\NormalTok{new\_data }\OtherTok{\textless{}{-}} \FunctionTok{runif}\NormalTok{(}\FunctionTok{ncol}\NormalTok{(tse))}

\CommentTok{\# store new data as new variable in colData}
\FunctionTok{colData}\NormalTok{(tse)}\SpecialCharTok{$}\NormalTok{NewVariable }\OtherTok{\textless{}{-}}\NormalTok{ new\_data}

\CommentTok{\# view new variable}
\FunctionTok{head}\NormalTok{(tse}\SpecialCharTok{$}\NormalTok{NewVariable)}
\end{Highlighting}
\end{Shaded}

\begin{verbatim}
## [1] 0.2428 0.7404 0.4172 0.3880 0.4568 0.9046
\end{verbatim}

\hypertarget{merge-data}{%
\section{Merge data}\label{merge-data}}

\texttt{mia} package has \texttt{mergeSEs} function that merges multiple \texttt{SummarizedExperiment}
objects. For example, it is possible to combine multiple \texttt{TreeSE} objects which each
includes one sample.

\texttt{mergeSEs} works like \texttt{dplyr} joining functions. In fact, there are available
\texttt{dplyr-like} aliases of \texttt{mergeSEs}, such as \texttt{full\_join}.

\begin{Shaded}
\begin{Highlighting}[]
\CommentTok{\# Take subsets for demonstration purposes}
\NormalTok{tse1 }\OtherTok{\textless{}{-}}\NormalTok{ tse[, }\DecValTok{1}\NormalTok{]}
\NormalTok{tse2 }\OtherTok{\textless{}{-}}\NormalTok{ tse[, }\DecValTok{2}\NormalTok{]}
\NormalTok{tse3 }\OtherTok{\textless{}{-}}\NormalTok{ tse[, }\DecValTok{3}\NormalTok{]}
\NormalTok{tse4 }\OtherTok{\textless{}{-}}\NormalTok{ tse[}\DecValTok{1}\SpecialCharTok{:}\DecValTok{100}\NormalTok{, }\DecValTok{4}\NormalTok{]}
\end{Highlighting}
\end{Shaded}

\begin{Shaded}
\begin{Highlighting}[]
\CommentTok{\# With inner join, we want to include all shared rows. When using mergeSEs function}
\CommentTok{\# all samples are always preserved.}
\NormalTok{tse }\OtherTok{\textless{}{-}} \FunctionTok{mergeSEs}\NormalTok{(}\FunctionTok{list}\NormalTok{(tse1, tse2, tse3, tse4), }\AttributeTok{join =} \StringTok{"inner"}\NormalTok{)}
\NormalTok{tse}
\end{Highlighting}
\end{Shaded}

\begin{verbatim}
## class: TreeSummarizedExperiment 
## dim: 100 4 
## metadata(0):
## assays(1): counts
## rownames(100): 239672 243675 ... 549322 951
## rowData names(7): Kingdom Phylum ... Genus Species
## colnames(4): CC1 CL3 M31Fcsw SV1
## colData names(8): X.SampleID Primer ... Description NewVariable
## reducedDimNames(0):
## mainExpName: NULL
## altExpNames(0):
## rowLinks: a LinkDataFrame (100 rows)
## rowTree: 1 phylo tree(s) (19216 leaves)
## colLinks: NULL
## colTree: NULL
\end{verbatim}

\begin{Shaded}
\begin{Highlighting}[]
\CommentTok{\# Left join preserves all rows of the 1st object}
\NormalTok{tse }\OtherTok{\textless{}{-}}\NormalTok{ mia}\SpecialCharTok{::}\FunctionTok{left\_join}\NormalTok{(tse1, tse4, }\AttributeTok{missing\_values =} \DecValTok{0}\NormalTok{)}
\NormalTok{tse}
\end{Highlighting}
\end{Shaded}

\begin{verbatim}
## class: TreeSummarizedExperiment 
## dim: 19216 2 
## metadata(0):
## assays(1): counts
## rownames(19216): 239672 243675 ... 239967 254851
## rowData names(7): Kingdom Phylum ... Genus Species
## colnames(2): CL3 M31Fcsw
## colData names(8): X.SampleID Primer ... Description NewVariable
## reducedDimNames(0):
## mainExpName: NULL
## altExpNames(0):
## rowLinks: a LinkDataFrame (19216 rows)
## rowTree: 1 phylo tree(s) (19216 leaves)
## colLinks: NULL
## colTree: NULL
\end{verbatim}

\hypertarget{additional-functions}{%
\subsection{Additional functions}\label{additional-functions}}

\begin{itemize}
\tightlist
\item
  \href{https://microbiome.github.io/mia/reference/taxonomy-methods.html}{mapTaxonomy}
\item
  \href{https://microbiome.github.io/mia/reference/merge-methods.html}{mergeRows/mergeCols}
\end{itemize}

\hypertarget{quality-control}{%
\chapter{Exploration and quality Control}\label{quality-control}}

This chapter focuses on the quality control and exploration of
microbiome data and establishes commonly used descriptive
summaries. Familiarizing with the peculiarities of a given data set is
the essential basis for any data analysis and model building.

The dataset should not suffer from severe technical biases, and you
should at least be aware of potential challenges, such as outliers,
biases, unexpected patterns and so forth. Standard summaries and
visualizations can help, and the rest comes with experience. The
exploration and quality control can be iterative processes.

\begin{Shaded}
\begin{Highlighting}[]
\FunctionTok{library}\NormalTok{(mia)}
\end{Highlighting}
\end{Shaded}

\hypertarget{abundance}{%
\section{Abundance}\label{abundance}}

Abundance visualization is an important data exploration
approach. \texttt{miaViz} offers the function \texttt{plotAbundanceDensity} to plot
the most abundant taxa with several options.

Next, a few demonstrations are shown, using the \citep{Lahti2014}
dataset. A Jitter plot based on relative abundance data, similar to
the one presented at \citep{Salosensaari2021} supplementary figure 1, could
be visualized as follows:

\begin{Shaded}
\begin{Highlighting}[]
\CommentTok{\# Load example data}
\FunctionTok{library}\NormalTok{(miaTime)}
\FunctionTok{library}\NormalTok{(miaViz)}
\FunctionTok{data}\NormalTok{(hitchip1006)}
\NormalTok{tse }\OtherTok{\textless{}{-}}\NormalTok{ hitchip1006}

\CommentTok{\# Add relative abundances}
\NormalTok{tse }\OtherTok{\textless{}{-}} \FunctionTok{transformCounts}\NormalTok{(tse, }\AttributeTok{MARGIN =} \StringTok{"samples"}\NormalTok{, }\AttributeTok{method =} \StringTok{"relabundance"}\NormalTok{)}

\CommentTok{\# Use argument names}
\CommentTok{\# assay.type / assay.type / assay.type}
\CommentTok{\# depending on the mia package version}
\FunctionTok{plotAbundanceDensity}\NormalTok{(tse, }\AttributeTok{layout =} \StringTok{"jitter"}\NormalTok{, }\AttributeTok{assay.type =} \StringTok{"relabundance"}\NormalTok{,}
                     \AttributeTok{n =} \DecValTok{40}\NormalTok{, }\AttributeTok{point\_size=}\DecValTok{1}\NormalTok{, }\AttributeTok{point\_shape=}\DecValTok{19}\NormalTok{, }\AttributeTok{point\_alpha=}\FloatTok{0.1}\NormalTok{) }\SpecialCharTok{+} 
                     \FunctionTok{scale\_x\_log10}\NormalTok{(}\AttributeTok{label=}\NormalTok{scales}\SpecialCharTok{::}\NormalTok{percent)}
\end{Highlighting}
\end{Shaded}

\includegraphics{12_quality_control_files/figure-latex/unnamed-chunk-2-1.pdf}

The relative abundance values for the top-5 taxonomic features can be
visualized as a density plot over a log scaled axis, with
``nationality'' indicated by colors:

\begin{Shaded}
\begin{Highlighting}[]
\FunctionTok{plotAbundanceDensity}\NormalTok{(tse, }\AttributeTok{layout =} \StringTok{"density"}\NormalTok{, }\AttributeTok{assay.type =} \StringTok{"relabundance"}\NormalTok{,}
                     \AttributeTok{n =} \DecValTok{5}\NormalTok{, }\AttributeTok{colour\_by=}\StringTok{"nationality"}\NormalTok{, }\AttributeTok{point\_alpha=}\DecValTok{1}\SpecialCharTok{/}\DecValTok{10}\NormalTok{) }\SpecialCharTok{+}
    \FunctionTok{scale\_x\_log10}\NormalTok{()}
\end{Highlighting}
\end{Shaded}

\includegraphics{12_quality_control_files/figure-latex/unnamed-chunk-3-1.pdf}

\hypertarget{prevalence}{%
\section{Prevalence}\label{prevalence}}

Prevalence quantifies the frequency of samples where certain microbes
were detected (above a given detection threshold). The prevalence can
be given as sample size (N) or percentage (unit interval).

Investigating prevalence allows you either to focus on changes which
pertain to the majority of the samples, or identify rare microbes,
which may be \emph{conditionally abundant} in a small number of samples.

The population prevalence (frequency) at a 1\% relative abundance
threshold (\texttt{detection\ =\ 1/100} and \texttt{as\_relative\ =\ TRUE}), can look
like this.

\begin{Shaded}
\begin{Highlighting}[]
\FunctionTok{head}\NormalTok{(}\FunctionTok{getPrevalence}\NormalTok{(tse, }\AttributeTok{detection =} \DecValTok{1}\SpecialCharTok{/}\DecValTok{100}\NormalTok{, }\AttributeTok{sort =} \ConstantTok{TRUE}\NormalTok{, }\AttributeTok{as\_relative =} \ConstantTok{TRUE}\NormalTok{))}
\end{Highlighting}
\end{Shaded}

\begin{verbatim}
## Faecalibacterium prausnitzii et rel.           Ruminococcus obeum et rel. 
##                               0.9522                               0.9140 
##   Oscillospira guillermondii et rel.        Clostridium symbiosum et rel. 
##                               0.8801                               0.8714 
##     Subdoligranulum variable at rel.     Clostridium orbiscindens et rel. 
##                               0.8358                               0.8315
\end{verbatim}

The function arguments \texttt{detection} and \texttt{as\_relative} can also be used
to access, how many samples do pass a threshold for raw counts. Here,
the population prevalence (frequency) at the absolute abundance
threshold (\texttt{as\_relative\ =\ FALSE}) at read count 1 (\texttt{detection\ =\ 1}) is
accessed.

\begin{Shaded}
\begin{Highlighting}[]
\FunctionTok{head}\NormalTok{(}\FunctionTok{getPrevalence}\NormalTok{(tse, }\AttributeTok{detection =} \DecValTok{1}\NormalTok{, }\AttributeTok{sort =} \ConstantTok{TRUE}\NormalTok{, }\AttributeTok{assay.type =} \StringTok{"counts"}\NormalTok{,}
                   \AttributeTok{as\_relative =} \ConstantTok{FALSE}\NormalTok{))}
\end{Highlighting}
\end{Shaded}

\begin{verbatim}
##            Uncultured Mollicutes      Uncultured Clostridiales II 
##                                1                                1 
##       Uncultured Clostridiales I               Tannerella et rel. 
##                                1                                1 
##   Sutterella wadsworthia et rel. Subdoligranulum variable at rel. 
##                                1                                1
\end{verbatim}

If the output should be used for subsetting or storing the data in the
\texttt{rowData}, set \texttt{sort\ =\ FALSE}.

\hypertarget{prevalence-analysis}{%
\subsection{Prevalence analysis}\label{prevalence-analysis}}

To investigate microbiome prevalence at a selected taxonomic level, two
approaches are available.

First the data can be agglomerated to the taxonomic level and \texttt{getPrevalence}
applied on the resulting object.

\begin{Shaded}
\begin{Highlighting}[]
\CommentTok{\# Agglomerate taxa abundances to Phylum level, and add the new table}
\CommentTok{\# to the altExp slot}
\FunctionTok{altExp}\NormalTok{(tse,}\StringTok{"Phylum"}\NormalTok{) }\OtherTok{\textless{}{-}} \FunctionTok{agglomerateByRank}\NormalTok{(tse, }\StringTok{"Phylum"}\NormalTok{)}
\CommentTok{\# Check prevalence for the Phylum abundance table from the altExp slot}
\FunctionTok{head}\NormalTok{(}\FunctionTok{getPrevalence}\NormalTok{(}\FunctionTok{altExp}\NormalTok{(tse,}\StringTok{"Phylum"}\NormalTok{), }\AttributeTok{detection =} \DecValTok{1}\SpecialCharTok{/}\DecValTok{100}\NormalTok{, }\AttributeTok{sort =} \ConstantTok{TRUE}\NormalTok{,}
                   \AttributeTok{assay.type =} \StringTok{"counts"}\NormalTok{, }\AttributeTok{as\_relative =} \ConstantTok{TRUE}\NormalTok{))}
\end{Highlighting}
\end{Shaded}

\begin{verbatim}
##      Firmicutes   Bacteroidetes  Actinobacteria  Proteobacteria Verrucomicrobia 
##       1.0000000       0.9852302       0.4821894       0.2988705       0.1277150 
##   Cyanobacteria 
##       0.0008688
\end{verbatim}

Alternatively, the \texttt{rank} argument could be set to perform the
agglomeration on the fly.

\begin{Shaded}
\begin{Highlighting}[]
\FunctionTok{head}\NormalTok{(}\FunctionTok{getPrevalence}\NormalTok{(tse, }\AttributeTok{rank =} \StringTok{"Phylum"}\NormalTok{, }\AttributeTok{detection =} \DecValTok{1}\SpecialCharTok{/}\DecValTok{100}\NormalTok{, }\AttributeTok{sort =} \ConstantTok{TRUE}\NormalTok{,}
                   \AttributeTok{assay.type =} \StringTok{"counts"}\NormalTok{, }\AttributeTok{as\_relative =} \ConstantTok{TRUE}\NormalTok{))}
\end{Highlighting}
\end{Shaded}

\begin{verbatim}
##      Firmicutes   Bacteroidetes  Actinobacteria  Proteobacteria Verrucomicrobia 
##       1.0000000       0.9852302       0.4821894       0.2988705       0.1277150 
##   Cyanobacteria 
##       0.0008688
\end{verbatim}

Note that, by default, \texttt{na.rm\ =\ TRUE} is used for agglomeration in
\texttt{getPrevalence}, whereas the default for \texttt{agglomerateByRank} is
\texttt{FALSE} to prevent accidental data loss.

If you only need the names of the prevalent taxa, \texttt{getPrevalentTaxa}
is available. This returns the taxa that exceed the given prevalence
and detection thresholds.

\begin{Shaded}
\begin{Highlighting}[]
\FunctionTok{getPrevalentTaxa}\NormalTok{(tse, }\AttributeTok{detection =} \DecValTok{0}\NormalTok{, }\AttributeTok{prevalence =} \DecValTok{50}\SpecialCharTok{/}\DecValTok{100}\NormalTok{)}
\NormalTok{prev }\OtherTok{\textless{}{-}} \FunctionTok{getPrevalentTaxa}\NormalTok{(tse, }\AttributeTok{detection =} \DecValTok{0}\NormalTok{, }\AttributeTok{prevalence =} \DecValTok{50}\SpecialCharTok{/}\DecValTok{100}\NormalTok{,}
                         \AttributeTok{rank =} \StringTok{"Phylum"}\NormalTok{, }\AttributeTok{sort =} \ConstantTok{TRUE}\NormalTok{)}
\NormalTok{prev}
\end{Highlighting}
\end{Shaded}

Note that the \texttt{detection} and \texttt{prevalence} thresholds are not the same, since
\texttt{detection} can be applied to relative counts or absolute counts depending on
whether \texttt{as\_relative} is set \texttt{TRUE} or \texttt{FALSE}

The function `getPrevalentAbundance' can be used to check the total
relative abundance of the prevalent taxa (between 0 and 1).

\hypertarget{rare-taxa}{%
\subsection{Rare taxa}\label{rare-taxa}}

Related functions are available for the analysis of rare taxa
(\texttt{rareMembers}; \texttt{rareAbundance}; \texttt{lowAbundance}, \texttt{getRareTaxa},
\texttt{subsetByRareTaxa}).

\hypertarget{plotting-prevalence}{%
\subsection{Plotting prevalence}\label{plotting-prevalence}}

To plot the prevalence, add the prevalence of each taxon to
\texttt{rowData}. Here, we are analysing the Phylum level abundances, which
are stored in the \texttt{altExp} slot.

\begin{Shaded}
\begin{Highlighting}[]
\FunctionTok{rowData}\NormalTok{(}\FunctionTok{altExp}\NormalTok{(tse,}\StringTok{"Phylum"}\NormalTok{))}\SpecialCharTok{$}\NormalTok{prevalence }\OtherTok{\textless{}{-}} 
    \FunctionTok{getPrevalence}\NormalTok{(}\FunctionTok{altExp}\NormalTok{(tse,}\StringTok{"Phylum"}\NormalTok{), }\AttributeTok{detection =} \DecValTok{1}\SpecialCharTok{/}\DecValTok{100}\NormalTok{, }\AttributeTok{sort =} \ConstantTok{FALSE}\NormalTok{,}
                  \AttributeTok{assay.type =} \StringTok{"counts"}\NormalTok{, }\AttributeTok{as\_relative =} \ConstantTok{TRUE}\NormalTok{)}
\end{Highlighting}
\end{Shaded}

The prevalences can then be plotted using the plotting functions from
the \texttt{scater} package.

\begin{Shaded}
\begin{Highlighting}[]
\FunctionTok{library}\NormalTok{(scater)}
\FunctionTok{plotRowData}\NormalTok{(}\FunctionTok{altExp}\NormalTok{(tse,}\StringTok{"Phylum"}\NormalTok{), }\StringTok{"prevalence"}\NormalTok{, }\AttributeTok{colour\_by =} \StringTok{"Phylum"}\NormalTok{)}
\end{Highlighting}
\end{Shaded}

\includegraphics{12_quality_control_files/figure-latex/unnamed-chunk-7-1.pdf}

The prevalence can also be visualized on the taxonomic tree with the
\texttt{miaViz} package.

\begin{Shaded}
\begin{Highlighting}[]
\FunctionTok{altExps}\NormalTok{(tse) }\OtherTok{\textless{}{-}} \FunctionTok{splitByRanks}\NormalTok{(tse)}
\FunctionTok{altExps}\NormalTok{(tse) }\OtherTok{\textless{}{-}}
   \FunctionTok{lapply}\NormalTok{(}\FunctionTok{altExps}\NormalTok{(tse),}
          \ControlFlowTok{function}\NormalTok{(y)\{}
              \FunctionTok{rowData}\NormalTok{(y)}\SpecialCharTok{$}\NormalTok{prevalence }\OtherTok{\textless{}{-}} 
                  \FunctionTok{getPrevalence}\NormalTok{(y, }\AttributeTok{detection =} \DecValTok{1}\SpecialCharTok{/}\DecValTok{100}\NormalTok{, }\AttributeTok{sort =} \ConstantTok{FALSE}\NormalTok{,}
                                \AttributeTok{assay.type =} \StringTok{"counts"}\NormalTok{, }\AttributeTok{as\_relative =} \ConstantTok{TRUE}\NormalTok{)}
\NormalTok{              y}
\NormalTok{          \})}
\NormalTok{top\_phyla }\OtherTok{\textless{}{-}} \FunctionTok{getTopTaxa}\NormalTok{(}\FunctionTok{altExp}\NormalTok{(tse,}\StringTok{"Phylum"}\NormalTok{),}
                        \AttributeTok{method=}\StringTok{"prevalence"}\NormalTok{,}
                        \AttributeTok{top=}\NormalTok{5L,}
                        \AttributeTok{assay.type=}\StringTok{"counts"}\NormalTok{)}
\NormalTok{top\_phyla\_mean }\OtherTok{\textless{}{-}} \FunctionTok{getTopTaxa}\NormalTok{(}\FunctionTok{altExp}\NormalTok{(tse,}\StringTok{"Phylum"}\NormalTok{),}
                             \AttributeTok{method=}\StringTok{"mean"}\NormalTok{,}
                             \AttributeTok{top=}\NormalTok{5L,}
                             \AttributeTok{assay.type=}\StringTok{"counts"}\NormalTok{)}
\NormalTok{x }\OtherTok{\textless{}{-}} \FunctionTok{unsplitByRanks}\NormalTok{(tse, }\AttributeTok{ranks =} \FunctionTok{taxonomyRanks}\NormalTok{(tse)[}\DecValTok{1}\SpecialCharTok{:}\DecValTok{6}\NormalTok{])}
\NormalTok{x }\OtherTok{\textless{}{-}} \FunctionTok{addTaxonomyTree}\NormalTok{(x)}
\end{Highlighting}
\end{Shaded}

After some preparation, the data is assembled and can be plotted with
\texttt{plotRowTree}.

\begin{Shaded}
\begin{Highlighting}[]
\FunctionTok{library}\NormalTok{(miaViz)}
\FunctionTok{plotRowTree}\NormalTok{(x[}\FunctionTok{rowData}\NormalTok{(x)}\SpecialCharTok{$}\NormalTok{Phylum }\SpecialCharTok{\%in\%}\NormalTok{ top\_phyla,],}
            \AttributeTok{edge\_colour\_by =} \StringTok{"Phylum"}\NormalTok{,}
            \AttributeTok{tip\_colour\_by =} \StringTok{"prevalence"}\NormalTok{,}
            \AttributeTok{node\_colour\_by =} \StringTok{"prevalence"}\NormalTok{)}
\end{Highlighting}
\end{Shaded}

\begin{figure}
\centering
\includegraphics{12_quality_control_files/figure-latex/plot-prev-prev-1.pdf}
\caption{\label{fig:plot-prev-prev}Prevalence of top phyla as judged by prevalence}
\end{figure}

\begin{Shaded}
\begin{Highlighting}[]
\FunctionTok{plotRowTree}\NormalTok{(x[}\FunctionTok{rowData}\NormalTok{(x)}\SpecialCharTok{$}\NormalTok{Phylum }\SpecialCharTok{\%in\%}\NormalTok{ top\_phyla\_mean,],}
            \AttributeTok{edge\_colour\_by =} \StringTok{"Phylum"}\NormalTok{,}
            \AttributeTok{tip\_colour\_by =} \StringTok{"prevalence"}\NormalTok{,}
            \AttributeTok{node\_colour\_by =} \StringTok{"prevalence"}\NormalTok{)}
\end{Highlighting}
\end{Shaded}

\begin{figure}
\centering
\includegraphics{12_quality_control_files/figure-latex/plot-prev-mean-1.pdf}
\caption{\label{fig:plot-prev-mean}Prevalence of top phyla as judged by mean abundance}
\end{figure}

\hypertarget{qc}{%
\section{Quality control}\label{qc}}

Next, let us load the \texttt{GlobalPatterns} data set to illustrate standard
microbiome data summaries.

\begin{Shaded}
\begin{Highlighting}[]
\FunctionTok{library}\NormalTok{(mia)}
\FunctionTok{data}\NormalTok{(}\StringTok{"GlobalPatterns"}\NormalTok{, }\AttributeTok{package=}\StringTok{"mia"}\NormalTok{)}
\NormalTok{tse }\OtherTok{\textless{}{-}}\NormalTok{ GlobalPatterns }
\end{Highlighting}
\end{Shaded}

\hypertarget{top-taxa}{%
\subsection{Top taxa}\label{top-taxa}}

The \texttt{getTopTaxa} identifies top taxa in the data.

\begin{Shaded}
\begin{Highlighting}[]
\CommentTok{\# Pick the top taxa}
\NormalTok{top\_features }\OtherTok{\textless{}{-}} \FunctionTok{getTopTaxa}\NormalTok{(tse, }\AttributeTok{method=}\StringTok{"median"}\NormalTok{, }\AttributeTok{top=}\DecValTok{10}\NormalTok{)}

\CommentTok{\# Check the information for these}
\FunctionTok{rowData}\NormalTok{(tse)[top\_features, }\FunctionTok{taxonomyRanks}\NormalTok{(tse)]}
\end{Highlighting}
\end{Shaded}

\begin{verbatim}
## DataFrame with 10 rows and 7 columns
##            Kingdom         Phylum               Class             Order
##        <character>    <character>         <character>       <character>
## 549656    Bacteria  Cyanobacteria         Chloroplast     Stramenopiles
## 331820    Bacteria  Bacteroidetes         Bacteroidia     Bacteroidales
## 317182    Bacteria  Cyanobacteria         Chloroplast     Stramenopiles
## 94166     Bacteria Proteobacteria Gammaproteobacteria    Pasteurellales
## 279599    Bacteria  Cyanobacteria    Nostocophycideae        Nostocales
## 158660    Bacteria  Bacteroidetes         Bacteroidia     Bacteroidales
## 329744    Bacteria Actinobacteria      Actinobacteria   Actinomycetales
## 326977    Bacteria Actinobacteria      Actinobacteria Bifidobacteriales
## 248140    Bacteria  Bacteroidetes         Bacteroidia     Bacteroidales
## 550960    Bacteria Proteobacteria Gammaproteobacteria Enterobacteriales
##                    Family           Genus                Species
##               <character>     <character>            <character>
## 549656                 NA              NA                     NA
## 331820     Bacteroidaceae     Bacteroides                     NA
## 317182                 NA              NA                     NA
## 94166     Pasteurellaceae     Haemophilus Haemophilusparainflu..
## 279599        Nostocaceae  Dolichospermum                     NA
## 158660     Bacteroidaceae     Bacteroides                     NA
## 329744             ACK-M1              NA                     NA
## 326977 Bifidobacteriaceae Bifidobacterium Bifidobacteriumadole..
## 248140     Bacteroidaceae     Bacteroides      Bacteroidescaccae
## 550960 Enterobacteriaceae     Providencia                     NA
\end{verbatim}

\hypertarget{library-size-read-count}{%
\subsection{Library size / read count}\label{library-size-read-count}}

The total counts/sample can be calculated using \texttt{perCellQCMetrics}/\texttt{addPerCellQC} from the \texttt{scater} package. The former one
just calculates the values, whereas the latter one directly adds them to
\texttt{colData}.

\begin{Shaded}
\begin{Highlighting}[]
\FunctionTok{library}\NormalTok{(scater)}
\FunctionTok{perCellQCMetrics}\NormalTok{(tse)}
\end{Highlighting}
\end{Shaded}

\begin{verbatim}
## DataFrame with 26 rows and 3 columns
##               sum  detected     total
##         <numeric> <numeric> <numeric>
## CL3        864077      6964    864077
## CC1       1135457      7679   1135457
## SV1        697509      5729    697509
## M31Fcsw   1543451      2667   1543451
## M11Fcsw   2076476      2574   2076476
## ...           ...       ...       ...
## TS28       937466      2679    937466
## TS29      1211071      2629   1211071
## Even1     1216137      4213   1216137
## Even2      971073      3130    971073
## Even3     1078241      2776   1078241
\end{verbatim}

\begin{Shaded}
\begin{Highlighting}[]
\NormalTok{tse }\OtherTok{\textless{}{-}} \FunctionTok{addPerCellQC}\NormalTok{(tse)}
\FunctionTok{colData}\NormalTok{(tse)}
\end{Highlighting}
\end{Shaded}

\begin{verbatim}
## DataFrame with 26 rows and 10 columns
##         X.SampleID   Primer Final_Barcode Barcode_truncated_plus_T
##           <factor> <factor>      <factor>                 <factor>
## CL3        CL3      ILBC_01        AACGCA                   TGCGTT
## CC1        CC1      ILBC_02        AACTCG                   CGAGTT
## SV1        SV1      ILBC_03        AACTGT                   ACAGTT
## M31Fcsw    M31Fcsw  ILBC_04        AAGAGA                   TCTCTT
## M11Fcsw    M11Fcsw  ILBC_05        AAGCTG                   CAGCTT
## ...            ...      ...           ...                      ...
## TS28         TS28   ILBC_25        ACCAGA                   TCTGGT
## TS29         TS29   ILBC_26        ACCAGC                   GCTGGT
## Even1        Even1  ILBC_27        ACCGCA                   TGCGGT
## Even2        Even2  ILBC_28        ACCTCG                   CGAGGT
## Even3        Even3  ILBC_29        ACCTGT                   ACAGGT
##         Barcode_full_length SampleType
##                    <factor>   <factor>
## CL3             CTAGCGTGCGT      Soil 
## CC1             CATCGACGAGT      Soil 
## SV1             GTACGCACAGT      Soil 
## M31Fcsw         TCGACATCTCT      Feces
## M11Fcsw         CGACTGCAGCT      Feces
## ...                     ...        ...
## TS28            GCATCGTCTGG      Feces
## TS29            CTAGTCGCTGG      Feces
## Even1           TGACTCTGCGG      Mock 
## Even2           TCTGATCGAGG      Mock 
## Even3           AGAGAGACAGG      Mock 
##                                        Description       sum  detected
##                                           <factor> <numeric> <numeric>
## CL3     Calhoun South Carolina Pine soil, pH 4.9      864077      6964
## CC1     Cedar Creek Minnesota, grassland, pH 6.1     1135457      7679
## SV1     Sevilleta new Mexico, desert scrub, pH 8.3    697509      5729
## M31Fcsw M3, Day 1, fecal swab, whole body study      1543451      2667
## M11Fcsw M1, Day 1, fecal swab, whole body study      2076476      2574
## ...                                            ...       ...       ...
## TS28                                       Twin #1    937466      2679
## TS29                                       Twin #2   1211071      2629
## Even1                                      Even1     1216137      4213
## Even2                                      Even2      971073      3130
## Even3                                      Even3     1078241      2776
##             total
##         <numeric>
## CL3        864077
## CC1       1135457
## SV1        697509
## M31Fcsw   1543451
## M11Fcsw   2076476
## ...           ...
## TS28       937466
## TS29      1211071
## Even1     1216137
## Even2      971073
## Even3     1078241
\end{verbatim}

The distribution of calculated library sizes can be visualized as a
histogram (left), or by sorting the samples by library size (right).

\begin{Shaded}
\begin{Highlighting}[]
\FunctionTok{library}\NormalTok{(ggplot2)}

\NormalTok{p1 }\OtherTok{\textless{}{-}} \FunctionTok{ggplot}\NormalTok{(}\FunctionTok{as.data.frame}\NormalTok{(}\FunctionTok{colData}\NormalTok{(tse))) }\SpecialCharTok{+}
        \FunctionTok{geom\_histogram}\NormalTok{(}\FunctionTok{aes}\NormalTok{(}\AttributeTok{x =}\NormalTok{ sum), }\AttributeTok{color =} \StringTok{"black"}\NormalTok{, }\AttributeTok{fill =} \StringTok{"gray"}\NormalTok{, }\AttributeTok{bins =} \DecValTok{30}\NormalTok{) }\SpecialCharTok{+}
        \FunctionTok{labs}\NormalTok{(}\AttributeTok{x =} \StringTok{"Library size"}\NormalTok{, }\AttributeTok{y =} \StringTok{"Frequency (n)"}\NormalTok{) }\SpecialCharTok{+} 
        \CommentTok{\# scale\_x\_log10(breaks = scales::trans\_breaks("log10", function(x) 10\^{}x), }
        \CommentTok{\# labels = scales::trans\_format("log10", scales::math\_format(10\^{}.x))) +}
        \FunctionTok{theme\_bw}\NormalTok{() }\SpecialCharTok{+}
        \FunctionTok{theme}\NormalTok{(}\AttributeTok{panel.grid.major =} \FunctionTok{element\_blank}\NormalTok{(), }\CommentTok{\# Removes the grid}
          \AttributeTok{panel.grid.minor =} \FunctionTok{element\_blank}\NormalTok{(),}
          \AttributeTok{panel.border =} \FunctionTok{element\_blank}\NormalTok{(),}
          \AttributeTok{panel.background =} \FunctionTok{element\_blank}\NormalTok{(),}
          \AttributeTok{axis.line =} \FunctionTok{element\_line}\NormalTok{(}\AttributeTok{colour =} \StringTok{"black"}\NormalTok{)) }\CommentTok{\# Adds y{-}axis}

\FunctionTok{library}\NormalTok{(dplyr)}
\NormalTok{df }\OtherTok{\textless{}{-}} \FunctionTok{as.data.frame}\NormalTok{(}\FunctionTok{colData}\NormalTok{(tse)) }\SpecialCharTok{\%\textgreater{}\%}
        \FunctionTok{arrange}\NormalTok{(sum) }\SpecialCharTok{\%\textgreater{}\%}
        \FunctionTok{mutate}\NormalTok{(}\AttributeTok{index =} \DecValTok{1}\SpecialCharTok{:}\FunctionTok{n}\NormalTok{())}
\NormalTok{p2 }\OtherTok{\textless{}{-}} \FunctionTok{ggplot}\NormalTok{(df, }\FunctionTok{aes}\NormalTok{(}\AttributeTok{y =}\NormalTok{ index, }\AttributeTok{x =}\NormalTok{ sum}\SpecialCharTok{/}\FloatTok{1e6}\NormalTok{)) }\SpecialCharTok{+}
        \FunctionTok{geom\_point}\NormalTok{() }\SpecialCharTok{+}  
        \FunctionTok{labs}\NormalTok{(}\AttributeTok{x =} \StringTok{"Library size (million reads)"}\NormalTok{, }\AttributeTok{y =} \StringTok{"Sample index"}\NormalTok{) }\SpecialCharTok{+}  
        \FunctionTok{theme\_bw}\NormalTok{() }\SpecialCharTok{+}
        \FunctionTok{theme}\NormalTok{(}\AttributeTok{panel.grid.major =} \FunctionTok{element\_blank}\NormalTok{(), }\CommentTok{\# Removes the grid}
          \AttributeTok{panel.grid.minor =} \FunctionTok{element\_blank}\NormalTok{(),}
          \AttributeTok{panel.border =} \FunctionTok{element\_blank}\NormalTok{(),}
          \AttributeTok{panel.background =} \FunctionTok{element\_blank}\NormalTok{(),}
          \AttributeTok{axis.line =} \FunctionTok{element\_line}\NormalTok{(}\AttributeTok{colour =} \StringTok{"black"}\NormalTok{)) }\CommentTok{\# Adds y{-}axis}

\FunctionTok{library}\NormalTok{(patchwork)}
\NormalTok{p1 }\SpecialCharTok{+}\NormalTok{ p2}
\end{Highlighting}
\end{Shaded}

\begin{figure}
\centering
\includegraphics{12_quality_control_files/figure-latex/plot-viz-lib-size-1-1.pdf}
\caption{\label{fig:plot-viz-lib-size-1}Library size distribution.}
\end{figure}

Library sizes other variables from \texttt{colData} can be
visualized by using specified function called \texttt{plotColData}.

\begin{Shaded}
\begin{Highlighting}[]
\FunctionTok{library}\NormalTok{(ggplot2)}
\CommentTok{\# Sort samples by read count, order the factor levels, and store back to tse as DataFrame}
\CommentTok{\# }\AlertTok{TODO}\CommentTok{: plotColData could include an option for sorting samples based on colData variables}
\FunctionTok{colData}\NormalTok{(tse) }\OtherTok{\textless{}{-}} \FunctionTok{as.data.frame}\NormalTok{(}\FunctionTok{colData}\NormalTok{(tse)) }\SpecialCharTok{\%\textgreater{}\%}
                 \FunctionTok{arrange}\NormalTok{(X.SampleID) }\SpecialCharTok{\%\textgreater{}\%}
             \FunctionTok{mutate}\NormalTok{(}\AttributeTok{X.SampleID =} \FunctionTok{factor}\NormalTok{(X.SampleID, }\AttributeTok{levels=}\NormalTok{X.SampleID)) }\SpecialCharTok{\%\textgreater{}\%}
\NormalTok{         DataFrame}
\FunctionTok{plotColData}\NormalTok{(tse,}\StringTok{"sum"}\NormalTok{,}\StringTok{"X.SampleID"}\NormalTok{, }\AttributeTok{colour\_by =} \StringTok{"SampleType"}\NormalTok{) }\SpecialCharTok{+} 
    \FunctionTok{theme}\NormalTok{(}\AttributeTok{axis.text.x =} \FunctionTok{element\_text}\NormalTok{(}\AttributeTok{angle =} \DecValTok{45}\NormalTok{, }\AttributeTok{hjust=}\DecValTok{1}\NormalTok{)) }\SpecialCharTok{+}
    \FunctionTok{labs}\NormalTok{(}\AttributeTok{y =} \StringTok{"Library size (N)"}\NormalTok{, }\AttributeTok{x =} \StringTok{"Sample ID"}\NormalTok{)       }
\end{Highlighting}
\end{Shaded}

\begin{figure}
\centering
\includegraphics{12_quality_control_files/figure-latex/plot-viz-lib-size-2-1.pdf}
\caption{\label{fig:plot-viz-lib-size-2}Library sizes per sample.}
\end{figure}

\begin{Shaded}
\begin{Highlighting}[]
\FunctionTok{plotColData}\NormalTok{(tse,}\StringTok{"sum"}\NormalTok{,}\StringTok{"SampleType"}\NormalTok{, }\AttributeTok{colour\_by =} \StringTok{"SampleType"}\NormalTok{) }\SpecialCharTok{+} 
    \FunctionTok{theme}\NormalTok{(}\AttributeTok{axis.text.x =} \FunctionTok{element\_text}\NormalTok{(}\AttributeTok{angle =} \DecValTok{45}\NormalTok{, }\AttributeTok{hjust=}\DecValTok{1}\NormalTok{))}
\end{Highlighting}
\end{Shaded}

\begin{figure}
\centering
\includegraphics{12_quality_control_files/figure-latex/plot-viz-lib-size-3-1.pdf}
\caption{\label{fig:plot-viz-lib-size-3}Library sizes per sample type.}
\end{figure}

In addition, data can be rarefied with
\href{https://microbiome.github.io/mia/reference/subsampleCounts.html}{subsampleCounts},
which normalises the samples to an equal number of reads. However,
this practice has been discouraged for the analysis of differentially
abundant microorganisms (see \citep{mcmurdie2014waste}).

\hypertarget{contaminant-sequences}{%
\subsection{Contaminant sequences}\label{contaminant-sequences}}

Samples might be contaminated with exogenous sequences. The impact of
each contaminant can be estimated based on their frequencies and
concentrations across the samples.

The following \href{https://microbiome.github.io/mia/reference/isContaminant.html}{decontam
functions}
are based on the \citep{davis2018simple} and support such functionality:

\begin{itemize}
\tightlist
\item
  \texttt{isContaminant}, \texttt{isNotContaminant}
\item
  \texttt{addContaminantQC}, \texttt{addNotContaminantQC}
\end{itemize}

\hypertarget{taxonomic-information}{%
\chapter{Taxonomic Information}\label{taxonomic-information}}

\begin{Shaded}
\begin{Highlighting}[]
\FunctionTok{library}\NormalTok{(mia)}
\FunctionTok{data}\NormalTok{(}\StringTok{"GlobalPatterns"}\NormalTok{, }\AttributeTok{package=}\StringTok{"mia"}\NormalTok{)}
\NormalTok{tse }\OtherTok{\textless{}{-}}\NormalTok{ GlobalPatterns }
\end{Highlighting}
\end{Shaded}

Taxonomic information is a key part of analyzing microbiome data and without
it, any type of data analysis probably will not make much sense. However,
the degree of detail of taxonomic information differs depending on the dataset
and annotation data used.

Therefore, the mia package expects a loose assembly of taxonomic information
and assumes certain key aspects:

\begin{itemize}
\tightlist
\item
  Taxonomic information is given as character vectors or factors in the
  \texttt{rowData} of a \texttt{SummarizedExperiment} object.
\item
  The columns containing the taxonomic information must be named \texttt{domain},
  \texttt{kingdom}, \texttt{phylum}, \texttt{class}, \texttt{order}, \texttt{family}, \texttt{genus}, \texttt{species} or with
  a capital first letter.
\item
  the columns must be given in the order shown above
\item
  column can be omited, but the order must remain
\end{itemize}

\hypertarget{assigning-taxonomic-information.}{%
\section{Assigning taxonomic information.}\label{assigning-taxonomic-information.}}

There are a number of methods to assign taxonomic information. We like to give
a short introduction about the methods available without ranking one over the
other. This has to be your choice based on the result for the individual
dataset.

\hypertarget{dada2}{%
\subsection{dada2}\label{dada2}}

The dada2 package \citep{Callahan2016dada2} implements the \texttt{assignTaxonomy}
function, which takes as input the ASV sequences associated with each
row of data and a training dataset. For more information visit the
\href{https://benjjneb.github.io/dada2/assign.html}{dada2 homepage}.

\hypertarget{decipher}{%
\subsection{DECIPHER}\label{decipher}}

The DECIPHER package \citep{R-DECIPHER} implements the \texttt{IDTAXA} algorithm to assign
either taxonomic information or function information. For \texttt{mia}
only the first option is of interest for now and more information can be
found on the \href{http://www2.decipher.codes/Classification.html}{DECIPHER website}.

\hypertarget{functions-to-access-taxonomic-information}{%
\section{Functions to access taxonomic information}\label{functions-to-access-taxonomic-information}}

\texttt{checkTaxonomy} checks whether the taxonomic information is usable for \texttt{mia}

\begin{Shaded}
\begin{Highlighting}[]
\FunctionTok{checkTaxonomy}\NormalTok{(tse)}
\end{Highlighting}
\end{Shaded}

\begin{verbatim}
## [1] TRUE
\end{verbatim}

Since the \texttt{rowData} can contain other data, \texttt{taxonomyRanks} will return the
columns \texttt{mia} assumes to contain the taxonomic information.

\begin{Shaded}
\begin{Highlighting}[]
\FunctionTok{taxonomyRanks}\NormalTok{(tse)}
\end{Highlighting}
\end{Shaded}

\begin{verbatim}
## [1] "Kingdom" "Phylum"  "Class"   "Order"   "Family"  "Genus"   "Species"
\end{verbatim}

This can then be used to subset the \texttt{rowData} to columns needed.

\begin{Shaded}
\begin{Highlighting}[]
\FunctionTok{rowData}\NormalTok{(tse)[,}\FunctionTok{taxonomyRanks}\NormalTok{(tse)]}
\end{Highlighting}
\end{Shaded}

\begin{verbatim}
## DataFrame with 19216 rows and 7 columns
##            Kingdom        Phylum        Class        Order        Family
##        <character>   <character>  <character>  <character>   <character>
## 549322     Archaea Crenarchaeota Thermoprotei           NA            NA
## 522457     Archaea Crenarchaeota Thermoprotei           NA            NA
## 951        Archaea Crenarchaeota Thermoprotei Sulfolobales Sulfolobaceae
## 244423     Archaea Crenarchaeota        Sd-NA           NA            NA
## 586076     Archaea Crenarchaeota        Sd-NA           NA            NA
## ...            ...           ...          ...          ...           ...
## 278222    Bacteria           SR1           NA           NA            NA
## 463590    Bacteria           SR1           NA           NA            NA
## 535321    Bacteria           SR1           NA           NA            NA
## 200359    Bacteria           SR1           NA           NA            NA
## 271582    Bacteria           SR1           NA           NA            NA
##              Genus                Species
##        <character>            <character>
## 549322          NA                     NA
## 522457          NA                     NA
## 951     Sulfolobus Sulfolobusacidocalda..
## 244423          NA                     NA
## 586076          NA                     NA
## ...            ...                    ...
## 278222          NA                     NA
## 463590          NA                     NA
## 535321          NA                     NA
## 200359          NA                     NA
## 271582          NA                     NA
\end{verbatim}

\texttt{taxonomyRankEmpty} checks for empty values in the given \texttt{rank} and returns a
logical vector of \texttt{length(x)}.

\begin{Shaded}
\begin{Highlighting}[]
\FunctionTok{all}\NormalTok{(}\SpecialCharTok{!}\FunctionTok{taxonomyRankEmpty}\NormalTok{(tse, }\AttributeTok{rank =} \StringTok{"Kingdom"}\NormalTok{))}
\end{Highlighting}
\end{Shaded}

\begin{verbatim}
## [1] TRUE
\end{verbatim}

\begin{Shaded}
\begin{Highlighting}[]
\FunctionTok{table}\NormalTok{(}\FunctionTok{taxonomyRankEmpty}\NormalTok{(tse, }\AttributeTok{rank =} \StringTok{"Genus"}\NormalTok{))}
\end{Highlighting}
\end{Shaded}

\begin{verbatim}
## 
## FALSE  TRUE 
##  8008 11208
\end{verbatim}

\begin{Shaded}
\begin{Highlighting}[]
\FunctionTok{table}\NormalTok{(}\FunctionTok{taxonomyRankEmpty}\NormalTok{(tse, }\AttributeTok{rank =} \StringTok{"Species"}\NormalTok{))}
\end{Highlighting}
\end{Shaded}

\begin{verbatim}
## 
## FALSE  TRUE 
##  1413 17803
\end{verbatim}

\texttt{getTaxonomyLabels} is a multi-purpose function, which turns taxonomic
information into a character vector of \texttt{length(x)}

\begin{Shaded}
\begin{Highlighting}[]
\FunctionTok{head}\NormalTok{(}\FunctionTok{getTaxonomyLabels}\NormalTok{(tse))}
\end{Highlighting}
\end{Shaded}

\begin{verbatim}
## [1] "Class:Thermoprotei"               "Class:Thermoprotei_1"            
## [3] "Species:Sulfolobusacidocaldarius" "Class:Sd-NA"                     
## [5] "Class:Sd-NA_1"                    "Class:Sd-NA_2"
\end{verbatim}

By default, this will use the lowest non-empty information to construct a
string with the following scheme \texttt{level:value}. If all levels are the same,
this part is omitted, but can be added by setting \texttt{with\_rank\ =\ TRUE}.

\begin{Shaded}
\begin{Highlighting}[]
\NormalTok{phylum }\OtherTok{\textless{}{-}} \SpecialCharTok{!}\FunctionTok{is.na}\NormalTok{(}\FunctionTok{rowData}\NormalTok{(tse)}\SpecialCharTok{$}\NormalTok{Phylum) }\SpecialCharTok{\&} 
    \FunctionTok{vapply}\NormalTok{(}\FunctionTok{data.frame}\NormalTok{(}\FunctionTok{apply}\NormalTok{(}\FunctionTok{rowData}\NormalTok{(tse)[,}\FunctionTok{taxonomyRanks}\NormalTok{(tse)[}\DecValTok{3}\SpecialCharTok{:}\DecValTok{7}\NormalTok{]],1L,is.na)),all,}\FunctionTok{logical}\NormalTok{(}\DecValTok{1}\NormalTok{))}
\FunctionTok{head}\NormalTok{(}\FunctionTok{getTaxonomyLabels}\NormalTok{(tse[phylum,]))}
\end{Highlighting}
\end{Shaded}

\begin{verbatim}
## [1] "Crenarchaeota"    "Crenarchaeota_1"  "Crenarchaeota_2"  "Actinobacteria"  
## [5] "Actinobacteria_1" "Spirochaetes"
\end{verbatim}

\begin{Shaded}
\begin{Highlighting}[]
\FunctionTok{head}\NormalTok{(}\FunctionTok{getTaxonomyLabels}\NormalTok{(tse[phylum,], }\AttributeTok{with\_rank =} \ConstantTok{TRUE}\NormalTok{))}
\end{Highlighting}
\end{Shaded}

\begin{verbatim}
## [1] "Phylum:Crenarchaeota"    "Phylum:Crenarchaeota_1" 
## [3] "Phylum:Crenarchaeota_2"  "Phylum:Actinobacteria"  
## [5] "Phylum:Actinobacteria_1" "Phylum:Spirochaetes"
\end{verbatim}

By default the return value of \texttt{getTaxonomyLabels} contains only
unique elements by passing it through \texttt{make.unique}. This step can be
omitted by setting \texttt{make\_unique\ =\ FALSE}.

\begin{Shaded}
\begin{Highlighting}[]
\FunctionTok{head}\NormalTok{(}\FunctionTok{getTaxonomyLabels}\NormalTok{(tse[phylum,], }\AttributeTok{with\_rank =} \ConstantTok{TRUE}\NormalTok{, }\AttributeTok{make\_unique =} \ConstantTok{FALSE}\NormalTok{))}
\end{Highlighting}
\end{Shaded}

\begin{verbatim}
## [1] "Phylum:Crenarchaeota"  "Phylum:Crenarchaeota"  "Phylum:Crenarchaeota" 
## [4] "Phylum:Actinobacteria" "Phylum:Actinobacteria" "Phylum:Spirochaetes"
\end{verbatim}

To apply the loop resolving function \texttt{resolveLoop} from the
\texttt{TreeSummarizedExperiment} package \citep{R-TreeSummarizedExperiment} within
\texttt{getTaxonomyLabels}, set \texttt{resolve\_loops\ =\ TRUE}.

The function \texttt{getUniqueTaxa} gives a list of unique taxa for the
specified taxonomic rank.

\begin{Shaded}
\begin{Highlighting}[]
\FunctionTok{head}\NormalTok{(}\FunctionTok{getUniqueTaxa}\NormalTok{(tse, }\AttributeTok{rank =} \StringTok{"Phylum"}\NormalTok{))}
\end{Highlighting}
\end{Shaded}

\begin{verbatim}
## [1] "Crenarchaeota"  "Euryarchaeota"  "Actinobacteria" "Spirochaetes"  
## [5] "MVP-15"         "Proteobacteria"
\end{verbatim}

\hypertarget{generate-a-taxonomic-tree-on-the-fly}{%
\subsection{Generate a taxonomic tree on the fly}\label{generate-a-taxonomic-tree-on-the-fly}}

To create a taxonomic tree, \texttt{taxonomyTree} used the information and returns a
\texttt{phylo} object. Duplicate information from the \texttt{rowData} is removed.

\begin{Shaded}
\begin{Highlighting}[]
\FunctionTok{taxonomyTree}\NormalTok{(tse)}
\end{Highlighting}
\end{Shaded}

\begin{verbatim}
## 
## Phylogenetic tree with 1645 tips and 1089 internal nodes.
## 
## Tip labels:
##   Species:Cenarchaeumsymbiosum, Species:pIVWA5, Species:CandidatusNitrososphaeragargensis, Species:SCA1145, Species:SCA1170, Species:Sulfolobusacidocaldarius, ...
## Node labels:
##   root:ALL, Kingdom:Archaea, Phylum:Crenarchaeota, Class:C2, Class:Sd-NA, Class:Thaumarchaeota, ...
## 
## Rooted; includes branch lengths.
\end{verbatim}

\begin{Shaded}
\begin{Highlighting}[]
\NormalTok{tse }\OtherTok{\textless{}{-}} \FunctionTok{addTaxonomyTree}\NormalTok{(tse)}
\NormalTok{tse}
\end{Highlighting}
\end{Shaded}

\begin{verbatim}
## class: TreeSummarizedExperiment 
## dim: 19216 26 
## metadata(0):
## assays(1): counts
## rownames(19216): Class:Thermoprotei Class:Thermoprotei ... Phylum:SR1
##   Phylum:SR1
## rowData names(7): Kingdom Phylum ... Genus Species
## colnames(26): CL3 CC1 ... Even2 Even3
## colData names(7): X.SampleID Primer ... SampleType Description
## reducedDimNames(0):
## mainExpName: NULL
## altExpNames(0):
## rowLinks: a LinkDataFrame (19216 rows)
## rowTree: 1 phylo tree(s) (1645 leaves)
## colLinks: NULL
## colTree: NULL
\end{verbatim}

The implementation is based on the \texttt{toTree} function from the
\texttt{TreeSummarizedExperiment} package \citep{R-TreeSummarizedExperiment}.

\hypertarget{data-agglomeration}{%
\section{Data agglomeration}\label{data-agglomeration}}

One of the main applications of taxonomic information in regards to count data
is to agglomerate count data on taxonomic levels and track the influence of
changing conditions through these levels. For this \texttt{mia} contains the
\texttt{agglomerateByRank} function. The ideal location to store the agglomerated data
is as an alternative experiment.

\begin{Shaded}
\begin{Highlighting}[]
\NormalTok{tse }\OtherTok{\textless{}{-}} \FunctionTok{relAbundanceCounts}\NormalTok{(tse)}
\FunctionTok{altExp}\NormalTok{(tse, }\StringTok{"Family"}\NormalTok{) }\OtherTok{\textless{}{-}} \FunctionTok{agglomerateByRank}\NormalTok{(tse, }\AttributeTok{rank =} \StringTok{"Family"}\NormalTok{,}
                                           \AttributeTok{agglomerateTree =} \ConstantTok{TRUE}\NormalTok{)}
\FunctionTok{altExp}\NormalTok{(tse, }\StringTok{"Family"}\NormalTok{)}
\end{Highlighting}
\end{Shaded}

\begin{verbatim}
## class: TreeSummarizedExperiment 
## dim: 603 26 
## metadata(1): agglomerated_by_rank
## assays(2): counts relabundance
## rownames(603): Class:Thermoprotei Family:Sulfolobaceae ...
##   Family:Thermodesulfobiaceae Phylum:SR1
## rowData names(7): Kingdom Phylum ... Genus Species
## colnames(26): CL3 CC1 ... Even2 Even3
## colData names(7): X.SampleID Primer ... SampleType Description
## reducedDimNames(0):
## mainExpName: NULL
## altExpNames(0):
## rowLinks: a LinkDataFrame (603 rows)
## rowTree: 1 phylo tree(s) (496 leaves)
## colLinks: NULL
## colTree: NULL
\end{verbatim}

If multiple assays (counts and relabundance) exist, both will be agglomerated.

\begin{Shaded}
\begin{Highlighting}[]
\FunctionTok{assayNames}\NormalTok{(tse)}
\end{Highlighting}
\end{Shaded}

\begin{verbatim}
## [1] "counts"       "relabundance"
\end{verbatim}

\begin{Shaded}
\begin{Highlighting}[]
\FunctionTok{assayNames}\NormalTok{(}\FunctionTok{altExp}\NormalTok{(tse, }\StringTok{"Family"}\NormalTok{))}
\end{Highlighting}
\end{Shaded}

\begin{verbatim}
## [1] "counts"       "relabundance"
\end{verbatim}

\begin{Shaded}
\begin{Highlighting}[]
\FunctionTok{assay}\NormalTok{(}\FunctionTok{altExp}\NormalTok{(tse, }\StringTok{"Family"}\NormalTok{), }\StringTok{"relabundance"}\NormalTok{)[}\DecValTok{1}\SpecialCharTok{:}\DecValTok{5}\NormalTok{,}\DecValTok{1}\SpecialCharTok{:}\DecValTok{7}\NormalTok{]}
\end{Highlighting}
\end{Shaded}

\begin{verbatim}
##                            CL3       CC1 SV1 M31Fcsw M11Fcsw M31Plmr   M11Plmr
## Class:Thermoprotei   0.0000000 0.000e+00   0       0       0       0 0.000e+00
## Family:Sulfolobaceae 0.0000000 0.000e+00   0       0       0       0 2.305e-06
## Class:Sd-NA          0.0000000 0.000e+00   0       0       0       0 0.000e+00
## Order:NRP-J          0.0001991 2.070e-04   0       0       0       0 6.914e-06
## Family:SAGMA-X       0.0000000 6.165e-06   0       0       0       0 0.000e+00
\end{verbatim}

\begin{Shaded}
\begin{Highlighting}[]
\FunctionTok{assay}\NormalTok{(}\FunctionTok{altExp}\NormalTok{(tse, }\StringTok{"Family"}\NormalTok{), }\StringTok{"counts"}\NormalTok{)[}\DecValTok{1}\SpecialCharTok{:}\DecValTok{5}\NormalTok{,}\DecValTok{1}\SpecialCharTok{:}\DecValTok{7}\NormalTok{]}
\end{Highlighting}
\end{Shaded}

\begin{verbatim}
##                      CL3 CC1 SV1 M31Fcsw M11Fcsw M31Plmr M11Plmr
## Class:Thermoprotei     0   0   0       0       0       0       0
## Family:Sulfolobaceae   0   0   0       0       0       0       1
## Class:Sd-NA            0   0   0       0       0       0       0
## Order:NRP-J          172 235   0       0       0       0       3
## Family:SAGMA-X         0   7   0       0       0       0       0
\end{verbatim}

\texttt{altExpNames} now consists of \texttt{Family} level data. This can be extended to use
any taxonomic level listed in \texttt{mia::taxonomyRanks(tse)}.

\hypertarget{data-transformation}{%
\section{Data transformation}\label{data-transformation}}

Data transformations are common in microbiome analysis. Examples
include the logarithmic transformation, calculation of relative
abundances (percentages), and compositionality-aware transformations
such as the centered log-ratio transformation (clr).

In mia package, transformations are applied to abundance data. The transformed
abundance table is stored back to `assays'. mia includes transformation
function (`transformCounts()') which applies sample-wise or column-wise transformation when MARGIN = `samples', feature-wise or row-wise transformation when MARGIN = `features'.

For a complete list of available transformations and parameters, see function
\href{https://microbiome.github.io/mia/reference/transformCounts.html}{help}.

\begin{Shaded}
\begin{Highlighting}[]
\FunctionTok{assay}\NormalTok{(tse, }\StringTok{"pseudo"}\NormalTok{) }\OtherTok{\textless{}{-}} \FunctionTok{assay}\NormalTok{(tse, }\StringTok{"counts"}\NormalTok{) }\SpecialCharTok{+} \DecValTok{1}
\NormalTok{tse }\OtherTok{\textless{}{-}} \FunctionTok{transformCounts}\NormalTok{(tse, }\AttributeTok{assay.type =} \StringTok{"pseudo"}\NormalTok{, }\AttributeTok{method =} \StringTok{"relabundance"}\NormalTok{)}
\NormalTok{tse }\OtherTok{\textless{}{-}} \FunctionTok{transformCounts}\NormalTok{(}\AttributeTok{x =}\NormalTok{ tse, }\AttributeTok{assay.type =} \StringTok{"relabundance"}\NormalTok{, }\AttributeTok{method =} \StringTok{"clr"}\NormalTok{, }
                        \AttributeTok{pseudocount =} \DecValTok{1}\NormalTok{, }\AttributeTok{name =} \StringTok{"clr\_transformation"}\NormalTok{)}

\FunctionTok{head}\NormalTok{(}\FunctionTok{assay}\NormalTok{(tse, }\StringTok{"clr\_transformation"}\NormalTok{))}
\end{Highlighting}
\end{Shaded}

\begin{verbatim}
##                                         CL3        CC1        SV1    M31Fcsw
## Class:Thermoprotei               -5.078e-05 -5.105e-05 -5.055e-05 -4.975e-05
## Class:Thermoprotei               -5.078e-05 -5.105e-05 -5.055e-05 -4.975e-05
## Species:Sulfolobusacidocaldarius -5.078e-05 -5.105e-05 -5.055e-05 -4.975e-05
## Class:Sd-NA                      -5.078e-05 -5.105e-05 -5.055e-05 -4.975e-05
## Class:Sd-NA                      -5.078e-05 -5.105e-05 -5.055e-05 -4.975e-05
## Class:Sd-NA                      -5.078e-05 -5.105e-05 -5.055e-05 -4.975e-05
##                                     M11Fcsw    M31Plmr    M11Plmr    F21Plmr
## Class:Thermoprotei               -4.947e-05 -4.931e-05 -4.879e-05 -4.671e-05
## Class:Thermoprotei               -4.947e-05 -4.931e-05 -4.879e-05 -4.671e-05
## Species:Sulfolobusacidocaldarius -4.947e-05 -4.931e-05 -4.658e-05 -4.671e-05
## Class:Sd-NA                      -4.947e-05 -4.931e-05 -4.879e-05 -4.671e-05
## Class:Sd-NA                      -4.947e-05 -4.931e-05 -4.879e-05 -4.671e-05
## Class:Sd-NA                      -4.947e-05 -4.931e-05 -4.879e-05 -4.671e-05
##                                     M31Tong    M11Tong   LMEpi24M   SLEpi20M
## Class:Thermoprotei               -4.846e-05 -4.257e-05 -4.756e-05 -4.837e-05
## Class:Thermoprotei               -4.846e-05 -4.257e-05 -4.756e-05 -4.918e-05
## Species:Sulfolobusacidocaldarius -4.846e-05 -4.257e-05 -4.756e-05 -4.918e-05
## Class:Sd-NA                      -4.846e-05 -4.257e-05 -4.756e-05 -4.918e-05
## Class:Sd-NA                      -4.846e-05 -4.257e-05 -4.756e-05 -4.918e-05
## Class:Sd-NA                      -4.846e-05 -4.257e-05 -4.756e-05 -4.918e-05
##                                      AQC1cm     AQC4cm     AQC7cm        NP2
## Class:Thermoprotei               -2.385e-05 -4.438e-06  2.787e-05 -4.731e-05
## Class:Thermoprotei               -4.660e-05 -4.568e-05 -4.428e-05 -4.915e-05
## Species:Sulfolobusacidocaldarius -4.660e-05 -4.652e-05 -4.777e-05 -4.915e-05
## Class:Sd-NA                      -4.660e-05 -3.726e-05 -3.090e-05 -4.915e-05
## Class:Sd-NA                      -4.660e-05 -4.568e-05 -4.719e-05 -4.915e-05
## Class:Sd-NA                      -4.660e-05 -4.610e-05 -4.603e-05 -4.915e-05
##                                         NP3        NP5    TRRsed1    TRRsed2
## Class:Thermoprotei               -5.068e-05 -5.083e-05 -3.909e-05 -4.927e-05
## Class:Thermoprotei               -5.068e-05 -5.083e-05 -3.909e-05 -4.927e-05
## Species:Sulfolobusacidocaldarius -5.068e-05 -5.083e-05 -3.909e-05 -4.927e-05
## Class:Sd-NA                      -5.068e-05 -5.083e-05 -3.909e-05 -4.927e-05
## Class:Sd-NA                      -5.068e-05 -5.083e-05 -3.909e-05 -4.927e-05
## Class:Sd-NA                      -5.068e-05 -5.083e-05 -3.909e-05 -4.927e-05
##                                     TRRsed3       TS28       TS29      Even1
## Class:Thermoprotei               -4.829e-05 -5.016e-05 -4.934e-05 -5.046e-05
## Class:Thermoprotei               -4.829e-05 -5.016e-05 -4.934e-05 -5.046e-05
## Species:Sulfolobusacidocaldarius -4.829e-05 -5.016e-05 -4.934e-05 -5.046e-05
## Class:Sd-NA                      -4.829e-05 -5.016e-05 -4.934e-05 -5.046e-05
## Class:Sd-NA                      -4.829e-05 -5.016e-05 -4.934e-05 -5.046e-05
## Class:Sd-NA                      -4.829e-05 -5.016e-05 -4.934e-05 -5.046e-05
##                                       Even2      Even3
## Class:Thermoprotei               -5.017e-05 -5.034e-05
## Class:Thermoprotei               -5.017e-05 -5.034e-05
## Species:Sulfolobusacidocaldarius -5.017e-05 -5.034e-05
## Class:Sd-NA                      -5.017e-05 -5.034e-05
## Class:Sd-NA                      -5.017e-05 -5.034e-05
## Class:Sd-NA                      -5.017e-05 -5.034e-05
\end{verbatim}

\begin{itemize}
\tightlist
\item
  In `pa' transformation, abundance table is converted to present/absent table.
\end{itemize}

\begin{Shaded}
\begin{Highlighting}[]
\NormalTok{tse }\OtherTok{\textless{}{-}} \FunctionTok{transformCounts}\NormalTok{(tse, }\AttributeTok{method =} \StringTok{"pa"}\NormalTok{)}

\FunctionTok{head}\NormalTok{(}\FunctionTok{assay}\NormalTok{(tse, }\StringTok{"pa"}\NormalTok{))}
\end{Highlighting}
\end{Shaded}

\begin{verbatim}
##                                  CL3 CC1 SV1 M31Fcsw M11Fcsw M31Plmr M11Plmr
## Class:Thermoprotei                 0   0   0       0       0       0       0
## Class:Thermoprotei                 0   0   0       0       0       0       0
## Species:Sulfolobusacidocaldarius   0   0   0       0       0       0       1
## Class:Sd-NA                        0   0   0       0       0       0       0
## Class:Sd-NA                        0   0   0       0       0       0       0
## Class:Sd-NA                        0   0   0       0       0       0       0
##                                  F21Plmr M31Tong M11Tong LMEpi24M SLEpi20M
## Class:Thermoprotei                     0       0       0        0        1
## Class:Thermoprotei                     0       0       0        0        0
## Species:Sulfolobusacidocaldarius       0       0       0        0        0
## Class:Sd-NA                            0       0       0        0        0
## Class:Sd-NA                            0       0       0        0        0
## Class:Sd-NA                            0       0       0        0        0
##                                  AQC1cm AQC4cm AQC7cm NP2 NP3 NP5 TRRsed1
## Class:Thermoprotei                    1      1      1   1   0   0       0
## Class:Thermoprotei                    0      1      1   0   0   0       0
## Species:Sulfolobusacidocaldarius      0      0      0   0   0   0       0
## Class:Sd-NA                           0      1      1   0   0   0       0
## Class:Sd-NA                           0      1      1   0   0   0       0
## Class:Sd-NA                           0      1      1   0   0   0       0
##                                  TRRsed2 TRRsed3 TS28 TS29 Even1 Even2 Even3
## Class:Thermoprotei                     0       0    0    0     0     0     0
## Class:Thermoprotei                     0       0    0    0     0     0     0
## Species:Sulfolobusacidocaldarius       0       0    0    0     0     0     0
## Class:Sd-NA                            0       0    0    0     0     0     0
## Class:Sd-NA                            0       0    0    0     0     0     0
## Class:Sd-NA                            0       0    0    0     0     0     0
\end{verbatim}

\begin{Shaded}
\begin{Highlighting}[]
\CommentTok{\# list of abundance tables that assays slot contains}
\FunctionTok{assays}\NormalTok{(tse)}
\end{Highlighting}
\end{Shaded}

\begin{verbatim}
## List of length 5
## names(5): counts relabundance pseudo clr_transformation pa
\end{verbatim}

\hypertarget{pick-specific}{%
\section{Pick specific}\label{pick-specific}}

Retrieving of specific elements that are required for specific analysis. For
instance, extracting abundances for a specific taxa in all samples or all taxa
in one sample.

\hypertarget{abundances-of-all-taxa-in-specific-sample}{%
\subsection{Abundances of all taxa in specific sample}\label{abundances-of-all-taxa-in-specific-sample}}

\begin{Shaded}
\begin{Highlighting}[]
\NormalTok{taxa.abund.cc1 }\OtherTok{\textless{}{-}} \FunctionTok{getAbundanceSample}\NormalTok{(tse, }
                                     \AttributeTok{sample\_id =} \StringTok{"CC1"}\NormalTok{,}
                                     \AttributeTok{assay.type =} \StringTok{"counts"}\NormalTok{)}
\NormalTok{taxa.abund.cc1[}\DecValTok{1}\SpecialCharTok{:}\DecValTok{10}\NormalTok{]}
\end{Highlighting}
\end{Shaded}

\begin{verbatim}
##               Class:Thermoprotei               Class:Thermoprotei 
##                                0                                0 
## Species:Sulfolobusacidocaldarius                      Class:Sd-NA 
##                                0                                0 
##                      Class:Sd-NA                      Class:Sd-NA 
##                                0                                0 
##                      Order:NRP-J                      Order:NRP-J 
##                                1                                0 
##                      Order:NRP-J                      Order:NRP-J 
##                              194                                5
\end{verbatim}

\hypertarget{abundances-of-specific-taxa-in-all-samples}{%
\subsection{Abundances of specific taxa in all samples}\label{abundances-of-specific-taxa-in-all-samples}}

\begin{Shaded}
\begin{Highlighting}[]
\NormalTok{taxa.abundances }\OtherTok{\textless{}{-}} \FunctionTok{getAbundanceFeature}\NormalTok{(tse, }
                                      \AttributeTok{feature\_id =} \StringTok{"Phylum:Bacteroidetes"}\NormalTok{,}
                                      \AttributeTok{assay.type =} \StringTok{"counts"}\NormalTok{)}
\NormalTok{taxa.abundances[}\DecValTok{1}\SpecialCharTok{:}\DecValTok{10}\NormalTok{]}
\end{Highlighting}
\end{Shaded}

\begin{verbatim}
##     CL3     CC1     SV1 M31Fcsw M11Fcsw M31Plmr M11Plmr F21Plmr M31Tong M11Tong 
##       2      18       2       0       0       0       0       1       0       0
\end{verbatim}

\hypertarget{community-diversity}{%
\chapter{Community diversity}\label{community-diversity}}

Diversity estimates are a central topic in microbiome data analysis.

There are three commonly employed levels of diversity measurements,
which are trying to put a number on different aspects of the questions
associated with diversity \citep{Whittaker1960}.

Many different ways for estimating such diversity measurements have been
described in the literature. Which measurement is best or applicable for your
samples, is not the aim of the following sections.

\begin{Shaded}
\begin{Highlighting}[]
\FunctionTok{library}\NormalTok{(mia)}
\FunctionTok{data}\NormalTok{(}\StringTok{"GlobalPatterns"}\NormalTok{, }\AttributeTok{package=}\StringTok{"mia"}\NormalTok{)}
\NormalTok{tse }\OtherTok{\textless{}{-}}\NormalTok{ GlobalPatterns}
\end{Highlighting}
\end{Shaded}

\textbf{\emph{Alpha diversity}}, also sometimes interchangeably used with the
term \textbf{\emph{species diversity}}, summarizes the distribution of species
abundances in a given sample into a single number that depends on
species richness and evenness. Diversity indices measure the overall
community heterogeneity. A number of ecological diversity measures are
available. The Hill coefficient combines many standard indices into a
single equation that provides observed richness, inverse Simpson, and
Shannon diversity, and generalized diversity as special cases. In
general, diversity increases together with increasing richness and
evenness. Sometimes richness, phylogenetic diversity, evenness, dominance,
and rarity are considered to be variants of alpha diversity.

\textbf{Richness} refers to the total number of species in a community
(sample). The simplest richness index is the number of observed
species (observed richness). Assuming limited sampling from the
community, however, this may underestimate the true species
richness. Several estimators are available, including for instance ACE
\citep{Chao1992} and Chao1 \citep{Chao1984}. Richness estimates are unaffected
by species abundances.

\textbf{Phylogenetic diversity} was first proposed by \citep{Faith1992}. Unlike the
diversity measures mentioned above, Phylogenetic diversity (PD)
measure incorporates information from phylogenetic relationships
stored in \texttt{phylo} tree between species in a community (sample). The
Faith's PD is calculated as the sum of branch length of all species in
a community (sample).

\textbf{Evenness} focuses on species abundances, and can thus complement
the number of species. A typical evenness index is the Pielou's
evenness, which is Shannon diversity normalized by the observed
richness.

\textbf{Dominance} indices are in general negatively correlated with
diversity, and sometimes used in ecological literature. High
dominance is obtained when one or few species have a high share of
the total species abundance in the community.

\textbf{Rarity} indices characterize the concentration of taxa at low abundance.
Prevalence and detection thresholds determine rare taxa whose total concentration
is represented as a rarity index.

\hypertarget{estimation}{%
\section{Estimation}\label{estimation}}

Alpha diversity can be estimated with wrapper functions that interact
with other packages implementing the calculation, such as \emph{\texttt{vegan}}
\citep{R-vegan}.

\hypertarget{richness}{%
\subsection{Richness}\label{richness}}

Richness gives the number of features present within a community and can be calculated with \texttt{estimateRichness}. Each of the estimate diversity/richness/evenness/dominance functions adds the calculated measure(s) to the \texttt{colData} of the \texttt{SummarizedExperiment} under the given column \texttt{name}. Here, we calculate \texttt{observed} features as a measure of richness.

\begin{Shaded}
\begin{Highlighting}[]
\NormalTok{tse }\OtherTok{\textless{}{-}}\NormalTok{ mia}\SpecialCharTok{::}\FunctionTok{estimateRichness}\NormalTok{(tse, }
                             \AttributeTok{assay.type =} \StringTok{"counts"}\NormalTok{, }
                             \AttributeTok{index =} \StringTok{"observed"}\NormalTok{, }
                             \AttributeTok{name=}\StringTok{"observed"}\NormalTok{)}

\FunctionTok{head}\NormalTok{(}\FunctionTok{colData}\NormalTok{(tse)}\SpecialCharTok{$}\NormalTok{observed)}
\end{Highlighting}
\end{Shaded}

\begin{verbatim}
##     CL3     CC1     SV1 M31Fcsw M11Fcsw M31Plmr 
##    6964    7679    5729    2667    2574    3214
\end{verbatim}

This allows access to the values to be analyzed directly from the \texttt{colData}, for example
by plotting them using \texttt{plotColData} from the \emph{\texttt{scater}} package \citep{R-scater}.

\begin{Shaded}
\begin{Highlighting}[]
\FunctionTok{library}\NormalTok{(scater)}
\FunctionTok{plotColData}\NormalTok{(tse, }
            \StringTok{"observed"}\NormalTok{, }
            \StringTok{"SampleType"}\NormalTok{, }
            \AttributeTok{colour\_by =} \StringTok{"Final\_Barcode"}\NormalTok{) }\SpecialCharTok{+}
    \FunctionTok{theme}\NormalTok{(}\AttributeTok{axis.text.x =} \FunctionTok{element\_text}\NormalTok{(}\AttributeTok{angle=}\DecValTok{45}\NormalTok{,}\AttributeTok{hjust=}\DecValTok{1}\NormalTok{)) }\SpecialCharTok{+} 
  \FunctionTok{ylab}\NormalTok{(}\FunctionTok{expression}\NormalTok{(Richness[Observed]))}
\end{Highlighting}
\end{Shaded}

\begin{figure}
\centering
\includegraphics{14_alpha_diversity_files/figure-latex/plot-div-shannon-1.pdf}
\caption{\label{fig:plot-div-shannon}Shannon diversity estimates plotted grouped by sample type with colour-labeled barcode.}
\end{figure}

\hypertarget{estimate-diversity}{%
\subsection{Diversity}\label{estimate-diversity}}

The main function, \texttt{estimateDiversity}, calculates the selected
diversity index based on the selected assay data.

\begin{Shaded}
\begin{Highlighting}[]
\NormalTok{tse }\OtherTok{\textless{}{-}}\NormalTok{ mia}\SpecialCharTok{::}\FunctionTok{estimateDiversity}\NormalTok{(tse, }
                              \AttributeTok{assay.type =} \StringTok{"counts"}\NormalTok{,}
                              \AttributeTok{index =} \StringTok{"shannon"}\NormalTok{, }
                              \AttributeTok{name =} \StringTok{"shannon"}\NormalTok{)}
\FunctionTok{head}\NormalTok{(}\FunctionTok{colData}\NormalTok{(tse)}\SpecialCharTok{$}\NormalTok{shannon)}
\end{Highlighting}
\end{Shaded}

\begin{verbatim}
##     CL3     CC1     SV1 M31Fcsw M11Fcsw M31Plmr 
##   6.577   6.777   6.498   3.828   3.288   4.289
\end{verbatim}

Alpha diversities can be visualized with boxplot. Here, Shannon index is compared
between different sample type groups. Individual data points are visualized by
plotting them as points with \texttt{geom\_jitter}.

\texttt{geom\_signif} is used to test whether these differences are statistically significant.
It adds p-values to plot.

\begin{Shaded}
\begin{Highlighting}[]
\FunctionTok{library}\NormalTok{(ggsignif)}
\FunctionTok{library}\NormalTok{(ggplot2)}
\FunctionTok{library}\NormalTok{(patchwork)}
\FunctionTok{library}\NormalTok{(ggsignif)}

\CommentTok{\# Subsets the data. Takes only those samples that are from feces, skin, or tongue,}
\CommentTok{\# and creates data frame from the collected data}
\NormalTok{df }\OtherTok{\textless{}{-}} \FunctionTok{as.data.frame}\NormalTok{(}\FunctionTok{colData}\NormalTok{(tse)[}\FunctionTok{colData}\NormalTok{(tse)}\SpecialCharTok{$}\NormalTok{SampleType }\SpecialCharTok{\%in\%} 
                                  \FunctionTok{c}\NormalTok{(}\StringTok{"Feces"}\NormalTok{, }\StringTok{"Skin"}\NormalTok{, }\StringTok{"Tongue"}\NormalTok{), ])}

\CommentTok{\# Changes old levels with new levels}
\NormalTok{df}\SpecialCharTok{$}\NormalTok{SampleType }\OtherTok{\textless{}{-}} \FunctionTok{factor}\NormalTok{(df}\SpecialCharTok{$}\NormalTok{SampleType)}

\CommentTok{\# For significance testing, all different combinations are determined}
\NormalTok{comb }\OtherTok{\textless{}{-}} \FunctionTok{split}\NormalTok{(}\FunctionTok{t}\NormalTok{(}\FunctionTok{combn}\NormalTok{(}\FunctionTok{levels}\NormalTok{(df}\SpecialCharTok{$}\NormalTok{SampleType), }\DecValTok{2}\NormalTok{)), }
           \FunctionTok{seq}\NormalTok{(}\FunctionTok{nrow}\NormalTok{(}\FunctionTok{t}\NormalTok{(}\FunctionTok{combn}\NormalTok{(}\FunctionTok{levels}\NormalTok{(df}\SpecialCharTok{$}\NormalTok{SampleType), }\DecValTok{2}\NormalTok{)))))}

\FunctionTok{ggplot}\NormalTok{(df, }\FunctionTok{aes}\NormalTok{(}\AttributeTok{x =}\NormalTok{ SampleType, }\AttributeTok{y =}\NormalTok{ shannon)) }\SpecialCharTok{+}
  \CommentTok{\# Outliers are removed, because otherwise each data point would be plotted twice; }
  \CommentTok{\# as an outlier of boxplot and as a point of dotplot.}
  \FunctionTok{geom\_boxplot}\NormalTok{(}\AttributeTok{outlier.shape =} \ConstantTok{NA}\NormalTok{) }\SpecialCharTok{+} 
  \FunctionTok{geom\_jitter}\NormalTok{(}\AttributeTok{width =} \FloatTok{0.2}\NormalTok{) }\SpecialCharTok{+} 
  \FunctionTok{geom\_signif}\NormalTok{(}\AttributeTok{comparisons =}\NormalTok{ comb, }\AttributeTok{map\_signif\_level =} \ConstantTok{FALSE}\NormalTok{) }\SpecialCharTok{+}
  \FunctionTok{theme}\NormalTok{(}\AttributeTok{text =} \FunctionTok{element\_text}\NormalTok{(}\AttributeTok{size =} \DecValTok{10}\NormalTok{))}
\end{Highlighting}
\end{Shaded}

\includegraphics{14_alpha_diversity_files/figure-latex/visualize-shannon-1.pdf}

\hypertarget{faith-diversity}{%
\subsection{Faith phylogenetic diversity}\label{faith-diversity}}

The Faith index is returned by the function \texttt{estimateFaith}.

\begin{Shaded}
\begin{Highlighting}[]
\NormalTok{tse }\OtherTok{\textless{}{-}}\NormalTok{ mia}\SpecialCharTok{::}\FunctionTok{estimateFaith}\NormalTok{(tse,}
                          \AttributeTok{assay.type =} \StringTok{"counts"}\NormalTok{)}
\FunctionTok{head}\NormalTok{(}\FunctionTok{colData}\NormalTok{(tse)}\SpecialCharTok{$}\NormalTok{faith)}
\end{Highlighting}
\end{Shaded}

\begin{verbatim}
## [1] 250.5 262.3 208.5 117.9 119.8 135.8
\end{verbatim}

\textbf{Note}: because \texttt{tse} is a \texttt{TreeSummarizedExperiment} object, its phylogenetic tree is used by default. However, the optional argument \texttt{tree} must be provided if \texttt{tse} does not contain one.

Below a visual comparison between shannon and faith indices is shown with a violin plot.

\begin{Shaded}
\begin{Highlighting}[]
\NormalTok{plots }\OtherTok{\textless{}{-}} \FunctionTok{lapply}\NormalTok{(}\FunctionTok{c}\NormalTok{(}\StringTok{"shannon"}\NormalTok{, }\StringTok{"faith"}\NormalTok{),}
\NormalTok{                plotColData,}
                \AttributeTok{object =}\NormalTok{ tse, }\AttributeTok{colour\_by =} \StringTok{"SampleType"}\NormalTok{)}
\NormalTok{plots[[}\DecValTok{1}\NormalTok{]] }\SpecialCharTok{+}\NormalTok{ plots[[}\DecValTok{2}\NormalTok{]] }\SpecialCharTok{+}
  \FunctionTok{plot\_layout}\NormalTok{(}\AttributeTok{guides =} \StringTok{"collect"}\NormalTok{)}
\end{Highlighting}
\end{Shaded}

\includegraphics{14_alpha_diversity_files/figure-latex/phylo-div-2-1.pdf}

Alternatively, the phylogenetic diversity can be calculated by \texttt{mia::estimateDiversity}. This is a faster re-implementation of\\
the widely used function in \emph{\texttt{picante}} \citep[\citet{Kembel2010}]{R-picante}.

Load \texttt{picante} R package and get the \texttt{phylo} stored in \texttt{rowTree}.

\begin{Shaded}
\begin{Highlighting}[]
\NormalTok{tse }\OtherTok{\textless{}{-}}\NormalTok{ mia}\SpecialCharTok{::}\FunctionTok{estimateDiversity}\NormalTok{(tse, }
                              \AttributeTok{assay.type =} \StringTok{"counts"}\NormalTok{,}
                              \AttributeTok{index =} \StringTok{"faith"}\NormalTok{, }
                              \AttributeTok{name =} \StringTok{"faith"}\NormalTok{)}
\end{Highlighting}
\end{Shaded}

\hypertarget{evenness}{%
\subsection{Evenness}\label{evenness}}

Evenness can be calculated with \texttt{estimateEvenness}.

\begin{Shaded}
\begin{Highlighting}[]
\NormalTok{tse }\OtherTok{\textless{}{-}} \FunctionTok{estimateEvenness}\NormalTok{(tse, }
                        \AttributeTok{assay.type =} \StringTok{"counts"}\NormalTok{, }
                        \AttributeTok{index=}\StringTok{"simpson"}\NormalTok{)}
\FunctionTok{head}\NormalTok{(}\FunctionTok{colData}\NormalTok{(tse)}\SpecialCharTok{$}\NormalTok{simpson)}
\end{Highlighting}
\end{Shaded}

\begin{verbatim}
## [1] 0.026871 0.027197 0.047049 0.005179 0.004304 0.005011
\end{verbatim}

\hypertarget{dominance}{%
\subsection{Dominance}\label{dominance}}

Dominance can be calculated with \texttt{estimateDominance}. Here, the \texttt{Relative\ index} is calculated which is the relative abundance of the most dominant species in the sample.

\begin{Shaded}
\begin{Highlighting}[]
\NormalTok{tse }\OtherTok{\textless{}{-}} \FunctionTok{estimateDominance}\NormalTok{(tse, }
                         \AttributeTok{assay.type =} \StringTok{"counts"}\NormalTok{, }
                         \AttributeTok{index=}\StringTok{"relative"}\NormalTok{)}

\FunctionTok{head}\NormalTok{(}\FunctionTok{colData}\NormalTok{(tse)}\SpecialCharTok{$}\NormalTok{relative)}
\end{Highlighting}
\end{Shaded}

\begin{verbatim}
##     CL3     CC1     SV1 M31Fcsw M11Fcsw M31Plmr 
## 0.03910 0.03226 0.01690 0.22981 0.21778 0.22329
\end{verbatim}

\hypertarget{rarity}{%
\subsection{Rarity}\label{rarity}}

\texttt{mia} package provides one rarity index called log-modulo skewness. It can be
calculated with \texttt{estimateDiversity}.

\begin{Shaded}
\begin{Highlighting}[]
\NormalTok{tse }\OtherTok{\textless{}{-}}\NormalTok{ mia}\SpecialCharTok{::}\FunctionTok{estimateDiversity}\NormalTok{(tse, }
                              \AttributeTok{assay.type =} \StringTok{"counts"}\NormalTok{,}
                              \AttributeTok{index =} \StringTok{"log\_modulo\_skewness"}\NormalTok{)}

\FunctionTok{head}\NormalTok{(}\FunctionTok{colData}\NormalTok{(tse)}\SpecialCharTok{$}\NormalTok{log\_modulo\_skewness)}
\end{Highlighting}
\end{Shaded}

\begin{verbatim}
## [1] 2.061 2.061 2.061 2.061 2.061 2.061
\end{verbatim}

\hypertarget{divergence}{%
\subsection{Divergence}\label{divergence}}

Divergence can be evaluated with \texttt{estimateDivergence}. Reference and algorithm for the calculation of divergence can be specified as \texttt{reference} and \texttt{FUN}, respectively.

\begin{Shaded}
\begin{Highlighting}[]
\NormalTok{tse }\OtherTok{\textless{}{-}}\NormalTok{ mia}\SpecialCharTok{::}\FunctionTok{estimateDivergence}\NormalTok{(tse,}
                               \AttributeTok{assay.type =} \StringTok{"counts"}\NormalTok{,}
                               \AttributeTok{reference =} \StringTok{"median"}\NormalTok{,}
                               \AttributeTok{FUN =}\NormalTok{ vegan}\SpecialCharTok{::}\NormalTok{vegdist)}
\end{Highlighting}
\end{Shaded}

\hypertarget{visualization}{%
\section{Visualization}\label{visualization}}

A plot comparing all the diversity measures calculated above and stored in \texttt{colData} can then be constructed directly.

\begin{Shaded}
\begin{Highlighting}[]
\NormalTok{plots }\OtherTok{\textless{}{-}} \FunctionTok{lapply}\NormalTok{(}\FunctionTok{c}\NormalTok{(}\StringTok{"observed"}\NormalTok{, }\StringTok{"shannon"}\NormalTok{, }\StringTok{"simpson"}\NormalTok{, }\StringTok{"relative"}\NormalTok{, }\StringTok{"faith"}\NormalTok{, }\StringTok{"log\_modulo\_skewness"}\NormalTok{),}
\NormalTok{                plotColData,}
                \AttributeTok{object =}\NormalTok{ tse,}
                \AttributeTok{x =} \StringTok{"SampleType"}\NormalTok{,}
                \AttributeTok{colour\_by =} \StringTok{"SampleType"}\NormalTok{)}

\NormalTok{plots }\OtherTok{\textless{}{-}} \FunctionTok{lapply}\NormalTok{(plots, }\StringTok{"+"}\NormalTok{, }
                \FunctionTok{theme}\NormalTok{(}\AttributeTok{axis.text.x =} \FunctionTok{element\_blank}\NormalTok{(),}
                      \AttributeTok{axis.title.x =} \FunctionTok{element\_blank}\NormalTok{(),}
                      \AttributeTok{axis.ticks.x =} \FunctionTok{element\_blank}\NormalTok{()))}

\NormalTok{((plots[[}\DecValTok{1}\NormalTok{]] }\SpecialCharTok{|}\NormalTok{ plots[[}\DecValTok{2}\NormalTok{]] }\SpecialCharTok{|}\NormalTok{ plots[[}\DecValTok{3}\NormalTok{]]) }\SpecialCharTok{/} 
\NormalTok{(plots[[}\DecValTok{4}\NormalTok{]] }\SpecialCharTok{|}\NormalTok{ plots[[}\DecValTok{5}\NormalTok{]] }\SpecialCharTok{|}\NormalTok{ plots[[}\DecValTok{6}\NormalTok{]])) }\SpecialCharTok{+}
  \FunctionTok{plot\_layout}\NormalTok{(}\AttributeTok{guides =} \StringTok{"collect"}\NormalTok{)}
\end{Highlighting}
\end{Shaded}

\includegraphics{14_alpha_diversity_files/figure-latex/plot-all-diversities-1.pdf}

\hypertarget{community-similarity}{%
\chapter{Community similarity}\label{community-similarity}}

Where alpha diversity focuses on community variation within a
community (sample), beta diversity quantifies (dis-)similarites
between communities (samples). Some of the most popular beta diversity
measures in microbiome research include Bray-Curtis index (for
compositional data), Jaccard index (for presence / absence data,
ignoring abundance information), Aitchison distance (Euclidean
distance for clr transformed abundances, aiming to avoid the
compositionality bias), and the Unifrac distances (that take into
account the phylogenetic tree information). Only some of the commonly
used beta diversity measures are actual \emph{distances}; this is a
mathematically well-defined concept and many ecological beta diversity
measures, such as Bray-Curtis index, are not proper distances.
Therefore, the term dissimilarity or beta diversity is commonly used.

Technically, beta diversities are usually represented as \texttt{dist}
objects, which contain triangular data describing the distance between
each pair of samples. These distances can be further subjected to
ordination. Ordination is a common concept in ecology that aims to
reduce the dimensionality of the data for further evaluation or
visualization. Ordination techniques aim to capture as much of
essential information in the data as possible in a lower dimensional
representation. Dimension reduction is bound to loose information but
the common ordination techniques aim to preserve relevant information
of sample similarities in an optimal way, which is defined in
different ways by different methods. {[}TODO add references and/or link
to ordination chapter instead?{]}

Some of the most common ordination methods in microbiome research
include Principal Component Analysis (PCA), metric and non-metric
multi-dimensional scaling (MDS, NMDS), The MDS methods are also known
as Principal Coordinates Analysis (PCoA). Other recently popular
techniques include t-SNE and UMAP.

\hypertarget{explained-variance}{%
\section{Explained variance}\label{explained-variance}}

The percentage of explained variance is typically shown for PCA
ordination plots. This quantifies the proportion of overall variance
in the data that is captured by the PCA axes, or how well the
ordination axes reflect the original distances.

Sometimes a similar measure is shown for MDS/PCoA. The interpretation
is generally different, however, and hence we do not recommend using
it. PCA is a special case of PCoA with Euclidean distances. With
non-Euclidean dissimilarities PCoA uses a trick where the pointwise
dissimilarities are first cast into similarities in a Euclidean space
(with some information loss i.e.~stress) and then projected to the
maximal variance axes. In this case, the maximal variance axes do not
directly reflect the correspondence of the projected distances and
original distances, as they do for PCA.

In typical use cases, we would like to know how well the ordination
reflects the original similarity structures; then the quantity of
interest is the so-called ``stress'' function, which measures the
difference in pairwise similarities between the data points in the
original (high-dimensional) vs.~projected (low-dimensional) space.

Hence, we propose that for PCoA and other ordination methods, users
would report relative stress (varies in the unit interval; the smaller
the better). This can be calculated as shown below. For further
examples, check the \href{https://www.huber.embl.de/users/klaus/Teaching/statisticalMethods-lab.pdf}{note from Huber
lab}.

\begin{Shaded}
\begin{Highlighting}[]
\CommentTok{\# Example data}
\FunctionTok{library}\NormalTok{(mia)}
\FunctionTok{data}\NormalTok{(GlobalPatterns, }\AttributeTok{package=}\StringTok{"mia"}\NormalTok{)}

\CommentTok{\# Data matrix (features x samples)}
\NormalTok{tse }\OtherTok{\textless{}{-}}\NormalTok{ GlobalPatterns}
\NormalTok{tse }\OtherTok{\textless{}{-}} \FunctionTok{transformCounts}\NormalTok{(tse, }\AttributeTok{method =} \StringTok{"relabundance"}\NormalTok{)}

\CommentTok{\# Add group information Feces yes/no}
\FunctionTok{colData}\NormalTok{(tse)}\SpecialCharTok{$}\NormalTok{Group }\OtherTok{\textless{}{-}} \FunctionTok{colData}\NormalTok{(tse)}\SpecialCharTok{$}\NormalTok{SampleType}\SpecialCharTok{==}\StringTok{"Feces"}

\CommentTok{\# Quantify dissimilarities in the original feature space}
\FunctionTok{library}\NormalTok{(vegan)}
\NormalTok{x }\OtherTok{\textless{}{-}} \FunctionTok{assay}\NormalTok{(tse, }\StringTok{"relabundance"}\NormalTok{) }\CommentTok{\# Pick relabunance assay separately}
\NormalTok{d0 }\OtherTok{\textless{}{-}} \FunctionTok{as.matrix}\NormalTok{(}\FunctionTok{vegdist}\NormalTok{(}\FunctionTok{t}\NormalTok{(x), }\StringTok{"bray"}\NormalTok{))}

\CommentTok{\# PCoA Ordination}
\NormalTok{pcoa }\OtherTok{\textless{}{-}} \FunctionTok{as.data.frame}\NormalTok{(}\FunctionTok{cmdscale}\NormalTok{(d0, }\AttributeTok{k =} \DecValTok{2}\NormalTok{))}
\FunctionTok{names}\NormalTok{(pcoa) }\OtherTok{\textless{}{-}} \FunctionTok{c}\NormalTok{(}\StringTok{"PCoA1"}\NormalTok{, }\StringTok{"PCoA2"}\NormalTok{)}

\CommentTok{\# Quantify dissimilarities in the ordination space}
\NormalTok{dp }\OtherTok{\textless{}{-}} \FunctionTok{as.matrix}\NormalTok{(}\FunctionTok{dist}\NormalTok{(pcoa))}

\CommentTok{\# Calculate stress i.e. relative difference in the original and}
\CommentTok{\# projected dissimilarities}
\NormalTok{stress }\OtherTok{\textless{}{-}} \FunctionTok{sum}\NormalTok{((dp }\SpecialCharTok{{-}}\NormalTok{ d0)}\SpecialCharTok{\^{}}\DecValTok{2}\NormalTok{)}\SpecialCharTok{/}\FunctionTok{sum}\NormalTok{(d0}\SpecialCharTok{\^{}}\DecValTok{2}\NormalTok{)}
\end{Highlighting}
\end{Shaded}

Shepard plot visualizes the original versus projected (ordination)
dissimilarities between the data points:

\begin{Shaded}
\begin{Highlighting}[]
\NormalTok{ord }\OtherTok{\textless{}{-}} \FunctionTok{order}\NormalTok{(}\FunctionTok{as.vector}\NormalTok{(d0))}
\NormalTok{df }\OtherTok{\textless{}{-}} \FunctionTok{data.frame}\NormalTok{(}\AttributeTok{d0 =} \FunctionTok{as.vector}\NormalTok{(d0)[ord],}
                  \AttributeTok{dmds =} \FunctionTok{as.vector}\NormalTok{(dp)[ord])}

\FunctionTok{library}\NormalTok{(ggplot2)}
\FunctionTok{ggplot}\NormalTok{(}\FunctionTok{aes}\NormalTok{(}\AttributeTok{x =}\NormalTok{ d0, }\AttributeTok{y =}\NormalTok{ dmds), }\AttributeTok{data=}\NormalTok{df) }\SpecialCharTok{+} 
       \FunctionTok{geom\_smooth}\NormalTok{() }\SpecialCharTok{+}
       \FunctionTok{geom\_point}\NormalTok{() }\SpecialCharTok{+}       
       \FunctionTok{labs}\NormalTok{(}\AttributeTok{title =} \StringTok{"Shepard plot"}\NormalTok{,}
       \AttributeTok{x =} \StringTok{"Original distance"}\NormalTok{,}
       \AttributeTok{y =} \StringTok{"MDS distance"}\NormalTok{,       }
            \AttributeTok{subtitle =} \FunctionTok{paste}\NormalTok{(}\StringTok{"Stress:"}\NormalTok{, }\FunctionTok{round}\NormalTok{(stress, }\DecValTok{2}\NormalTok{))) }\SpecialCharTok{+}
  \FunctionTok{theme\_bw}\NormalTok{()}
\end{Highlighting}
\end{Shaded}

\includegraphics{20_beta_diversity_files/figure-latex/shepard-1.png}

\hypertarget{community-comparisons-by-beta-diversity-analysis}{%
\section{Community comparisons by beta diversity analysis}\label{community-comparisons-by-beta-diversity-analysis}}

A typical comparison of community composition starts with a visual
comparison of the groups on a 2D ordination.

Then we estimate relative abundances and MDS ordination based on
Bray-Curtis (BC) dissimilarity between the groups, and visualize the
results.

In the following examples dissimilarities are calculated by
functions supplied to the \texttt{FUN} argument. This function can be defined by
the user. It must return a \texttt{dist} function, which can then be used to
calculate reduced dimensions either via ordination methods (such as MDS
or NMDS), and the results can be stored in the \texttt{reducedDim}.

This entire process is wrapped in the \texttt{runMDS} and \texttt{runNMDS}
functions.

\begin{Shaded}
\begin{Highlighting}[]
\FunctionTok{library}\NormalTok{(scater)}

\CommentTok{\# Bray{-}Curtis is usually applied to relative abundances}
\NormalTok{tse }\OtherTok{\textless{}{-}} \FunctionTok{transformCounts}\NormalTok{(tse, }\AttributeTok{method =} \StringTok{"relabundance"}\NormalTok{)}
\CommentTok{\# Perform PCoA}
\NormalTok{tse }\OtherTok{\textless{}{-}} \FunctionTok{runMDS}\NormalTok{(tse, }\AttributeTok{FUN =}\NormalTok{ vegan}\SpecialCharTok{::}\NormalTok{vegdist, }\AttributeTok{method =} \StringTok{"bray"}\NormalTok{, }\AttributeTok{name =} \StringTok{"PCoA\_BC"}\NormalTok{, }\AttributeTok{assay.type =} \StringTok{"relabundance"}\NormalTok{)}
\end{Highlighting}
\end{Shaded}

Sample similarities can be visualized on a lower-dimensional display
(typically 2D) using the \texttt{plotReducedDim} function in the \texttt{scater}
package. This provides also further tools to incorporate additional
information using variations in color, shape or size. Are there
visible differences between the groups?

\begin{Shaded}
\begin{Highlighting}[]
\CommentTok{\# Create ggplot object}
\NormalTok{p }\OtherTok{\textless{}{-}} \FunctionTok{plotReducedDim}\NormalTok{(tse, }\StringTok{"PCoA\_BC"}\NormalTok{, }\AttributeTok{colour\_by =} \StringTok{"Group"}\NormalTok{)}

\CommentTok{\# Add explained variance for each axis}
\NormalTok{e }\OtherTok{\textless{}{-}} \FunctionTok{attr}\NormalTok{(}\FunctionTok{reducedDim}\NormalTok{(tse, }\StringTok{"PCoA\_BC"}\NormalTok{), }\StringTok{"eig"}\NormalTok{);}
\NormalTok{rel\_eig }\OtherTok{\textless{}{-}}\NormalTok{ e}\SpecialCharTok{/}\FunctionTok{sum}\NormalTok{(e[e}\SpecialCharTok{\textgreater{}}\DecValTok{0}\NormalTok{])          }
\NormalTok{p }\OtherTok{\textless{}{-}}\NormalTok{ p }\SpecialCharTok{+} \FunctionTok{labs}\NormalTok{(}\AttributeTok{x =} \FunctionTok{paste}\NormalTok{(}\StringTok{"PCoA 1 ("}\NormalTok{, }\FunctionTok{round}\NormalTok{(}\DecValTok{100} \SpecialCharTok{*}\NormalTok{ rel\_eig[[}\DecValTok{1}\NormalTok{]],}\DecValTok{1}\NormalTok{), }\StringTok{"\%"}\NormalTok{, }\StringTok{")"}\NormalTok{, }\AttributeTok{sep =} \StringTok{""}\NormalTok{),}
              \AttributeTok{y =} \FunctionTok{paste}\NormalTok{(}\StringTok{"PCoA 2 ("}\NormalTok{, }\FunctionTok{round}\NormalTok{(}\DecValTok{100} \SpecialCharTok{*}\NormalTok{ rel\_eig[[}\DecValTok{2}\NormalTok{]],}\DecValTok{1}\NormalTok{), }\StringTok{"\%"}\NormalTok{, }\StringTok{")"}\NormalTok{, }\AttributeTok{sep =} \StringTok{""}\NormalTok{))}

\FunctionTok{print}\NormalTok{(p)}
\end{Highlighting}
\end{Shaded}

\begin{figure}
\centering
\includegraphics{20_beta_diversity_files/figure-latex/plot-mds-bray-curtis-1.png}
\caption{\label{fig:plot-mds-bray-curtis}MDS plot based on the Bray-Curtis distances on the GlobalPattern dataset.}
\end{figure}

With additional tools from the \texttt{ggplot2} universe, comparisons can be
performed informing on the applicability to visualize sample similarities in a
meaningful way.

\begin{Shaded}
\begin{Highlighting}[]
\NormalTok{tse }\OtherTok{\textless{}{-}} \FunctionTok{runMDS}\NormalTok{(tse, }\AttributeTok{FUN =}\NormalTok{ vegan}\SpecialCharTok{::}\NormalTok{vegdist, }\AttributeTok{name =} \StringTok{"MDS\_euclidean"}\NormalTok{,}
             \AttributeTok{method =} \StringTok{"euclidean"}\NormalTok{, }\AttributeTok{assay.type =} \StringTok{"counts"}\NormalTok{)}
\NormalTok{tse }\OtherTok{\textless{}{-}} \FunctionTok{runNMDS}\NormalTok{(tse, }\AttributeTok{FUN =}\NormalTok{ vegan}\SpecialCharTok{::}\NormalTok{vegdist, }\AttributeTok{name =} \StringTok{"NMDS\_BC"}\NormalTok{)}
\end{Highlighting}
\end{Shaded}

\begin{verbatim}
## initial  value 47.733208 
## iter   5 value 33.853364
## iter  10 value 32.891200
## final  value 32.823570 
## converged
\end{verbatim}

\begin{Shaded}
\begin{Highlighting}[]
\NormalTok{tse }\OtherTok{\textless{}{-}} \FunctionTok{runNMDS}\NormalTok{(tse, }\AttributeTok{FUN =}\NormalTok{ vegan}\SpecialCharTok{::}\NormalTok{vegdist, }\AttributeTok{name =} \StringTok{"NMDS\_euclidean"}\NormalTok{,}
               \AttributeTok{method =} \StringTok{"euclidean"}\NormalTok{)}
\end{Highlighting}
\end{Shaded}

\begin{verbatim}
## initial  value 31.882673 
## final  value 31.882673 
## converged
\end{verbatim}

\begin{Shaded}
\begin{Highlighting}[]
\NormalTok{plots }\OtherTok{\textless{}{-}} \FunctionTok{lapply}\NormalTok{(}\FunctionTok{c}\NormalTok{(}\StringTok{"PCoA\_BC"}\NormalTok{, }\StringTok{"MDS\_euclidean"}\NormalTok{, }\StringTok{"NMDS\_BC"}\NormalTok{, }\StringTok{"NMDS\_euclidean"}\NormalTok{),}
\NormalTok{                plotReducedDim,}
                \AttributeTok{object =}\NormalTok{ tse,}
                \AttributeTok{colour\_by =} \StringTok{"Group"}\NormalTok{)}

\FunctionTok{library}\NormalTok{(patchwork)}
\NormalTok{plots[[}\DecValTok{1}\NormalTok{]] }\SpecialCharTok{+}\NormalTok{ plots[[}\DecValTok{2}\NormalTok{]] }\SpecialCharTok{+}\NormalTok{ plots[[}\DecValTok{3}\NormalTok{]] }\SpecialCharTok{+}\NormalTok{ plots[[}\DecValTok{4}\NormalTok{]] }\SpecialCharTok{+}
  \FunctionTok{plot\_layout}\NormalTok{(}\AttributeTok{guides =} \StringTok{"collect"}\NormalTok{)}
\end{Highlighting}
\end{Shaded}

\begin{figure}
\centering
\includegraphics{20_beta_diversity_files/figure-latex/plot-mds-nmds-comparison-1.png}
\caption{\label{fig:plot-mds-nmds-comparison}Comparison of MDS and NMDS plots based on the Bray-Curtis or euclidean distances on the GlobalPattern dataset.}
\end{figure}

The \emph{Unifrac} method is a special case, as it requires data on the
relationship of features in form on a \texttt{phylo} tree. \texttt{calculateUnifrac}
performs the calculation to return a \texttt{dist} object, which can again be
used within \texttt{runMDS}.

\begin{Shaded}
\begin{Highlighting}[]
\FunctionTok{library}\NormalTok{(scater)}
\NormalTok{tse }\OtherTok{\textless{}{-}} \FunctionTok{runMDS}\NormalTok{(tse, }\AttributeTok{FUN =}\NormalTok{ mia}\SpecialCharTok{::}\NormalTok{calculateUnifrac, }\AttributeTok{name =} \StringTok{"Unifrac"}\NormalTok{,}
              \AttributeTok{tree =} \FunctionTok{rowTree}\NormalTok{(tse),}
              \AttributeTok{ntop =} \FunctionTok{nrow}\NormalTok{(tse),}
             \AttributeTok{assay.type =} \StringTok{"counts"}\NormalTok{)}
\end{Highlighting}
\end{Shaded}

\begin{Shaded}
\begin{Highlighting}[]
\FunctionTok{plotReducedDim}\NormalTok{(tse, }\StringTok{"Unifrac"}\NormalTok{, }\AttributeTok{colour\_by =} \StringTok{"Group"}\NormalTok{)}
\end{Highlighting}
\end{Shaded}

\begin{figure}
\centering
\includegraphics{20_beta_diversity_files/figure-latex/plot-unifrac-1.png}
\caption{\label{fig:plot-unifrac}Unifrac distances scaled by MDS of the GlobalPattern dataset.}
\end{figure}

\hypertarget{other-ordination-methods}{%
\section{Other ordination methods}\label{other-ordination-methods}}

Other dimension reduction methods, such as \texttt{PCA}, \texttt{t-SNE} and \texttt{UMAP} are
inherited directly from the \texttt{scater} package.

\begin{Shaded}
\begin{Highlighting}[]
\NormalTok{tse }\OtherTok{\textless{}{-}} \FunctionTok{runPCA}\NormalTok{(tse, }\AttributeTok{name =} \StringTok{"PCA"}\NormalTok{, }\AttributeTok{assay.type =} \StringTok{"counts"}\NormalTok{, }\AttributeTok{ncomponents =} \DecValTok{10}\NormalTok{)}
\end{Highlighting}
\end{Shaded}

\begin{Shaded}
\begin{Highlighting}[]
\FunctionTok{plotReducedDim}\NormalTok{(tse, }\StringTok{"PCA"}\NormalTok{, }\AttributeTok{colour\_by =} \StringTok{"Group"}\NormalTok{)}
\end{Highlighting}
\end{Shaded}

\begin{figure}
\centering
\includegraphics{20_beta_diversity_files/figure-latex/plot-pca-1.png}
\caption{\label{fig:plot-pca}PCA plot on the GlobalPatterns data set containing sample from different sources.}
\end{figure}

As mentioned before, applicability of the different methods depends on your
sample set.

FIXME: let us switch to UMAP for the examples?

\begin{Shaded}
\begin{Highlighting}[]
\NormalTok{tse }\OtherTok{\textless{}{-}} \FunctionTok{runTSNE}\NormalTok{(tse, }\AttributeTok{name =} \StringTok{"TSNE"}\NormalTok{, }\AttributeTok{assay.type =} \StringTok{"counts"}\NormalTok{, }\AttributeTok{ncomponents =} \DecValTok{3}\NormalTok{)}
\end{Highlighting}
\end{Shaded}

\begin{Shaded}
\begin{Highlighting}[]
\FunctionTok{plotReducedDim}\NormalTok{(tse, }\StringTok{"TSNE"}\NormalTok{, }\AttributeTok{colour\_by =} \StringTok{"Group"}\NormalTok{, }\AttributeTok{ncomponents =} \FunctionTok{c}\NormalTok{(}\DecValTok{1}\SpecialCharTok{:}\DecValTok{3}\NormalTok{))}
\end{Highlighting}
\end{Shaded}

\begin{figure}
\centering
\includegraphics{20_beta_diversity_files/figure-latex/plot-tsne-1.png}
\caption{\label{fig:plot-tsne}t-SNE plot on the GlobalPatterns data set containing sample from different sources.}
\end{figure}

As a final note, \texttt{mia} provides functions for the evaluation of additional dissimilarity indices, such as:
* \texttt{calculateJSD}, \texttt{runJSD} (Jensen-Shannon divergence)
* \texttt{calculateNMDS}, \texttt{plotNMDS} (non-metric multi-dimensional scaling)
* \texttt{calculateCCA}, \texttt{runCCA} (Canonical Correspondence Analysis)
* \texttt{calculateRDA}, \texttt{runRDA} (Redundancy Analysis)
* \texttt{calculateOverlap}, \texttt{runOverlap} ()
* \texttt{calculateDPCoA}, \texttt{runDPCoA} (Double Principal Coordinate Analysis)

Redundancy analysis is similar to PCA, however, it takes into account covariates.
It aims to maximize the variance in respect of covariates. The results shows how much
each covariate affects.

\begin{Shaded}
\begin{Highlighting}[]
\CommentTok{\# Load libraryd packages}
\FunctionTok{library}\NormalTok{(}\StringTok{"vegan"}\NormalTok{)}
\FunctionTok{library}\NormalTok{(}\StringTok{"stringr"}\NormalTok{)}
\FunctionTok{library}\NormalTok{(}\StringTok{"knitr"}\NormalTok{)}
\CommentTok{\# Load data}
\FunctionTok{data}\NormalTok{(enterotype, }\AttributeTok{package=}\StringTok{"mia"}\NormalTok{)}
\CommentTok{\# Covariates that are being analyzed}
\NormalTok{variable\_names }\OtherTok{\textless{}{-}} \FunctionTok{c}\NormalTok{(}\StringTok{"ClinicalStatus"}\NormalTok{, }\StringTok{"Gender"}\NormalTok{, }\StringTok{"Age"}\NormalTok{)}

\CommentTok{\# Apply relative transform}
\NormalTok{enterotype }\OtherTok{\textless{}{-}} \FunctionTok{transformCounts}\NormalTok{(enterotype, }\AttributeTok{method =} \StringTok{"relabundance"}\NormalTok{)}

\CommentTok{\# Create a formula}
\NormalTok{formula }\OtherTok{\textless{}{-}} \FunctionTok{as.formula}\NormalTok{(}\FunctionTok{paste0}\NormalTok{(}\StringTok{"assay \textasciitilde{} "}\NormalTok{, }\FunctionTok{str\_c}\NormalTok{(variable\_names, }\AttributeTok{collapse =} \StringTok{" + "}\NormalTok{)) )}

\CommentTok{\# \# Perform RDA}
\NormalTok{rda }\OtherTok{\textless{}{-}} \FunctionTok{calculateRDA}\NormalTok{(enterotype, }\AttributeTok{assay.type =} \StringTok{"relabundance"}\NormalTok{,}
                    \AttributeTok{formula =}\NormalTok{ formula, }\AttributeTok{distance =} \StringTok{"bray"}\NormalTok{, }\AttributeTok{na.action =}\NormalTok{ na.exclude)}
\CommentTok{\# Get the rda object}
\NormalTok{rda }\OtherTok{\textless{}{-}} \FunctionTok{attr}\NormalTok{(rda, }\StringTok{"rda"}\NormalTok{)}
\CommentTok{\# Calculate p{-}value and variance for whole model}
\CommentTok{\# Recommendation: use 999 permutations instead of 99}
\FunctionTok{set.seed}\NormalTok{(}\DecValTok{436}\NormalTok{)}
\NormalTok{permanova }\OtherTok{\textless{}{-}} \FunctionTok{anova.cca}\NormalTok{(rda, }\AttributeTok{permutations =} \DecValTok{99}\NormalTok{)}
\CommentTok{\# Create a data.frame for results}
\NormalTok{rda\_info }\OtherTok{\textless{}{-}} \FunctionTok{as.data.frame}\NormalTok{(permanova)[}\StringTok{"Model"}\NormalTok{, ]}

\CommentTok{\# Calculate p{-}value and variance for each variable}
\CommentTok{\# by = "margin" {-}{-}\textgreater{} the order or variables does not matter}
\FunctionTok{set.seed}\NormalTok{(}\DecValTok{4585}\NormalTok{)}
\NormalTok{permanova }\OtherTok{\textless{}{-}} \FunctionTok{anova.cca}\NormalTok{(rda, }\AttributeTok{by =} \StringTok{"margin"}\NormalTok{,  }\AttributeTok{permutations =} \DecValTok{99}\NormalTok{)}
\CommentTok{\# Add results to data.frame}
\NormalTok{rda\_info }\OtherTok{\textless{}{-}} \FunctionTok{rbind}\NormalTok{(rda\_info, permanova)}

\CommentTok{\# Add info about total variance}
\NormalTok{rda\_info[ , }\StringTok{"Total variance"}\NormalTok{] }\OtherTok{\textless{}{-}}\NormalTok{ rda\_info[}\StringTok{"Model"}\NormalTok{, }\DecValTok{2}\NormalTok{] }\SpecialCharTok{+}
\NormalTok{    rda\_info[}\StringTok{"Residual"}\NormalTok{, }\DecValTok{2}\NormalTok{]}

\CommentTok{\# Add info about explained variance}
\NormalTok{rda\_info[ , }\StringTok{"Explained variance"}\NormalTok{] }\OtherTok{\textless{}{-}}\NormalTok{ rda\_info[ , }\DecValTok{2}\NormalTok{] }\SpecialCharTok{/} 
\NormalTok{    rda\_info[ , }\StringTok{"Total variance"}\NormalTok{]}

\CommentTok{\# Loop through variables, calculate homogeneity}
\NormalTok{homogeneity }\OtherTok{\textless{}{-}} \FunctionTok{list}\NormalTok{()}
\CommentTok{\# Get colDtaa}
\NormalTok{coldata }\OtherTok{\textless{}{-}} \FunctionTok{colData}\NormalTok{(enterotype)}
\CommentTok{\# Get assay}
\NormalTok{assay }\OtherTok{\textless{}{-}} \FunctionTok{t}\NormalTok{(}\FunctionTok{assay}\NormalTok{(enterotype, }\StringTok{"relabundance"}\NormalTok{))}
\ControlFlowTok{for}\NormalTok{( variable\_name }\ControlFlowTok{in} \FunctionTok{rownames}\NormalTok{(rda\_info) )\{}
    \CommentTok{\# If data is continuous or discrete}
    \ControlFlowTok{if}\NormalTok{( variable\_name }\SpecialCharTok{\%in\%} \FunctionTok{c}\NormalTok{(}\StringTok{"Model"}\NormalTok{, }\StringTok{"Residual"}\NormalTok{) }\SpecialCharTok{||}
        \FunctionTok{length}\NormalTok{(}\FunctionTok{unique}\NormalTok{(coldata[[variable\_name]])) }\SpecialCharTok{/}
        \FunctionTok{length}\NormalTok{(coldata[[variable\_name]]) }\SpecialCharTok{\textgreater{}} \FloatTok{0.2}\NormalTok{ )\{}
        \CommentTok{\# Do not calculate homogeneity for continuous data}
\NormalTok{        temp }\OtherTok{\textless{}{-}} \ConstantTok{NA}
\NormalTok{    \} }\ControlFlowTok{else}\NormalTok{\{}
        \CommentTok{\# Calculate homogeneity for discrete data}
        \CommentTok{\# Calculate homogeneity}
        \FunctionTok{set.seed}\NormalTok{(}\DecValTok{413}\NormalTok{)}
\NormalTok{        temp }\OtherTok{\textless{}{-}} \FunctionTok{anova}\NormalTok{(}
            \FunctionTok{betadisper}\NormalTok{( }
                \FunctionTok{vegdist}\NormalTok{(assay, }\AttributeTok{method =} \StringTok{"bray"}\NormalTok{),}
                \AttributeTok{group =}\NormalTok{ coldata[[variable\_name]] ),}
            \AttributeTok{permutations =}\NormalTok{ permutations )[}\StringTok{"Groups"}\NormalTok{, }\StringTok{"Pr(\textgreater{}F)"}\NormalTok{]}
\NormalTok{    \}}
    \CommentTok{\# Add info to the list}
\NormalTok{    homogeneity[[variable\_name]] }\OtherTok{\textless{}{-}}\NormalTok{ temp}
\NormalTok{\}}
\CommentTok{\# Add homogeneity to information}
\NormalTok{rda\_info[[}\StringTok{"Homogeneity p{-}value (NULL hyp: distinct/homogeneous {-}{-}\textgreater{} permanova suitable)"}\NormalTok{]] }\OtherTok{\textless{}{-}}
\NormalTok{    homogeneity}

\FunctionTok{kable}\NormalTok{(rda\_info)}
\end{Highlighting}
\end{Shaded}

\begin{tabular}{l|r|r|r|r|r|r|l}
\hline
  & Df & SumOfSqs & F & Pr(>F) & Total variance & Explained variance & Homogeneity p-value (NULL hyp: distinct/homogeneous --> permanova suitable)\\
\hline
Model & 6 & 1.1157 & 1.940 & 0.05 & 3.991 & 0.2795 & NA\\
\hline
ClinicalStatus & 4 & 0.5837 & 1.522 & 0.15 & 3.991 & 0.1463 & 0.044277....\\
\hline
Gender & 1 & 0.1679 & 1.751 & 0.10 & 3.991 & 0.0421 & 0.522999....\\
\hline
Age & 1 & 0.5245 & 5.471 & 0.01 & 3.991 & 0.1314 & 0.000369....\\
\hline
Residual & 30 & 2.8757 & NA & NA & 3.991 & 0.7205 & NA\\
\hline
\end{tabular}

\begin{Shaded}
\begin{Highlighting}[]
\CommentTok{\# Load ggord for plotting}
\FunctionTok{library}\NormalTok{(}\StringTok{"ggord"}\NormalTok{)}
\FunctionTok{library}\NormalTok{(}\StringTok{"ggplot2"}\NormalTok{)}

\CommentTok{\# Since na.exclude was used, if there were rows missing information, they were }
\CommentTok{\# dropped off. Subset coldata so that it matches with rda.}
\NormalTok{coldata }\OtherTok{\textless{}{-}}\NormalTok{ coldata[ }\FunctionTok{rownames}\NormalTok{(rda}\SpecialCharTok{$}\NormalTok{CCA}\SpecialCharTok{$}\NormalTok{wa), ]}

\CommentTok{\# Adjust names}
\CommentTok{\# Get labels of vectors}
\NormalTok{vec\_lab\_old }\OtherTok{\textless{}{-}} \FunctionTok{rownames}\NormalTok{(rda}\SpecialCharTok{$}\NormalTok{CCA}\SpecialCharTok{$}\NormalTok{biplot)}

\CommentTok{\# Loop through vector labels}
\NormalTok{vec\_lab }\OtherTok{\textless{}{-}} \FunctionTok{sapply}\NormalTok{(vec\_lab\_old, }\AttributeTok{FUN =} \ControlFlowTok{function}\NormalTok{(name)\{}
    \CommentTok{\# Get the variable name}
\NormalTok{    variable\_name }\OtherTok{\textless{}{-}}\NormalTok{ variable\_names[ }\FunctionTok{str\_detect}\NormalTok{(name, variable\_names) ]}
    \CommentTok{\# If the vector label includes also group name}
    \ControlFlowTok{if}\NormalTok{( }\SpecialCharTok{!}\FunctionTok{any}\NormalTok{(name }\SpecialCharTok{\%in\%}\NormalTok{ variable\_names) )\{}
        \CommentTok{\# Get the group names}
\NormalTok{        group\_name }\OtherTok{\textless{}{-}} \FunctionTok{unique}\NormalTok{( coldata[[variable\_name]] )[ }
        \FunctionTok{which}\NormalTok{( }\FunctionTok{paste0}\NormalTok{(variable\_name, }\FunctionTok{unique}\NormalTok{( coldata[[variable\_name]] )) }\SpecialCharTok{==}\NormalTok{ name ) ]}
        \CommentTok{\# Modify vector so that group is separated from variable name}
\NormalTok{        new\_name }\OtherTok{\textless{}{-}} \FunctionTok{paste0}\NormalTok{(variable\_name, }\StringTok{" \textbackslash{}U2012 "}\NormalTok{, group\_name)}
\NormalTok{    \} }\ControlFlowTok{else}\NormalTok{\{}
\NormalTok{        new\_name }\OtherTok{\textless{}{-}}\NormalTok{ name}
\NormalTok{    \}}
    \CommentTok{\# Add percentage how much this variable explains, and p{-}value}
\NormalTok{    new\_name }\OtherTok{\textless{}{-}} \FunctionTok{expr}\NormalTok{(}\FunctionTok{paste}\NormalTok{(}\SpecialCharTok{!!}\NormalTok{new\_name, }\StringTok{" ("}\NormalTok{, }
                           \SpecialCharTok{!!}\FunctionTok{format}\NormalTok{(}\FunctionTok{round}\NormalTok{( rda\_info[variable\_name, }\StringTok{"Explained variance"}\NormalTok{]}\SpecialCharTok{*}\DecValTok{100}\NormalTok{, }\DecValTok{1}\NormalTok{), }\AttributeTok{nsmall =} \DecValTok{1}\NormalTok{), }
                           \StringTok{"\%, "}\NormalTok{,}\FunctionTok{italic}\NormalTok{(}\StringTok{"P"}\NormalTok{), }\StringTok{" = "}\NormalTok{, }
                           \SpecialCharTok{!!}\FunctionTok{gsub}\NormalTok{(}\StringTok{"0}\SpecialCharTok{\textbackslash{}\textbackslash{}}\StringTok{."}\NormalTok{,}\StringTok{"}\SpecialCharTok{\textbackslash{}\textbackslash{}}\StringTok{."}\NormalTok{, }\FunctionTok{format}\NormalTok{(}\FunctionTok{round}\NormalTok{( rda\_info[variable\_name, }\StringTok{"Pr(\textgreater{}F)"}\NormalTok{], }\DecValTok{3}\NormalTok{), }
                                                       \AttributeTok{nsmall =} \DecValTok{3}\NormalTok{)), }\StringTok{")"}\NormalTok{))}

    \FunctionTok{return}\NormalTok{(new\_name)}
\NormalTok{\})}
\CommentTok{\# Add names}
\FunctionTok{names}\NormalTok{(vec\_lab) }\OtherTok{\textless{}{-}}\NormalTok{ vec\_lab\_old}

\CommentTok{\# Create labels for axis}
\NormalTok{xlab }\OtherTok{\textless{}{-}} \FunctionTok{paste0}\NormalTok{(}\StringTok{"RDA1 ("}\NormalTok{, }\FunctionTok{format}\NormalTok{(}\FunctionTok{round}\NormalTok{( rda}\SpecialCharTok{$}\NormalTok{CCA}\SpecialCharTok{$}\NormalTok{eig[[}\DecValTok{1}\NormalTok{]]}\SpecialCharTok{/}\NormalTok{rda}\SpecialCharTok{$}\NormalTok{CCA}\SpecialCharTok{$}\NormalTok{tot.chi}\SpecialCharTok{*}\DecValTok{100}\NormalTok{, }\DecValTok{1}\NormalTok{), }\AttributeTok{nsmall =} \DecValTok{1}\NormalTok{ ), }\StringTok{"\%)"}\NormalTok{)}
\NormalTok{ylab }\OtherTok{\textless{}{-}} \FunctionTok{paste0}\NormalTok{(}\StringTok{"RDA2 ("}\NormalTok{, }\FunctionTok{format}\NormalTok{(}\FunctionTok{round}\NormalTok{( rda}\SpecialCharTok{$}\NormalTok{CCA}\SpecialCharTok{$}\NormalTok{eig[[}\DecValTok{2}\NormalTok{]]}\SpecialCharTok{/}\NormalTok{rda}\SpecialCharTok{$}\NormalTok{CCA}\SpecialCharTok{$}\NormalTok{tot.chi}\SpecialCharTok{*}\DecValTok{100}\NormalTok{, }\DecValTok{1}\NormalTok{), }\AttributeTok{nsmall =} \DecValTok{1}\NormalTok{ ), }\StringTok{"\%)"}\NormalTok{)}

\CommentTok{\# Create a plot        }
\NormalTok{plot }\OtherTok{\textless{}{-}} \FunctionTok{ggord}\NormalTok{(rda, }\AttributeTok{grp\_in =}\NormalTok{ coldata[[}\StringTok{"ClinicalStatus"}\NormalTok{]], }\AttributeTok{vec\_lab =}\NormalTok{ vec\_lab,}
              \AttributeTok{alpha =} \FloatTok{0.5}\NormalTok{,}
              \AttributeTok{size =} \DecValTok{4}\NormalTok{, }\AttributeTok{addsize =} \SpecialCharTok{{-}}\DecValTok{4}\NormalTok{,}
              \CommentTok{\#ext= 0.7, }
              \AttributeTok{txt =} \FloatTok{3.5}\NormalTok{, }\AttributeTok{repel =} \ConstantTok{TRUE}\NormalTok{, }
              \CommentTok{\#coord\_fix = FALSE}
\NormalTok{          ) }\SpecialCharTok{+} 
    \CommentTok{\# Adjust titles and labels}
    \FunctionTok{guides}\NormalTok{(}\AttributeTok{colour =} \FunctionTok{guide\_legend}\NormalTok{(}\StringTok{"ClinicalStatus"}\NormalTok{),}
           \AttributeTok{fill =} \FunctionTok{guide\_legend}\NormalTok{(}\StringTok{"ClinicalStatus"}\NormalTok{),}
           \AttributeTok{group =} \FunctionTok{guide\_legend}\NormalTok{(}\StringTok{"ClinicalStatus"}\NormalTok{),}
           \AttributeTok{shape =} \FunctionTok{guide\_legend}\NormalTok{(}\StringTok{"ClinicalStatus"}\NormalTok{),}
           \AttributeTok{x =} \FunctionTok{guide\_axis}\NormalTok{(xlab),}
           \AttributeTok{y =} \FunctionTok{guide\_axis}\NormalTok{(ylab)) }\SpecialCharTok{+}
    \FunctionTok{theme}\NormalTok{( }\AttributeTok{axis.title =} \FunctionTok{element\_text}\NormalTok{(}\AttributeTok{size =} \DecValTok{10}\NormalTok{) )}
\NormalTok{plot}
\end{Highlighting}
\end{Shaded}

\includegraphics{20_beta_diversity_files/figure-latex/microbiome_RDA2-1.png}

From RDA plot, we can see that only age has significant affect on microbial profile.

\hypertarget{pcoa-genus}{%
\section{Visualizing the most dominant genus on PCoA}\label{pcoa-genus}}

In this section we visualize most dominant genus on PCoA. A similar visualization was proposed by Salosensaari et al. \citeyearpar{Salosensaari2021}.

Let us agglomerate the data at a Genus level and getting the dominant taxa per sample.

\begin{Shaded}
\begin{Highlighting}[]
\CommentTok{\# Agglomerate to genus level}
\NormalTok{tse\_Genus }\OtherTok{\textless{}{-}} \FunctionTok{agglomerateByRank}\NormalTok{(tse, }\AttributeTok{rank=}\StringTok{"Genus"}\NormalTok{)}
\CommentTok{\# Convert to relative abundances}
\NormalTok{tse\_Genus }\OtherTok{\textless{}{-}} \FunctionTok{transformCounts}\NormalTok{(tse, }\AttributeTok{method =} \StringTok{"relabundance"}\NormalTok{, }\AttributeTok{assay.type=}\StringTok{"counts"}\NormalTok{)}
\CommentTok{\# Add info on dominant genus per sample}
\NormalTok{tse\_Genus }\OtherTok{\textless{}{-}} \FunctionTok{addPerSampleDominantTaxa}\NormalTok{(tse\_Genus, }\AttributeTok{assay.type=}\StringTok{"relabundance"}\NormalTok{, }\AttributeTok{name =} \StringTok{"dominant\_taxa"}\NormalTok{)}
\end{Highlighting}
\end{Shaded}

Performing PCoA with Bray-Curtis dissimilarity.

\begin{Shaded}
\begin{Highlighting}[]
\NormalTok{tse\_Genus }\OtherTok{\textless{}{-}} \FunctionTok{runMDS}\NormalTok{(tse\_Genus, }\AttributeTok{FUN =}\NormalTok{ vegan}\SpecialCharTok{::}\NormalTok{vegdist,}
              \AttributeTok{name =} \StringTok{"PCoA\_BC"}\NormalTok{, }\AttributeTok{assay.type =} \StringTok{"relabundance"}\NormalTok{)}
\end{Highlighting}
\end{Shaded}

Getting top taxa and visualizing the abundance on PCoA.

\begin{Shaded}
\begin{Highlighting}[]
\CommentTok{\# Getting the top taxa}
\NormalTok{top\_taxa }\OtherTok{\textless{}{-}} \FunctionTok{getTopTaxa}\NormalTok{(tse\_Genus,}\AttributeTok{top =} \DecValTok{6}\NormalTok{, }\AttributeTok{assay.type =} \StringTok{"relabundance"}\NormalTok{)}

\CommentTok{\# Naming all the rest of non top{-}taxa as "Other"}
\NormalTok{most\_abundant }\OtherTok{\textless{}{-}} \FunctionTok{lapply}\NormalTok{(}\FunctionTok{colData}\NormalTok{(tse\_Genus)}\SpecialCharTok{$}\NormalTok{dominant\_taxa,}
                   \ControlFlowTok{function}\NormalTok{(x)\{}\ControlFlowTok{if}\NormalTok{ (x }\SpecialCharTok{\%in\%}\NormalTok{ top\_taxa) \{x\} }\ControlFlowTok{else}\NormalTok{ \{}\StringTok{"Other"}\NormalTok{\}\})}

\CommentTok{\# Storing the previous results as a new column within colData}
\FunctionTok{colData}\NormalTok{(tse\_Genus)}\SpecialCharTok{$}\NormalTok{most\_abundant }\OtherTok{\textless{}{-}} \FunctionTok{as.character}\NormalTok{(most\_abundant)}

\CommentTok{\# Calculating percentage of the most abundant}
\NormalTok{most\_abundant\_freq }\OtherTok{\textless{}{-}} \FunctionTok{table}\NormalTok{(}\FunctionTok{as.character}\NormalTok{(most\_abundant))}
\NormalTok{most\_abundant\_percent }\OtherTok{\textless{}{-}} \FunctionTok{round}\NormalTok{(most\_abundant\_freq}\SpecialCharTok{/}\FunctionTok{sum}\NormalTok{(most\_abundant\_freq)}\SpecialCharTok{*}\DecValTok{100}\NormalTok{, }\DecValTok{1}\NormalTok{)}

\CommentTok{\# Retrieving the explained variance}
\NormalTok{e }\OtherTok{\textless{}{-}} \FunctionTok{attr}\NormalTok{(}\FunctionTok{reducedDim}\NormalTok{(tse\_Genus, }\StringTok{"PCoA\_BC"}\NormalTok{), }\StringTok{"eig"}\NormalTok{);}
\NormalTok{var\_explained }\OtherTok{\textless{}{-}}\NormalTok{ e}\SpecialCharTok{/}\FunctionTok{sum}\NormalTok{(e[e}\SpecialCharTok{\textgreater{}}\DecValTok{0}\NormalTok{])}\SpecialCharTok{*}\DecValTok{100}

\CommentTok{\# Visualization}
\NormalTok{plot }\OtherTok{\textless{}{-}}\FunctionTok{plotReducedDim}\NormalTok{(tse\_Genus,}\StringTok{"PCoA\_BC"}\NormalTok{, }\AttributeTok{colour\_by =} \StringTok{"most\_abundant"}\NormalTok{) }\SpecialCharTok{+}
  \FunctionTok{scale\_colour\_manual}\NormalTok{(}\AttributeTok{values =} \FunctionTok{c}\NormalTok{(}\StringTok{"black"}\NormalTok{, }\StringTok{"blue"}\NormalTok{, }\StringTok{"lightblue"}\NormalTok{, }\StringTok{"darkgray"}\NormalTok{, }\StringTok{"magenta"}\NormalTok{, }\StringTok{"darkgreen"}\NormalTok{, }\StringTok{"red"}\NormalTok{),}
                      \AttributeTok{labels=}\FunctionTok{paste0}\NormalTok{(}\FunctionTok{names}\NormalTok{(most\_abundant\_percent),}\StringTok{"("}\NormalTok{,most\_abundant\_percent,}\StringTok{"\%)"}\NormalTok{))}\SpecialCharTok{+}
  \FunctionTok{labs}\NormalTok{(}\AttributeTok{x=}\FunctionTok{paste}\NormalTok{(}\StringTok{"PC 1 ("}\NormalTok{,}\FunctionTok{round}\NormalTok{(var\_explained[}\DecValTok{1}\NormalTok{],}\DecValTok{1}\NormalTok{),}\StringTok{"\%)"}\NormalTok{),}
       \AttributeTok{y=}\FunctionTok{paste}\NormalTok{(}\StringTok{"PC 2 ("}\NormalTok{,}\FunctionTok{round}\NormalTok{(var\_explained[}\DecValTok{2}\NormalTok{],}\DecValTok{1}\NormalTok{),}\StringTok{"\%)"}\NormalTok{),}
       \AttributeTok{color=}\StringTok{""}\NormalTok{)}
\NormalTok{plot}
\end{Highlighting}
\end{Shaded}

\includegraphics{20_beta_diversity_files/figure-latex/unnamed-chunk-7-1.png}

Note: A 3D interactive version of the earlier plot can be found from \ref{extras}.

Similarly let's visualize and compare the sub-population.

\begin{Shaded}
\begin{Highlighting}[]
\CommentTok{\# Calculating the frequencies and percentages for both categories}
\NormalTok{freq\_TRUE }\OtherTok{\textless{}{-}} \FunctionTok{table}\NormalTok{(}\FunctionTok{as.character}\NormalTok{(most\_abundant[}\FunctionTok{colData}\NormalTok{(tse\_Genus)}\SpecialCharTok{$}\NormalTok{Group}\SpecialCharTok{==}\ConstantTok{TRUE}\NormalTok{]))}
\NormalTok{freq\_FALSE }\OtherTok{\textless{}{-}} \FunctionTok{table}\NormalTok{(}\FunctionTok{as.character}\NormalTok{(most\_abundant[}\FunctionTok{colData}\NormalTok{(tse\_Genus)}\SpecialCharTok{$}\NormalTok{Group}\SpecialCharTok{==}\ConstantTok{FALSE}\NormalTok{]))}
\NormalTok{percent\_TRUE }\OtherTok{\textless{}{-}} \FunctionTok{round}\NormalTok{(freq\_TRUE}\SpecialCharTok{/}\FunctionTok{sum}\NormalTok{(freq\_TRUE)}\SpecialCharTok{*}\DecValTok{100}\NormalTok{, }\DecValTok{1}\NormalTok{)}
\NormalTok{percent\_FALSE }\OtherTok{\textless{}{-}} \FunctionTok{round}\NormalTok{(freq\_FALSE}\SpecialCharTok{/}\FunctionTok{sum}\NormalTok{(freq\_FALSE)}\SpecialCharTok{*}\DecValTok{100}\NormalTok{, }\DecValTok{1}\NormalTok{)}

\CommentTok{\# Visualization}
\FunctionTok{plotReducedDim}\NormalTok{(tse\_Genus[,}\FunctionTok{colData}\NormalTok{(tse\_Genus)}\SpecialCharTok{$}\NormalTok{Group}\SpecialCharTok{==}\ConstantTok{TRUE}\NormalTok{],}
                          \StringTok{"PCoA\_BC"}\NormalTok{, }\AttributeTok{colour\_by =} \StringTok{"most\_abundant"}\NormalTok{) }\SpecialCharTok{+}
  \FunctionTok{scale\_colour\_manual}\NormalTok{(}\AttributeTok{values =} \FunctionTok{c}\NormalTok{(}\StringTok{"black"}\NormalTok{, }\StringTok{"blue"}\NormalTok{, }\StringTok{"lightblue"}\NormalTok{, }\StringTok{"darkgray"}\NormalTok{, }\StringTok{"magenta"}\NormalTok{, }\StringTok{"darkgreen"}\NormalTok{, }\StringTok{"red"}\NormalTok{),}
                      \AttributeTok{labels=}\FunctionTok{paste0}\NormalTok{(}\FunctionTok{names}\NormalTok{(percent\_TRUE),}\StringTok{"("}\NormalTok{,percent\_TRUE,}\StringTok{"\%)"}\NormalTok{))}\SpecialCharTok{+}
  \FunctionTok{labs}\NormalTok{(}\AttributeTok{x=}\FunctionTok{paste}\NormalTok{(}\StringTok{"PC 1 ("}\NormalTok{,}\FunctionTok{round}\NormalTok{(var\_explained[}\DecValTok{1}\NormalTok{],}\DecValTok{1}\NormalTok{),}\StringTok{"\%)"}\NormalTok{),}
       \AttributeTok{y=}\FunctionTok{paste}\NormalTok{(}\StringTok{"PC 2 ("}\NormalTok{,}\FunctionTok{round}\NormalTok{(var\_explained[}\DecValTok{2}\NormalTok{],}\DecValTok{1}\NormalTok{),}\StringTok{"\%)"}\NormalTok{),}
       \AttributeTok{title =} \StringTok{"Group = TRUE"}\NormalTok{, }\AttributeTok{color=}\StringTok{""}\NormalTok{)}
\end{Highlighting}
\end{Shaded}

\includegraphics{20_beta_diversity_files/figure-latex/unnamed-chunk-8-1.png}

\begin{Shaded}
\begin{Highlighting}[]
\FunctionTok{plotReducedDim}\NormalTok{(tse\_Genus[,}\FunctionTok{colData}\NormalTok{(tse\_Genus)}\SpecialCharTok{$}\NormalTok{Group}\SpecialCharTok{==}\ConstantTok{FALSE}\NormalTok{],}
                          \StringTok{"PCoA\_BC"}\NormalTok{, }\AttributeTok{colour\_by =} \StringTok{"most\_abundant"}\NormalTok{) }\SpecialCharTok{+}
  \FunctionTok{scale\_colour\_manual}\NormalTok{(}\AttributeTok{values =} \FunctionTok{c}\NormalTok{(}\StringTok{"black"}\NormalTok{, }\StringTok{"blue"}\NormalTok{, }\StringTok{"lightblue"}\NormalTok{, }\StringTok{"darkgray"}\NormalTok{, }\StringTok{"magenta"}\NormalTok{, }\StringTok{"darkgreen"}\NormalTok{, }\StringTok{"red"}\NormalTok{),}
                      \AttributeTok{labels=}\FunctionTok{paste0}\NormalTok{(}\FunctionTok{names}\NormalTok{(percent\_FALSE),}\StringTok{"("}\NormalTok{,percent\_FALSE,}\StringTok{"\%)"}\NormalTok{))}\SpecialCharTok{+}
  \FunctionTok{labs}\NormalTok{(}\AttributeTok{x=}\FunctionTok{paste}\NormalTok{(}\StringTok{"PC 1 ("}\NormalTok{,}\FunctionTok{round}\NormalTok{(var\_explained[}\DecValTok{1}\NormalTok{],}\DecValTok{1}\NormalTok{),}\StringTok{"\%)"}\NormalTok{),}
       \AttributeTok{y=}\FunctionTok{paste}\NormalTok{(}\StringTok{"PC 2 ("}\NormalTok{,}\FunctionTok{round}\NormalTok{(var\_explained[}\DecValTok{2}\NormalTok{],}\DecValTok{1}\NormalTok{),}\StringTok{"\%)"}\NormalTok{),}
       \AttributeTok{title =} \StringTok{"Group = FALSE"}\NormalTok{, }\AttributeTok{color=}\StringTok{""}\NormalTok{)}
\end{Highlighting}
\end{Shaded}

\includegraphics{20_beta_diversity_files/figure-latex/unnamed-chunk-8-2.png}

\hypertarget{testing-differences-in-community-composition-between-sample-groups}{%
\subsection{Testing differences in community composition between sample groups}\label{testing-differences-in-community-composition-between-sample-groups}}

The permutational analysis of variance (PERMANOVA) \citep{Anderson2001} is
a widely used non-parametric multivariate method that can be used to
estimate the actual statistical significance of differences in the
observed community composition between two groups of
samples.

PERMANOVA evaluates the hypothesis that the centroids and dispersion
of the community are equivalent between the compared groups. A small
p-value indicates that the compared groups have, on average, a
different community composition.

This method is implemented in the \texttt{vegan} package in the function
\href{https://www.rdocumentation.org/packages/vegan/versions/2.4-2/topics/adonis}{\texttt{adonis2}}.

\textbf{Note:}

It is recommended to \texttt{by\ =\ "margin"}. It specifies that each variable's marginal
effect is analyzed individually.

When \texttt{by\ =\ "terms"} (the default) the order of variables matters;
each variable is analyzed sequentially, and the result is different when more than 1 variable is
introduced and their order is differs. (Check \href{https://microbiome.github.io/OMA/extras.html\#permanova-comparison}{comparison})

We can perform PERMANOVA with \texttt{adonis2} function or by first performing distance-based
redundancy analysis (dbRDA), and then applying permutational test for result of
redundancy analysis. Advantage of the latter approach is that by doing so we can get
coefficients: how much each taxa affect to the result.

\begin{Shaded}
\begin{Highlighting}[]
\FunctionTok{library}\NormalTok{(vegan)}
\CommentTok{\# Agglomerate data to Species level}
\NormalTok{tse }\OtherTok{\textless{}{-}} \FunctionTok{agglomerateByRank}\NormalTok{(tse, }\AttributeTok{rank =} \StringTok{"Species"}\NormalTok{)}

\CommentTok{\# Set seed for reproducibility}
\FunctionTok{set.seed}\NormalTok{(}\DecValTok{1576}\NormalTok{)}
\CommentTok{\# We choose 99 random permutations. Consider applying more (999 or 9999) in your}
\CommentTok{\# analysis. }
\NormalTok{permanova }\OtherTok{\textless{}{-}} \FunctionTok{adonis2}\NormalTok{(}\FunctionTok{t}\NormalTok{(}\FunctionTok{assay}\NormalTok{(tse,}\StringTok{"relabundance"}\NormalTok{)) }\SpecialCharTok{\textasciitilde{}}\NormalTok{ Group,}
                     \AttributeTok{by =} \StringTok{"margin"}\NormalTok{, }\CommentTok{\# each term (here only \textquotesingle{}Group\textquotesingle{}) analyzed individually}
                     \AttributeTok{data =} \FunctionTok{colData}\NormalTok{(tse),}
                     \AttributeTok{method =} \StringTok{"euclidean"}\NormalTok{,}
                     \AttributeTok{permutations =} \DecValTok{99}\NormalTok{)}

\CommentTok{\# Set seed for reproducibility}
\FunctionTok{set.seed}\NormalTok{(}\DecValTok{1576}\NormalTok{)}
\CommentTok{\# Perform dbRDA}
\NormalTok{dbrda }\OtherTok{\textless{}{-}} \FunctionTok{dbrda}\NormalTok{(}\FunctionTok{t}\NormalTok{(}\FunctionTok{assay}\NormalTok{(tse,}\StringTok{"relabundance"}\NormalTok{)) }\SpecialCharTok{\textasciitilde{}}\NormalTok{ Group, }
               \AttributeTok{data =} \FunctionTok{colData}\NormalTok{(tse))}
\CommentTok{\# Perform permutational analysis}
\NormalTok{permanova2 }\OtherTok{\textless{}{-}} \FunctionTok{anova.cca}\NormalTok{(dbrda,}
                        \AttributeTok{by =} \StringTok{"margin"}\NormalTok{, }\CommentTok{\# each term (here only \textquotesingle{}Group\textquotesingle{}) analyzed individually}
                        \AttributeTok{method =} \StringTok{"euclidean"}\NormalTok{,}
                        \AttributeTok{permutations =} \DecValTok{99}\NormalTok{)}

\CommentTok{\# Get p{-}values}
\NormalTok{p\_values }\OtherTok{\textless{}{-}} \FunctionTok{c}\NormalTok{( permanova[}\StringTok{"Group"}\NormalTok{, }\StringTok{"Pr(\textgreater{}F)"}\NormalTok{], permanova2[}\StringTok{"Group"}\NormalTok{, }\StringTok{"Pr(\textgreater{}F)"}\NormalTok{] )}
\NormalTok{p\_values }\OtherTok{\textless{}{-}}\FunctionTok{as.data.frame}\NormalTok{(p\_values)}
\FunctionTok{rownames}\NormalTok{(p\_values) }\OtherTok{\textless{}{-}} \FunctionTok{c}\NormalTok{(}\StringTok{"adonis2"}\NormalTok{, }\StringTok{"dbRDA+anova.cca"}\NormalTok{)}
\NormalTok{p\_values}
\end{Highlighting}
\end{Shaded}

\begin{verbatim}
##                 p_values
## adonis2             0.02
## dbRDA+anova.cca     0.02
\end{verbatim}

As we can see, the community composition is significantly different
between the groups (p \textless{} 0.05), and these two methods give equal p-values.

Let us visualize the model coefficients for species that exhibit the
largest differences between the groups. This gives some insights into
how the groups tend to differ from each other in terms of community
composition.

\begin{Shaded}
\begin{Highlighting}[]
\CommentTok{\# Add taxa info}
\FunctionTok{sppscores}\NormalTok{(dbrda) }\OtherTok{\textless{}{-}} \FunctionTok{t}\NormalTok{(}\FunctionTok{assay}\NormalTok{(tse,}\StringTok{"relabundance"}\NormalTok{))}
\CommentTok{\# Get coefficients}
\NormalTok{coef }\OtherTok{\textless{}{-}}\NormalTok{ dbrda}\SpecialCharTok{$}\NormalTok{CCA}\SpecialCharTok{$}\NormalTok{v}
\CommentTok{\# Get the taxa with biggest weights}
\NormalTok{top.coef }\OtherTok{\textless{}{-}} \FunctionTok{head}\NormalTok{( coef[}\FunctionTok{rev}\NormalTok{(}\FunctionTok{order}\NormalTok{(}\FunctionTok{abs}\NormalTok{(coef))), , }\AttributeTok{drop =} \ConstantTok{FALSE}\NormalTok{], }\DecValTok{20}\NormalTok{)}
\CommentTok{\# Sort weights in increasing order}
\NormalTok{top.coef }\OtherTok{\textless{}{-}}\NormalTok{ top.coef[ }\FunctionTok{order}\NormalTok{(top.coef), ]}
\CommentTok{\# Get top names}
\NormalTok{top\_names }\OtherTok{\textless{}{-}} \FunctionTok{names}\NormalTok{(top.coef)[ }\FunctionTok{order}\NormalTok{(}\FunctionTok{abs}\NormalTok{(top.coef), }\AttributeTok{decreasing =} \ConstantTok{TRUE}\NormalTok{) ]}
\end{Highlighting}
\end{Shaded}

\begin{Shaded}
\begin{Highlighting}[]
\FunctionTok{ggplot}\NormalTok{(}\FunctionTok{data.frame}\NormalTok{(}\AttributeTok{x =}\NormalTok{ top.coef,}
                  \AttributeTok{y =} \FunctionTok{factor}\NormalTok{(}\FunctionTok{names}\NormalTok{(top.coef),}
                                      \FunctionTok{unique}\NormalTok{(}\FunctionTok{names}\NormalTok{(top.coef)))),}
        \FunctionTok{aes}\NormalTok{(}\AttributeTok{x =}\NormalTok{ x, }\AttributeTok{y =}\NormalTok{ y)) }\SpecialCharTok{+}
    \FunctionTok{geom\_bar}\NormalTok{(}\AttributeTok{stat=}\StringTok{"identity"}\NormalTok{) }\SpecialCharTok{+}
    \FunctionTok{labs}\NormalTok{(}\AttributeTok{x=}\StringTok{""}\NormalTok{,}\AttributeTok{y=}\StringTok{""}\NormalTok{,}\AttributeTok{title=}\StringTok{"Top Taxa"}\NormalTok{) }\SpecialCharTok{+}
    \FunctionTok{theme\_bw}\NormalTok{()}
\end{Highlighting}
\end{Shaded}

\includegraphics{20_beta_diversity_files/figure-latex/plot-top-coef-anova-1.png}

In the above example, the largest differences between the two groups
can be attributed to \emph{Genus:Bacteroides} (elevated in the first
group) and \emph{Family:Ruminococcaceae} (elevated in the second
group), and many other co-varying species.

\hypertarget{checking-the-homogeneity-condition}{%
\subsection{Checking the homogeneity condition}\label{checking-the-homogeneity-condition}}

It is important to note that the application of PERMANOVA assumes
homogeneous group dispersions (variances). This can be tested with the
PERMDISP2 method \citep{Anderson2006} by using the same assay and distance
method than in PERMANOVA.

\begin{Shaded}
\begin{Highlighting}[]
\FunctionTok{anova}\NormalTok{( }\FunctionTok{betadisper}\NormalTok{(}\FunctionTok{vegdist}\NormalTok{(}\FunctionTok{t}\NormalTok{(}\FunctionTok{assay}\NormalTok{(tse, }\StringTok{"counts"}\NormalTok{))), }\FunctionTok{colData}\NormalTok{(tse)}\SpecialCharTok{$}\NormalTok{Group) )}
\end{Highlighting}
\end{Shaded}

\begin{verbatim}
## Analysis of Variance Table
## 
## Response: Distances
##           Df Sum Sq Mean Sq F value  Pr(>F)    
## Groups     1 0.2385  0.2385     103 3.6e-10 ***
## Residuals 24 0.0554  0.0023                    
## ---
## Signif. codes:  0 '***' 0.001 '**' 0.01 '*' 0.05 '.' 0.1 ' ' 1
\end{verbatim}

If the groups have similar dispersion, PERMANOVA can be seen as an
appropriate choice for comparing community compositions.

\hypertarget{further-reading}{%
\section{Further reading}\label{further-reading}}

\begin{itemize}
\item
  \href{http://bioconductor.org/books/release/OSCA/clustering.html}{How to extract information from clusters}
\item
  Chapter \ref{clustering} on community typing
\end{itemize}

\hypertarget{microbiome-community}{%
\chapter{Community composition}\label{microbiome-community}}

\begin{Shaded}
\begin{Highlighting}[]
\FunctionTok{library}\NormalTok{(mia)}
\FunctionTok{data}\NormalTok{(}\StringTok{"GlobalPatterns"}\NormalTok{, }\AttributeTok{package=}\StringTok{"mia"}\NormalTok{)}
\NormalTok{tse }\OtherTok{\textless{}{-}}\NormalTok{ GlobalPatterns}
\end{Highlighting}
\end{Shaded}

\hypertarget{visual-composition}{%
\section{Visualizing taxonomic composition}\label{visual-composition}}

\hypertarget{composition-barplot}{%
\subsection{Composition barplot}\label{composition-barplot}}

A typical way to visualize microbiome composition is by using
composition barplot. In the following, relative abundance is
calculated and top taxa are retrieved for the Phylum rank. Thereafter,
the barplot is visualized ordering rank by abundance values and
samples by ``Bacteroidetes'':

\begin{Shaded}
\begin{Highlighting}[]
\FunctionTok{library}\NormalTok{(miaViz)}
\CommentTok{\# Computing relative abundance}
\NormalTok{tse }\OtherTok{\textless{}{-}} \FunctionTok{relAbundanceCounts}\NormalTok{(tse)}

\CommentTok{\# Getting top taxa on a Phylum level}
\NormalTok{tse\_phylum }\OtherTok{\textless{}{-}} \FunctionTok{agglomerateByRank}\NormalTok{(tse, }\AttributeTok{rank =}\StringTok{"Phylum"}\NormalTok{, }\AttributeTok{onRankOnly=}\ConstantTok{TRUE}\NormalTok{)}
\NormalTok{top\_taxa }\OtherTok{\textless{}{-}} \FunctionTok{getTopTaxa}\NormalTok{(tse\_phylum,}\AttributeTok{top =} \DecValTok{5}\NormalTok{, }\AttributeTok{assay.type =} \StringTok{"relabundance"}\NormalTok{)}

\CommentTok{\# Renaming the "Phylum" rank to keep only top taxa and the rest to "Other"}
\NormalTok{phylum\_renamed }\OtherTok{\textless{}{-}} \FunctionTok{lapply}\NormalTok{(}\FunctionTok{rowData}\NormalTok{(tse)}\SpecialCharTok{$}\NormalTok{Phylum,}
                   \ControlFlowTok{function}\NormalTok{(x)\{}\ControlFlowTok{if}\NormalTok{ (x }\SpecialCharTok{\%in\%}\NormalTok{ top\_taxa) \{x\} }\ControlFlowTok{else}\NormalTok{ \{}\StringTok{"Other"}\NormalTok{\}\})}
\FunctionTok{rowData}\NormalTok{(tse)}\SpecialCharTok{$}\NormalTok{Phylum }\OtherTok{\textless{}{-}} \FunctionTok{as.character}\NormalTok{(phylum\_renamed)}

\CommentTok{\# Visualizing the composition barplot, with samples order by "Bacteroidetes"}
\FunctionTok{plotAbundance}\NormalTok{(tse, }\AttributeTok{assay.type=}\StringTok{"relabundance"}\NormalTok{, }\AttributeTok{rank =} \StringTok{"Phylum"}\NormalTok{,}
              \AttributeTok{order\_rank\_by=}\StringTok{"abund"}\NormalTok{, }
              \AttributeTok{order\_sample\_by =} \StringTok{"Bacteroidetes"}\NormalTok{)}
\end{Highlighting}
\end{Shaded}

\includegraphics{21_microbiome_community_files/figure-latex/unnamed-chunk-1-1.pdf}

\hypertarget{composition-heatmap}{%
\subsection{Composition heatmap}\label{composition-heatmap}}

Community composition can be visualized with heatmap, where the
horizontal axis represents samples and the vertical axis the
taxa. Color of each intersection point represents abundance of a taxon
in a specific sample.

Here, abundances are first CLR (centered log-ratio) transformed to
remove compositionality bias. Then Z transformation is applied to
CLR-transformed data. This shifts all taxa to zero mean and unit
variance, allowing visual comparison between taxa that have different
absolute abundance levels. After these rough visual exploration
techniques, we can visualize the abundances at Phylum level.

\begin{Shaded}
\begin{Highlighting}[]
\FunctionTok{library}\NormalTok{(ggplot2)}

\CommentTok{\# Add clr{-}transformation on samples}
\FunctionTok{assay}\NormalTok{(tse\_phylum, }\StringTok{"pseudo"}\NormalTok{) }\OtherTok{\textless{}{-}} \FunctionTok{assay}\NormalTok{(tse\_phylum, }\StringTok{"counts"}\NormalTok{) }\SpecialCharTok{+} \DecValTok{1}
\NormalTok{tse\_phylum }\OtherTok{\textless{}{-}} \FunctionTok{transformCounts}\NormalTok{(tse\_phylum, }\AttributeTok{assay.type =} \StringTok{"pseudo"}\NormalTok{,}
                              \AttributeTok{method =} \StringTok{"relabundance"}\NormalTok{)}

\NormalTok{tse\_phylum }\OtherTok{\textless{}{-}} \FunctionTok{transformCounts}\NormalTok{(tse\_phylum,}
                  \AttributeTok{assay.type =} \StringTok{"relabundance"}\NormalTok{,}
          \AttributeTok{method =} \StringTok{"clr"}\NormalTok{)}

\CommentTok{\# Add z{-}transformation on features (taxa)}
\NormalTok{tse\_phylum }\OtherTok{\textless{}{-}} \FunctionTok{transformCounts}\NormalTok{(tse\_phylum, }\AttributeTok{assay.type =} \StringTok{"clr"}\NormalTok{, }
                              \AttributeTok{MARGIN =} \StringTok{"features"}\NormalTok{,}
                              \AttributeTok{method =} \StringTok{"z"}\NormalTok{, }\AttributeTok{name =} \StringTok{"clr\_z"}\NormalTok{)}
\end{Highlighting}
\end{Shaded}

Visualize as heatmap.

\begin{Shaded}
\begin{Highlighting}[]
\CommentTok{\# Melt the assay for plotting purposes}
\NormalTok{df }\OtherTok{\textless{}{-}} \FunctionTok{meltAssay}\NormalTok{(tse\_phylum, }\AttributeTok{assay.type =} \StringTok{"clr\_z"}\NormalTok{)}

\CommentTok{\# Determines the scaling of colours}
\NormalTok{maxval }\OtherTok{\textless{}{-}} \FunctionTok{round}\NormalTok{(}\FunctionTok{max}\NormalTok{(}\FunctionTok{abs}\NormalTok{(df}\SpecialCharTok{$}\NormalTok{clr\_z)))}
\NormalTok{limits }\OtherTok{\textless{}{-}} \FunctionTok{c}\NormalTok{(}\SpecialCharTok{{-}}\NormalTok{maxval, maxval)}
\NormalTok{breaks }\OtherTok{\textless{}{-}} \FunctionTok{seq}\NormalTok{(}\AttributeTok{from =} \FunctionTok{min}\NormalTok{(limits), }\AttributeTok{to =} \FunctionTok{max}\NormalTok{(limits), }\AttributeTok{by =} \FloatTok{0.5}\NormalTok{)}
\NormalTok{colours }\OtherTok{\textless{}{-}} \FunctionTok{c}\NormalTok{(}\StringTok{"darkblue"}\NormalTok{, }\StringTok{"blue"}\NormalTok{, }\StringTok{"white"}\NormalTok{, }\StringTok{"red"}\NormalTok{, }\StringTok{"darkred"}\NormalTok{)}

\CommentTok{\# Creates a ggplot object}
\FunctionTok{ggplot}\NormalTok{(df, }\FunctionTok{aes}\NormalTok{(}\AttributeTok{x =}\NormalTok{ SampleID, }\AttributeTok{y =}\NormalTok{ FeatureID, }\AttributeTok{fill =}\NormalTok{ clr\_z)) }\SpecialCharTok{+}
  \FunctionTok{geom\_tile}\NormalTok{() }\SpecialCharTok{+}
  \FunctionTok{scale\_fill\_gradientn}\NormalTok{(}\AttributeTok{name =} \StringTok{"CLR + Z transform"}\NormalTok{, }
                       \AttributeTok{breaks =}\NormalTok{ breaks, }\AttributeTok{limits =}\NormalTok{ limits, }\AttributeTok{colours =}\NormalTok{ colours) }\SpecialCharTok{+} 
  \FunctionTok{theme}\NormalTok{(}\AttributeTok{text =} \FunctionTok{element\_text}\NormalTok{(}\AttributeTok{size=}\DecValTok{10}\NormalTok{),}
        \AttributeTok{axis.text.x =} \FunctionTok{element\_text}\NormalTok{(}\AttributeTok{angle=}\DecValTok{45}\NormalTok{, }\AttributeTok{hjust=}\DecValTok{1}\NormalTok{),}
        \AttributeTok{legend.key.size =} \FunctionTok{unit}\NormalTok{(}\DecValTok{1}\NormalTok{, }\StringTok{"cm"}\NormalTok{)) }\SpecialCharTok{+}
  \FunctionTok{labs}\NormalTok{(}\AttributeTok{x =} \StringTok{"Samples"}\NormalTok{, }\AttributeTok{y =} \StringTok{"Taxa"}\NormalTok{)}
\end{Highlighting}
\end{Shaded}

\includegraphics{21_microbiome_community_files/figure-latex/heatmapvisu-1.pdf}

\emph{pheatmap} is a package that provides methods to plot clustered heatmaps.

\begin{Shaded}
\begin{Highlighting}[]
\FunctionTok{library}\NormalTok{(pheatmap)}

\CommentTok{\# Takes subset: only samples from feces, skin, or tongue}
\NormalTok{tse\_phylum\_subset }\OtherTok{\textless{}{-}}\NormalTok{ tse\_phylum[ , }\FunctionTok{colData}\NormalTok{(tse\_phylum)}\SpecialCharTok{$}\NormalTok{SampleType }\SpecialCharTok{\%in\%} \FunctionTok{c}\NormalTok{(}\StringTok{"Feces"}\NormalTok{, }\StringTok{"Skin"}\NormalTok{, }\StringTok{"Tongue"}\NormalTok{) ]}

\CommentTok{\# Add clr{-}transformation}
\NormalTok{tse\_phylum\_subset }\OtherTok{\textless{}{-}} \FunctionTok{transformCounts}\NormalTok{(tse\_phylum\_subset,}
                         \AttributeTok{method =} \StringTok{"clr"}\NormalTok{,}
                 \AttributeTok{pseudocount =} \DecValTok{1}\NormalTok{)}

\NormalTok{tse\_phylum\_subset }\OtherTok{\textless{}{-}} \FunctionTok{transformCounts}\NormalTok{(tse\_phylum\_subset, }\AttributeTok{assay.type =} \StringTok{"clr"}\NormalTok{,}
                                     \AttributeTok{MARGIN =} \StringTok{"features"}\NormalTok{, }
                                     \AttributeTok{method =} \StringTok{"z"}\NormalTok{, }\AttributeTok{name =} \StringTok{"clr\_z"}\NormalTok{)}

\CommentTok{\# Get n most abundant taxa, and subsets the data by them}
\NormalTok{top\_taxa }\OtherTok{\textless{}{-}} \FunctionTok{getTopTaxa}\NormalTok{(tse\_phylum\_subset, }\AttributeTok{top =} \DecValTok{20}\NormalTok{)}
\NormalTok{tse\_phylum\_subset }\OtherTok{\textless{}{-}}\NormalTok{ tse\_phylum\_subset[top\_taxa, ]}

\CommentTok{\# Gets the assay table}
\NormalTok{mat }\OtherTok{\textless{}{-}} \FunctionTok{assay}\NormalTok{(tse\_phylum\_subset, }\StringTok{"clr\_z"}\NormalTok{)}

\CommentTok{\# Creates the heatmap}
\FunctionTok{pheatmap}\NormalTok{(mat)}
\end{Highlighting}
\end{Shaded}

\includegraphics{21_microbiome_community_files/figure-latex/pheatmap1-1.pdf}

We can create clusters by hierarchical clustering and add them to the plot.

\begin{Shaded}
\begin{Highlighting}[]
\FunctionTok{library}\NormalTok{(ape)}

\CommentTok{\# Hierarchical clustering}
\NormalTok{taxa\_hclust }\OtherTok{\textless{}{-}} \FunctionTok{hclust}\NormalTok{(}\FunctionTok{dist}\NormalTok{(mat), }\AttributeTok{method =} \StringTok{"complete"}\NormalTok{)}

\CommentTok{\# Creates a phylogenetic tree}
\NormalTok{taxa\_tree }\OtherTok{\textless{}{-}} \FunctionTok{as.phylo}\NormalTok{(taxa\_hclust)}
\end{Highlighting}
\end{Shaded}

\begin{Shaded}
\begin{Highlighting}[]
\FunctionTok{library}\NormalTok{(ggtree)}

\CommentTok{\# Plot taxa tree}
\NormalTok{taxa\_tree }\OtherTok{\textless{}{-}} \FunctionTok{ggtree}\NormalTok{(taxa\_tree) }\SpecialCharTok{+} 
  \FunctionTok{theme}\NormalTok{(}\AttributeTok{plot.margin=}\FunctionTok{margin}\NormalTok{(}\DecValTok{0}\NormalTok{,}\DecValTok{0}\NormalTok{,}\DecValTok{0}\NormalTok{,}\DecValTok{0}\NormalTok{)) }\CommentTok{\# removes margins}

\CommentTok{\# Get order of taxa in plot}
\NormalTok{taxa\_ordered }\OtherTok{\textless{}{-}} \FunctionTok{get\_taxa\_name}\NormalTok{(taxa\_tree)}

\NormalTok{taxa\_tree}
\end{Highlighting}
\end{Shaded}

\includegraphics{21_microbiome_community_files/figure-latex/pheatmap3-1.pdf}

Based on phylo tree, we decide to create three clusters.

\begin{Shaded}
\begin{Highlighting}[]
\CommentTok{\# Creates clusters}
\NormalTok{taxa\_clusters }\OtherTok{\textless{}{-}} \FunctionTok{cutree}\NormalTok{(}\AttributeTok{tree =}\NormalTok{ taxa\_hclust, }\AttributeTok{k =} \DecValTok{3}\NormalTok{)}

\CommentTok{\# Converts into data frame}
\NormalTok{taxa\_clusters }\OtherTok{\textless{}{-}} \FunctionTok{data.frame}\NormalTok{(}\AttributeTok{clusters =}\NormalTok{ taxa\_clusters)}
\NormalTok{taxa\_clusters}\SpecialCharTok{$}\NormalTok{clusters }\OtherTok{\textless{}{-}} \FunctionTok{factor}\NormalTok{(taxa\_clusters}\SpecialCharTok{$}\NormalTok{clusters)}

\CommentTok{\# Order data so that it\textquotesingle{}s same as in phylo tree}
\NormalTok{taxa\_clusters }\OtherTok{\textless{}{-}}\NormalTok{ taxa\_clusters[taxa\_ordered, , drop }\OtherTok{=} \ConstantTok{FALSE}\NormalTok{] }

\CommentTok{\# Prints taxa and their clusters}
\NormalTok{taxa\_clusters}
\end{Highlighting}
\end{Shaded}

\begin{verbatim}
##                  clusters
## Chloroflexi             3
## Actinobacteria          3
## Crenarchaeota           3
## Planctomycetes          3
## Gemmatimonadetes        3
## Thermi                  3
## Acidobacteria           3
## Spirochaetes            2
## Fusobacteria            2
## SR1                     2
## Cyanobacteria           2
## Proteobacteria          2
## Synergistetes           2
## Lentisphaerae           1
## Bacteroidetes           1
## Verrucomicrobia         1
## Tenericutes             1
## Firmicutes              1
## Euryarchaeota           1
## SAR406                  1
\end{verbatim}

\begin{Shaded}
\begin{Highlighting}[]
\CommentTok{\# Adds information to rowData}
\FunctionTok{rowData}\NormalTok{(tse\_phylum\_subset)}\SpecialCharTok{$}\NormalTok{clusters }\OtherTok{\textless{}{-}}\NormalTok{ taxa\_clusters[}\FunctionTok{order}\NormalTok{(}\FunctionTok{match}\NormalTok{(}\FunctionTok{rownames}\NormalTok{(taxa\_clusters), }\FunctionTok{rownames}\NormalTok{(tse\_phylum\_subset))), ]}

\CommentTok{\# Prints taxa and their clusters}
\FunctionTok{rowData}\NormalTok{(tse\_phylum\_subset)}\SpecialCharTok{$}\NormalTok{clusters}
\end{Highlighting}
\end{Shaded}

\begin{verbatim}
##  [1] 1 1 2 3 2 2 1 1 1 1 3 2 3 3 3 2 2 3 3 1
## Levels: 1 2 3
\end{verbatim}

\begin{Shaded}
\begin{Highlighting}[]
\CommentTok{\# Hierarchical clustering}
\NormalTok{sample\_hclust }\OtherTok{\textless{}{-}} \FunctionTok{hclust}\NormalTok{(}\FunctionTok{dist}\NormalTok{(}\FunctionTok{t}\NormalTok{(mat)), }\AttributeTok{method =} \StringTok{"complete"}\NormalTok{)}

\CommentTok{\# Creates a phylogenetic tree}
\NormalTok{sample\_tree }\OtherTok{\textless{}{-}} \FunctionTok{as.phylo}\NormalTok{(sample\_hclust)}

\CommentTok{\# Plot sample tree}
\NormalTok{sample\_tree }\OtherTok{\textless{}{-}} \FunctionTok{ggtree}\NormalTok{(sample\_tree) }\SpecialCharTok{+} \FunctionTok{layout\_dendrogram}\NormalTok{() }\SpecialCharTok{+} 
  \FunctionTok{theme}\NormalTok{(}\AttributeTok{plot.margin=}\FunctionTok{margin}\NormalTok{(}\DecValTok{0}\NormalTok{,}\DecValTok{0}\NormalTok{,}\DecValTok{0}\NormalTok{,}\DecValTok{0}\NormalTok{)) }\CommentTok{\# removes margins}

\CommentTok{\# Get order of samples in plot}
\NormalTok{samples\_ordered }\OtherTok{\textless{}{-}} \FunctionTok{rev}\NormalTok{(}\FunctionTok{get\_taxa\_name}\NormalTok{(sample\_tree))}

\NormalTok{sample\_tree}
\end{Highlighting}
\end{Shaded}

\includegraphics{21_microbiome_community_files/figure-latex/pheatmap6-1.pdf}

\begin{Shaded}
\begin{Highlighting}[]
\CommentTok{\# Creates clusters}
\NormalTok{sample\_clusters }\OtherTok{\textless{}{-}} \FunctionTok{factor}\NormalTok{(}\FunctionTok{cutree}\NormalTok{(}\AttributeTok{tree =}\NormalTok{ sample\_hclust, }\AttributeTok{k =} \DecValTok{3}\NormalTok{))}

\CommentTok{\# Converts into data frame}
\NormalTok{sample\_data }\OtherTok{\textless{}{-}} \FunctionTok{data.frame}\NormalTok{(}\AttributeTok{clusters =}\NormalTok{ sample\_clusters)}

\CommentTok{\# Order data so that it\textquotesingle{}s same as in phylo tree}
\NormalTok{sample\_data }\OtherTok{\textless{}{-}}\NormalTok{ sample\_data[samples\_ordered, , drop }\OtherTok{=} \ConstantTok{FALSE}\NormalTok{] }

\CommentTok{\# Order data based on }
\NormalTok{tse\_phylum\_subset }\OtherTok{\textless{}{-}}\NormalTok{ tse\_phylum\_subset[ , }\FunctionTok{rownames}\NormalTok{(sample\_data)]}

\CommentTok{\# Add sample type data}
\NormalTok{sample\_data}\SpecialCharTok{$}\NormalTok{sample\_types }\OtherTok{\textless{}{-}} \FunctionTok{unfactor}\NormalTok{(}\FunctionTok{colData}\NormalTok{(tse\_phylum\_subset)}\SpecialCharTok{$}\NormalTok{SampleType)}

\NormalTok{sample\_data}
\end{Highlighting}
\end{Shaded}

\begin{verbatim}
##         clusters sample_types
## M11Plmr        2         Skin
## M31Plmr        2         Skin
## F21Plmr        2         Skin
## M31Fcsw        1        Feces
## M11Fcsw        1        Feces
## TS28           3        Feces
## TS29           3        Feces
## M31Tong        3       Tongue
## M11Tong        3       Tongue
\end{verbatim}

Now we can create heatmap with additional annotations.

\begin{Shaded}
\begin{Highlighting}[]
\CommentTok{\# Determines the scaling of colorss}
\CommentTok{\# Scale colors}
\NormalTok{breaks }\OtherTok{\textless{}{-}} \FunctionTok{seq}\NormalTok{(}\SpecialCharTok{{-}}\FunctionTok{ceiling}\NormalTok{(}\FunctionTok{max}\NormalTok{(}\FunctionTok{abs}\NormalTok{(mat))), }\FunctionTok{ceiling}\NormalTok{(}\FunctionTok{max}\NormalTok{(}\FunctionTok{abs}\NormalTok{(mat))), }
              \AttributeTok{length.out =} \FunctionTok{ifelse}\NormalTok{( }\FunctionTok{max}\NormalTok{(}\FunctionTok{abs}\NormalTok{(mat))}\SpecialCharTok{\textgreater{}}\DecValTok{5}\NormalTok{, }\DecValTok{2}\SpecialCharTok{*}\FunctionTok{ceiling}\NormalTok{(}\FunctionTok{max}\NormalTok{(}\FunctionTok{abs}\NormalTok{(mat))), }\DecValTok{10}\NormalTok{ ) )}
\NormalTok{colors }\OtherTok{\textless{}{-}} \FunctionTok{colorRampPalette}\NormalTok{(}\FunctionTok{c}\NormalTok{(}\StringTok{"darkblue"}\NormalTok{, }\StringTok{"blue"}\NormalTok{, }\StringTok{"white"}\NormalTok{, }\StringTok{"red"}\NormalTok{, }\StringTok{"darkred"}\NormalTok{))(}\FunctionTok{length}\NormalTok{(breaks)}\SpecialCharTok{{-}}\DecValTok{1}\NormalTok{)}

\FunctionTok{pheatmap}\NormalTok{(mat, }\AttributeTok{annotation\_row =}\NormalTok{ taxa\_clusters, }
         \AttributeTok{annotation\_col =}\NormalTok{ sample\_data,}
         \AttributeTok{breaks =}\NormalTok{ breaks,}
         \AttributeTok{color =}\NormalTok{ colors)}
\end{Highlighting}
\end{Shaded}

\includegraphics{21_microbiome_community_files/figure-latex/pheatmap8-1.pdf}

In addition, there are also other packages that provide functions for
more complex heatmaps, such as
\href{https://docs.ropensci.org/iheatmapr/articles/full_vignettes/iheatmapr.html}{\emph{iheatmapr}}
and ComplexHeatmap \citep{ComplexHeatmap}.
\href{http://www.bioconductor.org/packages/release/bioc/vignettes/sechm/inst/doc/sechm.html}{sechm}
package provides wrapper for \emph{ComplexHeatmap} and its usage is
explained in chapter \ref{viz-chapter} along with the \texttt{pheatmap}
package for clustered heatmaps.

\hypertarget{clustering}{%
\chapter{Community typing (clustering)}\label{clustering}}

\begin{Shaded}
\begin{Highlighting}[]
\FunctionTok{library}\NormalTok{(mia)}
\FunctionTok{data}\NormalTok{(}\StringTok{"GlobalPatterns"}\NormalTok{, }\AttributeTok{package =} \StringTok{"mia"}\NormalTok{)}
\FunctionTok{data}\NormalTok{(}\StringTok{"peerj13075"}\NormalTok{, }\AttributeTok{package =} \StringTok{"mia"}\NormalTok{)}
\NormalTok{tse }\OtherTok{\textless{}{-}}\NormalTok{ peerj13075}
\end{Highlighting}
\end{Shaded}

Clustering is an unsupervised machine learning technique. The idea of
it is to find clusters from the data. A cluster is a group of
features/samples that share pattern. For example, with clustering, we
can find group of samples that share similar community
composition. There are multiple clustering algorithms available.

\hypertarget{introduction-to-bluster-package}{%
\section{Introduction to bluster package}\label{introduction-to-bluster-package}}

\emph{bluster} is a Bioconductor package specialized in clustering. It offers
multiple algorithms such as hierarchical clustering, DBSCAN, K-means,
amongst others. The first thing to do when using this package is to load
it, and transform the data if necessary, depending on your analysis goals.

\begin{Shaded}
\begin{Highlighting}[]
\CommentTok{\# Load dependencies}
\FunctionTok{library}\NormalTok{(bluster)}

\CommentTok{\# Apply transformation}
\NormalTok{tse }\OtherTok{\textless{}{-}} \FunctionTok{transformCounts}\NormalTok{(tse, }\AttributeTok{method =} \StringTok{"relabundance"}\NormalTok{)}
\end{Highlighting}
\end{Shaded}

The clustering can be done on features or samples.

\hypertarget{sample-clustering}{%
\subsection{Sample clustering}\label{sample-clustering}}

To cluster samples, we simply need to transpose the assay so that
the samples are in rows.

\begin{Shaded}
\begin{Highlighting}[]
\NormalTok{x }\OtherTok{\textless{}{-}} \FunctionTok{t}\NormalTok{(}\FunctionTok{assay}\NormalTok{(tse, }\StringTok{"relabundance"}\NormalTok{))}
\end{Highlighting}
\end{Shaded}

Then, to perform clustering, use the \texttt{clusterRows} function. Depending
on its parameters, it will perform a different algorithm. Here, we'll
perform a hierarchical clustering.

\begin{Shaded}
\begin{Highlighting}[]
\CommentTok{\# Simple use of the hierarchical clustering which sets the cut height to}
\CommentTok{\# half the dendrogram height.}
\NormalTok{hclust.out }\OtherTok{\textless{}{-}} \FunctionTok{clusterRows}\NormalTok{(x, }\FunctionTok{HclustParam}\NormalTok{())}

\CommentTok{\# Add cluster indices to the TSE}
\FunctionTok{colData}\NormalTok{(tse)}\SpecialCharTok{$}\NormalTok{clusters }\OtherTok{\textless{}{-}}\NormalTok{ hclust.out}

\CommentTok{\# Checking the result}
\NormalTok{hclust.out}
\end{Highlighting}
\end{Shaded}

\begin{verbatim}
##  ID1  ID2  ID3  ID4  ID5  ID6  ID7  ID8  ID9 ID10 ID11 ID12 ID13 ID14 ID15 ID16 
##    1    2    3    4    5    5    2    2    6    7    5    5    6    5    6    5 
## ID17 ID18 ID19 ID20 ID21 ID22 ID23 ID24 ID25 ID26 ID27 ID28 ID29 ID30 ID31 ID32 
##    5    8    5    5    9   10   11    5    3    1   12   10   10    3    1    6 
## ID33 ID34 ID35 ID36 ID37 ID38 ID39 ID40 ID41 ID42 ID43 ID44 ID45 ID46 ID47 ID48 
##    2    2    6    2    6   11   11   13    7    5    4    6   11    1   11    2 
## ID49 ID50 ID51 ID52 ID53 ID54 ID55 ID56 ID57 ID58 
##    8    8    6   10    9   14   15    3   10   16 
## Levels: 1 2 3 4 5 6 7 8 9 10 11 12 13 14 15 16
\end{verbatim}

Once the clustering on the samples is done, we can also plot the clusters.

\begin{Shaded}
\begin{Highlighting}[]
\FunctionTok{library}\NormalTok{(scater)}
\CommentTok{\# Add the MDS dimensions for plotting}
\NormalTok{tse }\OtherTok{\textless{}{-}} \FunctionTok{runMDS}\NormalTok{(tse,}
  \AttributeTok{assay.type =} \StringTok{"relabundance"}\NormalTok{,}
  \AttributeTok{FUN =}\NormalTok{ vegan}\SpecialCharTok{::}\NormalTok{vegdist,}
  \AttributeTok{method =} \StringTok{"bray"}
\NormalTok{)}

\CommentTok{\# Plot the clusters}
\FunctionTok{plotReducedDim}\NormalTok{(tse, }\StringTok{"MDS"}\NormalTok{, }\AttributeTok{colour\_by =} \StringTok{"clusters"}\NormalTok{)}
\end{Highlighting}
\end{Shaded}

\includegraphics{24_clustering_files/figure-latex/bluster_sample_plot-1.pdf}

To change the clustering parameters, simply change the parameters in
\texttt{HclustParam} function. To see the different parameters, check the
\href{https://rdrr.io/github/LTLA/bluster/man/HclustParam-class.html}{HclustParam documentation}.

\begin{Shaded}
\begin{Highlighting}[]
\CommentTok{\# More complex use of the algorithm with another method and cut height.}
\CommentTok{\# We also add the full parameter to get the full hierarchical clustering}
\CommentTok{\# information to plot the dendrogram}
\NormalTok{hclust.out }\OtherTok{\textless{}{-}} \FunctionTok{clusterRows}\NormalTok{(x, }\FunctionTok{HclustParam}\NormalTok{(}\AttributeTok{method =} \StringTok{"complete"}\NormalTok{), }\AttributeTok{full =} \ConstantTok{TRUE}\NormalTok{)}

\CommentTok{\# Add data to the TSE}
\FunctionTok{colData}\NormalTok{(tse)}\SpecialCharTok{$}\NormalTok{clusters }\OtherTok{\textless{}{-}}\NormalTok{ hclust.out}\SpecialCharTok{$}\NormalTok{clusters}

\CommentTok{\# Get the dendrogram object}
\NormalTok{dendro }\OtherTok{\textless{}{-}} \FunctionTok{as.dendrogram}\NormalTok{(hclust.out}\SpecialCharTok{$}\NormalTok{objects}\SpecialCharTok{$}\NormalTok{hclust)}

\CommentTok{\# Plot the dendrogram}
\FunctionTok{plot}\NormalTok{(dendro)}
\end{Highlighting}
\end{Shaded}

\includegraphics{24_clustering_files/figure-latex/bluster_sample_hclust2-1.pdf}

Alternatively, it's also possible to use the \texttt{clusterCells} function from
the \emph{scran} package.This function uses directly the tree as a parameter
instead of the assay itself. Here, we will use the PAM algorithm.

\begin{Shaded}
\begin{Highlighting}[]
\FunctionTok{library}\NormalTok{(scran)}
\NormalTok{pam.out }\OtherTok{\textless{}{-}} \FunctionTok{clusterCells}\NormalTok{(tse,}
  \AttributeTok{assay.type =} \StringTok{"relabundance"}\NormalTok{,}
  \AttributeTok{BLUSPARAM =} \FunctionTok{PamParam}\NormalTok{(}\AttributeTok{centers =} \DecValTok{5}\NormalTok{)}
\NormalTok{)}
\end{Highlighting}
\end{Shaded}

\hypertarget{taxa-clustering}{%
\subsection{Taxa clustering}\label{taxa-clustering}}

Similarly to samples, it's also possible to cluster taxa.
Prior to clustering, the data should be modified with compositionally
aware transformations such as CLR. Then, simply follow the same steps as
with samples, using the \texttt{clusterRows} function, then putting the cluster
indices in rowData. At the end, you can also merge the rows using the
clusters indices.

\hypertarget{hierarchical-clustering}{%
\section{Hierarchical clustering}\label{hierarchical-clustering}}

Hierarchical clustering aims to find hiearchy between
samples/features. There are to approaches: agglomerative (``bottom-up'')
and divisive (``top-down'').

In agglomerative approach, each observation is first unique cluster.
Algorithm continues by agglomerating similar clusters. Divisive
approach starts with one cluster that contains all the
observations. Clusters are splitted recursively to clusters that
differ the most. Clustering ends when each cluster contains only one
observation.

Hierarchical clustering can be visualized with dendrogram tree. In each
splitting point, the three is divided into two clusters leading to
hierarchy.

Let's load data from mia package.

\begin{Shaded}
\begin{Highlighting}[]
\FunctionTok{library}\NormalTok{(mia)}
\FunctionTok{library}\NormalTok{(vegan)}

\CommentTok{\# Load experimental data}
\FunctionTok{data}\NormalTok{(peerj13075)}
\NormalTok{(tse }\OtherTok{\textless{}{-}}\NormalTok{ peerj13075)}
\end{Highlighting}
\end{Shaded}

\begin{verbatim}
## class: TreeSummarizedExperiment 
## dim: 674 58 
## metadata(0):
## assays(1): counts
## rownames(674): OTU1 OTU2 ... OTU2567 OTU2569
## rowData names(6): kingdom phylum ... family genus
## colnames(58): ID1 ID2 ... ID57 ID58
## colData names(5): Sample Geographical_location Gender Age Diet
## reducedDimNames(0):
## mainExpName: NULL
## altExpNames(0):
## rowLinks: NULL
## rowTree: NULL
## colLinks: NULL
## colTree: NULL
\end{verbatim}

Hierarchical clustering requires 2 steps. In the fist step, dissimilarities are
calculated. In prior to that, data transformation is applied if needed. Since
sequencing data is compositional, relative transformation is applied.
In the second step, clustering is performed based on dissimilarities.

\begin{Shaded}
\begin{Highlighting}[]
\FunctionTok{library}\NormalTok{(NbClust)}
\FunctionTok{library}\NormalTok{(cobiclust)}

\CommentTok{\# Apply transformation}
\NormalTok{tse }\OtherTok{\textless{}{-}} \FunctionTok{transformCounts}\NormalTok{(tse, }\AttributeTok{method =} \StringTok{"relabundance"}\NormalTok{)}
\CommentTok{\# Get the assay}
\NormalTok{assay }\OtherTok{\textless{}{-}} \FunctionTok{assay}\NormalTok{(tse, }\StringTok{"relabundance"}\NormalTok{)}
\CommentTok{\# Transpose assay {-}{-}\textgreater{} samples are now in rows {-}{-}\textgreater{} we are clustering samples}
\NormalTok{assay }\OtherTok{\textless{}{-}} \FunctionTok{t}\NormalTok{(assay)}

\CommentTok{\# Calculate distances}
\NormalTok{diss }\OtherTok{\textless{}{-}} \FunctionTok{vegdist}\NormalTok{(assay, }\AttributeTok{method =} \StringTok{"bray"}\NormalTok{)}

\CommentTok{\# Perform hierarchical clustering}
\NormalTok{hc }\OtherTok{\textless{}{-}} \FunctionTok{hclust}\NormalTok{(diss, }\AttributeTok{method =} \StringTok{"complete"}\NormalTok{)}

\CommentTok{\# To visualize, convert hclust object into dendrogram object}
\NormalTok{dendro }\OtherTok{\textless{}{-}} \FunctionTok{as.dendrogram}\NormalTok{(hc)}

\CommentTok{\# Plot dendrogram}
\FunctionTok{plot}\NormalTok{(dendro)}
\end{Highlighting}
\end{Shaded}

\includegraphics{24_clustering_files/figure-latex/hclust2-1.pdf}

We can use dendrogram to determine the number of clusters. Usually the
tree is splitted where the branch length is the largest. However, as
we can see from the dendrogram, clusters are not clear. Algorithms are
available to identify the optimal number of clusters.

\begin{Shaded}
\begin{Highlighting}[]
\CommentTok{\# Determine the optimal number of clusters}
\NormalTok{res }\OtherTok{\textless{}{-}} \FunctionTok{NbClust}\NormalTok{(}
  \AttributeTok{diss =}\NormalTok{ diss, }\AttributeTok{distance =} \ConstantTok{NULL}\NormalTok{, }\AttributeTok{method =} \StringTok{"ward.D2"}\NormalTok{,}
  \AttributeTok{index =} \StringTok{"silhouette"}
\NormalTok{)}
\end{Highlighting}
\end{Shaded}

\begin{verbatim}
## 
##  Only frey, mcclain, cindex, sihouette and dunn can be computed. To compute the other indices, data matrix is needed
\end{verbatim}

\begin{Shaded}
\begin{Highlighting}[]
\NormalTok{res}\SpecialCharTok{$}\NormalTok{Best.nc}
\end{Highlighting}
\end{Shaded}

\begin{verbatim}
## Number_clusters     Value_Index 
##         15.0000          0.4543
\end{verbatim}

Based on the result, let's divide observations into 15 clusters.

\begin{Shaded}
\begin{Highlighting}[]
\FunctionTok{library}\NormalTok{(dendextend)}

\CommentTok{\# Find clusters}
\FunctionTok{cutree}\NormalTok{(hc, }\AttributeTok{k =} \DecValTok{15}\NormalTok{)}
\end{Highlighting}
\end{Shaded}

\begin{verbatim}
##  ID1  ID2  ID3  ID4  ID5  ID6  ID7  ID8  ID9 ID10 ID11 ID12 ID13 ID14 ID15 ID16 
##    1    2    3    4    5    5    2    2    2    6    5    5    2    5    2    5 
## ID17 ID18 ID19 ID20 ID21 ID22 ID23 ID24 ID25 ID26 ID27 ID28 ID29 ID30 ID31 ID32 
##    5    4    5    5    7    8    9    5    3    1    8    8    8    3    1    2 
## ID33 ID34 ID35 ID36 ID37 ID38 ID39 ID40 ID41 ID42 ID43 ID44 ID45 ID46 ID47 ID48 
##    2    2    2    2    2    9   10   11    6    5    4    2    9    1   12    2 
## ID49 ID50 ID51 ID52 ID53 ID54 ID55 ID56 ID57 ID58 
##    4    4    2    8    7   13   14    3    8   15
\end{verbatim}

\begin{Shaded}
\begin{Highlighting}[]
\CommentTok{\# Making colors for 6 clusters}
\NormalTok{col\_val\_map }\OtherTok{\textless{}{-}}\NormalTok{ randomcoloR}\SpecialCharTok{::}\FunctionTok{distinctColorPalette}\NormalTok{(}\DecValTok{15}\NormalTok{) }\SpecialCharTok{\%\textgreater{}\%}
  \FunctionTok{as.list}\NormalTok{() }\SpecialCharTok{\%\textgreater{}\%}
  \FunctionTok{setNames}\NormalTok{(}\FunctionTok{paste0}\NormalTok{(}\StringTok{"clust\_"}\NormalTok{, }\FunctionTok{seq}\NormalTok{(}\DecValTok{15}\NormalTok{)))}

\NormalTok{dend }\OtherTok{\textless{}{-}} \FunctionTok{color\_branches}\NormalTok{(dendro, }\AttributeTok{k =} \DecValTok{15}\NormalTok{, }\AttributeTok{col =} \FunctionTok{unlist}\NormalTok{(col\_val\_map))}
\FunctionTok{labels}\NormalTok{(dend) }\OtherTok{\textless{}{-}} \ConstantTok{NULL}
\FunctionTok{plot}\NormalTok{(dend)}
\end{Highlighting}
\end{Shaded}

\includegraphics{24_clustering_files/figure-latex/hclust4-1.pdf}

\hypertarget{k-means-clustering}{%
\section{K-means clustering}\label{k-means-clustering}}

Hierarchical clustering did not yield clusters. Let's try k-means
clustering instead. Here observations are divided into clusters so
that the distances between observations and cluster centers are
minimized; an observation belongs to cluster whose center is the
nearest.

The algorithm starts by dividing observation to random clusters whose
number is defined by user. The centroids of clusters are then
calculated. After that, observations' allocation to clusters are
updated so that the means are minimized. Again, centroid are
calculated, and algorithm continues iteratively until the assignments
do not change.

The number of clusters can be determined based on algorithm. Here we
utilize silhouette analysis.

\begin{Shaded}
\begin{Highlighting}[]
\FunctionTok{library}\NormalTok{(factoextra)}


\CommentTok{\# Convert dist object into matrix}
\NormalTok{diss }\OtherTok{\textless{}{-}} \FunctionTok{as.matrix}\NormalTok{(diss)}
\CommentTok{\# Perform silhouette analysis and plot the result}
\FunctionTok{fviz\_nbclust}\NormalTok{(diss, kmeans, }\AttributeTok{method =} \StringTok{"silhouette"}\NormalTok{)}
\end{Highlighting}
\end{Shaded}

\includegraphics{24_clustering_files/figure-latex/kmeans1-1.pdf}

Based on the result of silhouette analysis, we choose 3 to be the number of clusters
in k-means clustering.

\begin{Shaded}
\begin{Highlighting}[]
\FunctionTok{library}\NormalTok{(scater)}

\CommentTok{\# The first step is random, add seed for reproducibility}
\FunctionTok{set.seed}\NormalTok{(}\DecValTok{15463}\NormalTok{)}
\CommentTok{\# Perform k{-}means clustering with 3 clusters}
\NormalTok{km }\OtherTok{\textless{}{-}} \FunctionTok{kmeans}\NormalTok{(diss, }\DecValTok{3}\NormalTok{, }\AttributeTok{nstart =} \DecValTok{25}\NormalTok{)}
\CommentTok{\# Add the result to colData}
\FunctionTok{colData}\NormalTok{(tse)}\SpecialCharTok{$}\NormalTok{clusters }\OtherTok{\textless{}{-}} \FunctionTok{as.factor}\NormalTok{(km}\SpecialCharTok{$}\NormalTok{cluster)}

\CommentTok{\# Perform PCoA so that we can visualize clusters}
\NormalTok{tse }\OtherTok{\textless{}{-}} \FunctionTok{runMDS}\NormalTok{(tse, }\AttributeTok{assay.type =} \StringTok{"relabundance"}\NormalTok{, }\AttributeTok{FUN =}\NormalTok{ vegan}\SpecialCharTok{::}\NormalTok{vegdist, }\AttributeTok{method =} \StringTok{"bray"}\NormalTok{)}

\CommentTok{\# Plot PCoA and color clusters}
\FunctionTok{plotReducedDim}\NormalTok{(tse, }\StringTok{"MDS"}\NormalTok{, }\AttributeTok{colour\_by =} \StringTok{"clusters"}\NormalTok{)}
\end{Highlighting}
\end{Shaded}

\includegraphics{24_clustering_files/figure-latex/kmeans2-1.pdf}

\hypertarget{dirichlet-multinomial-mixtures-dmm}{%
\section{Dirichlet Multinomial Mixtures (DMM)}\label{dirichlet-multinomial-mixtures-dmm}}

This section focus on DMM analysis.

One technique that allows to search for groups of samples that are
similar to each other is the \href{https://journals.plos.org/plosone/article?id=10.1371/journal.pone.0030126}{Dirichlet-Multinomial Mixture
Model}. In
DMM, we first determine the number of clusters (k) that best fit the
data (model evidence) using Laplace approximation. After fitting the
model with k clusters, we obtain for each sample k probabilities that
reflect the probability that a sample belongs to the given cluster.

Let's cluster the data with DMM clustering.

\begin{Shaded}
\begin{Highlighting}[]
\CommentTok{\# Runs model and calculates the most likely number of clusters from 1 to 7.}
\CommentTok{\# Since this is a large dataset it takes long computational time.}
\CommentTok{\# For this reason we use only a subset of the data; agglomerated by Phylum as a rank.}
\NormalTok{tse }\OtherTok{\textless{}{-}}\NormalTok{ GlobalPatterns}
\NormalTok{tse }\OtherTok{\textless{}{-}} \FunctionTok{agglomerateByRank}\NormalTok{(tse, }\AttributeTok{rank =} \StringTok{"Phylum"}\NormalTok{, }\AttributeTok{agglomerateTree =} \ConstantTok{TRUE}\NormalTok{)}
\end{Highlighting}
\end{Shaded}

\begin{Shaded}
\begin{Highlighting}[]
\NormalTok{tse\_dmn }\OtherTok{\textless{}{-}}\NormalTok{ mia}\SpecialCharTok{::}\FunctionTok{runDMN}\NormalTok{(tse, }\AttributeTok{name =} \StringTok{"DMN"}\NormalTok{, }\AttributeTok{k =} \DecValTok{1}\SpecialCharTok{:}\DecValTok{7}\NormalTok{)}
\end{Highlighting}
\end{Shaded}

\begin{Shaded}
\begin{Highlighting}[]
\CommentTok{\# It is stored in metadata}
\NormalTok{tse\_dmn}
\end{Highlighting}
\end{Shaded}

\begin{verbatim}
## class: TreeSummarizedExperiment 
## dim: 67 26 
## metadata(2): agglomerated_by_rank DMN
## assays(1): counts
## rownames(67): Phylum:Crenarchaeota Phylum:Euryarchaeota ...
##   Phylum:Synergistetes Phylum:SR1
## rowData names(7): Kingdom Phylum ... Genus Species
## colnames(26): CL3 CC1 ... Even2 Even3
## colData names(7): X.SampleID Primer ... SampleType Description
## reducedDimNames(0):
## mainExpName: NULL
## altExpNames(0):
## rowLinks: a LinkDataFrame (67 rows)
## rowTree: 1 phylo tree(s) (66 leaves)
## colLinks: NULL
## colTree: NULL
\end{verbatim}

Return information on metadata that the object contains.

\begin{Shaded}
\begin{Highlighting}[]
\FunctionTok{names}\NormalTok{(}\FunctionTok{metadata}\NormalTok{(tse\_dmn))}
\end{Highlighting}
\end{Shaded}

\begin{verbatim}
## [1] "agglomerated_by_rank" "DMN"
\end{verbatim}

This returns a list of DMN objects for a closer investigation.

\begin{Shaded}
\begin{Highlighting}[]
\FunctionTok{getDMN}\NormalTok{(tse\_dmn)}
\end{Highlighting}
\end{Shaded}

\begin{verbatim}
## [[1]]
## class: DMN 
## k: 1 
## samples x taxa: 26 x 67 
## Laplace: 7715 BIC: 7802 AIC: 7760 
## 
## [[2]]
## class: DMN 
## k: 2 
## samples x taxa: 26 x 67 
## Laplace: 7673 BIC: 7927 AIC: 7842 
## 
## [[3]]
## class: DMN 
## k: 3 
## samples x taxa: 26 x 67 
## Laplace: 7690 BIC: 8076 AIC: 7948 
## 
## [[4]]
## class: DMN 
## k: 4 
## samples x taxa: 26 x 67 
## Laplace: 7751 BIC: 8274 AIC: 8103 
## 
## [[5]]
## class: DMN 
## k: 5 
## samples x taxa: 26 x 67 
## Laplace: 7854 BIC: 8553 AIC: 8340 
## 
## [[6]]
## class: DMN 
## k: 6 
## samples x taxa: 26 x 67 
## Laplace: 7926 BIC: 8796 AIC: 8540 
## 
## [[7]]
## class: DMN 
## k: 7 
## samples x taxa: 26 x 67 
## Laplace: 8003 BIC: 9051 AIC: 8752
\end{verbatim}

Show Laplace approximation (model evidence) for each model of the k models.

\begin{Shaded}
\begin{Highlighting}[]
\FunctionTok{library}\NormalTok{(miaViz)}
\FunctionTok{plotDMNFit}\NormalTok{(tse\_dmn, }\AttributeTok{type =} \StringTok{"laplace"}\NormalTok{)}
\end{Highlighting}
\end{Shaded}

\includegraphics{24_clustering_files/figure-latex/unnamed-chunk-4-1.pdf}

Return the model that has the best fit.

\begin{Shaded}
\begin{Highlighting}[]
\FunctionTok{getBestDMNFit}\NormalTok{(tse\_dmn, }\AttributeTok{type =} \StringTok{"laplace"}\NormalTok{)}
\end{Highlighting}
\end{Shaded}

\begin{verbatim}
## class: DMN 
## k: 2 
## samples x taxa: 26 x 67 
## Laplace: 7673 BIC: 7927 AIC: 7842
\end{verbatim}

\hypertarget{pcoa-for-asv-level-data-with-bray-curtis-with-dmm-clusters-shown-with-colors}{%
\subsection{PCoA for ASV-level data with Bray-Curtis; with DMM clusters shown with colors}\label{pcoa-for-asv-level-data-with-bray-curtis-with-dmm-clusters-shown-with-colors}}

Group samples and return DMNGroup object that contains a summary.
Patient status is used for grouping.

\begin{Shaded}
\begin{Highlighting}[]
\NormalTok{dmn\_group }\OtherTok{\textless{}{-}} \FunctionTok{calculateDMNgroup}\NormalTok{(tse\_dmn,}
  \AttributeTok{variable =} \StringTok{"SampleType"}\NormalTok{, }\AttributeTok{assay.type =} \StringTok{"counts"}\NormalTok{,}
  \AttributeTok{k =} \DecValTok{2}\NormalTok{, }\AttributeTok{seed =}\NormalTok{ .Machine}\SpecialCharTok{$}\NormalTok{integer.max}
\NormalTok{)}

\NormalTok{dmn\_group}
\end{Highlighting}
\end{Shaded}

\begin{verbatim}
## class: DMNGroup 
## summary:
##                    k samples taxa    NLE  LogDet Laplace    BIC  AIC
## Feces              2       4   67 1078.3 -106.14   901.2 1171.9 1213
## Freshwater         2       2   67  889.6  -97.17   717.0  936.4 1025
## Freshwater (creek) 2       3   67 1600.3  860.08  1906.3 1674.5 1735
## Mock               2       3   67 1008.4  -55.37   856.6 1082.5 1143
## Ocean              2       3   67 1096.7  -56.21   944.6 1170.9 1232
## Sediment (estuary) 2       3   67 1195.5   18.63  1080.8 1269.7 1331
## Skin               2       3   67  992.6  -84.81   826.2 1066.8 1128
## Soil               2       3   67 1380.3   11.21  1261.8 1454.5 1515
## Tongue             2       2   67  783.0 -107.74   605.1  829.8  918
\end{verbatim}

Mixture weights (rough measure of the cluster size).

\begin{Shaded}
\begin{Highlighting}[]
\NormalTok{DirichletMultinomial}\SpecialCharTok{::}\FunctionTok{mixturewt}\NormalTok{(}\FunctionTok{getBestDMNFit}\NormalTok{(tse\_dmn))}
\end{Highlighting}
\end{Shaded}

\begin{verbatim}
##       pi theta
## 1 0.5385 20.60
## 2 0.4615 15.28
\end{verbatim}

Samples-cluster assignment probabilities / how probable it is that sample belongs
to each cluster

\begin{Shaded}
\begin{Highlighting}[]
\FunctionTok{head}\NormalTok{(DirichletMultinomial}\SpecialCharTok{::}\FunctionTok{mixture}\NormalTok{(}\FunctionTok{getBestDMNFit}\NormalTok{(tse\_dmn)))}
\end{Highlighting}
\end{Shaded}

\begin{verbatim}
##              [,1]      [,2]
## CL3     1.000e+00 5.004e-17
## CC1     1.000e+00 3.799e-22
## SV1     1.000e+00 2.021e-12
## M31Fcsw 7.309e-26 1.000e+00
## M11Fcsw 1.061e-16 1.000e+00
## M31Plmr 9.991e-14 1.000e+00
\end{verbatim}

Contribution of each taxa to each component

\begin{Shaded}
\begin{Highlighting}[]
\FunctionTok{head}\NormalTok{(DirichletMultinomial}\SpecialCharTok{::}\FunctionTok{fitted}\NormalTok{(}\FunctionTok{getBestDMNFit}\NormalTok{(tse\_dmn)))}
\end{Highlighting}
\end{Shaded}

\begin{verbatim}
##                         [,1]      [,2]
## Phylum:Crenarchaeota  0.3043 0.1354653
## Phylum:Euryarchaeota  0.2314 0.1468632
## Phylum:Actinobacteria 1.2105 1.0600542
## Phylum:Spirochaetes   0.2141 0.1318414
## Phylum:MVP-15         0.0299 0.0007646
## Phylum:Proteobacteria 6.8425 1.8151526
\end{verbatim}

Get the assignment probabilities

\begin{Shaded}
\begin{Highlighting}[]
\NormalTok{prob }\OtherTok{\textless{}{-}}\NormalTok{ DirichletMultinomial}\SpecialCharTok{::}\FunctionTok{mixture}\NormalTok{(}\FunctionTok{getBestDMNFit}\NormalTok{(tse\_dmn))}
\CommentTok{\# Add column names}
\FunctionTok{colnames}\NormalTok{(prob) }\OtherTok{\textless{}{-}} \FunctionTok{c}\NormalTok{(}\StringTok{"comp1"}\NormalTok{, }\StringTok{"comp2"}\NormalTok{)}

\CommentTok{\# For each row, finds column that has the highest value. Then extract the column}
\CommentTok{\# names of highest values.}
\NormalTok{vec }\OtherTok{\textless{}{-}} \FunctionTok{colnames}\NormalTok{(prob)[}\FunctionTok{max.col}\NormalTok{(prob, }\AttributeTok{ties.method =} \StringTok{"first"}\NormalTok{)]}
\end{Highlighting}
\end{Shaded}

Computing the euclidean PCoA and storing it as a data frame

\begin{Shaded}
\begin{Highlighting}[]
\CommentTok{\# Does clr transformation. Pseudocount is added, because data contains zeros.}
\FunctionTok{assay}\NormalTok{(tse, }\StringTok{"pseudo"}\NormalTok{) }\OtherTok{\textless{}{-}} \FunctionTok{assay}\NormalTok{(tse, }\StringTok{"counts"}\NormalTok{) }\SpecialCharTok{+} \DecValTok{1}
\NormalTok{tse }\OtherTok{\textless{}{-}} \FunctionTok{transformCounts}\NormalTok{(tse, }\AttributeTok{assay.type =} \StringTok{"pseudo"}\NormalTok{, }\AttributeTok{method =} \StringTok{"relabundance"}\NormalTok{)}
\NormalTok{tse }\OtherTok{\textless{}{-}} \FunctionTok{transformCounts}\NormalTok{(tse, }\StringTok{"relabundance"}\NormalTok{, }\AttributeTok{method =} \StringTok{"clr"}\NormalTok{)}

\FunctionTok{library}\NormalTok{(scater)}

\CommentTok{\# Does principal coordinate analysis}
\NormalTok{df }\OtherTok{\textless{}{-}} \FunctionTok{calculateMDS}\NormalTok{(tse, }\AttributeTok{assay.type =} \StringTok{"clr"}\NormalTok{, }\AttributeTok{method =} \StringTok{"euclidean"}\NormalTok{)}

\CommentTok{\# Creates a data frame from principal coordinates}
\NormalTok{euclidean\_pcoa\_df }\OtherTok{\textless{}{-}} \FunctionTok{data.frame}\NormalTok{(}
  \AttributeTok{pcoa1 =}\NormalTok{ df[, }\DecValTok{1}\NormalTok{],}
  \AttributeTok{pcoa2 =}\NormalTok{ df[, }\DecValTok{2}\NormalTok{]}
\NormalTok{)}
\end{Highlighting}
\end{Shaded}

\begin{Shaded}
\begin{Highlighting}[]
\CommentTok{\# Creates a data frame that contains principal coordinates and DMM information}
\NormalTok{euclidean\_dmm\_pcoa\_df }\OtherTok{\textless{}{-}} \FunctionTok{cbind}\NormalTok{(euclidean\_pcoa\_df,}
  \AttributeTok{dmm\_component =}\NormalTok{ vec}
\NormalTok{)}
\CommentTok{\# Creates a plot}
\NormalTok{euclidean\_dmm\_plot }\OtherTok{\textless{}{-}} \FunctionTok{ggplot}\NormalTok{(}
  \AttributeTok{data =}\NormalTok{ euclidean\_dmm\_pcoa\_df,}
  \FunctionTok{aes}\NormalTok{(}
    \AttributeTok{x =}\NormalTok{ pcoa1, }\AttributeTok{y =}\NormalTok{ pcoa2,}
    \AttributeTok{color =}\NormalTok{ dmm\_component}
\NormalTok{  )}
\NormalTok{) }\SpecialCharTok{+}
  \FunctionTok{geom\_point}\NormalTok{() }\SpecialCharTok{+}
  \FunctionTok{labs}\NormalTok{(}
    \AttributeTok{x =} \StringTok{"Coordinate 1"}\NormalTok{,}
    \AttributeTok{y =} \StringTok{"Coordinate 2"}\NormalTok{,}
    \AttributeTok{title =} \StringTok{"PCoA with Aitchison distances"}
\NormalTok{  ) }\SpecialCharTok{+}
  \FunctionTok{theme}\NormalTok{(}\AttributeTok{title =} \FunctionTok{element\_text}\NormalTok{(}\AttributeTok{size =} \DecValTok{12}\NormalTok{)) }\CommentTok{\# makes titles smaller}

\NormalTok{euclidean\_dmm\_plot}
\end{Highlighting}
\end{Shaded}

\includegraphics{24_clustering_files/figure-latex/unnamed-chunk-12-1.pdf}

\hypertarget{community-detection}{%
\section{Community Detection}\label{community-detection}}

Another approach for discovering communities within the samples of the
data, is to run community detection algorithms after building a
graph. The following demonstration builds a graph based on the k
nearest-neighbors and performs the community detection on the fly.

\emph{\texttt{bluster}} \citep{R-bluster} package offers several clustering methods,
among which graph-based are present, enabling the community detection
task.

Installing package:

\begin{Shaded}
\begin{Highlighting}[]
\FunctionTok{library}\NormalTok{(bluster)}
\end{Highlighting}
\end{Shaded}

The algorithm used is ``short random walks'' \citep{Pons2006}. Graph is
constructed using different k values (the number of nearest neighbors
to consider during graph construction) using the robust centered log
ratio (rclr) assay data. Then plotting the communities using UMAP
\citep{McInnes2018} ordination as a visual exploration aid. In the
following demonstration we use the \texttt{enterotype} dataset from the
\citep{R-mia} package.

\begin{Shaded}
\begin{Highlighting}[]
\FunctionTok{library}\NormalTok{(bluster)}
\FunctionTok{library}\NormalTok{(patchwork) }\CommentTok{\# For arranging several plots as a grid}
\FunctionTok{library}\NormalTok{(scater)}

\FunctionTok{data}\NormalTok{(}\StringTok{"enterotype"}\NormalTok{, }\AttributeTok{package =} \StringTok{"mia"}\NormalTok{)}
\NormalTok{tse }\OtherTok{\textless{}{-}}\NormalTok{ enterotype}
\NormalTok{tse }\OtherTok{\textless{}{-}} \FunctionTok{transformCounts}\NormalTok{(tse, }\AttributeTok{method =} \StringTok{"rclr"}\NormalTok{)}

\CommentTok{\# Performing and storing UMAP}
\NormalTok{tse }\OtherTok{\textless{}{-}} \FunctionTok{runUMAP}\NormalTok{(tse, }\AttributeTok{name =} \StringTok{"UMAP"}\NormalTok{, }\AttributeTok{assay.type =} \StringTok{"rclr"}\NormalTok{)}

\NormalTok{k }\OtherTok{\textless{}{-}} \FunctionTok{c}\NormalTok{(}\DecValTok{2}\NormalTok{, }\DecValTok{3}\NormalTok{, }\DecValTok{5}\NormalTok{, }\DecValTok{10}\NormalTok{)}
\NormalTok{ClustAndPlot }\OtherTok{\textless{}{-}} \ControlFlowTok{function}\NormalTok{(x) \{}
  \CommentTok{\# Creating the graph and running the short random walks algorithm}
\NormalTok{  graph\_clusters }\OtherTok{\textless{}{-}} \FunctionTok{clusterRows}\NormalTok{(}\FunctionTok{t}\NormalTok{(}\FunctionTok{assays}\NormalTok{(tse)}\SpecialCharTok{$}\NormalTok{rclr), }\FunctionTok{NNGraphParam}\NormalTok{(}\AttributeTok{k =}\NormalTok{ x))}

  \CommentTok{\# Results of the clustering as a color for each sample}
  \FunctionTok{plotUMAP}\NormalTok{(tse, }\AttributeTok{colour\_by =} \FunctionTok{I}\NormalTok{(graph\_clusters)) }\SpecialCharTok{+}
    \FunctionTok{labs}\NormalTok{(}\AttributeTok{title =} \FunctionTok{paste0}\NormalTok{(}\StringTok{"k = "}\NormalTok{, x))}
\NormalTok{\}}

\CommentTok{\# Applying the function for different k values}
\NormalTok{plots }\OtherTok{\textless{}{-}} \FunctionTok{lapply}\NormalTok{(k, ClustAndPlot)}

\CommentTok{\# Displaying plots in a grid}
\NormalTok{(plots[[}\DecValTok{1}\NormalTok{]] }\SpecialCharTok{+}\NormalTok{ plots[[}\DecValTok{2}\NormalTok{]]) }\SpecialCharTok{/}\NormalTok{ (plots[[}\DecValTok{3}\NormalTok{]] }\SpecialCharTok{+}\NormalTok{ plots[[}\DecValTok{4}\NormalTok{]])}
\end{Highlighting}
\end{Shaded}

\includegraphics{24_clustering_files/figure-latex/unnamed-chunk-14-1.pdf}

Similarly, the \emph{\texttt{bluster}} \citep{R-bluster} package offers clustering
diagnostics that can be used for judging the clustering quality (see
\href{http://bioconductor.org/packages/release/bioc/vignettes/bluster/inst/doc/diagnostics.html}{Assorted clustering
diagnostics}).
In the following, Silhouette width as a diagnostic tool is computed
and results are visualized for each case presented earlier. For more
about Silhouettes read \citep{Rousseeuw1987}.

\begin{Shaded}
\begin{Highlighting}[]
\NormalTok{ClustDiagPlot }\OtherTok{\textless{}{-}} \ControlFlowTok{function}\NormalTok{(x) \{}
  \CommentTok{\# Getting the clustering results}
\NormalTok{  graph\_clusters }\OtherTok{\textless{}{-}} \FunctionTok{clusterRows}\NormalTok{(}\FunctionTok{t}\NormalTok{(}\FunctionTok{assays}\NormalTok{(tse)}\SpecialCharTok{$}\NormalTok{rclr), }\FunctionTok{NNGraphParam}\NormalTok{(}\AttributeTok{k =}\NormalTok{ x))}

  \CommentTok{\# Computing the diagnostic info}
\NormalTok{  sil }\OtherTok{\textless{}{-}} \FunctionTok{approxSilhouette}\NormalTok{(}\FunctionTok{t}\NormalTok{(}\FunctionTok{assays}\NormalTok{(tse)}\SpecialCharTok{$}\NormalTok{rclr), graph\_clusters)}

  \CommentTok{\# Plotting as a boxlpot to observe cluster separation}
  \FunctionTok{boxplot}\NormalTok{(}\FunctionTok{split}\NormalTok{(sil}\SpecialCharTok{$}\NormalTok{width, graph\_clusters), }\AttributeTok{main =} \FunctionTok{paste0}\NormalTok{(}\StringTok{"k = "}\NormalTok{, x))}
\NormalTok{\}}
\CommentTok{\# Applying the function for different k values}
\NormalTok{res }\OtherTok{\textless{}{-}} \FunctionTok{lapply}\NormalTok{(k, ClustDiagPlot)}
\end{Highlighting}
\end{Shaded}

\includegraphics{24_clustering_files/figure-latex/unnamed-chunk-15-1.pdf} \includegraphics{24_clustering_files/figure-latex/unnamed-chunk-15-2.pdf} \includegraphics{24_clustering_files/figure-latex/unnamed-chunk-15-3.pdf} \includegraphics{24_clustering_files/figure-latex/unnamed-chunk-15-4.pdf}

\hypertarget{biclustering}{%
\section{Biclustering}\label{biclustering}}

Biclustering methods cluster rows and columns simultaneously in order
to find subsets of correlated features/samples.

Here, we use following packages:

\begin{itemize}
\tightlist
\item
  \href{https://cran.r-project.org/web/packages/biclust/index.html}{biclust}
\item
  \href{https://besjournals.onlinelibrary.wiley.com/doi/abs/10.1111/2041-210X.13582}{cobiclust}
\end{itemize}

\emph{cobiclust} is especially developed for microbiome data whereas \emph{biclust} is more
general method. In this section, we show three different cases and example
solutions to apply biclustering to them.

\begin{enumerate}
\def\labelenumi{\arabic{enumi}.}
\tightlist
\item
  Taxa vs samples
\item
  Taxa vs biomolecule/biomarker
\item
  Taxa vs taxa
\end{enumerate}

Biclusters can be visualized using heatmap or boxplot, for
instance. For checking purposes, also scatter plot might be valid
choice.

Check more ideas for heatmaps from chapters \ref{viz-chapter} and
@ref(microbiome-community.

\hypertarget{taxa-vs-samples}{%
\subsection{Taxa vs samples}\label{taxa-vs-samples}}

When you have microbial abundance matrices, we suggest to use
\emph{cobiclust} which is designed for microbial data.

Load example data

\begin{Shaded}
\begin{Highlighting}[]
\FunctionTok{library}\NormalTok{(mia)}
\FunctionTok{data}\NormalTok{(}\StringTok{"HintikkaXOData"}\NormalTok{)}
\NormalTok{mae }\OtherTok{\textless{}{-}}\NormalTok{ HintikkaXOData}
\end{Highlighting}
\end{Shaded}

Only the most prevalent taxa are included in analysis.

\begin{Shaded}
\begin{Highlighting}[]
\CommentTok{\# Subset data in the first experiment}
\NormalTok{mae[[}\DecValTok{1}\NormalTok{]] }\OtherTok{\textless{}{-}} \FunctionTok{subsetByPrevalentTaxa}\NormalTok{(mae[[}\DecValTok{1}\NormalTok{]], }\AttributeTok{rank =} \StringTok{"Genus"}\NormalTok{, }\AttributeTok{prevalence =} \FloatTok{0.2}\NormalTok{, }\AttributeTok{detection =} \FloatTok{0.001}\NormalTok{)}
\CommentTok{\# clr{-}transform in the first experiment}
\NormalTok{mae[[}\DecValTok{1}\NormalTok{]] }\OtherTok{\textless{}{-}} \FunctionTok{transformCounts}\NormalTok{(mae[[}\DecValTok{1}\NormalTok{]], }\AttributeTok{method =} \StringTok{"relabundance"}\NormalTok{)}
\NormalTok{mae[[}\DecValTok{1}\NormalTok{]] }\OtherTok{\textless{}{-}} \FunctionTok{transformCounts}\NormalTok{(mae[[}\DecValTok{1}\NormalTok{]], }\StringTok{"relabundance"}\NormalTok{, }\AttributeTok{method =} \StringTok{"rclr"}\NormalTok{)}
\end{Highlighting}
\end{Shaded}

\emph{cobiclust} takes counts table as an input and gives \emph{cobiclust} object as an output.
It includes clusters for taxa and samples.

\begin{Shaded}
\begin{Highlighting}[]
\CommentTok{\# Do clustering; use counts table´}
\NormalTok{clusters }\OtherTok{\textless{}{-}} \FunctionTok{cobiclust}\NormalTok{(}\FunctionTok{assay}\NormalTok{(mae[[}\DecValTok{1}\NormalTok{]], }\StringTok{"counts"}\NormalTok{))}

\CommentTok{\# Get clusters}
\NormalTok{row\_clusters }\OtherTok{\textless{}{-}}\NormalTok{ clusters}\SpecialCharTok{$}\NormalTok{classification}\SpecialCharTok{$}\NormalTok{rowclass}
\NormalTok{col\_clusters }\OtherTok{\textless{}{-}}\NormalTok{ clusters}\SpecialCharTok{$}\NormalTok{classification}\SpecialCharTok{$}\NormalTok{colclass}

\CommentTok{\# Add clusters to rowdata and coldata}
\FunctionTok{rowData}\NormalTok{(mae[[}\DecValTok{1}\NormalTok{]])}\SpecialCharTok{$}\NormalTok{clusters }\OtherTok{\textless{}{-}} \FunctionTok{factor}\NormalTok{(row\_clusters)}
\FunctionTok{colData}\NormalTok{(mae[[}\DecValTok{1}\NormalTok{]])}\SpecialCharTok{$}\NormalTok{clusters }\OtherTok{\textless{}{-}} \FunctionTok{factor}\NormalTok{(col\_clusters)}

\CommentTok{\# Order data based on clusters}
\NormalTok{mae[[}\DecValTok{1}\NormalTok{]] }\OtherTok{\textless{}{-}}\NormalTok{ mae[[}\DecValTok{1}\NormalTok{]][}\FunctionTok{order}\NormalTok{(}\FunctionTok{rowData}\NormalTok{(mae[[}\DecValTok{1}\NormalTok{]])}\SpecialCharTok{$}\NormalTok{clusters), }\FunctionTok{order}\NormalTok{(}\FunctionTok{colData}\NormalTok{(mae[[}\DecValTok{1}\NormalTok{]])}\SpecialCharTok{$}\NormalTok{clusters)]}

\CommentTok{\# Print clusters}
\NormalTok{clusters}\SpecialCharTok{$}\NormalTok{classification}
\end{Highlighting}
\end{Shaded}

\begin{verbatim}
## $rowclass
##  [1] 1 1 1 1 2 2 1 1 1 1 1 1 2 2 2 2 1 2 1 1 2 1 2 2 1 1 2 1 1 1 1 1 2 1 1 2 1 1
## [39] 1 1 1 1 1 1 1 1 1 2 1 2 1 1 1 2 1 1 1
## 
## $colclass
##  C1  C2  C3  C4  C5  C6  C7  C8  C9 C10 C11 C12 C13 C14 C15 C16 C17 C18 C19 C20 
##   1   2   2   2   2   2   2   2   2   2   2   2   2   2   2   2   2   2   2   2 
## C21 C22 C23 C24 C25 C26 C27 C28 C29 C30 C31 C32 C33 C34 C35 C36 C37 C38 C39 C40 
##   2   3   3   3   3   3   3   3   3   3   3   3   3   3   3   3   3   3   3   1
\end{verbatim}

Next we can plot clusters. Annotated heatmap is a common choice.

\begin{Shaded}
\begin{Highlighting}[]
\FunctionTok{library}\NormalTok{(pheatmap)}
\CommentTok{\# z{-}transform for heatmap}
\NormalTok{mae[[}\DecValTok{1}\NormalTok{]] }\OtherTok{\textless{}{-}} \FunctionTok{transformCounts}\NormalTok{(mae[[}\DecValTok{1}\NormalTok{]],}
  \AttributeTok{assay.type =} \StringTok{"rclr"}\NormalTok{,}
  \AttributeTok{MARGIN =} \StringTok{"features"}\NormalTok{,}
  \AttributeTok{method =} \StringTok{"z"}\NormalTok{, }\AttributeTok{name =} \StringTok{"clr\_z"}
\NormalTok{)}

\CommentTok{\# Create annotations. When column names are equal, they should share levels.}
\CommentTok{\# Here samples include 3 clusters, and taxa 2. That is why we have to make}
\CommentTok{\# column names unique.}
\NormalTok{annotation\_col }\OtherTok{\textless{}{-}} \FunctionTok{data.frame}\NormalTok{(}\FunctionTok{colData}\NormalTok{(mae[[}\DecValTok{1}\NormalTok{]])[, }\StringTok{"clusters"}\NormalTok{, }\AttributeTok{drop =}\NormalTok{ F])}
\FunctionTok{colnames}\NormalTok{(annotation\_col) }\OtherTok{\textless{}{-}} \StringTok{"col\_clusters"}

\NormalTok{annotation\_row }\OtherTok{\textless{}{-}} \FunctionTok{data.frame}\NormalTok{(}\FunctionTok{rowData}\NormalTok{(mae[[}\DecValTok{1}\NormalTok{]])[, }\StringTok{"clusters"}\NormalTok{, }\AttributeTok{drop =}\NormalTok{ F])}
\FunctionTok{colnames}\NormalTok{(annotation\_row) }\OtherTok{\textless{}{-}} \StringTok{"row\_clusters"}
\end{Highlighting}
\end{Shaded}

Plot the heatmap.

\begin{Shaded}
\begin{Highlighting}[]
\FunctionTok{pheatmap}\NormalTok{(}\FunctionTok{assay}\NormalTok{(mae[[}\DecValTok{1}\NormalTok{]], }\StringTok{"clr\_z"}\NormalTok{),}
  \AttributeTok{cluster\_rows =}\NormalTok{ F, }\AttributeTok{cluster\_cols =}\NormalTok{ F,}
  \AttributeTok{annotation\_col =}\NormalTok{ annotation\_col,}
  \AttributeTok{annotation\_row =}\NormalTok{ annotation\_row}
\NormalTok{)}
\end{Highlighting}
\end{Shaded}

\includegraphics{24_clustering_files/figure-latex/cobiclust_3b-1.pdf}

Boxplot is commonly used to summarize the results:

\begin{Shaded}
\begin{Highlighting}[]
\FunctionTok{library}\NormalTok{(ggplot2)}
\FunctionTok{library}\NormalTok{(patchwork)}

\CommentTok{\# ggplot requires data in melted format}
\NormalTok{melt\_assay }\OtherTok{\textless{}{-}} \FunctionTok{meltAssay}\NormalTok{(mae[[}\DecValTok{1}\NormalTok{]], }\AttributeTok{assay.type =} \StringTok{"rclr"}\NormalTok{, }\AttributeTok{add\_col\_data =}\NormalTok{ T, }\AttributeTok{add\_row\_data =}\NormalTok{ T)}

\CommentTok{\# patchwork two plots side{-}by{-}side}
\NormalTok{p1 }\OtherTok{\textless{}{-}} \FunctionTok{ggplot}\NormalTok{(melt\_assay) }\SpecialCharTok{+}
  \FunctionTok{geom\_boxplot}\NormalTok{(}\FunctionTok{aes}\NormalTok{(}\AttributeTok{x =}\NormalTok{ clusters.x, }\AttributeTok{y =}\NormalTok{ rclr)) }\SpecialCharTok{+}
  \FunctionTok{labs}\NormalTok{(}\AttributeTok{x =} \StringTok{"Taxa clusters"}\NormalTok{)}

\NormalTok{p2 }\OtherTok{\textless{}{-}} \FunctionTok{ggplot}\NormalTok{(melt\_assay) }\SpecialCharTok{+}
  \FunctionTok{geom\_boxplot}\NormalTok{(}\FunctionTok{aes}\NormalTok{(}\AttributeTok{x =}\NormalTok{ clusters.y, }\AttributeTok{y =}\NormalTok{ rclr)) }\SpecialCharTok{+}
  \FunctionTok{labs}\NormalTok{(}\AttributeTok{x =} \StringTok{"Sample clusters"}\NormalTok{)}

\NormalTok{p1 }\SpecialCharTok{+}\NormalTok{ p2}
\end{Highlighting}
\end{Shaded}

\includegraphics{24_clustering_files/figure-latex/cobiclust_4-1.pdf}

\hypertarget{taxa-vs-biomolecules}{%
\subsection{Taxa vs biomolecules}\label{taxa-vs-biomolecules}}

Here, we analyze cross-correlation between taxa and metabolites. This
is a case, where we use \emph{biclust} method which is suitable for numeric
matrices in general.

\begin{Shaded}
\begin{Highlighting}[]
\CommentTok{\# Samples must be in equal order}
\CommentTok{\# (Only 1st experiment  was ordered in cobiclust step leading to unequal order)}
\NormalTok{mae[[}\DecValTok{1}\NormalTok{]] }\OtherTok{\textless{}{-}}\NormalTok{ mae[[}\DecValTok{1}\NormalTok{]][, }\FunctionTok{colnames}\NormalTok{(mae[[}\DecValTok{2}\NormalTok{]])]}

\CommentTok{\# Make rownames unique since it is require by other steps}
\FunctionTok{rownames}\NormalTok{(mae[[}\DecValTok{1}\NormalTok{]]) }\OtherTok{\textless{}{-}} \FunctionTok{make.unique}\NormalTok{(}\FunctionTok{rownames}\NormalTok{(mae[[}\DecValTok{1}\NormalTok{]]))}
\CommentTok{\# Calculate correlations}
\NormalTok{corr }\OtherTok{\textless{}{-}} \FunctionTok{getExperimentCrossCorrelation}\NormalTok{(mae, }\DecValTok{1}\NormalTok{, }\DecValTok{2}\NormalTok{,}
  \AttributeTok{assay.type1 =} \StringTok{"rclr"}\NormalTok{,}
  \AttributeTok{assay.type2 =} \StringTok{"nmr"}\NormalTok{,}
  \AttributeTok{mode =} \StringTok{"matrix"}\NormalTok{,}
  \AttributeTok{cor\_threshold =} \FloatTok{0.2}
\NormalTok{)}
\end{Highlighting}
\end{Shaded}

\emph{biclust} takes matrix as an input and returns \emph{biclust} object.

\begin{Shaded}
\begin{Highlighting}[]
\CommentTok{\# Set seed for reproducibility}
\FunctionTok{set.seed}\NormalTok{(}\DecValTok{3973}\NormalTok{)}

\CommentTok{\# Find biclusters}
\FunctionTok{library}\NormalTok{(biclust)}
\NormalTok{bc }\OtherTok{\textless{}{-}} \FunctionTok{biclust}\NormalTok{(corr,}
  \AttributeTok{method =} \FunctionTok{BCPlaid}\NormalTok{(), }\AttributeTok{fit.model =}\NormalTok{ y }\SpecialCharTok{\textasciitilde{}}\NormalTok{ m,}
  \AttributeTok{background =} \ConstantTok{TRUE}\NormalTok{, }\AttributeTok{shuffle =} \DecValTok{100}\NormalTok{, }\AttributeTok{back.fit =} \DecValTok{0}\NormalTok{, }\AttributeTok{max.layers =} \DecValTok{10}\NormalTok{,}
  \AttributeTok{iter.startup =} \DecValTok{10}\NormalTok{, }\AttributeTok{iter.layer =} \DecValTok{100}\NormalTok{, }\AttributeTok{verbose =} \ConstantTok{FALSE}
\NormalTok{)}

\NormalTok{bc}
\end{Highlighting}
\end{Shaded}

\begin{verbatim}
## 
## An object of class Biclust 
## 
## call:
##  biclust(x = corr, method = BCPlaid(), fit.model = y ~ m, background = TRUE, 
##      shuffle = 100, back.fit = 0, max.layers = 10, iter.startup = 10, 
##      iter.layer = 100, verbose = FALSE)
## 
## There was no cluster found
\end{verbatim}

The object includes cluster information. However compared to
\emph{cobiclust}, \emph{biclust} object includes only information about clusters
that were found, not general cluster.

Meaning that if one cluster size of 5 features was found out of 20 features,
those 15 features do not belong to any cluster. That is why we have to create an
additional cluster for features/samples that are not assigned into any cluster.

\begin{Shaded}
\begin{Highlighting}[]
\CommentTok{\# Functions for obtaining biclust information}

\CommentTok{\# Get clusters for rows and columns}
\NormalTok{.get\_biclusters\_from\_biclust }\OtherTok{\textless{}{-}} \ControlFlowTok{function}\NormalTok{(bc, assay) \{}
  \CommentTok{\# Get cluster information for columns and rows}
\NormalTok{  bc\_columns }\OtherTok{\textless{}{-}} \FunctionTok{t}\NormalTok{(bc}\SpecialCharTok{@}\NormalTok{NumberxCol)}
\NormalTok{  bc\_columns }\OtherTok{\textless{}{-}} \FunctionTok{data.frame}\NormalTok{(bc\_columns)}
\NormalTok{  bc\_rows }\OtherTok{\textless{}{-}}\NormalTok{ bc}\SpecialCharTok{@}\NormalTok{RowxNumber}
\NormalTok{  bc\_rows }\OtherTok{\textless{}{-}} \FunctionTok{data.frame}\NormalTok{(bc\_rows)}

  \CommentTok{\# Get data into right format}
\NormalTok{  bc\_columns }\OtherTok{\textless{}{-}} \FunctionTok{.manipulate\_bc\_data}\NormalTok{(bc\_columns, assay, }\StringTok{"col"}\NormalTok{)}
\NormalTok{  bc\_rows }\OtherTok{\textless{}{-}} \FunctionTok{.manipulate\_bc\_data}\NormalTok{(bc\_rows, assay, }\StringTok{"row"}\NormalTok{)}

  \FunctionTok{return}\NormalTok{(}\FunctionTok{list}\NormalTok{(}\AttributeTok{bc\_columns =}\NormalTok{ bc\_columns, }\AttributeTok{bc\_rows =}\NormalTok{ bc\_rows))}
\NormalTok{\}}

\CommentTok{\# Input clusters, and how many observations there should be, i.e.,}
\CommentTok{\# the number of samples or features}
\NormalTok{.manipulate\_bc\_data }\OtherTok{\textless{}{-}} \ControlFlowTok{function}\NormalTok{(bc\_clusters, assay, row\_col) \{}
  \CommentTok{\# Get right dimension}
\NormalTok{  dim }\OtherTok{\textless{}{-}} \FunctionTok{ifelse}\NormalTok{(row\_col }\SpecialCharTok{==} \StringTok{"col"}\NormalTok{, }\FunctionTok{ncol}\NormalTok{(assay), }\FunctionTok{nrow}\NormalTok{(assay))}
  \CommentTok{\# Get column/row names}
  \ControlFlowTok{if}\NormalTok{ (row\_col }\SpecialCharTok{==} \StringTok{"col"}\NormalTok{) \{}
\NormalTok{    names }\OtherTok{\textless{}{-}} \FunctionTok{colnames}\NormalTok{(assay)}
\NormalTok{  \} }\ControlFlowTok{else}\NormalTok{ \{}
\NormalTok{    names }\OtherTok{\textless{}{-}} \FunctionTok{rownames}\NormalTok{(assay)}
\NormalTok{  \}}

  \CommentTok{\# If no clusters were found, create one. Otherwise create additional}
  \CommentTok{\# cluster which}
  \CommentTok{\# contain those samples that are not included in clusters that were found.}
  \ControlFlowTok{if}\NormalTok{ (}\FunctionTok{nrow}\NormalTok{(bc\_clusters) }\SpecialCharTok{!=}\NormalTok{ dim) \{}
\NormalTok{    bc\_clusters }\OtherTok{\textless{}{-}} \FunctionTok{data.frame}\NormalTok{(}\AttributeTok{cluster =} \FunctionTok{rep}\NormalTok{(}\ConstantTok{TRUE}\NormalTok{, dim))}
\NormalTok{  \} }\ControlFlowTok{else}\NormalTok{ \{}
    \CommentTok{\# Create additional cluster that includes those samples/features that}
    \CommentTok{\# are not included in other clusters.}
\NormalTok{    vec }\OtherTok{\textless{}{-}} \FunctionTok{ifelse}\NormalTok{(}\FunctionTok{rowSums}\NormalTok{(bc\_clusters) }\SpecialCharTok{\textgreater{}} \DecValTok{0}\NormalTok{, }\ConstantTok{FALSE}\NormalTok{, }\ConstantTok{TRUE}\NormalTok{)}
    \CommentTok{\# If additional cluster contains samples, then add it}
    \ControlFlowTok{if}\NormalTok{ (}\FunctionTok{any}\NormalTok{(vec)) \{}
\NormalTok{      bc\_clusters }\OtherTok{\textless{}{-}} \FunctionTok{cbind}\NormalTok{(bc\_clusters, vec)}
\NormalTok{    \}}
\NormalTok{  \}}
  \CommentTok{\# Adjust row and column names}
  \FunctionTok{rownames}\NormalTok{(bc\_clusters) }\OtherTok{\textless{}{-}}\NormalTok{ names}
  \FunctionTok{colnames}\NormalTok{(bc\_clusters) }\OtherTok{\textless{}{-}} \FunctionTok{paste0}\NormalTok{(}\StringTok{"cluster\_"}\NormalTok{, }\DecValTok{1}\SpecialCharTok{:}\FunctionTok{ncol}\NormalTok{(bc\_clusters))}
  \FunctionTok{return}\NormalTok{(bc\_clusters)}
\NormalTok{\}}
\end{Highlighting}
\end{Shaded}

\begin{Shaded}
\begin{Highlighting}[]
\CommentTok{\# Get biclusters}
\NormalTok{bcs }\OtherTok{\textless{}{-}} \FunctionTok{.get\_biclusters\_from\_biclust}\NormalTok{(bc, corr)}

\NormalTok{bicluster\_rows }\OtherTok{\textless{}{-}}\NormalTok{ bcs}\SpecialCharTok{$}\NormalTok{bc\_rows}
\NormalTok{bicluster\_columns }\OtherTok{\textless{}{-}}\NormalTok{ bcs}\SpecialCharTok{$}\NormalTok{bc\_columns}

\CommentTok{\# Print biclusters for rows}
\FunctionTok{head}\NormalTok{(bicluster\_rows)}
\end{Highlighting}
\end{Shaded}

\begin{verbatim}
##                                    cluster_1
## D_5__Ruminiclostridium 5                TRUE
## D_5__Lachnoclostridium                  TRUE
## D_5__Holdemania                         TRUE
## D_5__Anaerostipes                       TRUE
## D_5__uncultured_3                       TRUE
## D_5__Ruminococcaceae NK4A214 group      TRUE
\end{verbatim}

Let's collect information for the scatter plot.

\begin{Shaded}
\begin{Highlighting}[]
\CommentTok{\# Function for obtaining sample{-}wise sum, mean, median, and mean variance}
\CommentTok{\# for each cluster}

\NormalTok{.sum\_mean\_median\_var }\OtherTok{\textless{}{-}} \ControlFlowTok{function}\NormalTok{(tse1, tse2, assay.type1, assay.type2, clusters1, clusters2) \{}
\NormalTok{  list }\OtherTok{\textless{}{-}} \FunctionTok{list}\NormalTok{()}
  \CommentTok{\# Create a data frame that includes all the information}
  \ControlFlowTok{for}\NormalTok{ (i }\ControlFlowTok{in} \DecValTok{1}\SpecialCharTok{:}\FunctionTok{ncol}\NormalTok{(clusters1)) \{}
    \CommentTok{\# Subset data based on cluster}
\NormalTok{    tse\_subset1 }\OtherTok{\textless{}{-}}\NormalTok{ tse1[clusters1[, i], ]}
\NormalTok{    tse\_subset2 }\OtherTok{\textless{}{-}}\NormalTok{ tse2[clusters2[, i], ]}
    \CommentTok{\# Get assay}
\NormalTok{    assay1 }\OtherTok{\textless{}{-}} \FunctionTok{assay}\NormalTok{(tse\_subset1, assay.type1)}
\NormalTok{    assay2 }\OtherTok{\textless{}{-}} \FunctionTok{assay}\NormalTok{(tse\_subset2, assay.type2)}
    \CommentTok{\# Calculate sum, mean, median, and mean variance}
\NormalTok{    sum1 }\OtherTok{\textless{}{-}} \FunctionTok{colSums2}\NormalTok{(assay1, }\AttributeTok{na.rm =}\NormalTok{ T)}
\NormalTok{    mean1 }\OtherTok{\textless{}{-}} \FunctionTok{colMeans2}\NormalTok{(assay1, }\AttributeTok{na.rm =}\NormalTok{ T)}
\NormalTok{    median1 }\OtherTok{\textless{}{-}} \FunctionTok{colMedians}\NormalTok{(assay1, }\AttributeTok{na.rm =}\NormalTok{ T)}
\NormalTok{    var1 }\OtherTok{\textless{}{-}} \FunctionTok{colVars}\NormalTok{(assay1, }\AttributeTok{na.rm =}\NormalTok{ T)}

\NormalTok{    sum2 }\OtherTok{\textless{}{-}} \FunctionTok{colSums2}\NormalTok{(assay2, }\AttributeTok{na.rm =}\NormalTok{ T)}
\NormalTok{    mean2 }\OtherTok{\textless{}{-}} \FunctionTok{colMeans2}\NormalTok{(assay2, }\AttributeTok{na.rm =}\NormalTok{ T)}
\NormalTok{    median2 }\OtherTok{\textless{}{-}} \FunctionTok{colMedians}\NormalTok{(assay2, }\AttributeTok{na.rm =}\NormalTok{ T)}
\NormalTok{    var2 }\OtherTok{\textless{}{-}} \FunctionTok{colVars}\NormalTok{(assay2, }\AttributeTok{na.rm =}\NormalTok{ T)}

\NormalTok{    list[[i]] }\OtherTok{\textless{}{-}} \FunctionTok{data.frame}\NormalTok{(}
      \AttributeTok{sample =} \FunctionTok{colnames}\NormalTok{(tse1), sum1, sum2, mean1, mean2,}
\NormalTok{      median1, median2, var1, var2}
\NormalTok{    )}
\NormalTok{  \}}

  \FunctionTok{return}\NormalTok{(list)}
\NormalTok{\}}

\CommentTok{\# Calculate info}
\NormalTok{df }\OtherTok{\textless{}{-}} \FunctionTok{.sum\_mean\_median\_var}\NormalTok{(mae[[}\DecValTok{1}\NormalTok{]], mae[[}\DecValTok{2}\NormalTok{]], }\StringTok{"rclr"}\NormalTok{, }\StringTok{"nmr"}\NormalTok{, bicluster\_rows, bicluster\_columns)}
\end{Highlighting}
\end{Shaded}

Now we can create a scatter plot. X-axis includes median clr abundance
of microbiome and y-axis median absolute concentration of each
metabolite. Each data point represents a single sample.

From the plots, we can see that there is low negative correlation in
both cluster 1 and 3. This means that when abundance of bacteria
belonging to cluster 1 or 3 is higher, the concentration of
metabolites of cluster 1 or 3 is lower, and vice versa.

\begin{Shaded}
\begin{Highlighting}[]
\NormalTok{pics }\OtherTok{\textless{}{-}} \FunctionTok{list}\NormalTok{()}
\ControlFlowTok{for}\NormalTok{ (i }\ControlFlowTok{in} \FunctionTok{seq\_along}\NormalTok{(df)) \{}
\NormalTok{  pics[[i]] }\OtherTok{\textless{}{-}} \FunctionTok{ggplot}\NormalTok{(df[[i]]) }\SpecialCharTok{+}
    \FunctionTok{geom\_point}\NormalTok{(}\FunctionTok{aes}\NormalTok{(}\AttributeTok{x =}\NormalTok{ median1, }\AttributeTok{y =}\NormalTok{ median2)) }\SpecialCharTok{+}
    \FunctionTok{labs}\NormalTok{(}
      \AttributeTok{title =} \FunctionTok{paste0}\NormalTok{(}\StringTok{"Cluster "}\NormalTok{, i),}
      \AttributeTok{x =} \StringTok{"Taxa (rclr median)"}\NormalTok{,}
      \AttributeTok{y =} \StringTok{"Metabolites (abs. median)"}
\NormalTok{    )}
  \FunctionTok{print}\NormalTok{(pics[[i]])}
\NormalTok{\}}
\end{Highlighting}
\end{Shaded}

\includegraphics[width=0.33\linewidth]{24_clustering_files/figure-latex/biclust_6-1}

\begin{Shaded}
\begin{Highlighting}[]
\CommentTok{\# pics[[1]] + pics[[2]] + pics[[3]]}
\end{Highlighting}
\end{Shaded}

\emph{pheatmap} does not allow boolean values, so they must be converted into factors.

\begin{Shaded}
\begin{Highlighting}[]
\NormalTok{bicluster\_columns }\OtherTok{\textless{}{-}} \FunctionTok{data.frame}\NormalTok{(}\FunctionTok{apply}\NormalTok{(bicluster\_columns, }\DecValTok{2}\NormalTok{, as.factor))}
\NormalTok{bicluster\_rows }\OtherTok{\textless{}{-}} \FunctionTok{data.frame}\NormalTok{(}\FunctionTok{apply}\NormalTok{(bicluster\_rows, }\DecValTok{2}\NormalTok{, as.factor))}
\end{Highlighting}
\end{Shaded}

Again, we can plot clusters with heatmap.

\begin{Shaded}
\begin{Highlighting}[]
\CommentTok{\# Adjust colors for all clusters}
\ControlFlowTok{if}\NormalTok{ (}\FunctionTok{ncol}\NormalTok{(bicluster\_rows) }\SpecialCharTok{\textgreater{}} \FunctionTok{ncol}\NormalTok{(bicluster\_columns)) \{}
\NormalTok{  cluster\_names }\OtherTok{\textless{}{-}} \FunctionTok{colnames}\NormalTok{(bicluster\_rows)}
\NormalTok{\} }\ControlFlowTok{else}\NormalTok{ \{}
\NormalTok{  cluster\_names }\OtherTok{\textless{}{-}} \FunctionTok{colnames}\NormalTok{(bicluster\_columns)}
\NormalTok{\}}
\NormalTok{annotation\_colors }\OtherTok{\textless{}{-}} \FunctionTok{list}\NormalTok{()}
\ControlFlowTok{for}\NormalTok{ (name }\ControlFlowTok{in}\NormalTok{ cluster\_names) \{}
\NormalTok{  annotation\_colors[[name]] }\OtherTok{\textless{}{-}} \FunctionTok{c}\NormalTok{(}\StringTok{"TRUE"} \OtherTok{=} \StringTok{"red"}\NormalTok{, }\StringTok{"FALSE"} \OtherTok{=} \StringTok{"white"}\NormalTok{)}
\NormalTok{\}}

\CommentTok{\# Create a heatmap}
\FunctionTok{pheatmap}\NormalTok{(corr,}
  \AttributeTok{cluster\_cols =}\NormalTok{ F, }\AttributeTok{cluster\_rows =}\NormalTok{ F,}
  \AttributeTok{annotation\_col =}\NormalTok{ bicluster\_columns,}
  \AttributeTok{annotation\_row =}\NormalTok{ bicluster\_rows,}
  \AttributeTok{annotation\_colors =}\NormalTok{ annotation\_colors}
\NormalTok{)}
\end{Highlighting}
\end{Shaded}

\includegraphics{24_clustering_files/figure-latex/biclust_8-1.pdf}

\hypertarget{taxa-vs-taxa}{%
\subsection{Taxa vs taxa}\label{taxa-vs-taxa}}

Third and final example deals with situation where we want to analyze
correlation between taxa. \emph{biclust} is suitable for this.

\begin{Shaded}
\begin{Highlighting}[]
\CommentTok{\# Calculate cross{-}correlation}
\NormalTok{corr }\OtherTok{\textless{}{-}} \FunctionTok{getExperimentCrossCorrelation}\NormalTok{(mae, }\DecValTok{1}\NormalTok{, }\DecValTok{1}\NormalTok{,}
  \AttributeTok{assay.type1 =} \StringTok{"rclr"}\NormalTok{, }\AttributeTok{assay.type2 =} \StringTok{"rclr"}\NormalTok{,}
  \AttributeTok{mode =} \StringTok{"matrix"}\NormalTok{,}
  \AttributeTok{cor\_threshold =} \FloatTok{0.2}\NormalTok{, }\AttributeTok{verbose =}\NormalTok{ F, }\AttributeTok{show\_warning =}\NormalTok{ F}
\NormalTok{)}

\CommentTok{\# Find biclusters}
\FunctionTok{library}\NormalTok{(biclust)}
\NormalTok{bc }\OtherTok{\textless{}{-}} \FunctionTok{biclust}\NormalTok{(corr,}
  \AttributeTok{method =} \FunctionTok{BCPlaid}\NormalTok{(), }\AttributeTok{fit.model =}\NormalTok{ y }\SpecialCharTok{\textasciitilde{}}\NormalTok{ m,}
  \AttributeTok{background =} \ConstantTok{TRUE}\NormalTok{, }\AttributeTok{shuffle =} \DecValTok{100}\NormalTok{, }\AttributeTok{back.fit =} \DecValTok{0}\NormalTok{, }\AttributeTok{max.layers =} \DecValTok{10}\NormalTok{,}
  \AttributeTok{iter.startup =} \DecValTok{10}\NormalTok{, }\AttributeTok{iter.layer =} \DecValTok{100}\NormalTok{, }\AttributeTok{verbose =} \ConstantTok{FALSE}
\NormalTok{)}
\end{Highlighting}
\end{Shaded}

\begin{Shaded}
\begin{Highlighting}[]
\CommentTok{\# Get biclusters}
\NormalTok{bcs }\OtherTok{\textless{}{-}} \FunctionTok{.get\_biclusters\_from\_biclust}\NormalTok{(bc, corr)}

\NormalTok{bicluster\_rows }\OtherTok{\textless{}{-}}\NormalTok{ bcs}\SpecialCharTok{$}\NormalTok{bc\_rows}
\NormalTok{bicluster\_columns }\OtherTok{\textless{}{-}}\NormalTok{ bcs}\SpecialCharTok{$}\NormalTok{bc\_columns}
\end{Highlighting}
\end{Shaded}

\begin{Shaded}
\begin{Highlighting}[]
\CommentTok{\# Create a column that combines information}
\CommentTok{\# If row/column includes in multiple clusters, cluster numbers are separated with "\_\&\_"}
\NormalTok{bicluster\_columns}\SpecialCharTok{$}\NormalTok{clusters }\OtherTok{\textless{}{-}} \FunctionTok{apply}\NormalTok{(}
\NormalTok{  bicluster\_columns, }\DecValTok{1}\NormalTok{,}
  \ControlFlowTok{function}\NormalTok{(x) \{}
    \FunctionTok{paste}\NormalTok{(}\FunctionTok{paste}\NormalTok{(}\FunctionTok{which}\NormalTok{(x)), }\AttributeTok{collapse =} \StringTok{"\_\&\_"}\NormalTok{)}
\NormalTok{  \}}
\NormalTok{)}
\NormalTok{bicluster\_columns }\OtherTok{\textless{}{-}}\NormalTok{ bicluster\_columns[, }\StringTok{"clusters"}\NormalTok{, drop }\OtherTok{=} \ConstantTok{FALSE}\NormalTok{]}

\NormalTok{bicluster\_rows}\SpecialCharTok{$}\NormalTok{clusters }\OtherTok{\textless{}{-}} \FunctionTok{apply}\NormalTok{(}
\NormalTok{  bicluster\_rows, }\DecValTok{1}\NormalTok{,}
  \ControlFlowTok{function}\NormalTok{(x) \{}
    \FunctionTok{paste}\NormalTok{(}\FunctionTok{paste}\NormalTok{(}\FunctionTok{which}\NormalTok{(x)), }\AttributeTok{collapse =} \StringTok{"\_\&\_"}\NormalTok{)}
\NormalTok{  \}}
\NormalTok{)}
\NormalTok{bicluster\_rows }\OtherTok{\textless{}{-}}\NormalTok{ bicluster\_rows[, }\StringTok{"clusters"}\NormalTok{, drop }\OtherTok{=} \ConstantTok{FALSE}\NormalTok{]}
\end{Highlighting}
\end{Shaded}

\begin{Shaded}
\begin{Highlighting}[]
\CommentTok{\# Convert boolean values into factor}
\NormalTok{bicluster\_columns }\OtherTok{\textless{}{-}} \FunctionTok{data.frame}\NormalTok{(}\FunctionTok{apply}\NormalTok{(bicluster\_columns, }\DecValTok{2}\NormalTok{, as.factor))}
\NormalTok{bicluster\_rows }\OtherTok{\textless{}{-}} \FunctionTok{data.frame}\NormalTok{(}\FunctionTok{apply}\NormalTok{(bicluster\_rows, }\DecValTok{2}\NormalTok{, as.factor))}

\FunctionTok{pheatmap}\NormalTok{(corr,}
  \AttributeTok{cluster\_cols =}\NormalTok{ F, }\AttributeTok{cluster\_rows =}\NormalTok{ F,}
  \AttributeTok{annotation\_col =}\NormalTok{ bicluster\_columns,}
  \AttributeTok{annotation\_row =}\NormalTok{ bicluster\_rows}
\NormalTok{)}
\end{Highlighting}
\end{Shaded}

\includegraphics{24_clustering_files/figure-latex/biclust_12-1.pdf}

\hypertarget{additional-community-typing}{%
\section{Additional Community Typing}\label{additional-community-typing}}

For more community typing techniques applied to the `SprockettTHData'
data set, see the attached .Rmd file.

Link:

\begin{itemize}
\tightlist
\item
  \href{add-comm-typing.Rmd}{Rmd}
\end{itemize}

\hypertarget{differential-abundance}{%
\chapter{Differential abundance}\label{differential-abundance}}

\hypertarget{differential-abundance-analysis}{%
\section{Differential abundance analysis}\label{differential-abundance-analysis}}

This section provides an overview and examples of \emph{differential
abundance analysis (DAA)} based on one of the \href{https://microbiome.github.io/mia/reference/mia-datasets.html}{openly available
datasets}
in mia to illustrate how to perform differential abundance analysis
(DAA). DAA identifies differences in the abundances of individual
taxonomic groups between two or more groups (e.g.~treatment vs
control). This can be performed at any phylogenetic level.

We perform DAA to identify biomarkers and/or gain understanding of a
complex system by looking at its isolated components. For example,
identifying that a bacterial taxon is different between a patient
group with disease \emph{X} vs a healthy control group might lead to
important insights into the pathophysiology. Changes in the microbiota
might be cause or a consequence of a disease. Either way, it can
help to understand the system as a whole. Be aware that this approach
has also been criticized recently \citep{Quinn2021}.

\hypertarget{examples-and-tools}{%
\subsection{Examples and tools}\label{examples-and-tools}}

Due to the complex data characteristics of microbiome sequencing data,
differential abundance analysis of microbiome data faces many
statistical challenges \citep{Yang2022}, including:

\begin{itemize}
\item
  Highly variable. The abundance of a specific taxon could range over
  several orders of magnitude.
\item
  Zero-inflated. In a typical microbiome dataset, more than 70\% of the
  values are zeros. Zeros could be due to either physical absence
  (structural zeros) or insufficient sampling effort (sampling zeros).
\item
  Compositional. Increase or decrease in the (absolute) abundance of
  one taxon at the sampling site will lead to apparent changes in the
  relative abundances of other taxa in the sample.
\end{itemize}

As summarized in \citet{Yang2022}, to address the above statistical
chanllenegs:

\begin{itemize}
\item
  Over-dispersed count models has been proposed to address zero
  inflation, such as the negative binomial model used by edgeR
  \citep{Chen2016} and DESeq2 \citep{Love2014}, the beta-binomial model used by
  corncorb \citep{Martin2021}.
\item
  Zero-inflated mixture models has aslo been proposed to address zero
  inflation, such as zero-inflated log-normal/normal mixture model
  used by metagenomeSeq \citep{Paulson2017} and RAIDA \citep{Sohn2015},
  zero-inflated beta-binomial model used by ZIBB \citep{ZIBB2018}, and
  zero-inflated negative binomial model used by Omnibus
  \citep{Omnibus2018}.
\item
  Bayesian methods have been used to impute the zeros for methods
  working on proportion data, accounting for sampling variability and
  sequencing depth variation. Examples include ALDEx2 \citep{Gloor2016} and
  eBay \citep{Liu2020}.
\item
  Other methods use the pseudo-count approach to impute the zeros,
  such as MaAsLin2 \citep{Mallick2020} and ANCOMBC \citep{ancombc2020}.
\item
  Different strategies have been used to address compositional
  effects, including:

  \begin{itemize}
  \item
    Robust normalization. For example, trimmed mean of M-values (TMM)
    normalization used by edgeR, relative log expression (RLE)
    normalization used by DESeq2 \citep{Love2014}, cumulative sum scaling
    (CSS) normalization used by metagenomeSeq, centered log-ratio
    transformation (CLR) normalization used by ALDEx2 \citep{Gloor2016} and
    geometric mean of pairwise ratios (GMPR) normalization used by
    Omnibus \citep{Omnibus2018}. Wrench normalization \citep{Kumar2018} corrects
    the compositional bias by an empirical Bayes approach, which has
    been recommended in metagenomeSeq \citep{Paulson2017} tutorial.
  \item
    Reference taxa approach used by DACOMP \citep{Brill2019} and RAIDA \citep{Sohn2015}.
  \item
    Analyzing the pattern of pairwise log ratios, such as ANCOM \citep{Mandal2015}.
  \item
    Bias-correction used by ANCOMBC \citep{ancombc2020}.
  \end{itemize}
\end{itemize}

Some of the popular tools for differential abundance analysis include:

\begin{itemize}
\tightlist
\item
  ALDEx2 \citep{Gloor2016}
\item
  ANCOMBC \citep{ancombc2020}
\item
  corncob \citep{Martin2021}
\item
  DESeq2 \citep{Love2014}
\item
  edgeR \citep{Chen2016}
\item
  lefser \citep{Khlebrodova2021}
\item
  MaAsLin2 \citep{Mallick2020}
\item
  metagenomeSeq \citep{Paulson2017}
\item
  limma \citep{Ritchie2015}
\item
  LinDA \citep{Zhou2022}
\item
  ZicoSeq \citep{Yang2022}
\item
  LDM \citep{Hu2020}
\item
  RAIDA \citep{Sohn2015}
\item
  DACOMP \citep{Brill2019}
\item
  Omnibus \citep{Omnibus2018}
\item
  eBay \citep{Liu2020}
\item
  ZINQ \citep{Ling2021}
\item
  ANCOM \citep{Mandal2015}
\item
  fastANCOM \citep{fastANCOM2022}
\item
  \href{https://www.rdocumentation.org/packages/stats/versions/3.6.2/topics/t.test}{t-test}\\
\item
  \href{https://www.rdocumentation.org/packages/stats/versions/3.6.2/topics/wilcox.test}{Wilcoxon test}
\end{itemize}

We recommend to have a look at \citet{Nearing2022} who compared all these
methods across 38 different datasets. Because different methods use
different approaches (parametric vs non-parametric, different
normalization techiniques, assumptions etc.), the results may differ
between methods, sometimes substantially as \citet{Nearing2022} pointed
out. More recently \citet{Yang2022} comprehensively evaluated these methods
via a semi-parametric framework and 106 real datasets. \citet{Yang2022} also
concluded that different DA methods can sometimes produce discordant
results, opening to the possibility for cherry-picking tools in favor
of one's own hypothesis. Therefore it is highly recommended to pick
several methods to assess how robust and potentially reproducible your
findings are with different methods.

In this section we demonstrate the use of four methods that can be
recommended based on recent literature (ANCOM-BC \citep{ancombc2020},
\emph{ALDEx2} \citep{Gloor2016}, \emph{Maaslin2} \citep{Mallick2020}, \emph{LinDA} \citep{Zhou2022}
and \emph{ZicoSeq} \citep{Yang2022}).

The purpose of this section is to show how to perform DAA in R, not
how to correctly do causal inference. Depending on your experimental
setup and your theory, you must determine how to specify any model
exactly. E.g., there might be confounding factors that might drive
(the absence of) differences between the shown groups that we ignore
here for simplicity. Or your dataset is repeated sampling design,
matched-pair design or the general longitudianl design. We will
demonstrate how to include covariates in those models. We picked a
dataset that merely has microbial abundances in a TSE object as well
as a grouping variable in the sample data. We simplify the examples by
only including two of the three groups.

\begin{Shaded}
\begin{Highlighting}[]
\FunctionTok{library}\NormalTok{(mia)}
\FunctionTok{library}\NormalTok{(patchwork)}
\FunctionTok{library}\NormalTok{(tidySummarizedExperiment)}
\FunctionTok{library}\NormalTok{(knitr)}
\FunctionTok{library}\NormalTok{(tidyverse)}
\FunctionTok{library}\NormalTok{(ALDEx2)}
\FunctionTok{library}\NormalTok{(Maaslin2)}
\FunctionTok{library}\NormalTok{(MicrobiomeStat)}
\FunctionTok{library}\NormalTok{(ANCOMBC)}
\FunctionTok{library}\NormalTok{(GUniFrac)}

\CommentTok{\# set random seed because some tools can randomly vary and then produce }
\CommentTok{\# different results:}
\FunctionTok{set.seed}\NormalTok{(}\DecValTok{13253}\NormalTok{)}

\CommentTok{\# we use a demo dataset and restrict it to two geo locations}
\CommentTok{\# for easy illustration}
\FunctionTok{data}\NormalTok{(peerj13075)}
\NormalTok{tse0 }\OtherTok{\textless{}{-}}\NormalTok{ peerj13075}
\NormalTok{tse0 }\OtherTok{\textless{}{-}}\NormalTok{ tse0[ ,tse0}\SpecialCharTok{$}\NormalTok{Geographical\_location }\SpecialCharTok{\%in\%} \FunctionTok{c}\NormalTok{(}\StringTok{"Pune"}\NormalTok{, }\StringTok{"Nashik"}\NormalTok{)]}
\CommentTok{\# Let us make this a factor}
\NormalTok{tse0}\SpecialCharTok{$}\NormalTok{Geographical\_location }\OtherTok{\textless{}{-}} \FunctionTok{factor}\NormalTok{(tse0}\SpecialCharTok{$}\NormalTok{Geographical\_location)}

\CommentTok{\# how many observations do we have per group?}
\FunctionTok{as.data.frame}\NormalTok{(}\FunctionTok{colData}\NormalTok{(tse0)) }\SpecialCharTok{\%\textgreater{}\%} 
\FunctionTok{count}\NormalTok{(Geographical\_location) }\SpecialCharTok{\%\textgreater{}\%}
  \FunctionTok{kable}\NormalTok{()}
\end{Highlighting}
\end{Shaded}

\begin{tabular}{l|r}
\hline
Geographical\_location & n\\
\hline
Nashik & 11\\
\hline
Pune & 36\\
\hline
\end{tabular}

\hypertarget{prevalence-filtering}{%
\subsection{Prevalence Filtering}\label{prevalence-filtering}}

Before we jump to our analyses, we may want to perform some data
manipulation.

Let us here do aggregation to genus level, add relative abundance
assay, and perform prevalence filtering.

\begin{Shaded}
\begin{Highlighting}[]
\NormalTok{tse }\OtherTok{\textless{}{-}} \FunctionTok{agglomerateByRank}\NormalTok{(tse0, }\AttributeTok{rank =} \StringTok{"genus"}\NormalTok{) }\SpecialCharTok{\%\textgreater{}\%}
       \FunctionTok{transformCounts}\NormalTok{(}\AttributeTok{assay.type =} \StringTok{"counts"}\NormalTok{,}
                       \AttributeTok{method =} \StringTok{"relabundance"}\NormalTok{,}
               \AttributeTok{MARGIN =} \StringTok{"samples"}\NormalTok{) }\SpecialCharTok{\%\textgreater{}\%}
       \CommentTok{\# subset based on the relative abundance assay              }
       \FunctionTok{subsetByPrevalentTaxa}\NormalTok{(}\AttributeTok{detection =} \DecValTok{0}\NormalTok{,}
                             \AttributeTok{prevalence =} \DecValTok{10}\SpecialCharTok{/}\DecValTok{100}\NormalTok{,}
                 \AttributeTok{assay.type =} \StringTok{"relabundance"}\NormalTok{)}

\CommentTok{\# Add also clr abundances}
\NormalTok{tse }\OtherTok{\textless{}{-}} \FunctionTok{transformCounts}\NormalTok{(tse, }\AttributeTok{method=}\StringTok{"clr"}\NormalTok{, }\AttributeTok{pseudocount=}\DecValTok{1}\NormalTok{) }\CommentTok{\# not bale to run}
\end{Highlighting}
\end{Shaded}

Regarding prevalence filtering, \citet{Nearing2022} found that applying a 10\%
threshold for the prevalence of the taxa generally resulted in more
robust results. Some tools have builtin arguments for that. By
applying the threshold to our input data, we can make sure it is
applied for all tools.

\hypertarget{aldex2}{%
\subsection{ALDEx2}\label{aldex2}}

In this section, we will show how to perform a simple ALDEx2 analysis.
If you wanted to pick a single method, this method could be recommended to use.
According to the developers experience, it tends to identify the common
features identified by other methods. This statement is in line with a recent
independent evaluation by \citet{Nearing2022}.\\
Please also have a look at the more extensive
\href{https://bioconductor.org/packages/release/bioc/vignettes/ALDEx2/inst/doc/ALDEx2_vignette.html}{vignette}
that covers this flexible tool in more depth. ALDEx2 estimates technical
variation within each sample per taxon by utilizing the Dirichlet distribution.
It furthermore applies the centered-log-ratio transformation (or closely
related log-ratio transforms). Depending on the experimental setup, it will
perform a two sample Welch's T-test and Wilcoxon-test or a one-way ANOVA and
Kruskal-Wallis-test. For more complex study designs, there is a possibility to
utilize the \texttt{glm} functionality within ALDEx2.

The Benjamini-Hochberg procedure is applied by default to correct for
multiple testing. Below we show a simple example that illustrates the
workflow.

\begin{Shaded}
\begin{Highlighting}[]
\CommentTok{\# Generate Monte Carlo samples of the Dirichlet distribution for each sample.}
\CommentTok{\# Convert each instance using the centered log{-}ratio transform.}
\CommentTok{\# This is the input for all further analyses.}
\FunctionTok{set.seed}\NormalTok{(}\DecValTok{254}\NormalTok{)}
\NormalTok{x }\OtherTok{\textless{}{-}} \FunctionTok{aldex.clr}\NormalTok{(}\FunctionTok{assay}\NormalTok{(tse), tse}\SpecialCharTok{$}\NormalTok{Geographical\_location)     }
\end{Highlighting}
\end{Shaded}

The t-test:

\begin{Shaded}
\begin{Highlighting}[]
\CommentTok{\# calculates expected values of the Welch\textquotesingle{}s t{-}test and Wilcoxon rank}
\CommentTok{\# test on the data returned by aldex.clr}
\NormalTok{x\_tt }\OtherTok{\textless{}{-}} \FunctionTok{aldex.ttest}\NormalTok{(x, }\AttributeTok{paired.test =} \ConstantTok{FALSE}\NormalTok{, }\AttributeTok{verbose =} \ConstantTok{FALSE}\NormalTok{)}
\end{Highlighting}
\end{Shaded}

Effect sizes:

\begin{Shaded}
\begin{Highlighting}[]
\CommentTok{\# Determines the median clr abundance of the feature in all samples and in}
\CommentTok{\# groups, the median difference between the two groups, the median variation}
\CommentTok{\# within each group and the effect size, which is the median of the ratio}
\CommentTok{\# of the between group difference and the larger of the variance within groups}
\NormalTok{x\_effect }\OtherTok{\textless{}{-}} \FunctionTok{aldex.effect}\NormalTok{(x, }\AttributeTok{CI =} \ConstantTok{TRUE}\NormalTok{, }\AttributeTok{verbose =} \ConstantTok{FALSE}\NormalTok{)}

\CommentTok{\# combine all outputs }
\NormalTok{aldex\_out }\OtherTok{\textless{}{-}} \FunctionTok{data.frame}\NormalTok{(x\_tt, x\_effect)}
\end{Highlighting}
\end{Shaded}

Now, we can create a so called Bland-Altman or MA plot (left). It shows the
association between the relative abundance and the magnitude of the difference
per sample. Next to that, we can also create a plot that shows the dispersion
on the x-axis instead of log-ratio abundance. Red dots represent genera that are
differentially abundant (\(q \leq 0.1\)) between the 2 groups. Black points are
rare taxa and grey ones are abundant taxa. The dashed line represent an effect
size of 1. See \citet{Gloor2016} to learn more about these plots.

\begin{Shaded}
\begin{Highlighting}[]
\FunctionTok{par}\NormalTok{(}\AttributeTok{mfrow =} \FunctionTok{c}\NormalTok{(}\DecValTok{1}\NormalTok{, }\DecValTok{2}\NormalTok{))}
  \FunctionTok{aldex.plot}\NormalTok{(}
\NormalTok{    aldex\_out, }
    \AttributeTok{type =} \StringTok{"MA"}\NormalTok{, }
    \AttributeTok{test =} \StringTok{"welch"}\NormalTok{, }
    \AttributeTok{xlab =} \StringTok{"Log{-}ratio abundance"}\NormalTok{,}
    \AttributeTok{ylab =} \StringTok{"Difference"}\NormalTok{,}
    \AttributeTok{cutoff =} \FloatTok{0.05}
\NormalTok{  )}
  \FunctionTok{aldex.plot}\NormalTok{(}
\NormalTok{    aldex\_out, }
    \AttributeTok{type =} \StringTok{"MW"}\NormalTok{, }
    \AttributeTok{test =} \StringTok{"welch"}\NormalTok{,}
    \AttributeTok{xlab =} \StringTok{"Dispersion"}\NormalTok{,}
    \AttributeTok{ylab =} \StringTok{"Difference"}\NormalTok{,}
    \AttributeTok{cutoff =} \FloatTok{0.05}
\NormalTok{  )}
\end{Highlighting}
\end{Shaded}

\includegraphics{30_differential_abundance_files/figure-latex/unnamed-chunk-1-1.pdf}

The evaluation as differential abundant in above plots is based on the
corrected p-value. According to the ALDEx2 developers, the safest
approach is to identify those features where the 95\% CI of the effect
size does not cross 0. As we can see in below table, this is not the
case for any of the identified genera (see overlap column, which
indicates the proportion of overlap). Also, the authors recommend to
focus on effect sizes and CIs rather than interpreting the p-value. To
keep the comparison simple, we will here use the p-value as decision
criterion. But please be aware that the effect size together with the
CI is a better answer to the question we are typically interested in
(see also \href{https://www.nature.com/articles/d41586-019-00857-9}{this
article}).

\begin{Shaded}
\begin{Highlighting}[]
\FunctionTok{rownames\_to\_column}\NormalTok{(aldex\_out, }\StringTok{"genus"}\NormalTok{) }\SpecialCharTok{\%\textgreater{}\%}
  \FunctionTok{filter}\NormalTok{(wi.eBH }\SpecialCharTok{\textless{}=} \FloatTok{0.05}\NormalTok{)  }\SpecialCharTok{\%\textgreater{}\%} \CommentTok{\# here we chose the wilcoxon output rather than tt}
\NormalTok{  dplyr}\SpecialCharTok{::}\FunctionTok{select}\NormalTok{(genus, we.eBH, wi.eBH, effect, overlap) }\SpecialCharTok{\%\textgreater{}\%}
  \FunctionTok{kable}\NormalTok{()}
\end{Highlighting}
\end{Shaded}

\begin{tabular}{l|r|r|r|r}
\hline
genus & we.eBH & wi.eBH & effect & overlap\\
\hline
Anaerococcus & 0.0540 & 0.0150 & 0.9595 & 0.1546\\
\hline
Calditerricola & 0.0769 & 0.0299 & -0.7162 & 0.1702\\
\hline
Chitinivibrio & 0.1216 & 0.0484 & -0.7700 & 0.1776\\
\hline
Corynebacterium & 0.0280 & 0.0035 & 1.1857 & 0.1037\\
\hline
Desulfosporomusa & 0.0851 & 0.0359 & -0.8604 & 0.1733\\
\hline
Geobacillus & 0.0370 & 0.0081 & -1.0962 & 0.1293\\
\hline
Jeotgalicoccus & 0.0276 & 0.0251 & -0.9052 & 0.1676\\
\hline
Paenibacillus & 0.0837 & 0.0345 & -0.9380 & 0.1932\\
\hline
Virgibacillus & 0.1103 & 0.0443 & -0.8750 & 0.1960\\
\hline
\end{tabular}

\hypertarget{ancom-bc}{%
\subsection{ANCOM-BC}\label{ancom-bc}}

The analysis of composition of microbiomes with bias correction
(ANCOM-BC) \citep{ancombc2020} is a recently developed method for differential
abundance testing. It is based on an earlier published approach
\citep{Mandal2015}. The previous version of ANCOM was among the methods
that produced the most consistent results and is probably a
conservative approach \citep{Nearing2022}. However, the new ANCOM-BC
method operates quite differently compared to the former ANCOM method.

As the only method, ANCOM-BC incorporates the so called \emph{sampling
fraction} into the model. The latter term could be empirically
estimated by the ratio of the library size to the microbial
load. According to the authors, variations in this sampling fraction
would bias differential abundance analyses if ignored. Furthermore,
this method provides p-values and confidence intervals for each
taxon. It also controls the FDR and it is computationally simple to
implement.

Note that the original method was implemented in the \texttt{ancombc()} function (see
\href{https://www.bioconductor.org/packages/release/bioc/vignettes/ANCOMBC/inst/doc/ANCOMBC.html}{extended tutorial}).
The method has since then been updated and new features have been added to enable
multi-group comparisons and repeated measurements among other improvements.
We do not cover the more advanced features of ANCOMBC in this tutorial
as these features are documented in detail in this
\href{https://www.bioconductor.org/packages/release/bioc/vignettes/ANCOMBC/inst/doc/ANCOMBC2.html}{tutorial}.

We now proceed with a simple example. First, we specify a formula. In this
formula, other covariates could potentially be included to adjust for
confounding. We show this further below. Again, please make sure to check the
\href{https://rdrr.io/github/FrederickHuangLin/ANCOMBC/man/ancombc.html}{function documentation}
as well as the linked tutorials to learn about the additional arguments
that we specify.

\begin{Shaded}
\begin{Highlighting}[]
\CommentTok{\# Run ANCOM{-}BC }
\NormalTok{out }\OtherTok{\textless{}{-}} \FunctionTok{ancombc2}\NormalTok{(}
  \AttributeTok{data =}\NormalTok{ tse,}
  \AttributeTok{assay\_name =} \StringTok{"counts"}\NormalTok{, }
  \AttributeTok{tax\_level =} \StringTok{"genus"}\NormalTok{, }
  \AttributeTok{fix\_formula =} \StringTok{"Geographical\_location"}\NormalTok{, }
  \AttributeTok{p\_adj\_method =} \StringTok{"fdr"}\NormalTok{, }
  \AttributeTok{lib\_cut =} \DecValTok{0}\NormalTok{,}
  \AttributeTok{prv\_cut =} \DecValTok{0}\NormalTok{,}
  \AttributeTok{group =} \StringTok{"Geographical\_location"}\NormalTok{, }
  \AttributeTok{struc\_zero =} \ConstantTok{TRUE}\NormalTok{, }
  \AttributeTok{neg\_lb =} \ConstantTok{TRUE}\NormalTok{,}
  \AttributeTok{alpha =} \FloatTok{0.05}\NormalTok{, }
  \AttributeTok{global =} \ConstantTok{TRUE} \CommentTok{\# multi group comparison will be deactivated automatically }
\NormalTok{)}
\end{Highlighting}
\end{Shaded}

\begin{Shaded}
\begin{Highlighting}[]
\CommentTok{\# store the FDR adjusted results [test on v2.0.3] }
\NormalTok{ancombc\_result }\OtherTok{\textless{}{-}} \FunctionTok{cbind.data.frame}\NormalTok{(}\AttributeTok{taxid =}\NormalTok{ out}\SpecialCharTok{$}\NormalTok{res}\SpecialCharTok{$}\NormalTok{taxon,}
                       \AttributeTok{ancombc =} \FunctionTok{as.vector}\NormalTok{(out}\SpecialCharTok{$}\NormalTok{res}\SpecialCharTok{$}\NormalTok{q\_Geographical\_locationPune))}
\end{Highlighting}
\end{Shaded}

\begin{Shaded}
\begin{Highlighting}[]
\CommentTok{\# store the FDR adjusted results [test on v1.2.2]}
\NormalTok{ancombc\_result }\OtherTok{\textless{}{-}}\NormalTok{ out}\SpecialCharTok{$}\NormalTok{res }\SpecialCharTok{\%\textgreater{}\%}\NormalTok{ dplyr}\SpecialCharTok{::}\FunctionTok{select}\NormalTok{(}\FunctionTok{starts\_with}\NormalTok{(}\FunctionTok{c}\NormalTok{(}\StringTok{"taxon"}\NormalTok{, }\StringTok{"lfc"}\NormalTok{, }\StringTok{"q"}\NormalTok{)))}
\end{Highlighting}
\end{Shaded}

The object \texttt{out} contains all model output. Again, see the
\href{https://rdrr.io/github/FrederickHuangLin/ANCOMBC/man/ancombc.html}{documentation of the
function}
under \textbf{Value} for details. Our question whether taxa are
differentially abundant can be answered by looking at the \texttt{res}
object, which contains dataframes with the coefficients, standard
errors, p-values and q-values. Below we show the first entries of this
dataframe.

\begin{Shaded}
\begin{Highlighting}[]
\FunctionTok{kable}\NormalTok{(}\FunctionTok{head}\NormalTok{(ancombc\_result))}
\end{Highlighting}
\end{Shaded}

\begin{tabular}{l|r|r|r|r}
\hline
taxon & lfc\_(Intercept) & lfc\_Geographical\_locationPune & q\_(Intercept) & q\_Geographical\_locationPune\\
\hline
Abyssicoccus & 0.0397 & -0.0569 & 0.8731 & 0.8468\\
\hline
Acidaminococcus & 0.6872 & -0.9022 & 0.0032 & 0.0004\\
\hline
Acinetobacter & 0.1241 & -0.1671 & 0.8972 & 0.8698\\
\hline
Actinomyces & 0.1345 & -0.1806 & 0.6548 & 0.5551\\
\hline
Actinoplanes & 0.2713 & -0.3593 & 0.2760 & 0.1476\\
\hline
Aerococcus & 0.0234 & -0.0357 & 0.8985 & 0.8698\\
\hline
\end{tabular}

\hypertarget{maaslin2}{%
\subsection{MaAsLin2}\label{maaslin2}}

Let us next illustrate MaAsLin2 \citep{Mallick2020}. This method is based on
generalized linear models and flexible for different study designs
and covariate structures. For details, check their
\href{https://github.com/biobakery/biobakery/wiki/maaslin2}{Biobakery tutorial}.

\begin{Shaded}
\begin{Highlighting}[]
\CommentTok{\# maaslin expects features as columns and samples as rows }
\CommentTok{\# for both the abundance table as well as metadata }

\CommentTok{\# We can specify different GLMs/normalizations/transforms.}
\CommentTok{\# Let us use similar settings as in Nearing et al. (2021):}
\NormalTok{maaslin2\_out }\OtherTok{\textless{}{-}} \FunctionTok{Maaslin2}\NormalTok{(}
  \FunctionTok{t}\NormalTok{(}\FunctionTok{assay}\NormalTok{(tse)),}
  \FunctionTok{data.frame}\NormalTok{(}\FunctionTok{colData}\NormalTok{(tse)),}
  \AttributeTok{output =} \StringTok{"DAA example"}\NormalTok{,}
  \AttributeTok{transform =} \StringTok{"AST"}\NormalTok{,}
  \AttributeTok{fixed\_effects =} \StringTok{"Geographical\_location"}\NormalTok{,}
  \CommentTok{\# random\_effects = c(...), \# you can also fit MLM by specifying random effects}
  \CommentTok{\# specifying a ref is especially important if you have more than 2 levels}
  \AttributeTok{reference =} \StringTok{"Geographical\_location,Pune"}\NormalTok{,  }
  \AttributeTok{normalization =} \StringTok{"TSS"}\NormalTok{,}
  \AttributeTok{standardize =} \ConstantTok{FALSE}\NormalTok{,}
  \AttributeTok{min\_prevalence =} \DecValTok{0} \CommentTok{\# prev filterin already done}
\NormalTok{)}
\end{Highlighting}
\end{Shaded}

Which genera are identified as differentially abundant? (leave out ``head'' to see all).

\begin{Shaded}
\begin{Highlighting}[]
\FunctionTok{kable}\NormalTok{(}\FunctionTok{head}\NormalTok{(}\FunctionTok{filter}\NormalTok{(maaslin2\_out}\SpecialCharTok{$}\NormalTok{results, qval }\SpecialCharTok{\textless{}=} \FloatTok{0.05}\NormalTok{)))}
\end{Highlighting}
\end{Shaded}

\begin{tabular}{l|l|l|r|r|r|l|r|r|r}
\hline
feature & metadata & value & coef & stderr & pval & name & qval & N & N.not.zero\\
\hline
Fructobacillus & Geographical\_location & Nashik & 0.0080 & 0.0011 & 0 & Geographical\_locationNashik & 0 & 47 & 9\\
\hline
Desulfosporomusa & Geographical\_location & Nashik & 0.0373 & 0.0059 & 0 & Geographical\_locationNashik & 0 & 47 & 13\\
\hline
Geobacillus & Geographical\_location & Nashik & 0.1294 & 0.0207 & 0 & Geographical\_locationNashik & 0 & 47 & 27\\
\hline
Pullulanibacillus & Geographical\_location & Nashik & 0.0395 & 0.0062 & 0 & Geographical\_locationNashik & 0 & 47 & 9\\
\hline
Chitinivibrio & Geographical\_location & Nashik & 0.0274 & 0.0045 & 0 & Geographical\_locationNashik & 0 & 47 & 10\\
\hline
Thermoanaerobacter & Geographical\_location & Nashik & 0.0089 & 0.0015 & 0 & Geographical\_locationNashik & 0 & 47 & 10\\
\hline
\end{tabular}

This will create a folder that is called like in the output specified
above. It contains also figures to visualize difference between
significant taxa.

\hypertarget{linda}{%
\subsection{LinDA}\label{linda}}

Lastly, we cover linear models for differential abundance analysis of
microbiome compositional data (\citet{Zhou2022}). This is very similar to
ANCOMBC with few differences: 1) LinDA correct for the compositional
bias differently using the mode of all regression coefficients. 2) it
is faster (100x-1000x than ANCOMBC and according to the authors); 3)
it supports hierarchical models. The latest ANCOMBC versions are also
supporting hierarchical models. Nevertheless, LinDA seems a promising
tool that achieves a very good power/fdr trade-off together with
ANCOMBC according to the review. The speed improvements might make it
critical especially for datasets that have higher sample or feature
set sizes.

\begin{Shaded}
\begin{Highlighting}[]
\NormalTok{meta }\OtherTok{\textless{}{-}} \FunctionTok{as.data.frame}\NormalTok{(}\FunctionTok{colData}\NormalTok{(tse)) }\SpecialCharTok{\%\textgreater{}\%}\NormalTok{ dplyr}\SpecialCharTok{::}\FunctionTok{select}\NormalTok{(Geographical\_location)}
\NormalTok{linda.res }\OtherTok{\textless{}{-}} \FunctionTok{linda}\NormalTok{(}
  \FunctionTok{as.data.frame}\NormalTok{(}\FunctionTok{assay}\NormalTok{(tse)), }
\NormalTok{  meta, }
  \AttributeTok{formula =} \StringTok{\textquotesingle{}\textasciitilde{}Geographical\_location\textquotesingle{}}\NormalTok{, }
  \AttributeTok{alpha =} \FloatTok{0.05}\NormalTok{, }
  \AttributeTok{prev.filter =} \DecValTok{0}\NormalTok{, }
  \AttributeTok{mean.abund.filter =} \DecValTok{0}\NormalTok{)}
\end{Highlighting}
\end{Shaded}

\begin{verbatim}
## 0  features are filtered!
## The filtered data has  47  samples and  262  features will be tested!
## Pseudo-count approach is used.
## Fit linear models ...
## Completed.
\end{verbatim}

\begin{Shaded}
\begin{Highlighting}[]
\NormalTok{linda\_out }\OtherTok{\textless{}{-}}\NormalTok{ linda.res}\SpecialCharTok{$}\NormalTok{output}\SpecialCharTok{$}\NormalTok{Geographical\_locationPune}
\end{Highlighting}
\end{Shaded}

\begin{Shaded}
\begin{Highlighting}[]
\CommentTok{\# to scan the table for genera where H0 could be rejected:}
\FunctionTok{kable}\NormalTok{(}\FunctionTok{head}\NormalTok{(}\FunctionTok{filter}\NormalTok{(}\FunctionTok{as.data.frame}\NormalTok{(linda\_out), reject)))}
\end{Highlighting}
\end{Shaded}

\begin{tabular}{l|r|r|r|r|r|r|l|r}
\hline
  & baseMean & log2FoldChange & lfcSE & stat & pvalue & padj & reject & df\\
\hline
Acidaminococcus & 1194.7 & -1.9084 & 0.3579 & -5.332 & 0.0000 & 0.0000 & TRUE & 45\\
\hline
Aciditerrimonas & 393.5 & -0.6655 & 0.2184 & -3.048 & 0.0039 & 0.0165 & TRUE & 45\\
\hline
Actinomadura & 836.9 & -1.7985 & 0.3596 & -5.001 & 0.0000 & 0.0001 & TRUE & 45\\
\hline
Agromyces & 938.1 & -1.7530 & 0.3910 & -4.483 & 0.0001 & 0.0004 & TRUE & 45\\
\hline
Aminivibrio & 416.9 & -0.7489 & 0.2352 & -3.185 & 0.0026 & 0.0128 & TRUE & 45\\
\hline
Amycolatopsis & 556.1 & -0.9299 & 0.3320 & -2.801 & 0.0075 & 0.0302 & TRUE & 45\\
\hline
\end{tabular}

\hypertarget{zicoseq}{%
\subsection{ZicoSeq}\label{zicoseq}}

Subsequently, we add a linear model and permutation-based method, see
details at \href{https://cran.r-project.org/web/packages/GUniFrac/vignettes/ZicoSeq.html}{tutorial}.

This approach has been assessed to exhibit high power and a low false
discovery rate, which has the following components:

\begin{enumerate}
\def\labelenumi{\arabic{enumi}.}
\item
  Winsorization to decrease the influence of outliers;
\item
  Posterior sampling based on a beta mixture prior to address
  sampling variability and zero inflation;
\item
  Reference-based multiple-stage normalization to address
  compositional effects;
\end{enumerate}

\begin{Shaded}
\begin{Highlighting}[]
\FunctionTok{set.seed}\NormalTok{(}\DecValTok{123}\NormalTok{)}
\NormalTok{meta }\OtherTok{\textless{}{-}} \FunctionTok{as.data.frame}\NormalTok{(}\FunctionTok{colData}\NormalTok{(tse))}
\NormalTok{zicoseq.obj }\OtherTok{\textless{}{-}}\NormalTok{ GUniFrac}\SpecialCharTok{::}\FunctionTok{ZicoSeq}\NormalTok{(}\AttributeTok{meta.dat =}\NormalTok{ meta, }
                                 \AttributeTok{feature.dat =} \FunctionTok{as.matrix}\NormalTok{(}\FunctionTok{assay}\NormalTok{(tse)),}
                                 \AttributeTok{grp.name =} \StringTok{\textquotesingle{}Geographical\_location\textquotesingle{}}\NormalTok{,}
                                 \AttributeTok{adj.name =} \ConstantTok{NULL}\NormalTok{, }
                                 \AttributeTok{feature.dat.type =} \StringTok{\textquotesingle{}count\textquotesingle{}}\NormalTok{,}
                                 \AttributeTok{prev.filter =} \DecValTok{0}\NormalTok{,}
                                 \AttributeTok{perm.no =} \DecValTok{999}\NormalTok{,}
                                 \AttributeTok{mean.abund.filter =} \DecValTok{0}\NormalTok{,}
                                 \AttributeTok{max.abund.filter =} \DecValTok{0}\NormalTok{,}
                                 \AttributeTok{return.feature.dat =}\NormalTok{ T)}
\end{Highlighting}
\end{Shaded}

\begin{verbatim}
## 0  features are filtered!
## The data has  47  samples and  262  features will be tested!
## On average,  1  outlier counts will be replaced for each feature!
## Fitting beta mixture ...
## Finding the references ...
## Permutation testing ...
## ...................................................................................................
## ...................................................................................................
## ...................................................................................................
## ...................................................................................................
## ...................................................................................................
## ...................................................................................................
## Completed!
\end{verbatim}

\begin{Shaded}
\begin{Highlighting}[]
\NormalTok{zicoseq\_out }\OtherTok{\textless{}{-}} \FunctionTok{cbind.data.frame}\NormalTok{(}\AttributeTok{p.raw=}\NormalTok{zicoseq.obj}\SpecialCharTok{$}\NormalTok{p.raw, }\AttributeTok{p.adj.fdr=}\NormalTok{zicoseq.obj}\SpecialCharTok{$}\NormalTok{p.adj.fdr) }
\end{Highlighting}
\end{Shaded}

\begin{Shaded}
\begin{Highlighting}[]
\FunctionTok{kable}\NormalTok{(}\FunctionTok{head}\NormalTok{(}\FunctionTok{filter}\NormalTok{(zicoseq\_out, p.adj.fdr}\SpecialCharTok{\textless{}}\FloatTok{0.05}\NormalTok{)))}
\end{Highlighting}
\end{Shaded}

\begin{tabular}{l|r|r}
\hline
  & p.raw & p.adj.fdr\\
\hline
Actinomadura & 0.001 & 0.0001\\
\hline
Agromyces & 0.001 & 0.0038\\
\hline
Aneurinibacillus & 0.001 & 0.0071\\
\hline
Anoxybacillus & 0.001 & 0.0001\\
\hline
Chitinispirillum & 0.001 & 0.0100\\
\hline
Chitinivibrio & 0.001 & 0.0001\\
\hline
\end{tabular}

\begin{Shaded}
\begin{Highlighting}[]
\DocumentationTok{\#\# x{-}axis is the effect size: R2 * direction of coefficient}
\FunctionTok{ZicoSeq.plot}\NormalTok{(}\AttributeTok{ZicoSeq.obj =}\NormalTok{ zicoseq.obj,}
             \AttributeTok{meta.dat =}\NormalTok{ meta,}
         \AttributeTok{pvalue.type =}\StringTok{\textquotesingle{}p.adj.fdr\textquotesingle{}}\NormalTok{)}
\end{Highlighting}
\end{Shaded}

\includegraphics{30_differential_abundance_files/figure-latex/ZicoSeqplot-1.pdf}

\hypertarget{comparison-of-methods}{%
\subsection{Comparison of methods}\label{comparison-of-methods}}

The different methods yield somewhat different results but they could
be also expected to overlap to a substantial degree. As an exercirse,
you can compare the outcomes between the different methods in terms of
effect sizes, significances, or other aspects that are comparable
between the methods.

\hypertarget{confounding-variables}{%
\section{Confounding variables}\label{confounding-variables}}

Confounders are common in experimental research. In general, these can be
classified into 3 types:

\begin{itemize}
\item
  Biological confounder, such as age, sex, etc.
\item
  Technical confounder that caused by data collection, storage, DNA
  extraction, sequencing process, etc.
\item
  Confounder caused by experimental models, such as cage effect,
  sample background, etc.
\end{itemize}

Adjusting confounder is necessary and important to reach a valid
conclusion. To perform causal inference, it is crucial that the method
is able to include covariates in the model. This is not possible with
e.g.~the Wilcoxon test. Other methods such as DESeq2, edgeR, ANCOMBC,
LDM, Aldex2, Corncob, MaAsLin2, ZicoSeq, fastANCOM and ZINQ allow
this. Below we show how to include a confounder/covariate in ANCOMBC,
LinDA and ZicoSeq.

\hypertarget{ancombc}{%
\subsection{ANCOMBC}\label{ancombc}}

\begin{Shaded}
\begin{Highlighting}[]
\CommentTok{\# perform the analysis }
\NormalTok{ancombc\_cov }\OtherTok{\textless{}{-}} \FunctionTok{ancombc2}\NormalTok{(}
  \AttributeTok{data =}\NormalTok{ tse,}
  \AttributeTok{assay\_name =} \StringTok{"counts"}\NormalTok{,}
  \AttributeTok{tax\_level =} \StringTok{"genus"}\NormalTok{,}
  \AttributeTok{fix\_formula =} \StringTok{"Geographical\_location + Age"}\NormalTok{, }
  \AttributeTok{p\_adj\_method =} \StringTok{"fdr"}\NormalTok{, }
  \AttributeTok{lib\_cut =} \DecValTok{0}\NormalTok{, }
  \AttributeTok{group =} \StringTok{"Geographical\_location"}\NormalTok{, }
  \AttributeTok{struc\_zero =} \ConstantTok{TRUE}\NormalTok{, }
  \AttributeTok{neg\_lb =} \ConstantTok{TRUE}\NormalTok{,}
  \AttributeTok{alpha =} \FloatTok{0.05}\NormalTok{, }
  \AttributeTok{global =} \ConstantTok{TRUE} \CommentTok{\# multi group comparison will be deactivated automatically }
\NormalTok{)}
\CommentTok{\# now the model answers the question: holding Age constant, are }
\CommentTok{\# bacterial taxa differentially abundant? Or, if that is of interest,}
\CommentTok{\# holding phenotype constant, is Age associated with bacterial abundance?}
\CommentTok{\# Again we only show the first 6 entries.}
\end{Highlighting}
\end{Shaded}

\begin{Shaded}
\begin{Highlighting}[]
\NormalTok{tab }\OtherTok{\textless{}{-}}\NormalTok{ ancombc\_cov}\SpecialCharTok{$}\NormalTok{res }\SpecialCharTok{\%\textgreater{}\%}\NormalTok{ dplyr}\SpecialCharTok{::}\FunctionTok{select}\NormalTok{(}\FunctionTok{starts\_with}\NormalTok{(}\FunctionTok{c}\NormalTok{(}\StringTok{"taxon"}\NormalTok{, }\StringTok{"lfc"}\NormalTok{, }\StringTok{"q"}\NormalTok{)))}
\FunctionTok{kable}\NormalTok{(}\FunctionTok{head}\NormalTok{(tab))}
\end{Highlighting}
\end{Shaded}

\begin{tabular}{l|r|r|r|r|r|r|r|r}
\hline
taxon & lfc\_(Intercept) & lfc\_Geographical\_locationPune & lfc\_AgeElderly & lfc\_AgeMiddle\_age & q\_(Intercept) & q\_Geographical\_locationPune & q\_AgeElderly & q\_AgeMiddle\_age\\
\hline
Abyssicoccus & 0.0395 & -0.0947 & 0.0895 & 0.0127 & 0.8841 & 0.9271 & 0.9118 & 0.9948\\
\hline
Acidaminococcus & 0.7025 & -0.7557 & -0.2431 & -0.1576 & 0.0032 & 0.0388 & 0.8895 & 0.9774\\
\hline
Acinetobacter & 0.0202 & -1.0765 & 1.4995 & 1.1535 & 0.9880 & 0.6353 & 0.7718 & 0.7886\\
\hline
Actinomyces & 0.1762 & 0.1239 & -0.4581 & -0.4485 & 0.5453 & 0.8893 & 0.5720 & 0.6120\\
\hline
Actinoplanes & 0.3027 & -0.1600 & -0.2766 & -0.3340 & 0.2111 & 0.7623 & 0.8167 & 0.6120\\
\hline
Aerococcus & 0.0021 & -0.1441 & 0.1355 & 0.2462 & 0.9904 & 0.8209 & 0.8895 & 0.7886\\
\hline
\end{tabular}

\hypertarget{linda-1}{%
\subsection{LinDA}\label{linda-1}}

\begin{Shaded}
\begin{Highlighting}[]
\NormalTok{linda\_cov }\OtherTok{\textless{}{-}} \FunctionTok{linda}\NormalTok{(}
  \FunctionTok{as.data.frame}\NormalTok{(}\FunctionTok{assay}\NormalTok{(tse, }\StringTok{"counts"}\NormalTok{)), }
  \FunctionTok{as.data.frame}\NormalTok{(}\FunctionTok{colData}\NormalTok{(tse)), }
  \AttributeTok{formula =} \StringTok{\textquotesingle{}\textasciitilde{} Geographical\_location + Age\textquotesingle{}}\NormalTok{, }
  \AttributeTok{alpha =} \FloatTok{0.05}\NormalTok{, }
  \AttributeTok{prev.filter =} \DecValTok{0}\NormalTok{, }
  \AttributeTok{mean.abund.filter =} \DecValTok{0}\NormalTok{)}
\end{Highlighting}
\end{Shaded}

\begin{verbatim}
## 0  features are filtered!
## The filtered data has  47  samples and  262  features will be tested!
## Pseudo-count approach is used.
## Fit linear models ...
## Completed.
\end{verbatim}

\begin{Shaded}
\begin{Highlighting}[]
\NormalTok{linda.res }\OtherTok{\textless{}{-}}\NormalTok{ linda\_cov}\SpecialCharTok{$}\NormalTok{output}\SpecialCharTok{$}\NormalTok{Geographical\_locationPune}
\end{Highlighting}
\end{Shaded}

\begin{Shaded}
\begin{Highlighting}[]
\FunctionTok{kable}\NormalTok{(}\FunctionTok{head}\NormalTok{(}\FunctionTok{filter}\NormalTok{(linda.res, reject}\SpecialCharTok{==}\NormalTok{T)))}
\end{Highlighting}
\end{Shaded}

\begin{tabular}{l|r|r|r|r|r|r|l|r}
\hline
  & baseMean & log2FoldChange & lfcSE & stat & pvalue & padj & reject & df\\
\hline
Acidaminococcus & 1140.5 & -1.438 & 0.4390 & -3.276 & 0.0021 & 0.0182 & TRUE & 43\\
\hline
Actinomadura & 804.7 & -1.520 & 0.4477 & -3.394 & 0.0015 & 0.0139 & TRUE & 43\\
\hline
Anaerococcus & 939.0 & 6.795 & 1.3978 & 4.861 & 0.0000 & 0.0004 & TRUE & 43\\
\hline
Aneurinibacillus & 988.0 & -1.687 & 0.5426 & -3.110 & 0.0033 & 0.0256 & TRUE & 43\\
\hline
Anoxybacillus & 1712.8 & -2.727 & 0.5574 & -4.893 & 0.0000 & 0.0004 & TRUE & 43\\
\hline
Brachybacterium & 403.0 & 2.060 & 0.7303 & 2.821 & 0.0072 & 0.0450 & TRUE & 43\\
\hline
\end{tabular}

\hypertarget{zicoseq-1}{%
\subsection{ZicoSeq}\label{zicoseq-1}}

\begin{Shaded}
\begin{Highlighting}[]
\FunctionTok{set.seed}\NormalTok{(}\DecValTok{123}\NormalTok{)}
\NormalTok{zicoseq.obj }\OtherTok{\textless{}{-}}\NormalTok{ GUniFrac}\SpecialCharTok{::}\FunctionTok{ZicoSeq}\NormalTok{(}\AttributeTok{meta.dat =} \FunctionTok{as.data.frame}\NormalTok{(}\FunctionTok{colData}\NormalTok{(tse)) , }
                                 \AttributeTok{feature.dat =} \FunctionTok{as.matrix}\NormalTok{(}\FunctionTok{assay}\NormalTok{(tse)),}
                                 \AttributeTok{grp.name =} \StringTok{\textquotesingle{}Geographical\_location\textquotesingle{}}\NormalTok{,}
                                 \AttributeTok{adj.name =} \StringTok{\textquotesingle{}Gender\textquotesingle{}}\NormalTok{, }
                                 \AttributeTok{feature.dat.type =} \StringTok{\textquotesingle{}count\textquotesingle{}}\NormalTok{,}
                                 \AttributeTok{prev.filter =} \DecValTok{0}\NormalTok{,}
                                 \AttributeTok{perm.no =} \DecValTok{999}\NormalTok{,}
                                 \AttributeTok{mean.abund.filter =} \DecValTok{0}\NormalTok{,}
                                 \AttributeTok{max.abund.filter =} \DecValTok{0}\NormalTok{,}
                                 \AttributeTok{return.feature.dat =}\NormalTok{ T)}
\end{Highlighting}
\end{Shaded}

\begin{verbatim}
## 0  features are filtered!
## The data has  47  samples and  262  features will be tested!
## On average,  1  outlier counts will be replaced for each feature!
## Fitting beta mixture ...
## Finding the references ...
## Permutation testing ...
## ...................................................................................................
## ...................................................................................................
## ...................................................................................................
## ...................................................................................................
## ...................................................................................................
## ...................................................................................................
## Completed!
\end{verbatim}

\begin{Shaded}
\begin{Highlighting}[]
\NormalTok{zicoseq\_out }\OtherTok{\textless{}{-}} \FunctionTok{cbind.data.frame}\NormalTok{(}\AttributeTok{p.raw=}\NormalTok{zicoseq.obj}\SpecialCharTok{$}\NormalTok{p.raw,}
                                \AttributeTok{p.adj.fdr=}\NormalTok{zicoseq.obj}\SpecialCharTok{$}\NormalTok{p.adj.fdr) }
\end{Highlighting}
\end{Shaded}

\begin{Shaded}
\begin{Highlighting}[]
\FunctionTok{kable}\NormalTok{(}\FunctionTok{head}\NormalTok{(}\FunctionTok{filter}\NormalTok{(zicoseq\_out, p.adj.fdr}\SpecialCharTok{\textless{}}\FloatTok{0.05}\NormalTok{)))}
\end{Highlighting}
\end{Shaded}

\begin{tabular}{l|r|r}
\hline
  & p.raw & p.adj.fdr\\
\hline
Actinomadura & 0.001 & 0.0004\\
\hline
Agromyces & 0.001 & 0.0030\\
\hline
Aneurinibacillus & 0.001 & 0.0036\\
\hline
Anoxybacillus & 0.001 & 0.0001\\
\hline
Chitinispirillum & 0.001 & 0.0089\\
\hline
Chitinivibrio & 0.001 & 0.0001\\
\hline
\end{tabular}

\hypertarget{tree-based-methods}{%
\section{Tree-based methods}\label{tree-based-methods}}

Let us next cover phylogeny-aware methods to perform group-wise
associations.

\hypertarget{group-wise-associations-testing-based-on-balances}{%
\subsection{Group-wise associations testing based on balances}\label{group-wise-associations-testing-based-on-balances}}

For testing associations based on balances, check the philr
R/Bioconductor package.

\hypertarget{machine_learning}{%
\chapter{Machine learning}\label{machine_learning}}

Machine learning (ML) is a part of artificial intelligence. There are multiple
definitions, but ``machine'' refers to computation and ``learning'' to improving
performance based on the data by finding patterns from it. Machine learning
includes wide variety of methods from simple statistical methods to more
complex methods such as neural-networks.

Machine learning can be divided into supervised and unsupervised machine learning.
Supervised ML is used to predict outcome based on the data. Unsupervised ML is used,
for example, to reduce dimensionality (e.g.~PCA) and to find clusters from the
data (e.g., k-means clustering).

\hypertarget{supervised-machine-learning}{%
\section{Supervised machine learning}\label{supervised-machine-learning}}

``Supervised'' means that the training data is introduced before. The training data
contains labels (e.g., patient status), and the model is fitted based on the
training data. After fitting, the model is utilized to predict labels of data whose
labels are not known.

\begin{Shaded}
\begin{Highlighting}[]
\FunctionTok{library}\NormalTok{(mia)}

\CommentTok{\# Load experimental data}
\FunctionTok{data}\NormalTok{(peerj13075, }\AttributeTok{package=}\StringTok{"mia"}\NormalTok{)}
\NormalTok{tse }\OtherTok{\textless{}{-}}\NormalTok{ peerj13075}
\end{Highlighting}
\end{Shaded}

Let's first preprocess the data.

\begin{Shaded}
\begin{Highlighting}[]
\CommentTok{\# Agglomerate data}
\NormalTok{tse }\OtherTok{\textless{}{-}} \FunctionTok{agglomerateByRank}\NormalTok{(tse, }\AttributeTok{rank =} \StringTok{"order"}\NormalTok{)}

\CommentTok{\# Apply CLR transform}
\NormalTok{tse }\OtherTok{\textless{}{-}} \FunctionTok{transformCounts}\NormalTok{(tse, }\AttributeTok{assay.type =} \StringTok{"counts"}\NormalTok{, }\AttributeTok{method =} \StringTok{"clr"}\NormalTok{,}
                       \AttributeTok{MARGIN=}\StringTok{"samples"}\NormalTok{, }\AttributeTok{pseudocount=}\DecValTok{1}\NormalTok{)}

\CommentTok{\# Get assay}
\NormalTok{assay }\OtherTok{\textless{}{-}} \FunctionTok{assay}\NormalTok{(tse, }\StringTok{"clr"}\NormalTok{)}
\CommentTok{\# Transpose assay}
\NormalTok{assay }\OtherTok{\textless{}{-}} \FunctionTok{t}\NormalTok{(assay)}

\CommentTok{\# Convert into data.frame}
\NormalTok{df }\OtherTok{\textless{}{-}} \FunctionTok{as.data.frame}\NormalTok{(assay)}

\CommentTok{\# Add labels to assay}
\NormalTok{labels }\OtherTok{\textless{}{-}} \FunctionTok{colData}\NormalTok{(tse)}\SpecialCharTok{$}\NormalTok{Diet}
\NormalTok{labels }\OtherTok{\textless{}{-}} \FunctionTok{as.factor}\NormalTok{(labels)}
\NormalTok{df}\SpecialCharTok{$}\NormalTok{diet }\OtherTok{\textless{}{-}}\NormalTok{ labels }

\NormalTok{df[}\DecValTok{5}\NormalTok{, }\DecValTok{5}\NormalTok{]}
\end{Highlighting}
\end{Shaded}

\begin{verbatim}
## [1] -0.4612
\end{verbatim}

In the example below, we use \href{https://journals.asm.org/doi/10.1128/mBio.00434-20}{mikropml}
package. We try to predict the diet type based on the data.

\begin{Shaded}
\begin{Highlighting}[]
\FunctionTok{library}\NormalTok{(mikropml)}

\CommentTok{\# Run random forest }
\NormalTok{results }\OtherTok{\textless{}{-}} \FunctionTok{run\_ml}\NormalTok{(df, }\StringTok{"rf"}\NormalTok{, }\AttributeTok{outcome\_colname =} \StringTok{"diet"}\NormalTok{, }
                  \AttributeTok{kfold =} \DecValTok{2}\NormalTok{, }\AttributeTok{cv\_times =} \DecValTok{5}\NormalTok{, }\AttributeTok{training\_frac =} \FloatTok{0.8}\NormalTok{)}

\CommentTok{\# Print result}
\FunctionTok{confusionMatrix}\NormalTok{(}\AttributeTok{data =}\NormalTok{ results}\SpecialCharTok{$}\NormalTok{trained\_model}\SpecialCharTok{$}\NormalTok{finalModel}\SpecialCharTok{$}\NormalTok{predicted, }
                \AttributeTok{reference =}\NormalTok{ results}\SpecialCharTok{$}\NormalTok{trained\_model}\SpecialCharTok{$}\NormalTok{finalModel}\SpecialCharTok{$}\NormalTok{y)}
\end{Highlighting}
\end{Shaded}

\begin{verbatim}
## Confusion Matrix and Statistics
## 
##           Reference
## Prediction Mixed Veg
##      Mixed    12  10
##      Veg      11  14
##                                         
##                Accuracy : 0.553         
##                  95% CI : (0.401, 0.698)
##     No Information Rate : 0.511         
##     P-Value [Acc > NIR] : 0.331         
##                                         
##                   Kappa : 0.105         
##                                         
##  Mcnemar's Test P-Value : 1.000         
##                                         
##             Sensitivity : 0.522         
##             Specificity : 0.583         
##          Pos Pred Value : 0.545         
##          Neg Pred Value : 0.560         
##              Prevalence : 0.489         
##          Detection Rate : 0.255         
##    Detection Prevalence : 0.468         
##       Balanced Accuracy : 0.553         
##                                         
##        'Positive' Class : Mixed         
## 
\end{verbatim}

mikropml offers easier interface to \href{https://cran.r-project.org/web/packages/caret/index.html}{caret}
package. However, we can also use it directly.

Let's use xgboost model which is another commonly used algorithm in bioinformatics.

\begin{Shaded}
\begin{Highlighting}[]
\CommentTok{\# Set seed for reproducibility}
\FunctionTok{set.seed}\NormalTok{(}\DecValTok{6358}\NormalTok{)}

\CommentTok{\# Specify train control}
\NormalTok{train\_control }\OtherTok{\textless{}{-}} \FunctionTok{trainControl}\NormalTok{(}\AttributeTok{method =} \StringTok{"cv"}\NormalTok{, }\AttributeTok{number =} \DecValTok{5}\NormalTok{,}
                              \AttributeTok{classProbs =} \ConstantTok{TRUE}\NormalTok{, }
                              \AttributeTok{savePredictions =} \StringTok{"final"}\NormalTok{,}
                              \AttributeTok{allowParallel =} \ConstantTok{TRUE}\NormalTok{)}

\CommentTok{\# Specify hyperparameter tuning grid}
\NormalTok{tune\_grid }\OtherTok{\textless{}{-}} \FunctionTok{expand.grid}\NormalTok{(}\AttributeTok{nrounds =} \FunctionTok{c}\NormalTok{(}\DecValTok{50}\NormalTok{, }\DecValTok{100}\NormalTok{, }\DecValTok{200}\NormalTok{),}
                         \AttributeTok{max\_depth =} \FunctionTok{c}\NormalTok{(}\DecValTok{6}\NormalTok{, }\DecValTok{8}\NormalTok{, }\DecValTok{10}\NormalTok{),}
                         \AttributeTok{colsample\_bytree =} \FunctionTok{c}\NormalTok{(}\FloatTok{0.6}\NormalTok{, }\FloatTok{0.8}\NormalTok{, }\DecValTok{1}\NormalTok{),}
                         \AttributeTok{eta =} \FunctionTok{c}\NormalTok{(}\FloatTok{0.1}\NormalTok{, }\FloatTok{0.3}\NormalTok{),}
                         \AttributeTok{gamma =} \DecValTok{0}\NormalTok{,}
                         \AttributeTok{min\_child\_weight =} \FunctionTok{c}\NormalTok{(}\DecValTok{3}\NormalTok{, }\DecValTok{4}\NormalTok{, }\DecValTok{5}\NormalTok{),}
                         \AttributeTok{subsample =} \FunctionTok{c}\NormalTok{(}\FloatTok{0.6}\NormalTok{, }\FloatTok{0.8}\NormalTok{)}
\NormalTok{                         )}

\CommentTok{\# Train the model, use LOOCV to evaluate performance}
\NormalTok{model }\OtherTok{\textless{}{-}} \FunctionTok{train}\NormalTok{(}\AttributeTok{x =}\NormalTok{ assay, }
               \AttributeTok{y =}\NormalTok{ labels, }
               \AttributeTok{method =} \StringTok{"xgbTree"}\NormalTok{,}
               \AttributeTok{objective =} \StringTok{"binary:logistic"}\NormalTok{,}
               \AttributeTok{trControl =}\NormalTok{ train\_control,}
               \AttributeTok{tuneGrid =}\NormalTok{ tune\_grid,}
               \AttributeTok{metric =} \StringTok{"AUC"}\NormalTok{,}
               \AttributeTok{verbosity =} \DecValTok{0}
\NormalTok{)}
\end{Highlighting}
\end{Shaded}

Let's create ROC curve which is a commonly used method in binary classification.
For unbalanced data, you might want to plot precision-recall curve.

\begin{Shaded}
\begin{Highlighting}[]
\FunctionTok{library}\NormalTok{(MLeval)}

\CommentTok{\# Calculate different evaluation metrics}
\NormalTok{res }\OtherTok{\textless{}{-}} \FunctionTok{evalm}\NormalTok{(model, }\AttributeTok{showplots =} \ConstantTok{FALSE}\NormalTok{)}

\CommentTok{\# Use patchwork to plot ROC and precision{-}recall curve side{-}by{-}side}
\FunctionTok{library}\NormalTok{(patchwork)}
\NormalTok{res}\SpecialCharTok{$}\NormalTok{roc }\SpecialCharTok{+}\NormalTok{ res}\SpecialCharTok{$}\NormalTok{proc }\SpecialCharTok{+} 
    \FunctionTok{plot\_layout}\NormalTok{(}\AttributeTok{guides =} \StringTok{"collect"}\NormalTok{) }\SpecialCharTok{\&} \FunctionTok{theme}\NormalTok{(}\AttributeTok{legend.position =} \StringTok{\textquotesingle{}bottom\textquotesingle{}}\NormalTok{)}
\end{Highlighting}
\end{Shaded}

\includegraphics{40_machine_learning_files/figure-latex/super5-1.pdf}

\hypertarget{unsupervised-machine-learning}{%
\section{Unsupervised machine learning}\label{unsupervised-machine-learning}}

``Unsupervised'' means that the labels (e.g., patient status is not known),
and patterns are learned based only the abundance table, for instance.
Unsupervised ML is also known as a data mining where patterns are extracted
from big datasets.

For unsupervised machine learning, please refer to chapters that are listed below:

\begin{itemize}
\tightlist
\item
  Chapter \ref{clustering}
\item
  Chapter \ref{community-similarity}
\end{itemize}

\hypertarget{multi-assay-analyses}{%
\chapter{Multi-assay analyses}\label{multi-assay-analyses}}

\begin{Shaded}
\begin{Highlighting}[]
\FunctionTok{library}\NormalTok{(mia)}
\end{Highlighting}
\end{Shaded}

Multi-omics approaches integrate data from multiple sources. For
example, we can integrate taxonomic abundance profiles with
metabolomic or other biomolecular profiling data to observe
associations, make predictions, or aim at causal
inferences. Integrating evidence across multiple sources can lead to
enhanced predictions, more holistic understanding, or facilitate the
discovery of novel biomarkers. In this section we demonstrate common
multi-assay data integration tasks.

Cross-correlation analysis is a straightforward approach that can
reveal assocation strengths and types between data sets. For instance,
we can analyze if higher presence of a specific taxon equals to higher
levels of a biomolecule.

The analyse can be facilitated by the multi-assay data containers,
\emph{TreeSummarizedExperiment} and \emph{MultiAssayExperiment}. These are
scalable and contain different types of data in a single container,
making this framework particularly suited for multi-assay microbiome
data incorporating different types of complementary data sources in a
single reproducible workflow. Th different solutions for varying data
integration needs are discussed in more detail in Section
\ref{containers}. Another experiment can be stored in \emph{altExp} slot
of SE data container or both experiments can be stored side-by-side in
MAE data container (see the sections \ref{alt-exp} and \ref{mae} to
learn more about altExp and MAE objects, respectively). Different
experiments are first imported into these data containers similarly to
the case when only one experiment is present. After that, the
different experiments can be combined into the same multi-assay data
container. The result is one TreeSE object with alternative experiment in
altExp slot, or MAE object with multiple experiment in its experiment
slot, for instance.

As an example data, we use data from the following publication: Hintikka L
\emph{et al.} (2021) Xylo-oligosaccharides in prevention of hepatic
steatosis and adipose tissue inflammation: associating taxonomic and
metabolomic patterns in fecal microbiota with biclustering

\citep{Hintikka2021}.

In this study, mice were fed with high-fat and low-fat diets with or
without prebiotics. The purpose of this was to study whether prebiotics
reduce negative impacts of a high-fat diet.

This example data can be loaded from microbiomeDataSets. The data is
already in MAE format. It includes three different experiments:
microbial abundance data, metabolite concentrations, and data about
different biomarkers.
=======
\citep{Hintikka2021}. In this article, mice were fed with high-fat and
low-fat diets with or without prebiotics. The purpose of this was to
study if prebiotics would reduce the negative impact of high-fat diet.

This example data is readily available in the MultiAssayExperiment
format. It includes three different experiments: microbial abundance
data, metabolite concentrations, and data about different
biomarkers. If you like to construct the same data object from the
original files instead, \href{https://microbiome.github.io/OMA/containers.html\#loading-experimental-microbiome-data}{Here}
you can find help for importing data into a SE object.

\begin{Shaded}
\begin{Highlighting}[]
\CommentTok{\# Load the data}
\FunctionTok{data}\NormalTok{(HintikkaXOData, }\AttributeTok{package =} \StringTok{"mia"}\NormalTok{)}
\NormalTok{mae }\OtherTok{\textless{}{-}}\NormalTok{ HintikkaXOData}
\NormalTok{mae}
\end{Highlighting}
\end{Shaded}

\begin{verbatim}
## A MultiAssayExperiment object of 3 listed
##  experiments with user-defined names and respective classes.
##  Containing an ExperimentList class object of length 3:
##  [1] microbiota: TreeSummarizedExperiment with 12706 rows and 40 columns
##  [2] metabolites: TreeSummarizedExperiment with 38 rows and 40 columns
##  [3] biomarkers: TreeSummarizedExperiment with 39 rows and 40 columns
## Functionality:
##  experiments() - obtain the ExperimentList instance
##  colData() - the primary/phenotype DataFrame
##  sampleMap() - the sample coordination DataFrame
##  `$`, `[`, `[[` - extract colData columns, subset, or experiment
##  *Format() - convert into a long or wide DataFrame
##  assays() - convert ExperimentList to a SimpleList of matrices
##  exportClass() - save data to flat files
\end{verbatim}

\begin{Shaded}
\begin{Highlighting}[]
\FunctionTok{library}\NormalTok{(stringr)}
\CommentTok{\# Drop off those bacteria that do not include information in Phylum or lower levels}
\NormalTok{mae[[}\DecValTok{1}\NormalTok{]] }\OtherTok{\textless{}{-}}\NormalTok{ mae[[}\DecValTok{1}\NormalTok{]][}\SpecialCharTok{!}\FunctionTok{is.na}\NormalTok{(}\FunctionTok{rowData}\NormalTok{(mae[[}\DecValTok{1}\NormalTok{]])}\SpecialCharTok{$}\NormalTok{Phylum), ]}
\CommentTok{\# Clean taxonomy data, so that names do not include additional characters}
\FunctionTok{rowData}\NormalTok{(mae[[}\DecValTok{1}\NormalTok{]]) }\OtherTok{\textless{}{-}} \FunctionTok{DataFrame}\NormalTok{(}\FunctionTok{apply}\NormalTok{(}\FunctionTok{rowData}\NormalTok{(mae[[}\DecValTok{1}\NormalTok{]]), }\DecValTok{2}\NormalTok{, }
\NormalTok{                                     str\_remove, }\AttributeTok{pattern =} \StringTok{".\_[0{-}9]\_\_"}\NormalTok{))}
\CommentTok{\# Microbiome data}
\NormalTok{mae[[}\DecValTok{1}\NormalTok{]]}
\end{Highlighting}
\end{Shaded}

\begin{verbatim}
## class: TreeSummarizedExperiment 
## dim: 12613 40 
## metadata(0):
## assays(1): counts
## rownames(12613): GAYR01026362.62.2014 CVJT01000011.50.2173 ...
##   JRJTB:03787:02429 JRJTB:03787:02478
## rowData names(7): Phylum Class ... Species OTU
## colnames(40): C1 C2 ... C39 C40
## colData names(0):
## reducedDimNames(0):
## mainExpName: NULL
## altExpNames(0):
## rowLinks: NULL
## rowTree: NULL
## colLinks: NULL
## colTree: NULL
\end{verbatim}

\begin{Shaded}
\begin{Highlighting}[]
\CommentTok{\# Metabolite data}
\NormalTok{mae[[}\DecValTok{2}\NormalTok{]]}
\end{Highlighting}
\end{Shaded}

\begin{verbatim}
## class: TreeSummarizedExperiment 
## dim: 38 40 
## metadata(0):
## assays(1): nmr
## rownames(38): Butyrate Acetate ... Malonate 1,3-dihydroxyacetone
## rowData names(0):
## colnames(40): C1 C2 ... C39 C40
## colData names(0):
## reducedDimNames(0):
## mainExpName: NULL
## altExpNames(0):
## rowLinks: NULL
## rowTree: NULL
## colLinks: NULL
## colTree: NULL
\end{verbatim}

\begin{Shaded}
\begin{Highlighting}[]
\CommentTok{\# Biomarker data}
\NormalTok{mae[[}\DecValTok{3}\NormalTok{]]}
\end{Highlighting}
\end{Shaded}

\begin{verbatim}
## class: TreeSummarizedExperiment 
## dim: 39 40 
## metadata(0):
## assays(1): signals
## rownames(39): Triglycerides_liver CLSs_epi ... NPY_serum Glycogen_liver
## rowData names(0):
## colnames(40): C1 C2 ... C39 C40
## colData names(0):
## reducedDimNames(0):
## mainExpName: NULL
## altExpNames(0):
## rowLinks: NULL
## rowTree: NULL
## colLinks: NULL
## colTree: NULL
\end{verbatim}

\hypertarget{cross-correlation-analysis}{%
\section{Cross-correlation Analysis}\label{cross-correlation-analysis}}

Next we can do the cross-correlation analysis. Let us analyse if
individual bacteria genera correlates with concentrations of
individual metabolites. This helps to answer the question: ``If this
bacteria is present, is this metabolite's concentration then low or
high''?

\begin{Shaded}
\begin{Highlighting}[]
\CommentTok{\# Agglomerate microbiome data at family level}
\NormalTok{mae[[}\DecValTok{1}\NormalTok{]] }\OtherTok{\textless{}{-}} \FunctionTok{agglomerateByPrevalence}\NormalTok{(mae[[}\DecValTok{1}\NormalTok{]], }\AttributeTok{rank =} \StringTok{"Family"}\NormalTok{)}
\CommentTok{\# Does log10 transform for microbiome data}
\NormalTok{mae[[}\DecValTok{1}\NormalTok{]] }\OtherTok{\textless{}{-}} \FunctionTok{transformCounts}\NormalTok{(mae[[}\DecValTok{1}\NormalTok{]], }\AttributeTok{method =} \StringTok{"log10"}\NormalTok{, }\AttributeTok{pseudocount =} \DecValTok{1}\NormalTok{)}

\CommentTok{\# Give unique names so that we do not have problems when we are creating a plot}
\FunctionTok{rownames}\NormalTok{(mae[[}\DecValTok{1}\NormalTok{]]) }\OtherTok{\textless{}{-}} \FunctionTok{getTaxonomyLabels}\NormalTok{(mae[[}\DecValTok{1}\NormalTok{]])}

\CommentTok{\# Cross correlates data sets}
\NormalTok{correlations }\OtherTok{\textless{}{-}} \FunctionTok{testExperimentCrossCorrelation}\NormalTok{(mae, }
                                               \AttributeTok{experiment1 =} \DecValTok{1}\NormalTok{,}
                                               \AttributeTok{experiment2 =} \DecValTok{2}\NormalTok{,}
                                               \AttributeTok{assay.type1 =} \StringTok{"log10"}\NormalTok{, }
                                               \AttributeTok{assay.type2 =} \StringTok{"nmr"}\NormalTok{,}
                                               \AttributeTok{method =} \StringTok{"spearman"}\NormalTok{, }
                                               \AttributeTok{p\_adj\_threshold =} \ConstantTok{NULL}\NormalTok{,}
                                               \AttributeTok{cor\_threshold =} \ConstantTok{NULL}\NormalTok{,}
                                               \CommentTok{\# Remove when mia is fixed}
                                               \AttributeTok{mode =} \StringTok{"matrix"}\NormalTok{,}
                                               \AttributeTok{sort =} \ConstantTok{TRUE}\NormalTok{,}
                                               \AttributeTok{show\_warnings =} \ConstantTok{FALSE}\NormalTok{)}
\end{Highlighting}
\end{Shaded}

Creates the heatmap

\begin{Shaded}
\begin{Highlighting}[]
\FunctionTok{library}\NormalTok{(}\StringTok{"ComplexHeatmap"}\NormalTok{) }

\CommentTok{\# Create a heatmap and store it}
\NormalTok{plot }\OtherTok{\textless{}{-}} \FunctionTok{Heatmap}\NormalTok{(correlations}\SpecialCharTok{$}\NormalTok{cor,}
                \CommentTok{\# Print values to cells}
                \AttributeTok{cell\_fun =} \ControlFlowTok{function}\NormalTok{(j, i, x, y, width, height, fill) \{}
                    \CommentTok{\# If the p{-}value is under threshold}
                    \ControlFlowTok{if}\NormalTok{( }\SpecialCharTok{!}\FunctionTok{is.na}\NormalTok{(correlations}\SpecialCharTok{$}\NormalTok{p\_adj[i, j]) }\SpecialCharTok{\&}\NormalTok{ correlations}\SpecialCharTok{$}\NormalTok{p\_adj[i, j] }\SpecialCharTok{\textless{}} \FloatTok{0.05}\NormalTok{ )\{}
                        \CommentTok{\# Print "X"}
                        \FunctionTok{grid.text}\NormalTok{(}\FunctionTok{sprintf}\NormalTok{(}\StringTok{"\%s"}\NormalTok{, }\StringTok{"X"}\NormalTok{), x, y, }\AttributeTok{gp =} \FunctionTok{gpar}\NormalTok{(}\AttributeTok{fontsize =} \DecValTok{8}\NormalTok{, }\AttributeTok{col =} \StringTok{"black"}\NormalTok{))}
\NormalTok{                        \}}
\NormalTok{                    \},}
                \AttributeTok{heatmap\_legend\_param =} \FunctionTok{list}\NormalTok{(}\AttributeTok{title =} \StringTok{""}\NormalTok{, }\AttributeTok{legend\_height =} \FunctionTok{unit}\NormalTok{(}\DecValTok{5}\NormalTok{, }\StringTok{"cm"}\NormalTok{))}
\NormalTok{                )}
\NormalTok{plot}
\end{Highlighting}
\end{Shaded}

\includegraphics{23_multi-assay_analyses_files/figure-latex/cross-correlation6-1.pdf}

\hypertarget{multi-omics-factor-analysis}{%
\section{Multi-Omics Factor Analysis}\label{multi-omics-factor-analysis}}

Multi-Omics Factor Analysis \citep{Argelaguet2018} (MOFA) is an
unsupervised method for integrating multi-omic data sets in a
downstream analysis. It could be seen as a generalization of
principal component analysis. Yet, with the ability to infer a latent
(low-dimensional) representation, shared among the multiple (-omics)
data sets in hand.

We use the R \href{https://biofam.github.io/MOFA2/index.html}{MOFA2}
package for the analysis, and
\href{https://biofam.github.io/MOFA2/installation.html}{install} the
corresponding dependencies.

\begin{Shaded}
\begin{Highlighting}[]
\FunctionTok{library}\NormalTok{(MOFA2)}

\CommentTok{\# For inter{-}operability between Python and R, and setting Python dependencies,}
\CommentTok{\# reticulate package is needed}
\FunctionTok{library}\NormalTok{(reticulate)}
\CommentTok{\# Let us assume that these have been installed already.}
\CommentTok{\#reticulate::install\_miniconda(force = TRUE)}
\CommentTok{\#reticulate::use\_miniconda(condaenv = "env1", required = FALSE)}
\CommentTok{\#reticulate::py\_install(packages = c("mofapy2"), pip = TRUE, python\_version=3.6)}
\end{Highlighting}
\end{Shaded}

The \texttt{mae} object could be used straight to create the MOFA model. Yet,
we transform our assays since the model assumes normality per
default. We can also use Poisson or Bernoulli distributions among others.

\begin{Shaded}
\begin{Highlighting}[]
\FunctionTok{library}\NormalTok{(MOFA2)}
\CommentTok{\# For simplicity, classify all high{-}fat diets as high{-}fat, and all the low{-}fat }
\CommentTok{\# diets as low{-}fat diets}
\FunctionTok{colData}\NormalTok{(mae)}\SpecialCharTok{$}\NormalTok{Diet }\OtherTok{\textless{}{-}} \FunctionTok{ifelse}\NormalTok{(}\FunctionTok{colData}\NormalTok{(mae)}\SpecialCharTok{$}\NormalTok{Diet }\SpecialCharTok{==} \StringTok{"High{-}fat"} \SpecialCharTok{|} 
                              \FunctionTok{colData}\NormalTok{(mae)}\SpecialCharTok{$}\NormalTok{Diet }\SpecialCharTok{==} \StringTok{"High{-}fat + XOS"}\NormalTok{, }
                            \StringTok{"High{-}fat"}\NormalTok{, }\StringTok{"Low{-}fat"}\NormalTok{)}

\CommentTok{\# Removing duplicates at the microbiome data}
\CommentTok{\# which are also in form e.g. "Ambiguous" and "uncultured" taxa}
\NormalTok{mae[[}\DecValTok{1}\NormalTok{]] }\OtherTok{\textless{}{-}}\NormalTok{ mae[[}\DecValTok{1}\NormalTok{]][}\SpecialCharTok{!}\FunctionTok{duplicated}\NormalTok{(}\FunctionTok{rownames}\NormalTok{(}\FunctionTok{assay}\NormalTok{(mae[[}\DecValTok{1}\NormalTok{]]))), ]}

\CommentTok{\# Transforming microbiome data with rclr}
\NormalTok{mae[[}\DecValTok{1}\NormalTok{]] }\OtherTok{\textless{}{-}} \FunctionTok{transformCounts}\NormalTok{(mae[[}\DecValTok{1}\NormalTok{]], }\AttributeTok{method =} \StringTok{"relabundance"}\NormalTok{)}
\NormalTok{mae[[}\DecValTok{1}\NormalTok{]] }\OtherTok{\textless{}{-}} \FunctionTok{transformCounts}\NormalTok{(mae[[}\DecValTok{1}\NormalTok{]], }\AttributeTok{assay.type =} \StringTok{"relabundance"}\NormalTok{, }\AttributeTok{method =} \StringTok{"rclr"}\NormalTok{)}

\CommentTok{\# Transforming metabolomic data with log10}
\NormalTok{mae[[}\DecValTok{2}\NormalTok{]] }\OtherTok{\textless{}{-}} \FunctionTok{transformCounts}\NormalTok{(mae[[}\DecValTok{2}\NormalTok{]], }\AttributeTok{assay.type =} \StringTok{"nmr"}\NormalTok{,}
                            \AttributeTok{MARGIN =} \StringTok{"samples"}\NormalTok{,}
                            \AttributeTok{method =} \StringTok{"log10"}\NormalTok{)}

\CommentTok{\# Transforming biomarker data with z{-}transform}
\NormalTok{mae[[}\DecValTok{3}\NormalTok{]] }\OtherTok{\textless{}{-}} \FunctionTok{transformCounts}\NormalTok{(mae[[}\DecValTok{3}\NormalTok{]], }\AttributeTok{assay.type =} \StringTok{"signals"}\NormalTok{,}
                            \AttributeTok{MARGIN =} \StringTok{"features"}\NormalTok{,}
                            \AttributeTok{method =} \StringTok{"z"}\NormalTok{, }\AttributeTok{pseudocount =} \DecValTok{1}\NormalTok{)}

\CommentTok{\# Removing assays no longer needed}
\FunctionTok{assay}\NormalTok{(mae[[}\DecValTok{1}\NormalTok{]], }\StringTok{"counts"}\NormalTok{) }\OtherTok{\textless{}{-}} \ConstantTok{NULL}
\FunctionTok{assay}\NormalTok{(mae[[}\DecValTok{1}\NormalTok{]], }\StringTok{"log10"}\NormalTok{) }\OtherTok{\textless{}{-}} \ConstantTok{NULL}
\FunctionTok{assay}\NormalTok{(mae[[}\DecValTok{2}\NormalTok{]], }\StringTok{"nmr"}\NormalTok{) }\OtherTok{\textless{}{-}} \ConstantTok{NULL}
\FunctionTok{assay}\NormalTok{(mae[[}\DecValTok{3}\NormalTok{]], }\StringTok{"signals"}\NormalTok{) }\OtherTok{\textless{}{-}} \ConstantTok{NULL}

\CommentTok{\# Building our mofa model}
\NormalTok{model }\OtherTok{\textless{}{-}} \FunctionTok{create\_mofa\_from\_MultiAssayExperiment}\NormalTok{(mae,}
                                               \AttributeTok{groups =} \StringTok{"Diet"}\NormalTok{, }
                                               \AttributeTok{extract\_metadata =} \ConstantTok{TRUE}\NormalTok{)}
\NormalTok{model}
\end{Highlighting}
\end{Shaded}

\begin{verbatim}
## Untrained MOFA model with the following characteristics: 
##  Number of views: 3 
##  Views names: microbiota metabolites biomarkers 
##  Number of features (per view): 38 38 39 
##  Number of groups: 2 
##  Groups names: High-fat Low-fat 
##  Number of samples (per group): 20 20 
## 
\end{verbatim}

Model options could be defined as follows:

\begin{Shaded}
\begin{Highlighting}[]
\NormalTok{model\_opts }\OtherTok{\textless{}{-}} \FunctionTok{get\_default\_model\_options}\NormalTok{(model)}
\NormalTok{model\_opts}\SpecialCharTok{$}\NormalTok{num\_factors }\OtherTok{\textless{}{-}} \DecValTok{5}
\FunctionTok{head}\NormalTok{(model\_opts)}
\end{Highlighting}
\end{Shaded}

\begin{verbatim}
## $likelihoods
##  microbiota metabolites  biomarkers 
##  "gaussian"  "gaussian"  "gaussian" 
## 
## $num_factors
## [1] 5
## 
## $spikeslab_factors
## [1] FALSE
## 
## $spikeslab_weights
## [1] FALSE
## 
## $ard_factors
## [1] TRUE
## 
## $ard_weights
## [1] TRUE
\end{verbatim}

Model's training options are defined with the following:

\begin{Shaded}
\begin{Highlighting}[]
\NormalTok{train\_opts }\OtherTok{\textless{}{-}} \FunctionTok{get\_default\_training\_options}\NormalTok{(model)}
\FunctionTok{head}\NormalTok{(train\_opts)}
\end{Highlighting}
\end{Shaded}

\begin{verbatim}
## $maxiter
## [1] 1000
## 
## $convergence_mode
## [1] "fast"
## 
## $drop_factor_threshold
## [1] -1
## 
## $verbose
## [1] FALSE
## 
## $startELBO
## [1] 1
## 
## $freqELBO
## [1] 5
\end{verbatim}

Preparing and training the model:

\begin{Shaded}
\begin{Highlighting}[]
\NormalTok{model.prepared }\OtherTok{\textless{}{-}} \FunctionTok{prepare\_mofa}\NormalTok{(}
  \AttributeTok{object =}\NormalTok{ model,}
  \AttributeTok{model\_options =}\NormalTok{ model\_opts}
\NormalTok{)}

\CommentTok{\# Some systems may require the specification \textasciigrave{}use\_basilisk = TRUE\textasciigrave{}}
\CommentTok{\# so it has been added to the following code}
\NormalTok{model.trained }\OtherTok{\textless{}{-}} \FunctionTok{run\_mofa}\NormalTok{(model.prepared, }\AttributeTok{use\_basilisk =} \ConstantTok{TRUE}\NormalTok{)}
\end{Highlighting}
\end{Shaded}

Visualizing the variance explained:

\begin{Shaded}
\begin{Highlighting}[]
\FunctionTok{library}\NormalTok{(patchwork)}
\FunctionTok{library}\NormalTok{(ggplot2)}
\FunctionTok{wrap\_plots}\NormalTok{(}
    \FunctionTok{plot\_variance\_explained}\NormalTok{(model.trained, }\AttributeTok{x=}\StringTok{"view"}\NormalTok{, }\AttributeTok{y=}\StringTok{"factor"}\NormalTok{, }\AttributeTok{plot\_total =}\NormalTok{ T),}
    \AttributeTok{nrow =} \DecValTok{2}
\NormalTok{) }\SpecialCharTok{+} \FunctionTok{plot\_annotation}\NormalTok{(}\AttributeTok{title =} \StringTok{"Variance Explained per factor and assay"}\NormalTok{,}
                    \AttributeTok{theme =} \FunctionTok{theme}\NormalTok{(}\AttributeTok{plot.title =} \FunctionTok{element\_text}\NormalTok{(}\AttributeTok{hjust =} \FloatTok{0.5}\NormalTok{)))}
\end{Highlighting}
\end{Shaded}

\includegraphics{23_multi-assay_analyses_files/figure-latex/unnamed-chunk-5-1.pdf}

The top weights for each assay using all five factors:

\begin{Shaded}
\begin{Highlighting}[]
\NormalTok{plots }\OtherTok{\textless{}{-}} \FunctionTok{lapply}\NormalTok{(}\FunctionTok{c}\NormalTok{(}\StringTok{"microbiota"}\NormalTok{, }\StringTok{"metabolites"}\NormalTok{,}\StringTok{"biomarkers"}\NormalTok{), }\ControlFlowTok{function}\NormalTok{(name) \{}
    \FunctionTok{plot\_top\_weights}\NormalTok{(model.trained,}
                     \AttributeTok{view =}\NormalTok{ name,}
                     \AttributeTok{factors =} \StringTok{"all"}\NormalTok{,}
                     \AttributeTok{nfeatures =} \DecValTok{10}\NormalTok{) }\SpecialCharTok{+}
        \FunctionTok{labs}\NormalTok{(}\AttributeTok{title =} \FunctionTok{paste0}\NormalTok{(}\StringTok{"Top weights of the "}\NormalTok{, name,}\StringTok{" assay"}\NormalTok{))}
\NormalTok{\})}
\FunctionTok{wrap\_plots}\NormalTok{(plots, }\AttributeTok{nrow =} \DecValTok{3}\NormalTok{) }\SpecialCharTok{\&} \FunctionTok{theme}\NormalTok{(}\AttributeTok{text =} \FunctionTok{element\_text}\NormalTok{(}\AttributeTok{size =} \DecValTok{8}\NormalTok{))}
\end{Highlighting}
\end{Shaded}

\includegraphics{23_multi-assay_analyses_files/figure-latex/unnamed-chunk-6-1.pdf}

More tutorials and examples of using the package are found at \href{https://biofam.github.io/MOFA2/tutorials.html}{link}

\hypertarget{viz-chapter}{%
\chapter{Visualization}\label{viz-chapter}}

Data visualization will inevitably shape interpretation and motivate
the next steps of the analysis. A variety of visualization methods are
available for microbiome analysis but the application requires careful
attention to details. Knowledge on the available tools and their
limitations plays an important role in selecting the most suitable
methods to address a given question.

This chapter introduces the reader to a number of visualization
techniques found in this book, such as:

\begin{itemize}
\tightlist
\item
  barplots
\item
  boxplots
\item
  heatmaps
\item
  ordination charts
\item
  regression charts
\item
  trees
\end{itemize}

The toolkit which provides the essential plotting functionality
includes the following packages:

\begin{itemize}
\tightlist
\item
  \emph{patchwork}, \emph{cowplot}, \emph{ggpubr} and \emph{gridExtra}: plot layout and multi-panel plotting
\item
  \emph{miaViz}: specific visualization tools for \texttt{TreeSummaizedExperiment} objects
\item
  \emph{scater}: specific visualization tools for \texttt{SingleCellExperiment} objects
\item
  \emph{ggplot2}, \emph{pheatmap}, \emph{ggtree}, \emph{sechm}: composition heatmaps
\item
  \emph{ANCOMBC}, \emph{ALDEx2} and \emph{Maaslin2}: visual differential abundance
\item
  \emph{fido}: tree-based methods for differential abundance
\item
  \emph{plotly}: animated and 3D plotting
\end{itemize}

For systematic and extensive tutorials on the visual tools available
in \emph{mia}, readers can refer to the following material:

\begin{itemize}
\tightlist
\item
  \href{https://microbiome.github.io/tutorials/}{microbiome tutorials}
\end{itemize}

\hypertarget{pre-analysis-exploration}{%
\section{Pre-analysis exploration}\label{pre-analysis-exploration}}

\hypertarget{accessing-row-and-column-data}{%
\subsection{Accessing row and column data}\label{accessing-row-and-column-data}}

\texttt{SCE} and \texttt{TreeSE} objects contain multiple layers of information in the
form of rows, columns and meta data. The \emph{scater} package supports in
accessing, modifying and graphing the meta data related to features as
well as samples.

\begin{Shaded}
\begin{Highlighting}[]
\CommentTok{\# list row meta data}
\FunctionTok{names}\NormalTok{(}\FunctionTok{rowData}\NormalTok{(tse))}
\end{Highlighting}
\end{Shaded}

\begin{verbatim}
## [1] "Kingdom" "Phylum"  "Class"   "Order"   "Family"  "Genus"   "Species"
\end{verbatim}

\begin{Shaded}
\begin{Highlighting}[]
\CommentTok{\# list column meta data}
\FunctionTok{names}\NormalTok{(}\FunctionTok{colData}\NormalTok{(tse))}
\end{Highlighting}
\end{Shaded}

\begin{verbatim}
## [1] "X.SampleID"               "Primer"                  
## [3] "Final_Barcode"            "Barcode_truncated_plus_T"
## [5] "Barcode_full_length"      "SampleType"              
## [7] "Description"
\end{verbatim}

Such meta data can be directly plotted with the functions
\texttt{plotRowData} and \texttt{plotColData}.

\begin{Shaded}
\begin{Highlighting}[]
\CommentTok{\# obtain QC data}
\NormalTok{tse }\OtherTok{\textless{}{-}} \FunctionTok{addPerCellQC}\NormalTok{(tse)}
\NormalTok{tse }\OtherTok{\textless{}{-}} \FunctionTok{addPerFeatureQC}\NormalTok{(tse)}
\CommentTok{\# plot QC Mean against Species}
\FunctionTok{plotRowData}\NormalTok{(tse, }\StringTok{"mean"}\NormalTok{, }\StringTok{"Species"}\NormalTok{) }\SpecialCharTok{+}
  \FunctionTok{theme}\NormalTok{(}\AttributeTok{axis.text.x =} \FunctionTok{element\_blank}\NormalTok{()) }\SpecialCharTok{+}
  \FunctionTok{labs}\NormalTok{(}\AttributeTok{x =} \StringTok{"Species"}\NormalTok{, }\AttributeTok{y =} \StringTok{"QC Mean"}\NormalTok{)}
\end{Highlighting}
\end{Shaded}

\includegraphics{19_visualization_techniques_files/figure-latex/unnamed-chunk-3-1.pdf}

\begin{Shaded}
\begin{Highlighting}[]
\CommentTok{\# plot QC Sum against Sample ID, colour{-}labeled by Sample Type}
\FunctionTok{plotColData}\NormalTok{(tse, }\StringTok{"sum"}\NormalTok{, }\StringTok{"X.SampleID"}\NormalTok{, }\AttributeTok{colour\_by =} \StringTok{"SampleType"}\NormalTok{) }\SpecialCharTok{+}
  \FunctionTok{theme}\NormalTok{(}\AttributeTok{axis.text.x =} \FunctionTok{element\_text}\NormalTok{(}\AttributeTok{angle =} \DecValTok{45}\NormalTok{, }\AttributeTok{hjust =} \DecValTok{1}\NormalTok{)) }\SpecialCharTok{+}
  \FunctionTok{labs}\NormalTok{(}\AttributeTok{x =} \StringTok{"Sample ID"}\NormalTok{, }\AttributeTok{y =} \StringTok{"QC Sum"}\NormalTok{)}
\end{Highlighting}
\end{Shaded}

\includegraphics{19_visualization_techniques_files/figure-latex/unnamed-chunk-3-2.pdf}

Alternatively, they can be converted to a \texttt{data.frame} object and
passed to \texttt{ggplot}.

\begin{Shaded}
\begin{Highlighting}[]
\CommentTok{\# store colData into a data frame}
\NormalTok{coldata }\OtherTok{\textless{}{-}} \FunctionTok{as.data.frame}\NormalTok{(}\FunctionTok{colData}\NormalTok{(tse))}
\CommentTok{\# plot Number of Samples against Sampling Site}
\FunctionTok{ggplot}\NormalTok{(coldata, }\FunctionTok{aes}\NormalTok{(}\AttributeTok{x =}\NormalTok{ SampleType)) }\SpecialCharTok{+}
  \FunctionTok{geom\_bar}\NormalTok{(}\AttributeTok{width =} \FloatTok{0.5}\NormalTok{) }\SpecialCharTok{+}
  \FunctionTok{theme}\NormalTok{(}\AttributeTok{axis.text.x =} \FunctionTok{element\_text}\NormalTok{(}\AttributeTok{angle =} \DecValTok{45}\NormalTok{, }\AttributeTok{hjust =} \DecValTok{1}\NormalTok{)) }\SpecialCharTok{+}
  \FunctionTok{labs}\NormalTok{(}\AttributeTok{x =} \StringTok{"Sampling Site"}\NormalTok{,}
       \AttributeTok{y =} \StringTok{"Number of Samples"}\NormalTok{)}
\end{Highlighting}
\end{Shaded}

\includegraphics{19_visualization_techniques_files/figure-latex/unnamed-chunk-4-1.pdf}

Further methods of application can be found in the chapters \ref{qc}
and \ref{richness} and in a few \href{https://github.com/davismcc/scater_tutorials_open_data}{external
tutorials}
with open data. Additionally, \texttt{rowData} and \texttt{colData} allow
manipulation and subsetting of large data sets into smaller units, as
explained in chapter \ref{datamanipulation}.

\hypertarget{viewing-abundance-and-prevalence-patterns}{%
\subsection{Viewing abundance and prevalence patterns}\label{viewing-abundance-and-prevalence-patterns}}

Prior-to-analysis exploration may involve questions such as how microorganisms
are distributed across samples (abundance) and what microorganisms are present
in most of the samples (prevalence). The information on abundance and prevalence
can be summarized into a \textbf{jitter} or \textbf{density plot} and a \textbf{tree},
respectively, with the \emph{miaViz} package.

Specifically, the functions \texttt{plotAbundance}, \texttt{plotAbundanceDensity}
and \texttt{plotRowTree} are used, and examples on their usage are discussed
throughout chapter \ref{quality-control}.

\hypertarget{diversity-estimation}{%
\section{Diversity estimation}\label{diversity-estimation}}

Alpha diversity is commonly measured as one of the diversity indices
explained in chapter \ref{community-diversity}. Because the focus
lies on each sample separately, one-dimensional plots, such as
\textbf{scatter}, \textbf{violin} and \textbf{box plots}, are suitable.

Beta diversity is generally evaluated as one of the dissimilarity
indices reported in chapter \ref{community-similarity}. Unlike alpha
diversity, samples are compared collectively to estimate the
heterogeneity across them, therefore multidimensional plots, such as
\textbf{Shepard} and \textbf{ordination plots} are suitable.

\begin{longtable}[]{@{}
  >{\centering\arraybackslash}p{(\columnwidth - 4\tabcolsep) * \real{0.3125}}
  >{\centering\arraybackslash}p{(\columnwidth - 4\tabcolsep) * \real{0.3500}}
  >{\centering\arraybackslash}p{(\columnwidth - 4\tabcolsep) * \real{0.3375}}@{}}
\toprule()
\begin{minipage}[b]{\linewidth}\centering
\end{minipage} & \begin{minipage}[b]{\linewidth}\centering
alpha diversity
\end{minipage} & \begin{minipage}[b]{\linewidth}\centering
beta diversity
\end{minipage} \\
\midrule()
\endhead
used metrics & diversity indices & dissimilarity indices \\
& & \\
metric dimensionality & one-dimensional & multidimensional \\
& & \\
suitable visualization & scatter, violin, box plots & Shepard, ordination plots \\
\bottomrule()
\end{longtable}

In conclusion, visualization techniques for alpha and beta diversity
significantly differ from one another.

\hypertarget{alpha-diversity-with-scatter-violin-and-box-plots}{%
\subsection{Alpha diversity with scatter, violin and box plots}\label{alpha-diversity-with-scatter-violin-and-box-plots}}

The basic method to visualize the diversity values assigned to the
different samples in a \texttt{TSE} object includes the following, where each
data point represents one sample:

\begin{Shaded}
\begin{Highlighting}[]
\CommentTok{\# estimate shannon diversity index}
\NormalTok{tse }\OtherTok{\textless{}{-}}\NormalTok{ mia}\SpecialCharTok{::}\FunctionTok{estimateDiversity}\NormalTok{(tse, }
                              \AttributeTok{assay.type =} \StringTok{"counts"}\NormalTok{,}
                              \AttributeTok{index =} \StringTok{"shannon"}\NormalTok{, }
                              \AttributeTok{name =} \StringTok{"shannon"}\NormalTok{)}
\CommentTok{\# plot shannon diversity index, colour{-}labeled by Sample Type}
\FunctionTok{plotColData}\NormalTok{(tse, }\StringTok{"shannon"}\NormalTok{, }\AttributeTok{colour\_by =} \StringTok{"SampleType"}\NormalTok{)}
\end{Highlighting}
\end{Shaded}

\includegraphics{19_visualization_techniques_files/figure-latex/unnamed-chunk-5-1.pdf}

The several indices available for the evaluation of alpha diversity
often return slightly divergent results, which can be visually
compared with a multiple violin or box plot. For this purpose,
\texttt{plotColData} (for violin plots) or \texttt{ggplot} (for box plots) are
recursively applied to a number of diversity indices with the function
\texttt{lapply} and the multi-panel plotting functionality of the \emph{patchwork}
package is then exploited.

\begin{Shaded}
\begin{Highlighting}[]
\CommentTok{\# estimate faith diversity index}
\NormalTok{tse }\OtherTok{\textless{}{-}}\NormalTok{ mia}\SpecialCharTok{::}\FunctionTok{estimateFaith}\NormalTok{(tse,}
                          \AttributeTok{assay.type =} \StringTok{"counts"}\NormalTok{)}
\CommentTok{\# store colData into a data frame}
\NormalTok{coldata }\OtherTok{\textless{}{-}} \FunctionTok{as.data.frame}\NormalTok{(}\FunctionTok{colData}\NormalTok{(tse))}
\CommentTok{\# generate plots for shannon and faith indices}
\CommentTok{\# and store them into a list}
\NormalTok{plots }\OtherTok{\textless{}{-}} \FunctionTok{lapply}\NormalTok{(}\FunctionTok{c}\NormalTok{(}\StringTok{"shannon"}\NormalTok{, }\StringTok{"faith"}\NormalTok{),}
                \ControlFlowTok{function}\NormalTok{(i) }\FunctionTok{ggplot}\NormalTok{(coldata, }\FunctionTok{aes\_string}\NormalTok{(}\AttributeTok{y =}\NormalTok{ i)) }\SpecialCharTok{+}
                  \FunctionTok{geom\_boxplot}\NormalTok{() }\SpecialCharTok{+}
                  \FunctionTok{theme}\NormalTok{(}\AttributeTok{axis.text.x =} \FunctionTok{element\_blank}\NormalTok{(),}
                        \AttributeTok{axis.ticks.x =} \FunctionTok{element\_blank}\NormalTok{()))}
\CommentTok{\# combine plots with patchwork}
\NormalTok{plots[[}\DecValTok{1}\NormalTok{]] }\SpecialCharTok{+}\NormalTok{ plots[[}\DecValTok{2}\NormalTok{]]}
\end{Highlighting}
\end{Shaded}

\includegraphics{19_visualization_techniques_files/figure-latex/unnamed-chunk-6-1.pdf}

The analogous output in the form of a violin plot is obtained in
chapter \ref{faith-diversity}. In addition, box plots that group
samples according to certain information, such as origin, sex, age and
health condition, can be labeled with p-values for significant
differences with the package \emph{ggsignif} package, as shown in chapter
\ref{estimate-diversity}.

\hypertarget{beta-diversity-with-shepard-and-coordination-plots}{%
\subsection{Beta diversity with Shepard and coordination plots}\label{beta-diversity-with-shepard-and-coordination-plots}}

The \emph{scater} package offers the general function \texttt{plotReducedDim}. In
its basic form, it takes a \texttt{TSE} object and the results on sample
similarity stored in the same object, which can be evaluated with the
following coordination methods:

\begin{itemize}
\tightlist
\item
  \texttt{runMDS}
\item
  \texttt{runNMDS}
\item
  \texttt{runPCA}
\item
  \texttt{runTSNE}
\item
  \texttt{runUMAP}
\end{itemize}

Since these clustering techniques allow for multiple coordinates or
components, \textbf{coordination plots} can also span multiple dimensions,
which is explained in chapter \ref{extras}.

\begin{Shaded}
\begin{Highlighting}[]
\CommentTok{\# perform NMDS coordination method}
\NormalTok{tse }\OtherTok{\textless{}{-}} \FunctionTok{runNMDS}\NormalTok{(tse,}
               \AttributeTok{FUN =}\NormalTok{ vegan}\SpecialCharTok{::}\NormalTok{vegdist,}
               \AttributeTok{name =} \StringTok{"NMDS"}\NormalTok{)}
\end{Highlighting}
\end{Shaded}

\begin{verbatim}
## initial  value 47.733208 
## iter   5 value 33.853364
## iter  10 value 32.891200
## final  value 32.823570 
## converged
\end{verbatim}

\begin{Shaded}
\begin{Highlighting}[]
\CommentTok{\# plot results of a 2{-}component NMDS on tse,}
\CommentTok{\# coloured{-}scaled by shannon diversity index}
\FunctionTok{plotReducedDim}\NormalTok{(tse, }\StringTok{"NMDS"}\NormalTok{, }\AttributeTok{colour\_by =} \StringTok{"shannon"}\NormalTok{)}
\end{Highlighting}
\end{Shaded}

\includegraphics{19_visualization_techniques_files/figure-latex/unnamed-chunk-7-1.pdf}

Multiple combinations of coordinates or dimensions can also be integrated into a multi-panel arrangement.

\begin{Shaded}
\begin{Highlighting}[]
\CommentTok{\# perform MDS coordination method}
\NormalTok{tse }\OtherTok{\textless{}{-}} \FunctionTok{runMDS}\NormalTok{(tse,}
              \AttributeTok{FUN =}\NormalTok{ vegan}\SpecialCharTok{::}\NormalTok{vegdist,}
              \AttributeTok{method =} \StringTok{"bray"}\NormalTok{,}
              \AttributeTok{name =} \StringTok{"MDS"}\NormalTok{,}
              \AttributeTok{assay.type =} \StringTok{"counts"}\NormalTok{,}
              \AttributeTok{ncomponents =} \DecValTok{3}\NormalTok{)}
\CommentTok{\# plot results of a 3{-}component MDS on tse,}
\CommentTok{\# coloured{-}scaled by faith diversity index}
\FunctionTok{plotReducedDim}\NormalTok{(tse, }\StringTok{"MDS"}\NormalTok{, }\AttributeTok{ncomponents =} \FunctionTok{c}\NormalTok{(}\DecValTok{1}\SpecialCharTok{:}\DecValTok{3}\NormalTok{), }\AttributeTok{colour\_by =} \StringTok{"faith"}\NormalTok{)}
\end{Highlighting}
\end{Shaded}

\includegraphics{19_visualization_techniques_files/figure-latex/unnamed-chunk-8-1.pdf}

Similarly to iterating \texttt{plotColData} over indices of alpha diversity,
\texttt{lapply} can be used in combination with \emph{patchwork} to recursively
apply \texttt{plotReducedDim} and visually compare results among various
coordination methods.

\begin{Shaded}
\begin{Highlighting}[]
\CommentTok{\# generate plots for MDS and NMDS methods}
\CommentTok{\# and store them into a list}
\NormalTok{plots }\OtherTok{\textless{}{-}} \FunctionTok{lapply}\NormalTok{(}\FunctionTok{c}\NormalTok{(}\StringTok{"MDS"}\NormalTok{, }\StringTok{"NMDS"}\NormalTok{),}
\NormalTok{                plotReducedDim,}
                \AttributeTok{object =}\NormalTok{ tse,}
                \AttributeTok{colour\_by =} \StringTok{"shannon"}\NormalTok{)}
\CommentTok{\# combine plots with patchwork}
\NormalTok{plots[[}\DecValTok{1}\NormalTok{]] }\SpecialCharTok{+}\NormalTok{ plots[[}\DecValTok{2}\NormalTok{]] }\SpecialCharTok{+}
  \FunctionTok{plot\_layout}\NormalTok{(}\AttributeTok{guides =} \StringTok{"collect"}\NormalTok{)}
\end{Highlighting}
\end{Shaded}

\includegraphics{19_visualization_techniques_files/figure-latex/unnamed-chunk-9-1.pdf}

For similar examples, readers are referred to chapter
\ref{community-similarity}. Further material on the graphic
capabilities of \emph{patchwork} is available in its \href{https://patchwork.data-imaginist.com/articles/patchwork.html}{official package
tutorial}.

\hypertarget{statistical-analysis}{%
\section{Statistical analysis}\label{statistical-analysis}}

\hypertarget{heatmaps}{%
\subsection{Heatmaps}\label{heatmaps}}

As described in chapter \ref{visual-composition}, bar plots and
heatmaps can offer a useful insight into the composition of a
community. Simple methods involve the functions \texttt{plotAbundance} and
\texttt{geom\_tile} in combination with \texttt{scale\_fill\_gradientn} from the
packages \emph{miaViz} and \emph{ggplot2}, respectively.

For instance, below the composition of multiple samples (x axis) is
reported in terms of relative abundances (y axis) for the top 10 taxa
at the Order rank. Bar plots and heatmaps with analogous information
at the Phylum level are available in the aforementioned chapter.

\begin{Shaded}
\begin{Highlighting}[]
\CommentTok{\# agglomerate tse by Order}
\NormalTok{tse\_order }\OtherTok{\textless{}{-}} \FunctionTok{agglomerateByRank}\NormalTok{(tse,}
                                \AttributeTok{rank =} \StringTok{"Order"}\NormalTok{,}
                                \AttributeTok{onRankOnly =} \ConstantTok{TRUE}\NormalTok{)}
\CommentTok{\# transform counts into relative abundance}
\NormalTok{tse\_order }\OtherTok{\textless{}{-}} \FunctionTok{transformCounts}\NormalTok{(tse\_order,}
                              \AttributeTok{assay.type =} \StringTok{"counts"}\NormalTok{,}
                              \AttributeTok{method =} \StringTok{"relabundance"}\NormalTok{)}
\CommentTok{\# get top orders}
\NormalTok{top\_taxa }\OtherTok{\textless{}{-}} \FunctionTok{getTopTaxa}\NormalTok{(tse\_order,}
                       \AttributeTok{top =} \DecValTok{10}\NormalTok{,}
                       \AttributeTok{assay.type =} \StringTok{"relabundance"}\NormalTok{)}
\CommentTok{\# leave only names for top 10 orders and label the rest with "Other"}
\NormalTok{order\_renamed }\OtherTok{\textless{}{-}} \FunctionTok{lapply}\NormalTok{(}\FunctionTok{rowData}\NormalTok{(tse\_order)}\SpecialCharTok{$}\NormalTok{Order,}
                   \ControlFlowTok{function}\NormalTok{(x)\{}\ControlFlowTok{if}\NormalTok{ (x }\SpecialCharTok{\%in\%}\NormalTok{ top\_taxa) \{x\} }\ControlFlowTok{else}\NormalTok{ \{}\StringTok{"Other"}\NormalTok{\}\})}
\FunctionTok{rowData}\NormalTok{(tse\_order)}\SpecialCharTok{$}\NormalTok{Order }\OtherTok{\textless{}{-}} \FunctionTok{as.character}\NormalTok{(order\_renamed)}
\CommentTok{\# plot composition as a bar plot}
\FunctionTok{plotAbundance}\NormalTok{(tse\_order,}
              \AttributeTok{assay.type =} \StringTok{"relabundance"}\NormalTok{,}
              \AttributeTok{rank =} \StringTok{"Order"}\NormalTok{,}
              \AttributeTok{order\_rank\_by =} \StringTok{"abund"}\NormalTok{,}
              \AttributeTok{order\_sample\_by =} \StringTok{"Clostridiales"}\NormalTok{)}
\end{Highlighting}
\end{Shaded}

\includegraphics{19_visualization_techniques_files/figure-latex/plotAbundance1-1.pdf}

To add a sample annotation, you can combine plots that you get from the output
of \emph{plotAbundance}.

\begin{Shaded}
\begin{Highlighting}[]
\CommentTok{\# Create plots}
\NormalTok{plots }\OtherTok{\textless{}{-}} \FunctionTok{plotAbundance}\NormalTok{(tse\_order,}
            \AttributeTok{assay.type =} \StringTok{"relabundance"}\NormalTok{,}
        \AttributeTok{rank =} \StringTok{"Order"}\NormalTok{,}
            \AttributeTok{order\_rank\_by =} \StringTok{"abund"}\NormalTok{,}
        \AttributeTok{order\_sample\_by =} \StringTok{"Clostridiales"}\NormalTok{,}
            \AttributeTok{features =} \StringTok{"SampleType"}\NormalTok{)}

\CommentTok{\# Modify the legend of the first plot to be smaller }
\NormalTok{plots[[}\DecValTok{1}\NormalTok{]] }\OtherTok{\textless{}{-}}\NormalTok{ plots[[}\DecValTok{1}\NormalTok{]] }\SpecialCharTok{+}
    \FunctionTok{theme}\NormalTok{(}\AttributeTok{legend.key.size =} \FunctionTok{unit}\NormalTok{(}\FloatTok{0.3}\NormalTok{, }\StringTok{\textquotesingle{}cm\textquotesingle{}}\NormalTok{),}
          \AttributeTok{legend.text =} \FunctionTok{element\_text}\NormalTok{(}\AttributeTok{size =} \DecValTok{6}\NormalTok{),}
          \AttributeTok{legend.title =} \FunctionTok{element\_text}\NormalTok{(}\AttributeTok{size =} \DecValTok{8}\NormalTok{))}

\CommentTok{\# Modify the legend of the second plot to be smaller }
\NormalTok{plots[[}\DecValTok{2}\NormalTok{]] }\OtherTok{\textless{}{-}}\NormalTok{ plots[[}\DecValTok{2}\NormalTok{]] }\SpecialCharTok{+}
    \FunctionTok{theme}\NormalTok{(}\AttributeTok{legend.key.height =} \FunctionTok{unit}\NormalTok{(}\FloatTok{0.3}\NormalTok{, }\StringTok{\textquotesingle{}cm\textquotesingle{}}\NormalTok{),}
          \AttributeTok{legend.key.width =} \FunctionTok{unit}\NormalTok{(}\FloatTok{0.3}\NormalTok{, }\StringTok{\textquotesingle{}cm\textquotesingle{}}\NormalTok{),}
          \AttributeTok{legend.text =} \FunctionTok{element\_text}\NormalTok{(}\AttributeTok{size =} \DecValTok{6}\NormalTok{),}
          \AttributeTok{legend.title =} \FunctionTok{element\_text}\NormalTok{(}\AttributeTok{size =} \DecValTok{8}\NormalTok{),}
          \AttributeTok{legend.direction =} \StringTok{"vertical"}\NormalTok{)}

\CommentTok{\# Load required packages}
\FunctionTok{library}\NormalTok{(}\StringTok{"ggpubr"}\NormalTok{)}
\FunctionTok{library}\NormalTok{(}\StringTok{"patchwork"}\NormalTok{) }
\CommentTok{\# Combine legends}
\NormalTok{legend }\OtherTok{\textless{}{-}} \FunctionTok{wrap\_plots}\NormalTok{(}\FunctionTok{as\_ggplot}\NormalTok{(}\FunctionTok{get\_legend}\NormalTok{(plots[[}\DecValTok{1}\NormalTok{]])), }\FunctionTok{as\_ggplot}\NormalTok{(}\FunctionTok{get\_legend}\NormalTok{(plots[[}\DecValTok{2}\NormalTok{]])), }\AttributeTok{ncol =} \DecValTok{1}\NormalTok{) }

\CommentTok{\# Remove legends from the plots}
\NormalTok{plots[[}\DecValTok{1}\NormalTok{]] }\OtherTok{\textless{}{-}}\NormalTok{ plots[[}\DecValTok{1}\NormalTok{]] }\SpecialCharTok{+} \FunctionTok{theme}\NormalTok{(}\AttributeTok{legend.position =} \StringTok{"none"}\NormalTok{)}
\NormalTok{plots[[}\DecValTok{2}\NormalTok{]] }\OtherTok{\textless{}{-}}\NormalTok{ plots[[}\DecValTok{2}\NormalTok{]] }\SpecialCharTok{+} \FunctionTok{theme}\NormalTok{(}\AttributeTok{legend.position =} \StringTok{"none"}\NormalTok{, }\AttributeTok{axis.title.x=}\FunctionTok{element\_blank}\NormalTok{()) }

\CommentTok{\# Combine plots}
\NormalTok{plot }\OtherTok{\textless{}{-}} \FunctionTok{wrap\_plots}\NormalTok{(plots[[}\DecValTok{2}\NormalTok{]], plots[[}\DecValTok{1}\NormalTok{]], }\AttributeTok{ncol =} \DecValTok{1}\NormalTok{, }\AttributeTok{heights =} \FunctionTok{c}\NormalTok{(}\DecValTok{2}\NormalTok{, }\DecValTok{10}\NormalTok{))}
\CommentTok{\# Combine the plot with the legend}
\FunctionTok{wrap\_plots}\NormalTok{(plot, legend, }\AttributeTok{nrow =} \DecValTok{1}\NormalTok{, }\AttributeTok{widths =} \FunctionTok{c}\NormalTok{(}\DecValTok{2}\NormalTok{, }\DecValTok{1}\NormalTok{))}
\end{Highlighting}
\end{Shaded}

\includegraphics{19_visualization_techniques_files/figure-latex/plotAbundance2-1.pdf}

For more sophisticated visualizations than those produced with \texttt{plotAbundance}
and \emph{ggplot2}, the packages \emph{pheatmap} and \emph{sechm} provide methods to include
feature and sample clusters in a heatmap, along with further functionality.

\begin{Shaded}
\begin{Highlighting}[]
\CommentTok{\# Agglomerate tse by phylum}
\NormalTok{tse\_phylum }\OtherTok{\textless{}{-}} \FunctionTok{agglomerateByRank}\NormalTok{(tse,}
                                \AttributeTok{rank =} \StringTok{"Phylum"}\NormalTok{,}
                                \AttributeTok{onRankOnly =} \ConstantTok{TRUE}\NormalTok{)}

\CommentTok{\# Add clr{-}transformation on samples}
\NormalTok{tse\_phylum }\OtherTok{\textless{}{-}} \FunctionTok{transformCounts}\NormalTok{(tse\_phylum, }\AttributeTok{MARGIN =} \StringTok{"samples"}\NormalTok{, }\AttributeTok{method =} \StringTok{"clr"}\NormalTok{, }\AttributeTok{assay.type =} \StringTok{"counts"}\NormalTok{, }\AttributeTok{pseudocount=}\DecValTok{1}\NormalTok{)}

\CommentTok{\# Add z{-}transformation on features (taxa)}
\NormalTok{tse\_phylum }\OtherTok{\textless{}{-}} \FunctionTok{transformCounts}\NormalTok{(tse\_phylum, }\AttributeTok{assay.type =} \StringTok{"clr"}\NormalTok{,}
                              \AttributeTok{MARGIN =} \StringTok{"features"}\NormalTok{, }
                              \AttributeTok{method =} \StringTok{"z"}\NormalTok{, }\AttributeTok{name =} \StringTok{"clr\_z"}\NormalTok{)}

\CommentTok{\# Take subset: only samples from feces, skin, or tongue}
\NormalTok{tse\_phylum\_subset }\OtherTok{\textless{}{-}}\NormalTok{ tse\_phylum[ , tse\_phylum}\SpecialCharTok{$}\NormalTok{SampleType }\SpecialCharTok{\%in\%} \FunctionTok{c}\NormalTok{(}\StringTok{"Feces"}\NormalTok{, }\StringTok{"Skin"}\NormalTok{, }\StringTok{"Tongue"}\NormalTok{) ]}

\CommentTok{\# Add clr{-}transformation}
\NormalTok{tse\_phylum\_subset }\OtherTok{\textless{}{-}} \FunctionTok{transformCounts}\NormalTok{(tse\_phylum\_subset, }\AttributeTok{method =} \StringTok{"clr"}\NormalTok{,}
                                     \AttributeTok{MARGIN=}\StringTok{"samples"}\NormalTok{,}
                                     \AttributeTok{assay.type =} \StringTok{"counts"}\NormalTok{, }\AttributeTok{pseudocount=}\DecValTok{1}\NormalTok{)}
\CommentTok{\# Does z{-}transformation}
\NormalTok{tse\_phylum\_subset }\OtherTok{\textless{}{-}} \FunctionTok{transformCounts}\NormalTok{(tse\_phylum\_subset, }\AttributeTok{assay.type =} \StringTok{"clr"}\NormalTok{,}
                                     \AttributeTok{MARGIN =} \StringTok{"features"}\NormalTok{, }
                                     \AttributeTok{method =} \StringTok{"z"}\NormalTok{, }\AttributeTok{name =} \StringTok{"clr\_z"}\NormalTok{)}

\CommentTok{\# Get n most abundant taxa, and subsets the data by them}
\NormalTok{top\_taxa }\OtherTok{\textless{}{-}} \FunctionTok{getTopTaxa}\NormalTok{(tse\_phylum\_subset, }\AttributeTok{top =} \DecValTok{20}\NormalTok{)}
\NormalTok{tse\_phylum\_subset }\OtherTok{\textless{}{-}}\NormalTok{ tse\_phylum\_subset[top\_taxa, ]}

\CommentTok{\# Gets the assay table}
\NormalTok{mat }\OtherTok{\textless{}{-}} \FunctionTok{assay}\NormalTok{(tse\_phylum\_subset, }\StringTok{"clr\_z"}\NormalTok{)}

\CommentTok{\# Creates the heatmap}
\FunctionTok{pheatmap}\NormalTok{(mat)}
\end{Highlighting}
\end{Shaded}

\includegraphics{19_visualization_techniques_files/figure-latex/pheatmap1-1.pdf}

We can cluster both samples and features hierarchically and add them to the
x and y axes of the heatmap, respectively.

\begin{Shaded}
\begin{Highlighting}[]
\CommentTok{\# Hierarchical clustering}
\NormalTok{taxa\_hclust }\OtherTok{\textless{}{-}} \FunctionTok{hclust}\NormalTok{(}\FunctionTok{dist}\NormalTok{(mat), }\AttributeTok{method =} \StringTok{"complete"}\NormalTok{)}

\CommentTok{\# Creates a phylogenetic tree}
\NormalTok{taxa\_tree }\OtherTok{\textless{}{-}} \FunctionTok{as.phylo}\NormalTok{(taxa\_hclust)}

\CommentTok{\# Plot taxa tree}
\NormalTok{taxa\_tree }\OtherTok{\textless{}{-}} \FunctionTok{ggtree}\NormalTok{(taxa\_tree) }\SpecialCharTok{+} 
  \FunctionTok{theme}\NormalTok{(}\AttributeTok{plot.margin=}\FunctionTok{margin}\NormalTok{(}\DecValTok{0}\NormalTok{,}\DecValTok{0}\NormalTok{,}\DecValTok{0}\NormalTok{,}\DecValTok{0}\NormalTok{)) }\CommentTok{\# removes margins}

\CommentTok{\# Get order of taxa in plot}
\NormalTok{taxa\_ordered }\OtherTok{\textless{}{-}} \FunctionTok{get\_taxa\_name}\NormalTok{(taxa\_tree)}

\CommentTok{\# to view the tree, run}
\CommentTok{\# taxa\_tree}
\end{Highlighting}
\end{Shaded}

Based on phylo tree, we decide to create three clusters.

\begin{Shaded}
\begin{Highlighting}[]
\CommentTok{\# Creates clusters}
\NormalTok{taxa\_clusters }\OtherTok{\textless{}{-}} \FunctionTok{cutree}\NormalTok{(}\AttributeTok{tree =}\NormalTok{ taxa\_hclust, }\AttributeTok{k =} \DecValTok{3}\NormalTok{)}

\CommentTok{\# Converts into data frame}
\NormalTok{taxa\_clusters }\OtherTok{\textless{}{-}} \FunctionTok{data.frame}\NormalTok{(}\AttributeTok{clusters =}\NormalTok{ taxa\_clusters)}
\NormalTok{taxa\_clusters}\SpecialCharTok{$}\NormalTok{clusters }\OtherTok{\textless{}{-}} \FunctionTok{factor}\NormalTok{(taxa\_clusters}\SpecialCharTok{$}\NormalTok{clusters)}

\CommentTok{\# Order data so that it\textquotesingle{}s same as in phylo tree}
\NormalTok{taxa\_clusters }\OtherTok{\textless{}{-}}\NormalTok{ taxa\_clusters[taxa\_ordered, , drop }\OtherTok{=} \ConstantTok{FALSE}\NormalTok{] }

\CommentTok{\# Prints taxa and their clusters}
\NormalTok{taxa\_clusters}
\end{Highlighting}
\end{Shaded}

\begin{verbatim}
##                  clusters
## Chloroflexi             3
## Actinobacteria          3
## Crenarchaeota           3
## Planctomycetes          3
## Gemmatimonadetes        3
## Thermi                  3
## Acidobacteria           3
## Spirochaetes            2
## Fusobacteria            2
## SR1                     2
## Cyanobacteria           2
## Proteobacteria          2
## Synergistetes           2
## Lentisphaerae           1
## Bacteroidetes           1
## Verrucomicrobia         1
## Tenericutes             1
## Firmicutes              1
## Euryarchaeota           1
## SAR406                  1
\end{verbatim}

The information on the clusters is then added to the feature meta data.

\begin{Shaded}
\begin{Highlighting}[]
\CommentTok{\# Adds information to rowData}
\FunctionTok{rowData}\NormalTok{(tse\_phylum\_subset)}\SpecialCharTok{$}\NormalTok{clusters }\OtherTok{\textless{}{-}}\NormalTok{ taxa\_clusters[}\FunctionTok{order}\NormalTok{(}\FunctionTok{match}\NormalTok{(}\FunctionTok{rownames}\NormalTok{(taxa\_clusters), }\FunctionTok{rownames}\NormalTok{(tse\_phylum\_subset))), ]}

\CommentTok{\# Prints taxa and their clusters}
\FunctionTok{rowData}\NormalTok{(tse\_phylum\_subset)}\SpecialCharTok{$}\NormalTok{clusters}
\end{Highlighting}
\end{Shaded}

\begin{verbatim}
##  [1] 1 1 2 3 2 2 1 1 1 1 3 2 3 3 3 2 2 3 3 1
## Levels: 1 2 3
\end{verbatim}

Similarly, samples are hierarchically grouped into clusters, the most suitable
number of clusters for the plot is selected and the new information is stored
into the sample meta data.

\begin{Shaded}
\begin{Highlighting}[]
\CommentTok{\# Hierarchical clustering}
\NormalTok{sample\_hclust }\OtherTok{\textless{}{-}} \FunctionTok{hclust}\NormalTok{(}\FunctionTok{dist}\NormalTok{(}\FunctionTok{t}\NormalTok{(mat)), }\AttributeTok{method =} \StringTok{"complete"}\NormalTok{)}

\CommentTok{\# Creates a phylogenetic tree}
\NormalTok{sample\_tree }\OtherTok{\textless{}{-}} \FunctionTok{as.phylo}\NormalTok{(sample\_hclust)}

\CommentTok{\# Plot sample tree}
\NormalTok{sample\_tree }\OtherTok{\textless{}{-}} \FunctionTok{ggtree}\NormalTok{(sample\_tree) }\SpecialCharTok{+} \FunctionTok{layout\_dendrogram}\NormalTok{() }\SpecialCharTok{+} 
  \FunctionTok{theme}\NormalTok{(}\AttributeTok{plot.margin=}\FunctionTok{margin}\NormalTok{(}\DecValTok{0}\NormalTok{,}\DecValTok{0}\NormalTok{,}\DecValTok{0}\NormalTok{,}\DecValTok{0}\NormalTok{)) }\CommentTok{\# removes margins}

\CommentTok{\# Get order of samples in plot}
\NormalTok{samples\_ordered }\OtherTok{\textless{}{-}} \FunctionTok{rev}\NormalTok{(}\FunctionTok{get\_taxa\_name}\NormalTok{(sample\_tree))}

\CommentTok{\# to view the tree, run}
\CommentTok{\# sample\_tree}

\CommentTok{\# Creates clusters}
\NormalTok{sample\_clusters }\OtherTok{\textless{}{-}} \FunctionTok{factor}\NormalTok{(}\FunctionTok{cutree}\NormalTok{(}\AttributeTok{tree =}\NormalTok{ sample\_hclust, }\AttributeTok{k =} \DecValTok{3}\NormalTok{))}

\CommentTok{\# Converts into data frame}
\NormalTok{sample\_data }\OtherTok{\textless{}{-}} \FunctionTok{data.frame}\NormalTok{(}\AttributeTok{clusters =}\NormalTok{ sample\_clusters)}

\CommentTok{\# Order data so that it\textquotesingle{}s same as in phylo tree}
\NormalTok{sample\_data }\OtherTok{\textless{}{-}}\NormalTok{ sample\_data[samples\_ordered, , drop }\OtherTok{=} \ConstantTok{FALSE}\NormalTok{] }

\CommentTok{\# Order data based on }
\NormalTok{tse\_phylum\_subset }\OtherTok{\textless{}{-}}\NormalTok{ tse\_phylum\_subset[ , }\FunctionTok{rownames}\NormalTok{(sample\_data)]}

\CommentTok{\# Add sample type data}
\NormalTok{sample\_data}\SpecialCharTok{$}\NormalTok{sample\_types }\OtherTok{\textless{}{-}} \FunctionTok{unfactor}\NormalTok{(}\FunctionTok{colData}\NormalTok{(tse\_phylum\_subset)}\SpecialCharTok{$}\NormalTok{SampleType)}

\NormalTok{sample\_data}
\end{Highlighting}
\end{Shaded}

\begin{verbatim}
##         clusters sample_types
## M11Plmr        2         Skin
## M31Plmr        2         Skin
## F21Plmr        2         Skin
## M31Fcsw        1        Feces
## M11Fcsw        1        Feces
## TS28           3        Feces
## TS29           3        Feces
## M31Tong        3       Tongue
## M11Tong        3       Tongue
\end{verbatim}

Now we can create heatmap with additional annotations.

\begin{Shaded}
\begin{Highlighting}[]
\CommentTok{\# Determines the scaling of colorss}
\CommentTok{\# Scale colors}
\NormalTok{breaks }\OtherTok{\textless{}{-}} \FunctionTok{seq}\NormalTok{(}\SpecialCharTok{{-}}\FunctionTok{ceiling}\NormalTok{(}\FunctionTok{max}\NormalTok{(}\FunctionTok{abs}\NormalTok{(mat))), }\FunctionTok{ceiling}\NormalTok{(}\FunctionTok{max}\NormalTok{(}\FunctionTok{abs}\NormalTok{(mat))), }
              \AttributeTok{length.out =} \FunctionTok{ifelse}\NormalTok{( }\FunctionTok{max}\NormalTok{(}\FunctionTok{abs}\NormalTok{(mat))}\SpecialCharTok{\textgreater{}}\DecValTok{5}\NormalTok{, }\DecValTok{2}\SpecialCharTok{*}\FunctionTok{ceiling}\NormalTok{(}\FunctionTok{max}\NormalTok{(}\FunctionTok{abs}\NormalTok{(mat))), }\DecValTok{10}\NormalTok{ ) )}
\NormalTok{colors }\OtherTok{\textless{}{-}} \FunctionTok{colorRampPalette}\NormalTok{(}\FunctionTok{c}\NormalTok{(}\StringTok{"darkblue"}\NormalTok{, }\StringTok{"blue"}\NormalTok{, }\StringTok{"white"}\NormalTok{, }\StringTok{"red"}\NormalTok{, }\StringTok{"darkred"}\NormalTok{))(}\FunctionTok{length}\NormalTok{(breaks)}\SpecialCharTok{{-}}\DecValTok{1}\NormalTok{)}

\FunctionTok{pheatmap}\NormalTok{(mat, }\AttributeTok{annotation\_row =}\NormalTok{ taxa\_clusters, }
         \AttributeTok{annotation\_col =}\NormalTok{ sample\_data,}
         \AttributeTok{breaks =}\NormalTok{ breaks,}
         \AttributeTok{color =}\NormalTok{ colors)}
\end{Highlighting}
\end{Shaded}

\includegraphics{19_visualization_techniques_files/figure-latex/pheatmap6-1.pdf}

The package \emph{sechm} allows for further visual capabilities and flexibility.
In this case, the clustering step is automatically performed by the plotting
function and does not need to be executed in advance.

\begin{Shaded}
\begin{Highlighting}[]
\CommentTok{\# Stores annotation colros to metadata}
\FunctionTok{metadata}\NormalTok{(tse\_phylum\_subset)}\SpecialCharTok{$}\NormalTok{anno\_colors}\SpecialCharTok{$}\NormalTok{SampleType }\OtherTok{\textless{}{-}} \FunctionTok{c}\NormalTok{(}\AttributeTok{Feces =} \StringTok{"blue"}\NormalTok{, }
                                                        \AttributeTok{Skin =} \StringTok{"red"}\NormalTok{, }
                                                        \AttributeTok{Tongue =} \StringTok{"gray"}\NormalTok{)}

\CommentTok{\# Create a plot}
\FunctionTok{sechm}\NormalTok{(tse\_phylum\_subset, }
      \AttributeTok{features =} \FunctionTok{rownames}\NormalTok{(tse\_phylum\_subset), }
      \AttributeTok{assayName =} \StringTok{"clr"}\NormalTok{, }
      \AttributeTok{do.scale =} \ConstantTok{TRUE}\NormalTok{, }
      \AttributeTok{top\_annotation =} \FunctionTok{c}\NormalTok{(}\StringTok{"SampleType"}\NormalTok{), }
      \AttributeTok{gaps\_at =} \StringTok{"SampleType"}\NormalTok{,}
      \AttributeTok{cluster\_cols =} \ConstantTok{TRUE}\NormalTok{, }\AttributeTok{cluster\_rows =} \ConstantTok{TRUE}\NormalTok{)}
\end{Highlighting}
\end{Shaded}

\includegraphics{19_visualization_techniques_files/figure-latex/sechm-1.pdf}

It is also possible to create an analogous heatmap by just using the
\emph{ggplot2} package. However, a relatively long code is required to
generate an identical output.

\begin{Shaded}
\begin{Highlighting}[]
\CommentTok{\# Add feature names to column as a factor}
\NormalTok{taxa\_clusters}\SpecialCharTok{$}\NormalTok{Feature }\OtherTok{\textless{}{-}} \FunctionTok{rownames}\NormalTok{(taxa\_clusters)}
\NormalTok{taxa\_clusters}\SpecialCharTok{$}\NormalTok{Feature }\OtherTok{\textless{}{-}} \FunctionTok{factor}\NormalTok{(taxa\_clusters}\SpecialCharTok{$}\NormalTok{Feature, }\AttributeTok{levels =}\NormalTok{ taxa\_clusters}\SpecialCharTok{$}\NormalTok{Feature)}

\CommentTok{\# Create annotation plot}
\NormalTok{row\_annotation }\OtherTok{\textless{}{-}} \FunctionTok{ggplot}\NormalTok{(taxa\_clusters) }\SpecialCharTok{+} 
  \FunctionTok{geom\_tile}\NormalTok{(}\FunctionTok{aes}\NormalTok{(}\AttributeTok{x =} \ConstantTok{NA}\NormalTok{, }\AttributeTok{y =}\NormalTok{ Feature, }\AttributeTok{fill =}\NormalTok{ clusters)) }\SpecialCharTok{+}
  \FunctionTok{coord\_equal}\NormalTok{(}\AttributeTok{ratio =} \DecValTok{1}\NormalTok{) }\SpecialCharTok{+}
  \FunctionTok{theme}\NormalTok{(}
        \AttributeTok{axis.text.x=}\FunctionTok{element\_blank}\NormalTok{(),}
        \AttributeTok{axis.text.y=}\FunctionTok{element\_blank}\NormalTok{(),}
        \AttributeTok{axis.ticks.y=}\FunctionTok{element\_blank}\NormalTok{(),}
        \AttributeTok{axis.title.y=}\FunctionTok{element\_blank}\NormalTok{(),}
        \AttributeTok{axis.title.x =} \FunctionTok{element\_text}\NormalTok{(}\AttributeTok{angle =} \DecValTok{90}\NormalTok{, }\AttributeTok{vjust =} \FloatTok{0.5}\NormalTok{, }\AttributeTok{hjust=}\DecValTok{1}\NormalTok{),}
        \AttributeTok{plot.margin=}\FunctionTok{margin}\NormalTok{(}\DecValTok{0}\NormalTok{,}\DecValTok{0}\NormalTok{,}\DecValTok{0}\NormalTok{,}\DecValTok{0}\NormalTok{),}
\NormalTok{        ) }\SpecialCharTok{+}
      \FunctionTok{labs}\NormalTok{(}\AttributeTok{fill =} \StringTok{"Clusters"}\NormalTok{, }\AttributeTok{x =} \StringTok{"Clusters"}\NormalTok{)}

\CommentTok{\# to view the notation, run}
\CommentTok{\# row\_annotation}

\CommentTok{\# Add sample names to one of the columns}
\NormalTok{sample\_data}\SpecialCharTok{$}\NormalTok{sample }\OtherTok{\textless{}{-}} \FunctionTok{factor}\NormalTok{(}\FunctionTok{rownames}\NormalTok{(sample\_data), }\AttributeTok{levels =} \FunctionTok{rownames}\NormalTok{(sample\_data))}

\CommentTok{\# Create annotation plot}
\NormalTok{sample\_types\_annotation }\OtherTok{\textless{}{-}} \FunctionTok{ggplot}\NormalTok{(sample\_data) }\SpecialCharTok{+}
  \FunctionTok{scale\_y\_discrete}\NormalTok{(}\AttributeTok{position =} \StringTok{"right"}\NormalTok{, }\AttributeTok{expand =} \FunctionTok{c}\NormalTok{(}\DecValTok{0}\NormalTok{,}\DecValTok{0}\NormalTok{)) }\SpecialCharTok{+}
  \FunctionTok{geom\_tile}\NormalTok{(}\FunctionTok{aes}\NormalTok{(}\AttributeTok{y =} \ConstantTok{NA}\NormalTok{, }\AttributeTok{x =}\NormalTok{ sample, }\AttributeTok{fill =}\NormalTok{ sample\_types)) }\SpecialCharTok{+}
  \FunctionTok{coord\_equal}\NormalTok{(}\AttributeTok{ratio =} \DecValTok{1}\NormalTok{) }\SpecialCharTok{+}
  \FunctionTok{theme}\NormalTok{(}
        \AttributeTok{axis.text.x=}\FunctionTok{element\_blank}\NormalTok{(),}
        \AttributeTok{axis.text.y=}\FunctionTok{element\_blank}\NormalTok{(),}
        \AttributeTok{axis.title.x=}\FunctionTok{element\_blank}\NormalTok{(),}
        \AttributeTok{axis.ticks.x=}\FunctionTok{element\_blank}\NormalTok{(),}
        \AttributeTok{plot.margin=}\FunctionTok{margin}\NormalTok{(}\DecValTok{0}\NormalTok{,}\DecValTok{0}\NormalTok{,}\DecValTok{0}\NormalTok{,}\DecValTok{0}\NormalTok{),}
        \AttributeTok{axis.title.y.right =} \FunctionTok{element\_text}\NormalTok{(}\AttributeTok{angle=}\DecValTok{0}\NormalTok{, }\AttributeTok{vjust =} \FloatTok{0.5}\NormalTok{)}
\NormalTok{        ) }\SpecialCharTok{+}
      \FunctionTok{labs}\NormalTok{(}\AttributeTok{fill =} \StringTok{"Sample types"}\NormalTok{, }\AttributeTok{y =} \StringTok{"Sample types"}\NormalTok{)}
\CommentTok{\# to view the notation, run}
\CommentTok{\# sample\_types\_annotation}

\CommentTok{\# Create annotation plot}
\NormalTok{sample\_clusters\_annotation }\OtherTok{\textless{}{-}} \FunctionTok{ggplot}\NormalTok{(sample\_data) }\SpecialCharTok{+}
  \FunctionTok{scale\_y\_discrete}\NormalTok{(}\AttributeTok{position =} \StringTok{"right"}\NormalTok{, }\AttributeTok{expand =} \FunctionTok{c}\NormalTok{(}\DecValTok{0}\NormalTok{,}\DecValTok{0}\NormalTok{)) }\SpecialCharTok{+}
  \FunctionTok{geom\_tile}\NormalTok{(}\FunctionTok{aes}\NormalTok{(}\AttributeTok{y =} \ConstantTok{NA}\NormalTok{, }\AttributeTok{x =}\NormalTok{ sample, }\AttributeTok{fill =}\NormalTok{ clusters)) }\SpecialCharTok{+}
  \FunctionTok{coord\_equal}\NormalTok{(}\AttributeTok{ratio =} \DecValTok{1}\NormalTok{) }\SpecialCharTok{+}
  \FunctionTok{theme}\NormalTok{(}
        \AttributeTok{axis.text.x=}\FunctionTok{element\_blank}\NormalTok{(),}
        \AttributeTok{axis.text.y=}\FunctionTok{element\_blank}\NormalTok{(),}
        \AttributeTok{axis.title.x=}\FunctionTok{element\_blank}\NormalTok{(),}
        \AttributeTok{axis.ticks.x=}\FunctionTok{element\_blank}\NormalTok{(),}
        \AttributeTok{plot.margin=}\FunctionTok{margin}\NormalTok{(}\DecValTok{0}\NormalTok{,}\DecValTok{0}\NormalTok{,}\DecValTok{0}\NormalTok{,}\DecValTok{0}\NormalTok{),}
        \AttributeTok{axis.title.y.right =} \FunctionTok{element\_text}\NormalTok{(}\AttributeTok{angle=}\DecValTok{0}\NormalTok{, }\AttributeTok{vjust =} \FloatTok{0.5}\NormalTok{)}
\NormalTok{        ) }\SpecialCharTok{+}
      \FunctionTok{labs}\NormalTok{(}\AttributeTok{fill =} \StringTok{"Clusters"}\NormalTok{, }\AttributeTok{y =} \StringTok{"Clusters"}\NormalTok{)}
\CommentTok{\# to view the notation, run}
\CommentTok{\# sample\_clusters\_annotation}

\CommentTok{\# Order data based on clusters and sample types}
\NormalTok{mat }\OtherTok{\textless{}{-}}\NormalTok{ mat[}\FunctionTok{unfactor}\NormalTok{(taxa\_clusters}\SpecialCharTok{$}\NormalTok{Feature), }\FunctionTok{unfactor}\NormalTok{(sample\_data}\SpecialCharTok{$}\NormalTok{sample)]}

\CommentTok{\# ggplot requires data in melted format}
\NormalTok{melted\_mat }\OtherTok{\textless{}{-}} \FunctionTok{melt}\NormalTok{(mat)}
\FunctionTok{colnames}\NormalTok{(melted\_mat) }\OtherTok{\textless{}{-}} \FunctionTok{c}\NormalTok{(}\StringTok{"Taxa"}\NormalTok{, }\StringTok{"Sample"}\NormalTok{, }\StringTok{"clr\_z"}\NormalTok{)}

\CommentTok{\# Determines the scaling of colorss}
\NormalTok{maxval }\OtherTok{\textless{}{-}} \FunctionTok{round}\NormalTok{(}\FunctionTok{max}\NormalTok{(}\FunctionTok{abs}\NormalTok{(melted\_mat}\SpecialCharTok{$}\NormalTok{clr\_z)))}
\NormalTok{limits }\OtherTok{\textless{}{-}} \FunctionTok{c}\NormalTok{(}\SpecialCharTok{{-}}\NormalTok{maxval, maxval)}
\NormalTok{breaks }\OtherTok{\textless{}{-}} \FunctionTok{seq}\NormalTok{(}\AttributeTok{from =} \FunctionTok{min}\NormalTok{(limits), }\AttributeTok{to =} \FunctionTok{max}\NormalTok{(limits), }\AttributeTok{by =} \FloatTok{0.5}\NormalTok{)}
\NormalTok{colours }\OtherTok{\textless{}{-}} \FunctionTok{c}\NormalTok{(}\StringTok{"darkblue"}\NormalTok{, }\StringTok{"blue"}\NormalTok{, }\StringTok{"white"}\NormalTok{, }\StringTok{"red"}\NormalTok{, }\StringTok{"darkred"}\NormalTok{)}

\NormalTok{heatmap }\OtherTok{\textless{}{-}} \FunctionTok{ggplot}\NormalTok{(melted\_mat) }\SpecialCharTok{+} 
  \FunctionTok{geom\_tile}\NormalTok{(}\FunctionTok{aes}\NormalTok{(}\AttributeTok{x =}\NormalTok{ Sample, }\AttributeTok{y =}\NormalTok{ Taxa, }\AttributeTok{fill =}\NormalTok{ clr\_z)) }\SpecialCharTok{+}
  \FunctionTok{theme}\NormalTok{(}
    \AttributeTok{axis.title.y=}\FunctionTok{element\_blank}\NormalTok{(),}
    \AttributeTok{axis.title.x=}\FunctionTok{element\_blank}\NormalTok{(),}
    \AttributeTok{axis.ticks.y=}\FunctionTok{element\_blank}\NormalTok{(),}
    \AttributeTok{axis.text.x =} \FunctionTok{element\_text}\NormalTok{(}\AttributeTok{angle =} \DecValTok{90}\NormalTok{, }\AttributeTok{vjust =} \FloatTok{0.5}\NormalTok{, }\AttributeTok{hjust=}\DecValTok{1}\NormalTok{),}
    
    \AttributeTok{plot.margin=}\FunctionTok{margin}\NormalTok{(}\DecValTok{0}\NormalTok{,}\DecValTok{0}\NormalTok{,}\DecValTok{0}\NormalTok{,}\DecValTok{0}\NormalTok{), }\CommentTok{\# removes margins}
    \AttributeTok{legend.key.height=} \FunctionTok{unit}\NormalTok{(}\DecValTok{1}\NormalTok{, }\StringTok{\textquotesingle{}cm\textquotesingle{}}\NormalTok{)}
\NormalTok{    ) }\SpecialCharTok{+}
  \FunctionTok{scale\_fill\_gradientn}\NormalTok{(}\AttributeTok{name =} \StringTok{"CLR + Z transform"}\NormalTok{, }
                       \AttributeTok{breaks =}\NormalTok{ breaks, }
                       \AttributeTok{limits =}\NormalTok{ limits, }
                       \AttributeTok{colours =}\NormalTok{ colours) }\SpecialCharTok{+} 
  \FunctionTok{scale\_y\_discrete}\NormalTok{(}\AttributeTok{position =} \StringTok{"right"}\NormalTok{)}

\NormalTok{heatmap}
\end{Highlighting}
\end{Shaded}

\includegraphics{19_visualization_techniques_files/figure-latex/more_complex_heatmap-1.pdf}

\begin{Shaded}
\begin{Highlighting}[]
\FunctionTok{library}\NormalTok{(patchwork)}

\CommentTok{\# Create layout}
\NormalTok{design }\OtherTok{\textless{}{-}} \FunctionTok{c}\NormalTok{(}
\NormalTok{  patchwork}\SpecialCharTok{::}\FunctionTok{area}\NormalTok{(}\DecValTok{3}\NormalTok{, }\DecValTok{1}\NormalTok{, }\DecValTok{4}\NormalTok{, }\DecValTok{1}\NormalTok{),}
\NormalTok{  patchwork}\SpecialCharTok{::}\FunctionTok{area}\NormalTok{(}\DecValTok{1}\NormalTok{, }\DecValTok{2}\NormalTok{, }\DecValTok{1}\NormalTok{, }\DecValTok{3}\NormalTok{),}
\NormalTok{  patchwork}\SpecialCharTok{::}\FunctionTok{area}\NormalTok{(}\DecValTok{2}\NormalTok{, }\DecValTok{2}\NormalTok{, }\DecValTok{2}\NormalTok{, }\DecValTok{3}\NormalTok{),}
\NormalTok{  patchwork}\SpecialCharTok{::}\FunctionTok{area}\NormalTok{(}\DecValTok{3}\NormalTok{, }\DecValTok{2}\NormalTok{, }\DecValTok{4}\NormalTok{, }\DecValTok{3}\NormalTok{)}
\NormalTok{)}
\CommentTok{\# to view the design, run}
\CommentTok{\# plot(design)}

\CommentTok{\# Combine plots}
\NormalTok{plot }\OtherTok{\textless{}{-}}\NormalTok{ row\_annotation }\SpecialCharTok{+}\NormalTok{ sample\_clusters\_annotation }\SpecialCharTok{+}
\NormalTok{                         sample\_types\_annotation }\SpecialCharTok{+}
\NormalTok{             heatmap  }\SpecialCharTok{+}
    \FunctionTok{plot\_layout}\NormalTok{(}\AttributeTok{design =}\NormalTok{ design, }\AttributeTok{guides =} \StringTok{"collect"}\NormalTok{,}
                \CommentTok{\# Specify layout, collect legends}
                
                \CommentTok{\# Adjust widths and heights to align plots.}
                \CommentTok{\# When annotation plot is larger, it might not fit into}
        \CommentTok{\# its column/row.}
                \CommentTok{\# Then you need to make column/row larger.}
                
                \CommentTok{\# Relative widths and heights of each column and row:}
                \CommentTok{\# Currently, the width of the first column is 15 \% and the height of}
                \CommentTok{\# first two rows are 30 \% the size of others}
                
                \CommentTok{\# To get this work most of the times, you can adjust all sizes to be 1, i.e. equal, }
                \CommentTok{\# but then the gaps between plots are larger.}
                \AttributeTok{widths =} \FunctionTok{c}\NormalTok{(}\FloatTok{0.15}\NormalTok{, }\DecValTok{1}\NormalTok{, }\DecValTok{1}\NormalTok{),}
                \AttributeTok{heights =} \FunctionTok{c}\NormalTok{(}\FloatTok{0.3}\NormalTok{, }\FloatTok{0.3}\NormalTok{, }\DecValTok{1}\NormalTok{, }\DecValTok{1}\NormalTok{))}

\CommentTok{\# plot}
\end{Highlighting}
\end{Shaded}

\begin{Shaded}
\begin{Highlighting}[]
\CommentTok{\# Create layout}
\NormalTok{design }\OtherTok{\textless{}{-}} \FunctionTok{c}\NormalTok{(}
\NormalTok{  patchwork}\SpecialCharTok{::}\FunctionTok{area}\NormalTok{(}\DecValTok{4}\NormalTok{, }\DecValTok{1}\NormalTok{, }\DecValTok{5}\NormalTok{, }\DecValTok{1}\NormalTok{),}
\NormalTok{  patchwork}\SpecialCharTok{::}\FunctionTok{area}\NormalTok{(}\DecValTok{4}\NormalTok{, }\DecValTok{2}\NormalTok{, }\DecValTok{5}\NormalTok{, }\DecValTok{2}\NormalTok{),}
\NormalTok{  patchwork}\SpecialCharTok{::}\FunctionTok{area}\NormalTok{(}\DecValTok{1}\NormalTok{, }\DecValTok{3}\NormalTok{, }\DecValTok{1}\NormalTok{, }\DecValTok{4}\NormalTok{),}
\NormalTok{  patchwork}\SpecialCharTok{::}\FunctionTok{area}\NormalTok{(}\DecValTok{2}\NormalTok{, }\DecValTok{3}\NormalTok{, }\DecValTok{2}\NormalTok{, }\DecValTok{4}\NormalTok{),}
\NormalTok{  patchwork}\SpecialCharTok{::}\FunctionTok{area}\NormalTok{(}\DecValTok{3}\NormalTok{, }\DecValTok{3}\NormalTok{, }\DecValTok{3}\NormalTok{, }\DecValTok{4}\NormalTok{),}
\NormalTok{  patchwork}\SpecialCharTok{::}\FunctionTok{area}\NormalTok{(}\DecValTok{4}\NormalTok{, }\DecValTok{3}\NormalTok{, }\DecValTok{5}\NormalTok{, }\DecValTok{4}\NormalTok{)}
\NormalTok{)}

\CommentTok{\# to view the design, run}
\CommentTok{\# plot(design)}

\CommentTok{\# Combine plots}
\NormalTok{plot }\OtherTok{\textless{}{-}}\NormalTok{ taxa\_tree }\SpecialCharTok{+} 
\NormalTok{  row\_annotation }\SpecialCharTok{+}
\NormalTok{  sample\_tree }\SpecialCharTok{+} 
\NormalTok{  sample\_clusters\_annotation }\SpecialCharTok{+}
\NormalTok{  sample\_types\_annotation }\SpecialCharTok{+}
\NormalTok{  heatmap }\SpecialCharTok{+}
    \FunctionTok{plot\_layout}\NormalTok{(}\AttributeTok{design =}\NormalTok{ design, }\AttributeTok{guides =} \StringTok{"collect"}\NormalTok{, }\CommentTok{\# Specify layout, collect legends}
                \AttributeTok{widths =} \FunctionTok{c}\NormalTok{(}\FloatTok{0.2}\NormalTok{, }\FloatTok{0.15}\NormalTok{, }\DecValTok{1}\NormalTok{, }\DecValTok{1}\NormalTok{, }\DecValTok{1}\NormalTok{),}
                \AttributeTok{heights =} \FunctionTok{c}\NormalTok{(}\FloatTok{0.1}\NormalTok{, }\FloatTok{0.15}\NormalTok{, }\FloatTok{0.15}\NormalTok{, }\FloatTok{0.25}\NormalTok{, }\DecValTok{1}\NormalTok{, }\DecValTok{1}\NormalTok{))}

\NormalTok{plot}
\end{Highlighting}
\end{Shaded}

Heatmaps find several other applications in biclustering and
multi-assay analyses. These are discussed further in chapters
\ref{clustering} and \ref{multi-assay-analyses}.

\hypertarget{part-training}{%
\part{Training}\label{part-training}}

\hypertarget{training}{%
\chapter{Training}\label{training}}

The page provides practical information to support training and self-study.

\hypertarget{checklist}{%
\section{Checklist}\label{checklist}}

Brief checklist to prepare for training (see below for links).

\begin{itemize}
\tightlist
\item
  Install the recommended software
\item
  Watch the short online videos and familiarize with the other available material
\item
  Join Gitter online chat for support
\end{itemize}

\hypertarget{software}{%
\section{Recommended software}\label{software}}

We recommend to install and set up the relevant software packages on
your own computer as this will support later use. The essential
components to install include:

\begin{itemize}
\item
  \href{https://www.r-project.org/}{R (the latest official release)}
\item
  \href{https://www.rstudio.com/products/rstudio/download/}{RStudio};
  choose ``Rstudio Desktop'' to download the latest version. Check the
  \href{https://www.rstudio.com/}{Rstudio home page} for more
  information. RStudio is optional.
\item
  Install key R packages (Section \ref{packages} provides an installation script)
\item
  After a successful installation you can consider trying out examples
  from Section \ref{exercises} already before training. \textbf{You can run
  the workflows by simply copy-pasting examples.} You can then test
  further examples from this tutorial, modifying and applying these
  techniques to your own data. Plain source code for the individual chapters of this book are available via \href{https://github.com/microbiome/OMA/tree/master/R}{Github}
\end{itemize}

\hypertarget{material}{%
\section{Study material}\label{material}}

We encourage to familiarize with the material and test examples in advance.

\begin{itemize}
\item
  \href{https://www.youtube.com/playlist?list=PLjiXAZO27elAJEptP59BN3whVJ61XIkST}{Short online videos} on microbiome data science with R/Bioconductor
\item
  \href{https://microbiome.github.io/outreach/index.html}{Quarto presentations}
\item
  \href{https://github.com/microbiome/outreach}{Other outreach material}
\item
  \href{https://microbiome.github.io/OMA/}{Orchestrating Microbiome Analysis with R/Bioconductor (OMA)} (this book)
\item
  \href{https://microbiome.github.io/OMA/exercises.html}{Exercises} for self-study
\item
  \href{https://microbiome.github.io/OMA/resources.html}{Resources} and links to complementary external material
\end{itemize}

\hypertarget{support-and-resources}{%
\section{Support and resources}\label{support-and-resources}}

For online support on installation and other matters, join us at
\href{https://gitter.im/microbiome/miaverse?utm_source=badge\&utm_medium=badge\&utm_campaign=pr-badge\&utm_content=badge}{Gitter}.

You are also welcome to connect through various channels with our
broader \href{https://microbiome.github.io}{developer and user community}.

\hypertarget{coc}{%
\section{Code of Conduct}\label{coc}}

We support the \href{https://bioconductor.github.io/bioc_coc_multilingual/}{Bioconductor Code of Conduct}. The community values an open approach to science that promotes

\begin{itemize}
\tightlist
\item
  sharing of ideas, code, software and expertise
\item
  a kind and welcoming environment, diversity and inclusivity
\item
  community contributions and collaboration
\end{itemize}

\hypertarget{resources}{%
\chapter{Resources}\label{resources}}

\hypertarget{data-containers-1}{%
\section{Data containers}\label{data-containers-1}}

\hypertarget{resources-for-treesummarizedexperiment}{%
\subsection{Resources for TreeSummarizedExperiment}\label{resources-for-treesummarizedexperiment}}

\begin{itemize}
\tightlist
\item
  SingleCellExperiment \citep{R-SingleCellExperiment}

  \begin{itemize}
  \tightlist
  \item
    \href{https://bioconductor.org/packages/release/bioc/vignettes/SingleCellExperiment/inst/doc/intro.html}{Online tutorial}
  \item
    \href{https://bioconductor.org/packages/release/bioc/html/SingleCellExperiment.html}{Project page}
  \end{itemize}
\item
  SummarizedExperiment \citep{R-SummarizedExperiment}

  \begin{itemize}
  \tightlist
  \item
    \href{https://bioconductor.org/packages/release/bioc/vignettes/SummarizedExperiment/inst/doc/SummarizedExperiment.html}{Online tutorial}
  \item
    \href{https://bioconductor.org/packages/release/bioc/html/SummarizedExperiment.html}{Project page}
  \end{itemize}
\item
  TreeSummarizedExperiment \citep{R-TreeSummarizedExperiment}

  \begin{itemize}
  \tightlist
  \item
    \href{https://bioconductor.org/packages/release/bioc/vignettes/TreeSummarizedExperiment/inst/doc/Introduction_to_treeSummarizedExperiment.html}{Online tutorial}
  \item
    \href{https://www.bioconductor.org/packages/release/bioc/html/TreeSummarizedExperiment.html}{Project page}
  \item
    Publication: \citep{Huang2021}
  \end{itemize}
\end{itemize}

\hypertarget{other-relevant-containers}{%
\subsection{Other relevant containers}\label{other-relevant-containers}}

\begin{itemize}
\tightlist
\item
  \href{https://rdrr.io/bioc/S4Vectors/man/DataFrame-class.html}{DataFrame} which behaves similarly to \texttt{data.frame}, yet efficient and fast when used with large datasets.
\item
  \href{https://rdrr.io/bioc/Biostrings/man/DNAString-class.html}{DNAString} along with \texttt{DNAStringSet},\texttt{RNAString} and \texttt{RNAStringSet} efficient storage and handling of long biological sequences are offered within the Biostrings package \citep{R-Biostrings}.
\item
  GenomicRanges (\citep{GenomicRanges2013}) offers an efficient representation and manipulation of genomic annotations and alignments, see e.g.~\texttt{GRanges} and \texttt{GRangesList} at \href{https://bioconductor.org/packages/release/bioc/vignettes/GenomicRanges/inst/doc/GenomicRangesIntroduction.html}{An Introduction to the GenomicRangesPackage}.
\end{itemize}

\href{http://girke.bioinformatics.ucr.edu/GEN242/tutorials/rsequences/rsequences/}{NGS Analysis Basics} provides a walk-through of the above-mentioned features with detailed examples.

\hypertarget{alternative-containers-for-microbiome-data}{%
\subsection{Alternative containers for microbiome data}\label{alternative-containers-for-microbiome-data}}

The \texttt{phyloseq} package and class became the first widely used data
container for microbiome data science in R. Many methods for taxonomic
profiling data are readily available for this class. We provide here a
short description how \texttt{phyloseq} and \texttt{*Experiment} classes relate to
each other.

\texttt{assays} : This slot is similar to \texttt{otu\_table} in \texttt{phyloseq}. In a
\texttt{SummarizedExperiment} object multiple assays, raw
counts, transformed counts can be stored. See also
\citep{Ramos2017}
for storing data from multiple \texttt{experiments} such as
RNASeq, Proteomics, etc. \texttt{rowData} : This slot is
similar to \texttt{tax\_table} in \texttt{phyloseq} to store taxonomic
information. \texttt{colData} : This slot is similar to
\texttt{sample\_data} in \texttt{phyloseq} to store information
related to samples. \texttt{rowTree} : This slot is similar
to \texttt{phy\_tree} in \texttt{phyloseq} to store phylogenetic tree.

In this book, you will come across terms like \texttt{FeatureIDs} and
\texttt{SampleIDs}. \texttt{FeatureIDs} : These are basically OTU/ASV ids which are
row names in \texttt{assays} and \texttt{rowData}. \texttt{SampleIDs} : As the name
suggests, these are sample ids which are column names in \texttt{assays} and
row names in \texttt{colData}.

\texttt{FeatureIDs} and \texttt{SampleIDs} are used but the technical terms
\texttt{rownames} and \texttt{colnames} are encouraged to be used, since they relate
to actual objects we work with.

\hypertarget{resources-for-phyloseq}{%
\subsection{Resources for phyloseq}\label{resources-for-phyloseq}}

The (Tree)SummarizedExperiment objects can be converted into the alternative phyloseq format, for which further methods are available.

\begin{itemize}
\tightlist
\item
  \href{https://microsud.github.io/Tools-Microbiome-Analysis/}{List of R tools for microbiome analysis}
\item
  phyloseq \citep{McMurdie2013}
\item
  \href{http://microbiome.github.io/tutorials/}{microbiome tutorial}
\item
  \href{https://microsud.github.io/microbiomeutilities/}{microbiomeutilities}
\item
  Bioconductor Workflow for Microbiome Data Analysis: from raw reads to community analyses \citep{Callahan2016}.
\end{itemize}

\hypertarget{r-programming-resources}{%
\section{R programming resources}\label{r-programming-resources}}

If you are new to R, you could try \href{https://swirlstats.com/}{swirl}
for a kickstart to R programming. Further support resources are
available through the Bioconductor
project \citep{Huber2015}.

\begin{itemize}
\tightlist
\item
  R programming basics: \href{https://www.rstudio.com/wp-content/uploads/2016/10/r-cheat-sheet-3.pdf}{Base R}
\item
  Basics of R programming: \href{https://raw.githubusercontent.com/rstudio/cheatsheets/master/base-r.pdf}{Base R}
\item
  \href{https://www.rstudio.com/resources/cheatsheets/}{R cheat sheets}
\item
  R visualization with \href{https://www.rstudio.com/wp-content/uploads/2016/11/ggplot2-cheatsheet-2.1.pdf}{ggplot2}
\item
  \href{http://www.cookbook-r.com/Graphs/}{R graphics cookbook}
\end{itemize}

Rmarkdown

\begin{itemize}
\tightlist
\item
  Rmarkdown tips \citep{Xie2020}
\end{itemize}

RStudio

\begin{itemize}
\tightlist
\item
  \href{https://www.rstudio.com/wp-content/uploads/2016/01/rstudio-IDE-cheatsheet.pdf}{RStudio cheat sheet}
\end{itemize}

\hypertarget{bioc_intro}{%
\subsection{Bioconductor Classes}\label{bioc_intro}}

\textbf{S4 system}

S4 class system has brought several useful features to the
object-oriented programming paradigm within R, and it is constantly
deployed in R/Bioconductor packages \citep{Huber2015}.

~~Online Document:

\begin{itemize}
\tightlist
\item
  Hervé Pagès, \href{https://bioconductor.org/packages/release/bioc/vignettes/S4Vectors/inst/doc/S4QuickOverview.pdf}{A quick overview of the S4 class system}.
\item
  Laurent Gatto, \href{https://bioconductor.org/help/course-materials/2013/CSAMA2013/friday/afternoon/S4-tutorial.pdf}{A practical tutorial on S4 programming}
\item
  How S4 Methods Work \citep{Chambers2006}
\end{itemize}

~~Books:

\begin{itemize}
\tightlist
\item
  John M. Chambers. Software for Data Analysis: Programming with R. Springer, New York, 2008. ISBN-13 978-0387759357 \citep{Chambers2008}
\item
  I Robert Gentleman. R Programming for Bioinformatics. Chapman \& Hall/CRC, New York, 2008. ISBN-13 978-1420063677 \citep{gentleman2008r}
\end{itemize}

\hypertarget{quarto}{%
\section{Reproducible reporting with Quarto}\label{quarto}}

\hypertarget{learn-quarto}{%
\subsection{Learn Quarto}\label{learn-quarto}}

Reproducible reporting is the starting point for robust interactive
data science. Perform the following tasks:

\begin{itemize}
\item
  If you are entirely new to Quarto, take
  \href{https://quarto.org/docs/get-started/hello/rstudio.html}{this short tutorial}
  to get introduced to the most important functions within Quarto.
  Then experiment with different options from the
  \href{https://res.cloudinary.com/dyd911kmh/image/upload/v1676540721/Marketing/Blog/Quarto_Cheat_Sheet.pdf}{Quarto cheatsheet}.
\item
  Create a Quarto template in RStudio, and render it into a
  document (markdown, PDF, docx or other format). In case you are new
  to Quarto, its documentation provides guidelines to use Quarto with the
  R language (\href{https://quarto.org/docs/computations/r.html}{here})
  and the RStudio IDE (\href{https://quarto.org/docs/tools/rstudio.html}{here}).
\item
  Further examples are tips for Quarto are available in \href{https://appsilon.com/r-quarto-tutorial/}{this online tutorial}
  to interactive reproducible reporting.
\end{itemize}

\hypertarget{additional-material-on-rmarkdown}{%
\subsection{Additional material on Rmarkdown}\label{additional-material-on-rmarkdown}}

Being able to use Quarto in R partly relies on your previous knowledge of Rmarkdown. The following resources can help you get familiar with Rmarkdown:

\begin{itemize}
\tightlist
\item
  \href{https://www.markdowntutorial.com/}{online tutorial}
\item
  \href{https://www.rstudio.com/wp-content/uploads/2015/02/rmarkdown-cheatsheet.pdf}{cheatsheet}
\item
  \href{https://rmarkdown.rstudio.com/lesson-1.html}{documentation}
\item
  \href{https://rpubs.com/marschmi/RMarkdown}{Dr.~C Titus Brown's tutorial}
\end{itemize}

\textbf{Figure sources:}

\textbf{Original article}
- Huang R \emph{et al}. (2021) TreeSummarizedExperiment: a S4 class
for data with hierarchical structure. F1000Research 9:1246. \citep{Huang2021}

\textbf{Reference Sequence slot extension}
- Lahti L \emph{et al}. (2020) \href{https://doi.org/10.7490/\%20f1000research.1118447.1}{Upgrading the R/Bioconductor ecosystem for microbiome
research} F1000Research 9:1464 (slides).

\hypertarget{exercises}{%
\chapter{Exercises}\label{exercises}}

Here you can find assignments on different topics.

\textbf{Tips for exercises:}

\begin{itemize}
\tightlist
\item
  Add comments that explain what each line or lines of code do. This helps you and others to understand your code and find bugs. Furthermore, it is easier for you to reuse the code, and it promotes transparency.
\item
  Interpret results by comparing them to literature. List main findings, so that results can easily be understood by others without advanced data analytics knowledge.
\item
  Avoid hard-coding. Use variables which get values in the beginning of the pipeline. That way it is easier for you to modify parameters and reuse the code.
\end{itemize}

\hypertarget{workflows}{%
\section{Workflows}\label{workflows}}

\hypertarget{reproducible-reporting-with-quarto}{%
\subsection{Reproducible reporting with Quarto}\label{reproducible-reporting-with-quarto}}

\begin{enumerate}
\def\labelenumi{\arabic{enumi}.}
\tightlist
\item
  Create a new Quarto file
\end{enumerate}

\begin{itemize}
\tightlist
\item
  Rstudio has ready-made templates for this
\item
  \href{https://quarto.org/docs/tools/rstudio.html\#creating-documents}{Creating Quarto Documents}
\end{itemize}

\begin{enumerate}
\def\labelenumi{\arabic{enumi}.}
\setcounter{enumi}{1}
\tightlist
\item
  Add a code chunk and name it.
\item
  Render (or knit) the file into pdf or html format (Hint: \href{https://quarto.org/docs/tools/rstudio.html\#render-and-preview}{Quarto Rstudio rendering})
\item
  Import e.g., iris dataset, and add a dotplot with a title (Hint: \href{https://ggplot2.tidyverse.org/reference/geom_dotplot.html}{geom\_dotplot})
\item
  Create another code chunk and plot.
\item
  Adjust figure size and hide the code chunk from output report.
\end{enumerate}

\begin{itemize}
\tightlist
\item
  \href{https://quarto.org/docs/authoring/diagrams.html\#sizing}{Sizing}
\item
  \href{https://quarto.org/docs/computations/r.html\#chunk-options}{chunk-options}
\end{itemize}

\begin{enumerate}
\def\labelenumi{\arabic{enumi}.}
\setcounter{enumi}{6}
\tightlist
\item
  Add some text.
\item
  Add R commands within the text (Hint: \href{https://quarto.org/docs/visual-editor/technical.html\#code-chunks}{code-chunks})
\item
  Update HTML file from the qmd file (Hint: \href{https://quarto.org/docs/tools/rstudio.html\#render-and-preview}{Quarto Rstudio rendering})
\end{enumerate}

For tips on Quarto, see \href{https://quarto.org/docs/authoring/markdown-basics.html}{Quarto tutorial}

\hypertarget{data-containers-treese}{%
\section{Data containers: TreeSE}\label{data-containers-treese}}

\hypertarget{constructing-a-data-object}{%
\subsection{Constructing a data object}\label{constructing-a-data-object}}

Import data from CSV files to TreeSE (see shared data folder for example data sets).

\begin{enumerate}
\def\labelenumi{\arabic{enumi}.}
\tightlist
\item
  Import the data files in R
\item
  Construct a TreeSE data object (see \href{https://microbiome.github.io/OMA/containers.html\#loading-experimental-microbiome-data}{Ch. 2})
\item
  Check that importing is done correctly. E.g., choose random samples and features,
  and check that their values equal between raw files and TreeSE.
\end{enumerate}

Useful functions: DataFrame, TreeSummarizedExperiment, matrix, rownames, colnames, SimpleList

\hypertarget{importing-data}{%
\subsection{Importing data}\label{importing-data}}

You can also check the \href{https://microbiome.github.io/mia/reference/index.html}{function reference in the mia package}

\begin{enumerate}
\def\labelenumi{\arabic{enumi}.}
\tightlist
\item
  Import data from another format (functions: loadFromMetaphlan \textbar{} loadFromMothur \textbar{} loadFromQIIME2 \textbar{} makeTreeSummarizedExperimentFromBiom \textbar{} makeTreeSummarizedExperimentFromDADA2 \ldots)
\item
  Try out conversions between TreeSE and phyloseq data containers (makeTreeSummarizedExperimentFromPhyloseq; makephyloseqFromTreeSummarizedExperiment)
\end{enumerate}

\hypertarget{basic-summaries}{%
\subsection{Basic summaries}\label{basic-summaries}}

\begin{enumerate}
\def\labelenumi{\arabic{enumi}.}
\tightlist
\item
  Load experimental dataset from mia (e.g.~\texttt{peerj13075} with the \texttt{data()} command; see OMA \href{https://microbiome.github.io/OMA/containers.html\#example-data}{section 2.4 Demonstration Data}; also see \href{https://microbiome.github.io/OMA/containers.html\#assay-data}{loading experimental data}).
\item
  Check a summary about the TreeSE object loaded (Hint: \texttt{summary()})
\item
  What are the dimensions? (How many samples there are, and how many taxa in each taxonomic rank?) (Hint: material in \href{https://microbiome.github.io/OMA/containers.html\#data-science-framework}{Section 2} may help)
\item
  List sample and features names for the data (rownames, colnames..)
\end{enumerate}

\hypertarget{taxonomic-abundance-table-assay}{%
\subsection{Taxonomic abundance table (assay)}\label{taxonomic-abundance-table-assay}}

\begin{enumerate}
\def\labelenumi{\arabic{enumi}.}
\tightlist
\item
  (Load example data)
\item
  Fetch the list of available assays (Hints: \href{https://microbiome.github.io/OMA/containers.html\#assay-data}{assayNames})
\item
  Fetch the \texttt{counts} assay, and show part of it. (Hint: \href{https://microbiome.github.io/OMA/containers.html\#assay-data}{assay-data})
\end{enumerate}

\hypertarget{sample-side-information}{%
\subsection{Sample side information}\label{sample-side-information}}

\begin{enumerate}
\def\labelenumi{\arabic{enumi}.}
\tightlist
\item
  (Load example data)
\item
  Fetch and show data about samples (Hint: \href{https://microbiome.github.io/OMA/containers.html\#coldata}{colData})
\item
  Get abundance data for all taxa for a specific sample (sample names: function \texttt{colnames(tse)})
\end{enumerate}

\begin{itemize}
\tightlist
\item
  \href{https://microbiome.github.io/OMA/taxonomic-information.html\#abundances-of-all-taxa-in-specific-sample}{example}
\end{itemize}

\hypertarget{feature-side-information}{%
\subsection{Feature side information}\label{feature-side-information}}

\begin{enumerate}
\def\labelenumi{\arabic{enumi}.}
\tightlist
\item
  (Load example data)
\item
  Fetch and show data on the (taxonomic) features of the analyzed samples (Hint: \href{https://microbiome.github.io/OMA/containers.html\#rowdata}{rowData})
\item
  Get abundance data for all samples given a specific features (Hint: \href{https://microbiome.github.io/OMA/taxonomic-information.html\#abundances-of-specific-taxa-in-all-samples}{example})
\end{enumerate}

Optional:

\begin{enumerate}
\def\labelenumi{\arabic{enumi}.}
\setcounter{enumi}{3}
\tightlist
\item
  Create taxonomy tree based on the taxonomy mappings display its information:
\end{enumerate}

\begin{itemize}
\tightlist
\item
  \href{https://microbiome.github.io/OMA/taxonomic-information.html\#generate-a-taxonomic-tree-on-the-fly}{generate a taxonomic tree on the fly}
\item
  \href{https://microbiome.github.io/OMA/containers.html\#rowtree}{rowtree}
\end{itemize}

\hypertarget{other-elements}{%
\subsection{Other elements}\label{other-elements}}

Try to extract some of the \href{https://f1000research.com/articles/9-1246/v2}{other TreeSE elements}. These are not always included:

\begin{itemize}
\tightlist
\item
  Experiment metadata
\item
  Sample tree (colTree)
\item
  Phylogenetic tree (rowTree)
\item
  Feature sequences information (DNA sequence slot)
\end{itemize}

\hypertarget{data-manipulation}{%
\section{Data manipulation}\label{data-manipulation}}

\hypertarget{subsetting-1}{%
\subsection{Subsetting}\label{subsetting-1}}

\begin{enumerate}
\def\labelenumi{\arabic{enumi}.}
\tightlist
\item
  Subset the TreeSE object to specific samples
\item
  Subset the TreeSE object to specific features
\item
  Subset the TreeSE object to specific samples and features
\end{enumerate}

\hypertarget{library-sizes}{%
\subsection{Library sizes}\label{library-sizes}}

\begin{enumerate}
\def\labelenumi{\arabic{enumi}.}
\tightlist
\item
  Calculate library sizes
\item
  Subsample / rarify the counts (see: subsampleCounts)
\end{enumerate}

Useful functions: nrow, ncol, dim, summary, table, quantile, unique, addPerCellQC, agglomerateByRank

\hypertarget{prevalent-and-core-taxonomic-features}{%
\subsection{Prevalent and core taxonomic features}\label{prevalent-and-core-taxonomic-features}}

\begin{enumerate}
\def\labelenumi{\arabic{enumi}.}
\tightlist
\item
  Estimate prevalence for your chosen feature (row, taxonomic group)
\item
  Identify all prevalent features and subset the data accordingly
\item
  Report the thresholds and the dimensionality of the data before and after subsetting
\item
  Visualize prevalence
\end{enumerate}

Useful functions: getPrevalence, getPrevalentTaxa, subsetByPrevalentTaxa

\hypertarget{data-exploration}{%
\subsection{Data exploration}\label{data-exploration}}

\begin{enumerate}
\def\labelenumi{\arabic{enumi}.}
\tightlist
\item
  Summarize sample metadata variables. (How many age groups, how they are distributed? 0\%, 25\%, 50\%, 75\%, and 100\% quantiles of library size?)
\item
  Create two histograms. Another shows the distribution of absolute counts, another shows how CLR transformed values are distributed.
\item
  Visualize how relative abundances are distributed between taxa in samples.
\end{enumerate}

Useful functions: nrow, ncol, dim, summary, table, quantile, unique, transformCounts, ggplot, wilcox.test, agglomerateByRank, plotAbundance

\hypertarget{other-functions}{%
\subsection{Other functions}\label{other-functions}}

\begin{enumerate}
\def\labelenumi{\arabic{enumi}.}
\tightlist
\item
  Merge data objects (merge, mergeSEs)
\item
  Melt the data for visualization purposes (meltAssay)
\end{enumerate}

\hypertarget{transformations}{%
\subsection{Transformations}\label{transformations}}

\begin{enumerate}
\def\labelenumi{\arabic{enumi}.}
\tightlist
\item
  Transform abundance data with relative abundance and add a relative abundance assay (see \href{https://microbiome.github.io/OMA/taxonomic-information.html\#data-transformation}{data-transformation})
\item
  Transform abundance data with clr transformation and add a new assay
\item
  List the available assays by name
\item
  Pick one of the assays and show a subset of it
\item
  Subset the entire TreeSE data object, and check how this affects individual (transformed) assays
\end{enumerate}

Optional:

\begin{enumerate}
\def\labelenumi{\arabic{enumi}.}
\setcounter{enumi}{5}
\tightlist
\item
  If the data has phylogenetic tree, perform the phILR transformation
\end{enumerate}

\hypertarget{abundance-tables}{%
\section{Abundance tables}\label{abundance-tables}}

\hypertarget{taxonomic-levels}{%
\subsection{Taxonomic levels}\label{taxonomic-levels}}

\begin{enumerate}
\def\labelenumi{\arabic{enumi}.}
\tightlist
\item
  List the available taxonomic ranks in the data
\item
  Merge the data to Phylum level
\item
  Report dimensionality before and after aggomeration
\end{enumerate}

Optional:

\begin{enumerate}
\def\labelenumi{\arabic{enumi}.}
\setcounter{enumi}{3}
\tightlist
\item
  Perform CLR transformation on the data; does this affect agglomeration?
\item
  List full taxonomic information for some given taxa (Hint: \href{https://microbiome.github.io/mia/reference/taxonomy-methods.html}{mapTaxonomy})
\end{enumerate}

Useful functions: \href{https://microbiome.github.io/mia/reference/taxonomy-methods.html}{taxonomyRanks}, agglomerateByRank, mergeRows

\hypertarget{alternative-experiments-altexp}{%
\subsection{Alternative experiments (altExp)}\label{alternative-experiments-altexp}}

\begin{enumerate}
\def\labelenumi{\arabic{enumi}.}
\tightlist
\item
  Create taxonomic abundance tables for all different levels (splitByRanks)
\item
  Check the available alternative experiment (altExp) names before and after splitByRanks
\item
  Pick specific ``experiment'' (taxonomic rank) from specific altExp; and a specific assay
\end{enumerate}

Optional:

\begin{enumerate}
\def\labelenumi{\arabic{enumi}.}
\setcounter{enumi}{3}
\tightlist
\item
  Split the data based on other features (splitOn)
\end{enumerate}

\hypertarget{community-diversity-alpha-diversity}{%
\section{Community diversity (alpha diversity)}\label{community-diversity-alpha-diversity}}

\hypertarget{alpha-diversity-basics}{%
\subsection{Alpha diversity basics}\label{alpha-diversity-basics}}

\begin{enumerate}
\def\labelenumi{\arabic{enumi}.}
\tightlist
\item
  Calculate alpha diversity indices
\item
  Test if data agglomeration to higher taxonomic ranks affects the indices
\item
  Look for differences in alpha diversity between groups or correlation with a continuous variable
\end{enumerate}

Useful functions: estimateDiversity, colSums, agglomerateByRank, kruskal.test, cor

\hypertarget{alpha-diversity-extra}{%
\subsection{Alpha diversity extra}\label{alpha-diversity-extra}}

\begin{enumerate}
\def\labelenumi{\arabic{enumi}.}
\tightlist
\item
  Estimate Shannon diversity for the data
\item
  Try also another diversity index and compare the results with a scatterplot
\item
  Compare Shannon diversity between groups (boxplot)
\item
  Is diversity significantly different between vegan and mixed diet?
\item
  Calculate and visualize library size, compare with diversity
\end{enumerate}

Useful functions: estimateDiversity, colSums, geom\_point, geom\_boxplot

\hypertarget{community-composition-beta-diversity}{%
\section{Community composition (beta diversity)}\label{community-composition-beta-diversity}}

\hypertarget{beta-diversity-basics}{%
\subsection{Beta diversity basics}\label{beta-diversity-basics}}

\begin{enumerate}
\def\labelenumi{\arabic{enumi}.}
\tightlist
\item
  Visualize community variation with different methods (PCA, MDS, NMDS, etc.) with plotReduceDim and with different dissimilarities and transformations,plot also other than the first two axes.
\item
  Use PERMANOVA to test differences in beta diversity. You can also try including continuous and/or categorical covariates
\item
  If there are statistical differences in PERMANOVA, test PERMDISP2 (betadisper function)
\item
  Do clustering
\item
  Try RDA to test the variance explained by external variables
\end{enumerate}

\hypertarget{beta-diversity-extra}{%
\subsection{Beta diversity extra}\label{beta-diversity-extra}}

\begin{enumerate}
\def\labelenumi{\arabic{enumi}.}
\tightlist
\item
  Install the latest development version of mia from GitHub.
\item
  Load experimental dataset from mia.
\item
  Create a PCoA with Aitchison dissimilarities. How much coordinate 1 explains the differences? How about coordinate 2?
\item
  Create dbRDA with Bray-Curtis dissimilarities on relative abundances. Use PERMANOVA. Can differences between samples be explained with variables of sample meta data?
\item
  Analyze diets' association on beta diversity. Calculate dbRDA and then PERMANOVA. Visualize coefficients. Which taxa's abundances differ the most between samples?
\item
  Interpret your results. Is there association between community composition and location? What are those taxa that differ the most; find information from literature.
\end{enumerate}

Useful functions: runMDS, runRDA, anova.cca, transformCounts, agglomerateByRank, ggplot, plotReducedDim, vegan::adonis2

\hypertarget{differential-abundance-1}{%
\section{Differential abundance}\label{differential-abundance-1}}

\hypertarget{univariate-analyses}{%
\subsection{Univariate analyses}\label{univariate-analyses}}

\begin{enumerate}
\def\labelenumi{\arabic{enumi}.}
\tightlist
\item
  Get the abundances for an individual feature (taxonomic group / row)
\item
  Visualize the abundances per group with boxplot / jitterplot
\item
  Is the difference significant (Wilcoxon test)?
\item
  Is the difference significant (linear model with covariates)?
\item
  How do transformations affect the outcome (log10, clr..)?
\item
  Get p-values for all features (taxa), for instance with a for loop
\item
  Do multiple testing correction
\item
  Compare the results from different tests with a scatterplot
\end{enumerate}

Useful functions: {[}{]}, ggplot2::geom\_boxplot, ggplot2::geom\_jitter, wilcox.test, lm.test, transformCounts, p.adjust

\hypertarget{differential-abundance-analysis-1}{%
\subsection{Differential abundance analysis}\label{differential-abundance-analysis-1}}

\begin{enumerate}
\def\labelenumi{\arabic{enumi}.}
\tightlist
\item
  install the latest development version of mia from GitHub.
\item
  Load experimental dataset from mia.
\item
  Compare abundances of each taxa between groups. First, use Wilcoxon or Kruskall-Wallis test. Then use some other method dedicated to microbiome data.
\item
  Summarize findings by plotting a taxa vs samples heatmap. Add column annotation that tells the group of each sample, and row annotation that tells whether the difference of certain taxa was statistically significant.
\item
  Choose statistically significant taxa and visualize their abundances with boxplot \& jitterplot.
\end{enumerate}

Useful functions: wilcox.test, kruskal.test, ggplot, pheatmap, ComplexHeatMap::Heatmap, ancombc, aldex2, maaslin2, agglomerateByRank, transformCounts, subsetByPrevalentTaxa

\hypertarget{visualization-1}{%
\section{Visualization}\label{visualization-1}}

\hypertarget{multivariate-ordination}{%
\subsection{Multivariate ordination}\label{multivariate-ordination}}

\begin{enumerate}
\def\labelenumi{\arabic{enumi}.}
\tightlist
\item
  Load experimental dataset from mia.
\item
  Create PCoA with Bray-Curtis dissimilarities
\item
  Create PCA with Aitchison dissimilarities
\item
  Visualize and compare both
\item
  Test other transformations, dissimilarities, and ordination methods
\end{enumerate}

Useful functions: runMDS, runNMDS, transformCounts, ggplot, plotReducedDim

\hypertarget{heatmap-visualization}{%
\subsection{Heatmap visualization}\label{heatmap-visualization}}

\begin{enumerate}
\def\labelenumi{\arabic{enumi}.}
\tightlist
\item
  Load experimental dataset from mia.
\item
  Visualize abundances with heatmap
\item
  Visualize abundances with heatmap after CLR + Z transformation
\end{enumerate}

See the OMA book for examples.

\hypertarget{multiomics}{%
\section{Multiomics}\label{multiomics}}

\hypertarget{multiassayexperiment-mae-data-container}{%
\subsection{MultiAssayExperiment (MAE) data container}\label{multiassayexperiment-mae-data-container}}

\begin{enumerate}
\def\labelenumi{\arabic{enumi}.}
\tightlist
\item
  Create TreeSE data containers from individual CSV files.
\item
  Combine TreeSE into MAE.
\item
  Check that each individual experiment of MAE equals corresponding TreeSE.
\item
  Take a subset of MAE (e.g., 10 first samples), and observe the subsetted MAE.
\end{enumerate}

Useful functions: DataFrame, TreeSummarizedExperiment, matrix, rownames, colnames, MultiAssayExperiment, ExperimentList, SimpleList

\hypertarget{multi-omic-data-exploration}{%
\subsection{Multi-omic data exploration}\label{multi-omic-data-exploration}}

\begin{enumerate}
\def\labelenumi{\arabic{enumi}.}
\tightlist
\item
  Load experimental dataset from microbiomeDataSets (e.g., HintikkaXOData).
\item
  Analyze correlations between experiments. (Taxa vs lipids, Taxa vs biomarkers, Lipids vs biomarkers)
\item
  Agglomerate taxa data.
\item
  Apply CLR to taxa data, apply log10 to lipids and biomarkers.
\item
  Perform cross-correlation analyses and visualize results with heatmaps. (Use Spearman coefficients)
\item
  Is there significant correlations? Interpret your results.
\end{enumerate}

Useful functions: pheatmap, ComplexHeatMap::Heatmap, ggplot, transformCounts, testExperimentCrossAssociation

\hypertarget{part-appendix}{%
\part{Appendix}\label{part-appendix}}

\hypertarget{extras}{%
\chapter{Extra material}\label{extras}}

\hypertarget{permanova-comparison}{%
\section{PERMANOVA comparison}\label{permanova-comparison}}

Here we present two possible uses of the \texttt{adonis2} function which performs PERMANOVA. The
optional argument \texttt{by} has an effect on the statistical outcome, so its two options are
compared here.

\begin{Shaded}
\begin{Highlighting}[]
\CommentTok{\# import necessary packages}
\FunctionTok{library}\NormalTok{(gtools)}
\FunctionTok{library}\NormalTok{(purrr)}
\FunctionTok{library}\NormalTok{(vegan)}
\FunctionTok{library}\NormalTok{(gtools)}
\FunctionTok{library}\NormalTok{(purrr)}
\end{Highlighting}
\end{Shaded}

Let us load the \emph{enterotype} TSE object and run PERMANOVA for
different orders of three variables with two different approaches:
\texttt{by\ =\ "margin"} or \texttt{by\ =\ "terms"}.

\begin{Shaded}
\begin{Highlighting}[]
\CommentTok{\# load and prepare data}
\FunctionTok{library}\NormalTok{(mia)}
\FunctionTok{data}\NormalTok{(}\StringTok{"enterotype"}\NormalTok{, }\AttributeTok{package=}\StringTok{"mia"}\NormalTok{)}
\NormalTok{enterotype }\OtherTok{\textless{}{-}} \FunctionTok{transformCounts}\NormalTok{(enterotype, }\AttributeTok{method =} \StringTok{"relabundance"}\NormalTok{)}
\CommentTok{\# drop samples missing meta data}
\NormalTok{enterotype }\OtherTok{\textless{}{-}}\NormalTok{ enterotype[ , }\SpecialCharTok{!}\FunctionTok{rowSums}\NormalTok{(}\FunctionTok{is.na}\NormalTok{(}\FunctionTok{colData}\NormalTok{(enterotype)[, }\FunctionTok{c}\NormalTok{(}\StringTok{"Nationality"}\NormalTok{, }\StringTok{"Gender"}\NormalTok{, }\StringTok{"ClinicalStatus"}\NormalTok{)]) }\SpecialCharTok{\textgreater{}} \DecValTok{0}\NormalTok{)]}
\CommentTok{\# define variables and list all possible combinations}
\NormalTok{vars }\OtherTok{\textless{}{-}} \FunctionTok{c}\NormalTok{(}\StringTok{"Nationality"}\NormalTok{, }\StringTok{"Gender"}\NormalTok{, }\StringTok{"ClinicalStatus"}\NormalTok{)}
\NormalTok{var\_perm }\OtherTok{\textless{}{-}} \FunctionTok{permutations}\NormalTok{(}\AttributeTok{n =} \DecValTok{3}\NormalTok{, }\AttributeTok{r =} \DecValTok{3}\NormalTok{, vars)}
\NormalTok{formulas }\OtherTok{\textless{}{-}} \FunctionTok{apply}\NormalTok{(var\_perm, }\DecValTok{1}\NormalTok{, }\ControlFlowTok{function}\NormalTok{(row) purrr}\SpecialCharTok{::}\FunctionTok{reduce}\NormalTok{(row, }\ControlFlowTok{function}\NormalTok{(x, y) }\FunctionTok{paste}\NormalTok{(x, }\StringTok{"+"}\NormalTok{, y)))}
\CommentTok{\# create empty data.frames for further storing p{-}values}
\NormalTok{terms\_df }\OtherTok{\textless{}{-}} \FunctionTok{data.frame}\NormalTok{(}\StringTok{"Formula"} \OtherTok{=}\NormalTok{ formulas,}
                       \StringTok{"ClinicalStatus"} \OtherTok{=} \FunctionTok{rep}\NormalTok{(}\DecValTok{0}\NormalTok{, }\DecValTok{6}\NormalTok{),}
                       \StringTok{"Gender"} \OtherTok{=} \FunctionTok{rep}\NormalTok{(}\DecValTok{0}\NormalTok{, }\DecValTok{6}\NormalTok{),}
                       \StringTok{"Nationality"} \OtherTok{=} \FunctionTok{rep}\NormalTok{(}\DecValTok{0}\NormalTok{, }\DecValTok{6}\NormalTok{))}
\NormalTok{margin\_df }\OtherTok{\textless{}{-}} \FunctionTok{data.frame}\NormalTok{(}\StringTok{"Formula"} \OtherTok{=}\NormalTok{ formulas,}
                        \StringTok{"ClinicalStatus"} \OtherTok{=} \FunctionTok{rep}\NormalTok{(}\DecValTok{0}\NormalTok{, }\DecValTok{6}\NormalTok{),}
                        \StringTok{"Gender"} \OtherTok{=} \FunctionTok{rep}\NormalTok{(}\DecValTok{0}\NormalTok{, }\DecValTok{6}\NormalTok{),}
                        \StringTok{"Nationality"} \OtherTok{=} \FunctionTok{rep}\NormalTok{(}\DecValTok{0}\NormalTok{, }\DecValTok{6}\NormalTok{))}
\end{Highlighting}
\end{Shaded}

\begin{Shaded}
\begin{Highlighting}[]
\ControlFlowTok{for}\NormalTok{ (row\_idx }\ControlFlowTok{in} \DecValTok{1}\SpecialCharTok{:}\FunctionTok{nrow}\NormalTok{(var\_perm)) \{}
  
  \CommentTok{\# generate temporary formula (i.e. "assay \textasciitilde{} ClinicalStatus + Nationality + Gender")}
\NormalTok{  tmp\_formula }\OtherTok{\textless{}{-}}\NormalTok{ purrr}\SpecialCharTok{::}\FunctionTok{reduce}\NormalTok{(var\_perm[row\_idx, ], }\ControlFlowTok{function}\NormalTok{(x, y) }\FunctionTok{paste}\NormalTok{(x, }\StringTok{"+"}\NormalTok{, y))}
\NormalTok{  tmp\_formula }\OtherTok{\textless{}{-}} \FunctionTok{as.formula}\NormalTok{(}\FunctionTok{paste0}\NormalTok{(}\StringTok{\textquotesingle{}t(assay(enterotype, "relabundance")) \textasciitilde{} \textquotesingle{}}\NormalTok{,}
\NormalTok{                            tmp\_formula))}

  \CommentTok{\# multiple variables, default: by = "terms"}
  \FunctionTok{set.seed}\NormalTok{(}\DecValTok{75}\NormalTok{)}
\NormalTok{  with\_terms }\OtherTok{\textless{}{-}} \FunctionTok{adonis2}\NormalTok{(tmp\_formula, }
                \AttributeTok{by =} \StringTok{"terms"}\NormalTok{,}
                \AttributeTok{data =} \FunctionTok{colData}\NormalTok{(enterotype),}
                \AttributeTok{permutations =} \DecValTok{99}\NormalTok{)}
  
  \CommentTok{\# multiple variables, by = "margin"}
  \FunctionTok{set.seed}\NormalTok{(}\DecValTok{75}\NormalTok{)}
\NormalTok{  with\_margin }\OtherTok{\textless{}{-}} \FunctionTok{adonis2}\NormalTok{(tmp\_formula, }
                 \AttributeTok{by =} \StringTok{"margin"}\NormalTok{,}
                 \AttributeTok{data =} \FunctionTok{colData}\NormalTok{(enterotype),}
                 \AttributeTok{permutations =} \DecValTok{99}\NormalTok{)}

  \CommentTok{\# extract p{-}values}
\NormalTok{  terms\_p }\OtherTok{\textless{}{-}}\NormalTok{ with\_terms[[}\StringTok{"Pr(\textgreater{}F)"}\NormalTok{]]}
\NormalTok{  terms\_p }\OtherTok{\textless{}{-}}\NormalTok{ terms\_p[}\SpecialCharTok{!}\FunctionTok{is.na}\NormalTok{(terms\_p)]}
\NormalTok{  margin\_p }\OtherTok{\textless{}{-}}\NormalTok{ with\_margin[[}\StringTok{"Pr(\textgreater{}F)"}\NormalTok{]]}
\NormalTok{  margin\_p }\OtherTok{\textless{}{-}}\NormalTok{ margin\_p[}\SpecialCharTok{!}\FunctionTok{is.na}\NormalTok{(margin\_p)]}
  
  \CommentTok{\# store p{-}values into data.frames}
  \ControlFlowTok{for}\NormalTok{ (col\_idx }\ControlFlowTok{in} \DecValTok{1}\SpecialCharTok{:}\FunctionTok{ncol}\NormalTok{(var\_perm)) \{}
    
\NormalTok{    terms\_df[var\_perm[row\_idx, col\_idx]][row\_idx, ] }\OtherTok{\textless{}{-}}\NormalTok{ terms\_p[col\_idx]}
\NormalTok{    margin\_df[var\_perm[row\_idx, col\_idx]][row\_idx, ] }\OtherTok{\textless{}{-}}\NormalTok{ margin\_p[col\_idx]}
    
\NormalTok{  \}}
  
\NormalTok{\}}
\end{Highlighting}
\end{Shaded}

The following table displays the p-values for the three variables
ClinicalStatus, Gender and Nationality obtained by PERMANOVA with
\texttt{adonis2}. Note that the p-values remain identical when \texttt{by\ =\ "margin"}, but change with the order of the variables in the
formula when \texttt{by\ =\ "terms"} (default).

\begin{Shaded}
\begin{Highlighting}[]
\NormalTok{df }\OtherTok{\textless{}{-}}\NormalTok{ terms\_df }\SpecialCharTok{\%\textgreater{}\%}
\NormalTok{  dplyr}\SpecialCharTok{::}\FunctionTok{inner\_join}\NormalTok{(margin\_df, }\AttributeTok{by =} \StringTok{"Formula"}\NormalTok{, }\AttributeTok{suffix =} \FunctionTok{c}\NormalTok{(}\StringTok{" (terms)"}\NormalTok{, }\StringTok{" (margin)"}\NormalTok{))}

\NormalTok{knitr}\SpecialCharTok{::}\FunctionTok{kable}\NormalTok{(df)}
\end{Highlighting}
\end{Shaded}

\begin{tabular}{l|r|r|r|r|r|r}
\hline
Formula & ClinicalStatus (terms) & Gender (terms) & Nationality (terms) & ClinicalStatus (margin) & Gender (margin) & Nationality (margin)\\
\hline
ClinicalStatus + Gender + Nationality & 0.20 & 0.70 & 0.04 & 0.53 & 0.29 & 0.04\\
\hline
ClinicalStatus + Nationality + Gender & 0.20 & 0.29 & 0.05 & 0.53 & 0.29 & 0.04\\
\hline
Gender + ClinicalStatus + Nationality & 0.17 & 0.79 & 0.04 & 0.53 & 0.29 & 0.04\\
\hline
Gender + Nationality + ClinicalStatus & 0.53 & 0.79 & 0.03 & 0.53 & 0.29 & 0.04\\
\hline
Nationality + ClinicalStatus + Gender & 0.61 & 0.29 & 0.04 & 0.53 & 0.29 & 0.04\\
\hline
Nationality + Gender + ClinicalStatus & 0.53 & 0.39 & 0.04 & 0.53 & 0.29 & 0.04\\
\hline
\end{tabular}

\hypertarget{bayesian-multinomial-logistic-normal-models}{%
\section{Bayesian Multinomial Logistic-Normal Models}\label{bayesian-multinomial-logistic-normal-models}}

Analysis using such model could be performed with the function
\texttt{pibble} from the \texttt{fido} package, wihch is in form of a Multinomial
Logistic-Normal Linear Regression model; see
\href{https://jsilve24.github.io/fido/articles/introduction-to-fido.html}{vignette}
of package.

The following presents such an exemplary analysis based on the
data of \citet{Sprockett2020} available
through \texttt{microbiomeDataSets} package.

\begin{Shaded}
\begin{Highlighting}[]
\FunctionTok{library}\NormalTok{(fido)}
\end{Highlighting}
\end{Shaded}

Loading the libraries and importing data:

\begin{Shaded}
\begin{Highlighting}[]
\FunctionTok{library}\NormalTok{(fido)}
\end{Highlighting}
\end{Shaded}

\begin{Shaded}
\begin{Highlighting}[]
\FunctionTok{library}\NormalTok{(microbiomeDataSets)}
\NormalTok{tse }\OtherTok{\textless{}{-}} \FunctionTok{SprockettTHData}\NormalTok{()}
\end{Highlighting}
\end{Shaded}

We pick three covariates (``Sex'',``Age\_Years'',``Delivery\_Mode'') during this
analysis as an example, and beforehand we check for missing data:

\begin{Shaded}
\begin{Highlighting}[]
\FunctionTok{library}\NormalTok{(mia)}
\NormalTok{cov\_names }\OtherTok{\textless{}{-}} \FunctionTok{c}\NormalTok{(}\StringTok{"Sex"}\NormalTok{,}\StringTok{"Age\_Years"}\NormalTok{,}\StringTok{"Delivery\_Mode"}\NormalTok{)}
\NormalTok{na\_counts }\OtherTok{\textless{}{-}} \FunctionTok{apply}\NormalTok{(}\FunctionTok{is.na}\NormalTok{(}\FunctionTok{colData}\NormalTok{(tse)[,cov\_names]), }\DecValTok{2}\NormalTok{, sum)}
\NormalTok{na\_summary}\OtherTok{\textless{}{-}}\FunctionTok{as.data.frame}\NormalTok{(na\_counts,}\AttributeTok{row.names=}\NormalTok{cov\_names)}
\end{Highlighting}
\end{Shaded}

We drop missing values of the covariates:

\begin{Shaded}
\begin{Highlighting}[]
\NormalTok{tse }\OtherTok{\textless{}{-}}\NormalTok{ tse[ , }\SpecialCharTok{!}\FunctionTok{is.na}\NormalTok{(}\FunctionTok{colData}\NormalTok{(tse)}\SpecialCharTok{$}\NormalTok{Delivery\_Mode) ]}
\NormalTok{tse }\OtherTok{\textless{}{-}}\NormalTok{ tse[ , }\SpecialCharTok{!}\FunctionTok{is.na}\NormalTok{(}\FunctionTok{colData}\NormalTok{(tse)}\SpecialCharTok{$}\NormalTok{Age\_Years) ]}
\end{Highlighting}
\end{Shaded}

We agglomerate microbiome data to Phylum:

\begin{Shaded}
\begin{Highlighting}[]
\NormalTok{tse\_phylum }\OtherTok{\textless{}{-}} \FunctionTok{agglomerateByRank}\NormalTok{(tse, }\StringTok{"Phylum"}\NormalTok{)}
\end{Highlighting}
\end{Shaded}

We extract the counts assay and covariate data to build the model
matrix:

\begin{Shaded}
\begin{Highlighting}[]
\NormalTok{Y }\OtherTok{\textless{}{-}} \FunctionTok{assays}\NormalTok{(tse\_phylum)}\SpecialCharTok{$}\NormalTok{counts}
\CommentTok{\# design matrix}
\CommentTok{\# taking 3 covariates}
\NormalTok{sample\_data}\OtherTok{\textless{}{-}}\FunctionTok{as.data.frame}\NormalTok{(}\FunctionTok{colData}\NormalTok{(tse\_phylum)[,cov\_names])}
\NormalTok{X }\OtherTok{\textless{}{-}} \FunctionTok{t}\NormalTok{(}\FunctionTok{model.matrix}\NormalTok{(}\SpecialCharTok{\textasciitilde{}}\NormalTok{Sex}\SpecialCharTok{+}\NormalTok{Age\_Years}\SpecialCharTok{+}\NormalTok{Delivery\_Mode,}\AttributeTok{data=}\NormalTok{sample\_data))}
\end{Highlighting}
\end{Shaded}

Building the parameters for the \texttt{pibble} call to build the model; see more at \href{https://jsilve24.github.io/fido/articles/introduction-to-fido.html}{vignette}:

\begin{Shaded}
\begin{Highlighting}[]
\NormalTok{n\_taxa}\OtherTok{\textless{}{-}}\FunctionTok{nrow}\NormalTok{(Y)}
\NormalTok{upsilon }\OtherTok{\textless{}{-}}\NormalTok{ n\_taxa}\SpecialCharTok{+}\DecValTok{3}
\NormalTok{Omega }\OtherTok{\textless{}{-}} \FunctionTok{diag}\NormalTok{(n\_taxa)}
\NormalTok{G }\OtherTok{\textless{}{-}} \FunctionTok{cbind}\NormalTok{(}\FunctionTok{diag}\NormalTok{(n\_taxa}\DecValTok{{-}1}\NormalTok{), }\SpecialCharTok{{-}}\DecValTok{1}\NormalTok{)}
\NormalTok{Xi }\OtherTok{\textless{}{-}}\NormalTok{ (upsilon}\SpecialCharTok{{-}}\NormalTok{n\_taxa)}\SpecialCharTok{*}\NormalTok{G}\SpecialCharTok{\%*\%}\NormalTok{Omega}\SpecialCharTok{\%*\%}\FunctionTok{t}\NormalTok{(G)}
\NormalTok{Theta }\OtherTok{\textless{}{-}} \FunctionTok{matrix}\NormalTok{(}\DecValTok{0}\NormalTok{, n\_taxa}\DecValTok{{-}1}\NormalTok{, }\FunctionTok{nrow}\NormalTok{(X))}
\NormalTok{Gamma }\OtherTok{\textless{}{-}} \FunctionTok{diag}\NormalTok{(}\FunctionTok{nrow}\NormalTok{(X))}
\end{Highlighting}
\end{Shaded}

Automatically initializing the priors and visualizing their distributions:

\begin{Shaded}
\begin{Highlighting}[]
\NormalTok{priors }\OtherTok{\textless{}{-}} \FunctionTok{pibble}\NormalTok{(}\ConstantTok{NULL}\NormalTok{, X, upsilon, Theta, Gamma, Xi)}
\FunctionTok{names\_covariates}\NormalTok{(priors) }\OtherTok{\textless{}{-}} \FunctionTok{rownames}\NormalTok{(X)}
\FunctionTok{plot}\NormalTok{(priors, }\AttributeTok{pars=}\StringTok{"Lambda"}\NormalTok{) }\SpecialCharTok{+}\NormalTok{ ggplot2}\SpecialCharTok{::}\FunctionTok{xlim}\NormalTok{(}\FunctionTok{c}\NormalTok{(}\SpecialCharTok{{-}}\DecValTok{5}\NormalTok{, }\DecValTok{5}\NormalTok{))}
\end{Highlighting}
\end{Shaded}

\includegraphics{97_extra_materials_files/figure-latex/unnamed-chunk-10-1.pdf}

Estimating the posterior by including our response data \texttt{Y}.
Note: Some computational failures could occur (see \href{https://github-wiki-see.page/m/jsilve24/fido/wiki/Frequently-Asked-Questions}{discussion})
the arguments \texttt{multDirichletBoot} \texttt{calcGradHess} could be passed in such case.

\begin{Shaded}
\begin{Highlighting}[]
\NormalTok{priors}\SpecialCharTok{$}\NormalTok{Y }\OtherTok{\textless{}{-}}\NormalTok{ Y }
\NormalTok{posterior }\OtherTok{\textless{}{-}} \FunctionTok{refit}\NormalTok{(priors, }\AttributeTok{optim\_method=}\StringTok{"adam"}\NormalTok{, }\AttributeTok{multDirichletBoot=}\FloatTok{0.5}\NormalTok{) }\CommentTok{\#calcGradHess=FALSE}
\end{Highlighting}
\end{Shaded}

Printing a summary about the posterior:

\begin{Shaded}
\begin{Highlighting}[]
\FunctionTok{ppc\_summary}\NormalTok{(posterior)}
\end{Highlighting}
\end{Shaded}

\begin{verbatim}
## Proportions of Observations within 95% Credible Interval: 0.9979
\end{verbatim}

Plotting the summary of the posterior distributions of the regression parameters:

\begin{Shaded}
\begin{Highlighting}[]
\FunctionTok{names\_categories}\NormalTok{(posterior) }\OtherTok{\textless{}{-}} \FunctionTok{rownames}\NormalTok{(Y)}
\FunctionTok{plot}\NormalTok{(posterior,}\AttributeTok{par=}\StringTok{"Lambda"}\NormalTok{,}\AttributeTok{focus.cov=}\FunctionTok{rownames}\NormalTok{(X)[}\DecValTok{2}\SpecialCharTok{:}\DecValTok{4}\NormalTok{])}
\end{Highlighting}
\end{Shaded}

\includegraphics{97_extra_materials_files/figure-latex/unnamed-chunk-13-1.pdf}

Taking a closer look at ``Sex'' and ``Delivery\_Mode'':

\begin{Shaded}
\begin{Highlighting}[]
\FunctionTok{plot}\NormalTok{(posterior, }\AttributeTok{par=}\StringTok{"Lambda"}\NormalTok{, }\AttributeTok{focus.cov =} \FunctionTok{rownames}\NormalTok{(X)[}\FunctionTok{c}\NormalTok{(}\DecValTok{2}\NormalTok{,}\DecValTok{4}\NormalTok{)])}
\end{Highlighting}
\end{Shaded}

\includegraphics{97_extra_materials_files/figure-latex/unnamed-chunk-14-1.pdf}

\hypertarget{interactive-3d-plots}{%
\section{Interactive 3D Plots}\label{interactive-3d-plots}}

\begin{Shaded}
\begin{Highlighting}[]
\CommentTok{\# Installing libraryd packages}
\FunctionTok{library}\NormalTok{(rgl)}
\FunctionTok{library}\NormalTok{(plotly)}
\end{Highlighting}
\end{Shaded}

\begin{Shaded}
\begin{Highlighting}[]
\FunctionTok{library}\NormalTok{(knitr)}
\FunctionTok{library}\NormalTok{(rgl)}
\NormalTok{knitr}\SpecialCharTok{::}\NormalTok{knit\_hooks}\SpecialCharTok{$}\FunctionTok{set}\NormalTok{(}\AttributeTok{webgl =}\NormalTok{ hook\_webgl)}
\end{Highlighting}
\end{Shaded}

In this section we make a 3D version of the earlier Visualizing the most dominant genus on PCoA (see \ref{quality-control}), with the help of the plotly \citep{Sievert2020}.

\begin{Shaded}
\begin{Highlighting}[]
\CommentTok{\# Installing the package}
\FunctionTok{library}\NormalTok{(curatedMetagenomicData)}
\CommentTok{\# Importing necessary libraries}
\FunctionTok{library}\NormalTok{(curatedMetagenomicData)}
\FunctionTok{library}\NormalTok{(dplyr)}
\FunctionTok{library}\NormalTok{(DT)}
\FunctionTok{library}\NormalTok{(mia)}
\FunctionTok{library}\NormalTok{(scater)}

\CommentTok{\# Querying the data}
\NormalTok{tse }\OtherTok{\textless{}{-}}\NormalTok{ sampleMetadata }\SpecialCharTok{\%\textgreater{}\%}
    \FunctionTok{filter}\NormalTok{(age }\SpecialCharTok{\textgreater{}=} \DecValTok{18}\NormalTok{) }\SpecialCharTok{\%\textgreater{}\%} \CommentTok{\# taking only data of age 18 or above}
    \FunctionTok{filter}\NormalTok{(}\SpecialCharTok{!}\FunctionTok{is.na}\NormalTok{(alcohol)) }\SpecialCharTok{\%\textgreater{}\%} \CommentTok{\# excluding missing values}
    \FunctionTok{returnSamples}\NormalTok{(}\StringTok{"relative\_abundance"}\NormalTok{)}

\NormalTok{tse\_Genus }\OtherTok{\textless{}{-}} \FunctionTok{agglomerateByRank}\NormalTok{(tse, }\AttributeTok{rank=}\StringTok{"genus"}\NormalTok{)}
\NormalTok{tse\_Genus }\OtherTok{\textless{}{-}} \FunctionTok{addPerSampleDominantTaxa}\NormalTok{(tse\_Genus,}\AttributeTok{assay.type=}\StringTok{"relative\_abundance"}\NormalTok{, }\AttributeTok{name =} \StringTok{"dominant\_taxa"}\NormalTok{)}

\CommentTok{\# Performing PCoA with Bray{-}Curtis dissimilarity.}
\NormalTok{tse\_Genus }\OtherTok{\textless{}{-}} \FunctionTok{runMDS}\NormalTok{(tse\_Genus, }\AttributeTok{FUN =}\NormalTok{ vegan}\SpecialCharTok{::}\NormalTok{vegdist, }\AttributeTok{ncomponents =} \DecValTok{3}\NormalTok{,}
              \AttributeTok{name =} \StringTok{"PCoA\_BC"}\NormalTok{, }\AttributeTok{assay.type =} \StringTok{"relative\_abundance"}\NormalTok{)}

\CommentTok{\# Getting the 6 top taxa}
\NormalTok{top\_taxa }\OtherTok{\textless{}{-}} \FunctionTok{getTopTaxa}\NormalTok{(tse\_Genus,}\AttributeTok{top =} \DecValTok{6}\NormalTok{, }\AttributeTok{assay.type =} \StringTok{"relative\_abundance"}\NormalTok{)}

\CommentTok{\# Naming all the rest of non top{-}taxa as "Other"}
\NormalTok{most\_abundant }\OtherTok{\textless{}{-}} \FunctionTok{lapply}\NormalTok{(}\FunctionTok{colData}\NormalTok{(tse\_Genus)}\SpecialCharTok{$}\NormalTok{dominant\_taxa,}
                   \ControlFlowTok{function}\NormalTok{(x)\{}\ControlFlowTok{if}\NormalTok{ (x }\SpecialCharTok{\%in\%}\NormalTok{ top\_taxa) \{x\} }\ControlFlowTok{else}\NormalTok{ \{}\StringTok{"Other"}\NormalTok{\}\})}

\CommentTok{\# Storing the previous results as a new column within colData}
\FunctionTok{colData}\NormalTok{(tse\_Genus)}\SpecialCharTok{$}\NormalTok{most\_abundant }\OtherTok{\textless{}{-}} \FunctionTok{as.character}\NormalTok{(most\_abundant)}

\CommentTok{\# Calculating percentage of the most abundant}
\NormalTok{most\_abundant\_freq }\OtherTok{\textless{}{-}} \FunctionTok{table}\NormalTok{(}\FunctionTok{as.character}\NormalTok{(most\_abundant))}
\NormalTok{most\_abundant\_percent }\OtherTok{\textless{}{-}} \FunctionTok{round}\NormalTok{(most\_abundant\_freq}\SpecialCharTok{/}\FunctionTok{sum}\NormalTok{(most\_abundant\_freq)}\SpecialCharTok{*}\DecValTok{100}\NormalTok{, }\DecValTok{1}\NormalTok{)}

\CommentTok{\# Retrieving the explained variance}
\NormalTok{e }\OtherTok{\textless{}{-}} \FunctionTok{attr}\NormalTok{(}\FunctionTok{reducedDim}\NormalTok{(tse\_Genus, }\StringTok{"PCoA\_BC"}\NormalTok{), }\StringTok{"eig"}\NormalTok{);}
\NormalTok{var\_explained }\OtherTok{\textless{}{-}}\NormalTok{ e}\SpecialCharTok{/}\FunctionTok{sum}\NormalTok{(e[e}\SpecialCharTok{\textgreater{}}\DecValTok{0}\NormalTok{])}\SpecialCharTok{*}\DecValTok{100}
\end{Highlighting}
\end{Shaded}

\hypertarget{acknowledgments}{%
\chapter{Acknowledgments}\label{acknowledgments}}

This work would not have been possible without the countless
contributions and interactions with other researchers, developers, and
users. We express our gratitude to the entire Bioconductor community
for developing this high-quality open research software repository for
life science analytics, continuously pushing the limits in emerging
fields \citep[\citet{Huber2015}]{Gentleman2004}. The developers and contributors
of this online tutorial are listed in Chapter \ref{contributors}.

The base ecosystem of data containers, packages, and tutorials was set
up as a collaborative effort by Tuomas Borman, Henrik Eckermann,
Chouaib Benchraka, Chandler Ross, Shigdel Rajesh, Yağmur Şimşek,
Giulio Benedetti, Sudarshan Shetty, Felix Ernst, and \href{http://www.iki.fi/Leo.Lahti}{Leo
Lahti}.

The work has been supported by the COST Action network on Statistical
and Machine Learning Techniques for Human Microbiome Studies
(\href{https://www.ml4microbiome.eu/}{ML4microbiome}) \citep{MorenoIndias2021}.

The framework is based on the \emph{TreeSummarizedExperiment} data
container created by Ruizhu Huang and others
\citep{R-TreeSummarizedExperiment}, and on the MultiAssayExperiment by
Marcel Ramos et al. \citep{Ramos2017}. The idea of using these containers
as a basis for microbiome data science was initially advanced by the
groundwork of Domenick Braccia, Héctor Corrada Bravo and others, and
subsequently brought together with other microbiome data science
developers \citep{Shetty2019}.

Ample demonstration data resources have been made available as the
\href{https://waldronlab.io/curatedMetagenomicData/}{curatedMetagenomicData}
project by Edoardo Pasolli, Lucas Schiffer, Levi Waldron and others
\citep{Pasolli2017} adding important support.
A number of other contributors have advanced the ecosystem
further, and will be acknowledged in the individual
packages, \href{https://github.com/microbiome/OMA/graphs/contributors}{pull
requests},
\href{https://github.com/microbiome/OMA/issues}{issues}, and other work.

The work has drawn inspiration from many sources, most notably from
the work on \emph{phyloseq} by Paul McMurdie and Susan Holmes
\citep{McMurdie2013} who pioneered the work on rigorous and reproducible
microbiome data science ecosystems in R/Bioconductor. The phyloseq
framework continues to provide a vast array of complementary packages
and methods for microbiome studies, and we aim to support full
interoperability.

The open source books by Susan Holmes and Wolfgang Huber, Modern
Statistics for Modern Biology \citep{Holmes2019} and by Garret Grolemund
and Hadley Wickham, the R for Data Science \citep{Grolemund2017}, and
Richard McElreath's Statistical Rethinking and the associated online
resources by Solomon Kurz \citep{McElreath2020} are key references that
advanced reproducible data science training and dissemination. The
Orchestrating Single-Cell Analysis with Bioconductor, or \emph{OSCA} book
by Robert Amezquita, Aaron Lun, Stephanie Hicks, and Raphael Gottardo
\citep{Amezquita2020natmeth} has implemented closely related work on the
\emph{SummarizedExperiment} data container and its derivatives in the field
of single cell sequencing studies. Many approaches used in this book
have been derived from the \href{https://bioconductor.org/books/release/OSCA/}{OSCA
framework}, with various
adjustments and extensions dedicated to microbiome data science.

\hypertarget{sessioninfo}{%
\chapter*{Sessioninfo}\label{sessioninfo}}
\addcontentsline{toc}{chapter}{Sessioninfo}

\begin{Shaded}
\begin{Highlighting}[]
\FunctionTok{sessionInfo}\NormalTok{()}
\end{Highlighting}
\end{Shaded}

\begin{verbatim}
## R version 4.3.0 (2023-04-21)
## Platform: x86_64-pc-linux-gnu (64-bit)
## Running under: Ubuntu 22.04.2 LTS
## 
## Matrix products: default
## 
## 
## locale:
##  [1] LC_CTYPE=en_US.UTF-8       LC_NUMERIC=C              
##  [3] LC_TIME=en_US.UTF-8        LC_COLLATE=en_US.UTF-8    
##  [5] LC_MONETARY=en_US.UTF-8    LC_MESSAGES=en_US.UTF-8   
##  [7] LC_PAPER=en_US.UTF-8       LC_NAME=C                 
##  [9] LC_ADDRESS=C               LC_TELEPHONE=C            
## [11] LC_MEASUREMENT=en_US.UTF-8 LC_IDENTIFICATION=C       
## 
## time zone: UTC
## tzcode source: system (glibc)
## 
## attached base packages:
## [1] stats     graphics  grDevices utils     datasets  methods   base     
## 
## other attached packages:
## [1] BiocStyle_2.28.0 rebook_1.10.1   
## 
## loaded via a namespace (and not attached):
## [1] knitr_1.43      shiny_1.7.4     htmltools_0.5.5 rmarkdown_2.22 
## [5] bookdown_0.34   miniUI_0.1.1.1  tools_4.3.0
\end{verbatim}

  \bibliography{book.bib}

\end{document}
